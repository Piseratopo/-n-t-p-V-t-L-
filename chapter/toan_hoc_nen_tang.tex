\chapter{Kiến thức toán học nền tảng}

\ % Lùi đầu dòng

Chương này bao gồm các kiến thức toán học cần thiết để xây dựng lí thuyết của môn vật lí (hoặc ít nhất để đọc tài liệu này), giả sử rằng bạn đọc đã có một chút kiến thức đại số và hình học trung học phổ thông từ ghế nhà trường. Một điều cần lưu ý là chương này sẽ bao hàm những phần không nằm trong chương trình trung học phổ thông và có thể cả chương trình đại học. Mặc dù rằng là tác giả đã bao hàm rất nhiều toán trong chương, nhưng tác giả không có ý định viết để thay thế toàn bộ giáo trình toán. Các cuốn giải tích, đại số tuyến tính, hình học phẳng, hình học không gian, xác suất, và các cuốn giáo trình toán khác đều có vị trí đứng của chúng. Điều mà tác giả mong muốn tài liệu này có được chính là sự tổng hợp của kiến thức toán sao cho phù hợp với các ngành vật lí và sự bù đắp cho những lỗ hổng mà tác giả còn thấy ở tài liệu toán hiện hành ở Việt Nam. Kể như, trong tài liệu này, khi nhắc về hàm số, không có phần về đơn ánh hay toàn ánh. Những khái niệm này là vô cùng quan trọng nếu tập trung chứng minh chặt chẽ các tính chất liên quan đến hàm số, nhưng không phục vụ nhiều trong ứng dụng thực tiễn. Thay vào đó, tài liệu được đưa thêm những dạng bài tập, như các dạng bài liên quan đến hàm số rời rạc được cho dưới dạng bảng, mà bạn đọc ít khả năng nhìn thấy ở trong những tài liệu khác. Không phải dạng bài tập mới là để bạn đọc trở nên hứng thú hơn, bởi dĩ tác giả khi soạn đáp án còn thấy chán, mà điều quan trọng là tìm ra nguyên nhân từ cái chán đó, và tìm cách chấm dứt triệt để cái chán bằng việc kết nối các bài toán lại với nhau, và rút ra một quy luật tổng quát giữa chúng. Suy cho cùng, sau khi bạn đọc làm nhiều bài tập, tác giả kì vọng, hơn cả việc bạn đọc tính toán nhanh và thành thạo (đương nhiên điều này cũng rất tốt), chính là việc hiểu rõ bản chất của các mảng lí thuyết và từ đó ứng dụng vào các trường hợp khác nhau.

Thông thường, các tài liệu vật lí sẽ lược qua hay tối giản phần toán, với ba ngầm định. Thứ nhất, sẽ có tài liệu toán ứng dụng đi kèm với tài liệu vật lí. Thứ hai, vật lí không dùng nhiều đến lí thuyết toán chuyên sâu hay chứng minh chặt chẽ. Và thứ ba, vật lí không nên dùng đến các tính toán phức tạp mà nên tập trung nhiều vào phần thông hiểu lí thuyết và ứng dụng đời sống. Tuy nhiên, tác giả lại không định hướng tài liệu đi theo những quan điểm này. Các mô hình vật lí đều có toán học phụ trợ đằng sau và chứng minh toán học mới là thứ xây dựng mô hình để dự đoán tương lai. Lấy ví dụ, thuyết tương đối rộng của Anh-xtanh\footnote{Albert Einstein (1879 - 1955)}. Đây là thuyết có thể nói được kiểm chứng thực nghiệm nhiều lần nhất trong vật lí, và giống rất nhiều công trình vật lí hiện đại khác, được xây dựng từ bút, giấy, và nhiều công cụ toán và một chút góc nhìn sáng tạo của vật lí. Quay trở về hiện tại, theo tác giả, nếu như nhà vật lí hay kĩ sư mà không làm được toán cao cấp, thì có lẽ họ nên chuyển nghề. Cho nên, trong tài liệu này, tác giả không chỉ đưa nhiều toán, mà còn đưa ra toán theo con đường khác với con đường thông thường. Các lí thuyết bình thường được đặt ở cùng chỗ thì sẽ tách nhau ra, không phải là cố tình phức tạp hóa, mà là để thể hiện tính mạch lạc của toán, nhấn mạnh rằng toán có thể tư duy được chứ không chỉ là thuộc lòng một cách \dblquote{tôn giáo hóa}. Tác giả vẫn đưa một số lí thuyết dựa trên ngôn ngữ đời thường, nhưng nếu có thể, tác giả sẽ đưa định nghĩa hay chứng minh theo toán học thuần túy, dựa trên những lí thuyết đã có trước đó.

Có thể những kiến thức này đã cũ và bạn đọc chỉ muốn làm nóng lại kiến thức ở những phần cần thiết, thì bạn đọc có thể bỏ qua một vài phần của chương này. Nhưng nếu bạn đọc thấy những kiến thức này còn mới, còn nhiều lỗ hổng, thì bạn đọc nên đọc kĩ lưỡng. Hi vọng từ lí thuyết và bài tập, bạn đọc có thể hiểu được góc nhìn của tác giả về toán, và tự xây dựng cho mình một ma trận kiến thức riêng để phục vụ sau này.

\section{Hệ các số thực}

\ % Lùi đầu dòng

\subsection{Số thực}

\ % Lùi đầu dòng

Phần này đề cập các yếu tố đại số cơ bản của \defText{số thực}, cụ thể là những hệ thức mà trong đó số thực tương tác với một số hữu hạn các \defText{phép cộng} và \defText{phép nhân}. 

Gọi $\defMath{\mathbb{R}}$ là tập hợp số thực. Nếu $a, b, c$ đều thuộc $\mathbb{R}$, với phép cộng và phép nhân mang ý nghĩa thông thường, có:
\begin{itemize}
   \item $a + b$ và $a\times b$ (hay $a\cdot b$, $ab$) đều thuộc $\mathbb{R}$;
   \item $a+b=b+a$ và $ab=ba$ (\defText{tính giao hoán});
   \item $a+(b+c)=(a+b)+c$ và $a(bc)=(ab)c$ (\defText{tính kết hợp});
   \item $a(b+c)=ab+ac$ (\defText{tính phân phối});
   \item $a\times 1 = a$ (\defText{đơn vị});
   \item $a + 0 = a$ và $a\times 0 = 0$ (\defText{số không});
   \item $a + c = b + c \implies a = b$ (\defText{tính giản ước được});
   \item Nếu $c \neq 0$, $ac = bc \implies a = b$ (\defText{tính giản ước được}).
\end{itemize}

Mỗi $a$ chỉ tồn tại một \defText{số đối} $-a$ duy nhất sao cho $a + (-a) = 0$ và nếu $a\neq 0$, tồn tại một \defText{số nghịch đảo} $\frac{1}{a}$ duy nhất sao cho $a\times \frac{1}{a} = 1$. \defText{Phép trừ} được định nghĩa là $$\defMath{a-b = a + (-b)}$$ và \defText{phép chia} được định nghĩa là $$\defMath{\frac{a}{b} = a\times \frac{1}{b}}.$$ Trên tập số thực, không có nghịch đảo của $0$.

\subsection{Số tự nhiên và quy nạp toán học}

\ % Lùi đầu dòng

Một tập số thường xuyên được đề cập ngoài tập số thực là tập \defText{số tự nhiên} $\defMath{\mathbb{N}}$. 
\subsection{Số nguyên và số hữu tỉ}

\ % Lùi đầu dòng

Từ tập số tự nhiên, chúng ta có thể xây dựng tập \defText{số nguyên} bằng việc kết hợp các dạng số, $n$ và $-n$ với $n$ là số tự nhiên nào đó. Kí hiệu tập số nguyên là $\defMath{\mathbb{Z}}$. Trong một vài trường hợp, chúng ta sẽ chỉ quan tâm đến só dương, khi này, có tập số nguyên dương $\defMath{\mathbb{Z}^+}$ hay $\defMath{\mathbb{N}^*}$.

Mở rộng tập số nguyên, các số có dạng $\frac{a}{b}$ với $a$ là số nguyên và $b$ là số nguyên khác $0$ tạo thành tập \defText{số hữu tỉ} kí hiệu là $\defMath{\mathbb{Q}}$.

Để xây dựng số thực thì sẽ cần những khái niệm cao cấp hơn. Một số thực có thể được định nghĩa là giới hạn của một dãy số hữu tỉ. Các số thực không phải số hữu tỉ thì là \defText{số vô tỉ}. Tuy rằng hiện tại chúng ta chưa đề cập đến định nghĩa toán học chặt chẽ của số thực, nhưng khả năng cao là bạn đọc đã có làm quen với nhiều số thực như $\sqrt{2}$ hay $\pi$. Do đó, tác giả sẽ thừa nhận các tính chất của số thực, và sẽ xây dựng lại định nghĩa khi điều kiện cho phép.


% \input{\chapdir toan_hoc_nen_tang/do_thi.tex}
% \section{Hàm số đại số một biến và các phép biến đổi trên hàm}

\ % Lùi đầu dòng

% \input{\chapdir toan_hoc_nen_tang/ham_so_mot_bien/dinh_nghia.tex}
% \input{\chapdir toan_hoc_nen_tang/ham_so_mot_bien/ki_hieu_tong_tich.tex}
% \subsection{Hàm đa thức}

\ % Lùi đầu dòng

Một dạng hàm quen thuộc, được giới thiệu trong chương trình học trung học phổ thông, là đa thức. Nhưng trước khi chúng ta nhắc lại về đa thức, chúng ta sẽ nhắc lại về đơn thức. Hàm \defText{đơn thức}\footnote{Chúng ta sẽ đề cập đến đơn thức và đa thức nhiều biến khi đến phần hàm nhiều biến.} là một hàm được viết dưới dạng $$\defMath{f(x) = ax^n}$$ với $a$ là một số thực và biến $x$ được mũ lên một số nguyên không âm $n$. 

Hàm \defText{đa thức}, như cái tên của nó, được biểu diễn dưới dạng tổng của các đơn thức $$\defMath{f(x)=P_n(x)=\sum_{i = 0}^n \left(a_i x^i\right) = a_nx^n + a_{n-1}x^{n-1} + \cdots + a_1x + a_0}$$ với $n$ là một số nguyên không âm, $a_i$ là các số thực, gọi là các \defText{hệ số}, với mọi $i$ nguyên nằm trong đoạn $[0, n]$ và $a_n \neq 0$. Khi này, $n$ được gọi là \defText{bậc} của đa thức\label{def:ham_so_mot_bien:da_thuc:da_thuc}. Mọi giá trị $x \in \mathbb{R}$ đều thuộc tập xác định của hàm đa thức $f(x)$. Ví dụ:
\begin{itemize}
   \item $f(x) = 2x^2 + 3x + 1$ là một đa thức bậc $2$ với các hệ số $a_2 = 2$, $a_1 = 3$, $a_0 = 1$;
   \item $g(y) = y^3 - 4y$ là một đa thức bậc $3$ với các hệ số $b_3 = 1$, $b_2 = 0$, $b_1 = -4$, $b_0 = 0$;
   \item $h(z) = 5$ là một đa thức bậc $0$ với hệ số $c_0 = 5$;
\end{itemize}
Tính toán một số giá trị mẫu:
\begin{itemize}
   \item $p(1) = 7 \cdot 1^4 - 2 \cdot 1^2 + 9 = 14$ với $q(t)= 7t^4 - 2t^2 + 9$ là một đa thức bậc $4$ với các hệ số $d_4 = 7$, $d_3 = 0$, $d_2 = -2$, $d_1 = 0$, $d_0 = 9$;
   \item $q(2) = -3 \cdot 2 + 8 = 2$ với $q(r) = -3r + 8$ là một đa thức bậc $1$ với các hệ số $e_1 = -3$, $e_0 = 8$.
\end{itemize}
Khi đa thức có bậc bằng $0$, hay $f = P_0 = a_0$, thì được gọi là \defText{đa thức hằng} hay \defText{hàm hằng}. Một trường hợp đặc biệt là khi $f = 0$ (hay $f(x) = 0$ với mọi $x$). Nếu hàm này là đa thức, theo định nghĩa, hàm này chỉ có duy nhất hệ số đầu $a_0 = 0$. Tuy nhiên, cũng theo định nghĩa thì hệ số đầu phải khác $0$. Vì vậy, hàm không có bậc và không được gọi là đa thức. Nhưng, do hàm nhận giá trị cố định với mọi $x$ nên vẫn được gọi là hàm hằng \footnote{Đa số những nhà toán học không coi $f = 0$ là đa thức bậc $0$ do nhiều tính chất của đa thức bị phá vỡ khi gặp trường hợp này. Tuy nhiên, nhiều người vẫn coi $f = 0$ là đa thức không có bậc. Trong tài liệu này, tác giả không coi $0$ là đa thức, nhưng vẫn coi là hàm hằng.}.

\exercise Phác thảo đồ thị của những hàm sau:
\begin{multicols}{2}
   \begin{enumerate}
      \item $f(x) = x + 2$; 
      \item $f(x) = x^2 + 2x + 3$;
      \item $f(x) = -2x^2 + 5x - 6$;
      \item $f(x) = x^3 - 9x^2 + 24x - 16$;
      \item $f(x) = 2$;
      \item $f(x) = 36x^4 + 28x^3 - 3x^2 - 6x - 1$;
      \item $f(x) = -x^6 + x^2 - 4x - 2$;
      \item $f(x) = -x^7 + x$.
   \end{enumerate}
\end{multicols}

\solution

Bạn đọc có thể dùng những phần mềm vẽ đồ thị để nhanh chóng có hình vẽ. Tuy nhiên, nếu không có thiết bị điện tử thì bạn đọc vẫn có thể vẽ đồ thị bằng giấy và bút bằng cách lấy nhiều điểm ví dụ cho $x$ và tính toán giá trị $f(x)$ và sau đó nối chúng lại với nhau.

Bạn đọc có thể để ý rằng là không phải lúc nào cũng đặt gốc tọa độ ở vị trí chính giữa và tỉ lệ xích trên hai trục không phải là giống nhau. Trong nhiều trường hợp, việc ép đặt gốc ở giữa và giữ tỉ lệ giống nhau trên các trục sẽ làm cho đồ thị lệch ra khỏi khu vực vẽ. Điều quan trọng nhất của những bài vẽ đồ thị trong vật lí không chỉ là căn ke chính xác vị trí từng điểm, mà còn là nhận ra được dáng điệu của đồ thị và vị trí tương đối giữa các điểm trên đồ thị đó. Qua đó, chúng ta rút ra được những tính chất toán học cần thiết để phục vụ những yêu cầu cụ thể trong bài tập ứng dụng.

Dưới đây là đồ thị của các hàm đa thức trong bài:

{
   \begin{minipageindent}{0.48\textwidth}
      \begin{figure}[H]
         \centering
         \begin{tikzpicture}
            \draw[->] (-5, 0) -- (1, 0) node[right] {$x$};
            \draw[->] (0, -3) -- (0, 3) node[above] {$f(x)$};
            \draw[graph thickness, color=colorEmphasisCyan] plot[domain=-5:1] (\x, {\x + 2});
            \filldraw[color=colorEmphasisCyan] (0, 2) circle (\pointSize) node[below right] {$\left(0; 2\right)$};
            \filldraw[color=colorEmphasisCyan] (-2, 0) circle (\pointSize) node[below right] {$\left(-2; 0\right)$};
         \end{tikzpicture}
         \caption{Đồ thị của hàm $f(x) = x + 2$}
         \label{fig:ham_so:ham_da_thuc:x_2}
      \end{figure}
   \end{minipageindent}
   \hfill
   \begin{minipageindent}{0.48\textwidth}
      \begin{figure}[H]
         \centering
         \begin{tikzpicture}
            \draw[->] (-4, 0) -- (2, 0) node[right] {$x$};
            \draw[->] (0, 0) -- (0, 6) node[above] {$f(x)$};
            \draw[color=colorEmphasisCyan, graph thickness, smooth, samples=100] plot[domain=-3:1] (\x, {(\x + 1)^2 + 2});
            \filldraw[color=colorEmphasisCyan] (-1, 2) circle (\pointSize) node[below] {$\left(-1; 2\right)$};
            \filldraw[color=colorEmphasisCyan] (0, 3) circle (\pointSize) node[below right] {$\left(0; 3\right)$};
            \filldraw[color=colorEmphasisCyan] (-2, 3) circle (\pointSize) node[left] {$\left(-2; 3\right)$};
         \end{tikzpicture}
         \caption{Đồ thị của hàm $f(x) = x^2 + 2x + 3$}
         \label{fig:ham_so_mot_bien:da_thuc:x2_2x_3}
      \end{figure}
   \end{minipageindent}
}

{
   \begin{minipageindent}{0.48\textwidth}
      \begin{figure}[H]
         \centering
         \begin{tikzpicture}
            \draw[->] (-2, 0) -- (4, 0) node[right] {$x$};
            \draw[->] (0, -5) -- (0, 1) node[above] {$f(x)$};
            \draw[color=colorEmphasisCyan, graph thickness, smooth, samples=100] plot[domain=-0.186:2.686] (\x, {(-2*(\x)^2 + 5*(\x) - 4)});
            \draw[snake it, name path=A] (-2, -0.25) -- (4, -0.25);
            \draw[snake it, name path=B] (-2, -0.35) -- (4, -0.35);
            \tikzfillbetween[of=A and B]{white};
            \foreach \x/\y/\yy/\pos in {0/-4/-6/right, 1/-1/-3/above, 2/-2/-4/right} {
               \filldraw[color=colorEmphasisCyan] (\x, \y) circle (\pointSize) node[\pos] {$\left(\x; {\yy}\right)$};
            }
         \end{tikzpicture}
         \caption{Đồ thị của hàm $f(x) = -2x^2 + 5x - 6$}
         \label{fig:ham_so_mot_bien:da_thuc:t2x2_5x_t6}
      \end{figure}
   \end{minipageindent}
   \hfill
   \begin{minipageindent}{0.48\textwidth}
      \begin{figure}[H]
         \centering
         \begin{tikzpicture}
            \draw[->] (0, 0) -- (6, 0) node[right] {$x$};
            \draw[->] (0, -3) -- (0, 5) node[above] {$f(x)$};
            \draw[color=colorEmphasisCyan, graph thickness, smooth, samples=100] plot[domain=0.508:5.492] (\x, {((\x)^3 - 9*(\x)^2 + 24*(\x) - 16) / 2});
            \filldraw[color=colorEmphasisCyan] (2, 2) circle (\pointSize) node[above] {$\left(2; 4\right)$};
            \filldraw[color=colorEmphasisCyan] (4, 0) circle (\pointSize) node[below] {$\left(4; 0\right)$};
            \filldraw[color=colorEmphasisCyan] (1, 0) circle (\pointSize) node[below right] {$\left(1; 0\right)$};
            \filldraw[color=colorEmphasisCyan] (5, 2) circle (\pointSize) node[right] {$\left(5; 4\right)$};
         \end{tikzpicture}
         \caption{Đồ thị của hàm $f(x) = x^3 - 9x^2 + 24x - 16$}
         \label{fig:ham_so_mot_bien:da_thuc:x3_t9x2_24x_t16}
      \end{figure}
   \end{minipageindent}
}

Ở hình \ref{fig:ham_so_mot_bien:da_thuc:t2x2_5x_t6}, đồ thị có đoạn díc dắc. Ý nghĩa của cách vẽ này là để cắt bỏ phần đồ thị không cần thiết.

{
   \begin{minipageindent}{0.48\textwidth}
      \begin{figure}[H]
         \centering
         \begin{tikzpicture}
            \draw[->] (-3, 0) -- (3, 0) node[right] {$x$};
            \draw[->] (0, -1) -- (0, 3) node[above] {$f(x)$};
            \draw[graph thickness, color=colorEmphasisCyan] plot[domain=-3:3] (\x, {2});
            \filldraw[color=colorEmphasisCyan] (0, 2) circle (\pointSize) node[above left] {$\left(0; 2\right)$};
         \end{tikzpicture}
         \caption{Đồ thị của hàm $f(x) = 2$}
         \label{fig:ham_so:ham_da_thuc:2}
      \end{figure}
   \end{minipageindent}
   \hfill
   \begin{minipageindent}{0.48\textwidth}
      \begin{figure}[H]
         \centering
         \begin{tikzpicture}
            \draw[->] (-3, 0) -- (3, 0) node[right] {$x$};
            \draw[->] (0, -3) -- (0, 3) node[above] {$f(x)$};
            \draw[color=colorEmphasisCyan, graph thickness, smooth, samples=100] plot[domain=-2.491:1.669] (\x, {36*(\x/3)^4 + 28*(\x/3)^3 - 3*(\x/3)^2 - 6*(\x/3) - 1});
            \foreach \x/\y/\xlabel/\ylabel/\pos in {
               -1.5/0/{-0,5}/0/below,
               -0.721/0/{-0,24}/0/above,
               0/-1/0/-1/left,
               0.75/-2.109/{0,25}/{-2,11}/below,
               1.387/0/{0,462}/0/below right,
               -2.357/2/{-0,79}/2/right,
               1.5/1/{0,5}/1/right} {
               \filldraw[color=colorEmphasisCyan] (\x, \y) circle (\pointSize) node[\pos] {$\left(\xlabel;\ylabel\right)$};
            }
         \end{tikzpicture}
         \caption{Đồ thị của hàm $f(x) = 36x^4 + 28x^3 - 3x^2 - 6x - 1$}
         \label{fig:ham_so:ham_da_thuc:36x4_28x3_t3x2_t6x_t1}
      \end{figure}
   \end{minipageindent}
}

{
   \begin{minipageindent}{0.48\textwidth}
      \begin{figure}[H]
         \centering
         \begin{tikzpicture}
            \draw[->] (-3, 0) -- (3, 0) node[right] {$x$};
            \draw[->] (0, -4) -- (0, 2) node[above] {$f(x)$};
            \draw[color=colorEmphasisCyan, graph thickness, smooth, samples=80] plot[domain=-3:2.356] (\x, {(-(\x / 2)^6 + (\x / 2)^2 - 2*(\x) - 2)/2});
            \foreach \x/\y/\xlabel/\ylabel/\pos in {
               -3/-2.570/{-1,5}/{-5,14}/right,
               -2/1/-1/2/above,
               0/-1/0/-1/below left,
               2.356/-4/{1,18}/-8/left} {
               \filldraw[color=colorEmphasisCyan] (\x, \y) circle (\pointSize) node[\pos] {$\left(\xlabel;\ylabel\right)$};
            }
         \end{tikzpicture}
         \caption{Đồ thị của hàm $f(x) = -x^6 + x^2 - 4x - 2$}
         \label{fig:ham_so:ham_da_thuc:tx6_x2_t4x_t2}
      \end{figure}
   \end{minipageindent}
   \hfill
   \begin{minipageindent}{0.48\textwidth}
      \begin{figure}[H]
         \centering
         \begin{tikzpicture}
            \draw[->] (-3, 0) -- (3, 0) node[right] {$x$};
            \draw[->] (0, -3) -- (0, 3) node[above] {$f(x)$};
            \draw[color=colorEmphasisCyan, graph thickness, smooth, samples=80] plot[domain=-1.229:1.229] (\x, {-(\x)^7 + (\x)});
            \foreach \x/\y/\xlabel/\ylabel/\pos in {
               -1/0/-1/0/above left,
               0/0/0/0/below right,
               1/0/1/0/above right,
               0.723/0.62/{0,72}/{0,62}/above,
               -0.723/-0.62/{-0,72}/{-0,62}/below left}{
                  \filldraw[color=colorEmphasisCyan] (\x, \y) circle (\pointSize) node[\pos] {$\left(\xlabel;\ylabel\right)$};
            }
         \end{tikzpicture}
         \caption{Đồ thị của hàm $f(x) = -x^7 + x$}
         \label{fig:ham_so:ham_da_thuc:tx7_x}
      \end{figure}
   \end{minipageindent}
}

\exercise Giải những phương trình sau. Các phương trình đều có ẩn là $x \in \mathbb{R}$.
\begin{multicols}{2}
   \begin{enumerate}
      \item $3x - 7 = 0$;
      \item $x - 9 = 5x + 3$;
      \item $\frac{1}{v}\cdot x - \frac{1}{v} \cdot x_0 = t$, với $v$, $x_0$, $t$ là những tham số thực;
      \item $6x^2 - 5x - 21 = 0$;
      \item $5x^2 - 50x + 125 = 0$;
      \item $x^2 + 2x + 4 = 0$;
      \item $x^2 + 2x + 4 = 8$;
      \item $5x^2 - 20x + 20 = x^2 - 4$;
      \item $\frac{1}{2}kx^2 + \frac{1}{2}mv^2 = \frac{1}{2}kx_0^2$, với $k$, $m$, $v$, $x_0$ là những tham số thực;
      \item $x^3 - \frac{11}{6}\cdot x^2 + x - \frac{1}{6} = 0$;
      \item $2x^3 - 2x^2 + 2x - 2 = 6 + 6x^2$;
      \item $x^4+2x^3-x^2-2x=0$;
      \item $-x^4 -3x^2 = -5$;
      \item $x^4 + 1 = 3x^3 + x^2 + 3x$.
   \end{enumerate}
\end{multicols}

\solution

\setcounter{subexercise}{1}
\arabic{subexercise}. Biến đổi tương đương phương trình để có:
\begin{align*}
   3x - 7 &= 0 \\
   \iff 3x &= 7\\
   \iff x &= \frac{7}{3}.
\end{align*}
Vậy tập nghiệm của phương trình là $\displaystyle\left\{\frac{7}{3}\right\}$.

\stepcounter{subexercise}
\arabic{subexercise}. Chuyển số hạng có thừa số $x$ về một phía, và số hạng tự do về phía còn lại để được:
\begin{align*}
   x - 9 &= 5x + 3 \\
   \iff (x - 9) + (9 - 5x) &= (5x + 3) + (9 - 5x) \\ 
   \iff -4x &= 12 \\
   \iff x &= -3.
\end{align*}
Vậy tập nghiệm của phương trình là $\displaystyle\left\{-3\right\}$.

\stepcounter{subexercise}
\arabic{subexercise}. Để giải phương trình có chứa tham số, chúng ta cần viết lại ẩn $x$ dưới dạng một biểu thức chỉ chứa tham số và hằng số. Cụ thể,
\begin{align*}
   \frac{1}{v}\cdot x - \frac{1}{v} \cdot x_0 &= t \\
   \iff \frac{x}{v} &= t + \frac{x_0}{v} \\
   \iff x &= vt + x_0.
\end{align*}
Vậy nghiệm của phương trình là $\displaystyle\left\{vt + x_0\right\}$.

\stepcounter{subexercise}
\arabic{subexercise}. Nếu như bạn đọc chưa biết, nếu như một đa thức $f(x)$ nhận $x = a$ là nghiệm thì $f(x)$ có thể được viết thành tích của $(x - a)$ nhân một đa thức $g(x)$ với bậc nhỏ hơn $1$ so với $f(x)$. Và nếu $g(x)$ lại có nghiệm $x = b$ thì chúng ta có thể viết $g(x) = (x-b)h(x)$ và qua đó có thể viết lại $f(x) = (x-a)(x-b)h(x)$. Một cách tổng quát nhất, nếu như $f(x)$ là phương trình bậc $n$ có $n$ nghiệm $a_1, a_2, \cdots, a_n$ thì có thể viết lại $$f(x) = A \prod_{i=1}^{n} (x - a_i) = A(x - a_1)(x - a_2)\cdots (x - a_n)$$ với $A$ là hệ số của số hạng có bậc lớn nhất trong đa thức $f(x)$.

Nhẩm nghiệm (bằng cách bấm máy tính) phương trình thì có $x = -\frac{3}{2}$ và $x = \frac{7}{3}$. Chúng ta kì vọng có thể viết lại phương trình dưới dạng $6\left(x - \left(-\frac{3}{2}\right)\right)\left(x - \frac{7}{3}\right) = 0$. Thực vậy, thực hiện phân tích nhân tử để có:
\begin{align*}
   &6x^2 - 5x - 21 = 0 \\
   \iff &6x^2 - 14x + 9x - 21 = 0 \\
   \iff &2x(3x - 7) + 3(3x - 7) = 0 \\
   \iff &(2x + 3)(3x - 7) = 0 \\
   \iff &\left[
      \begin{aligned}
         2x + 3 &= 0 \\
         3x - 7 &= 0
      \end{aligned}
   \right.
   \iff \left[
      \begin{aligned}
         x &= -\frac{3}{2} \\
         x &= \frac{7}{3}
      \end{aligned}
   \right..
\end{align*}
Vậy phương trình có nghiệm là $\displaystyle\left\{-\frac{3}{2}; \frac{7}{3}\right\}$.

\stepcounter{subexercise}
\arabic{subexercise}.
\begin{align*}
   5x^2 - 50x + 125 &= 0 \\
   \iff 5\left(x^2 - 10x + 25\right) &= 0 \\
   \iff 5(x - 5)^2 &= 0 \\
   \iff x - 5 &= 0 \\
   \iff x &= 5.
\end{align*}

Vậy tập nghiệm của phương trình có một phần tử duy nhất $\displaystyle\left\{5\right\}$.

\stepcounter{subexercise}
\arabic{subexercise}. Với những phương trình liên quan tới đa thức bậc hai không thể nhẩm ngay được nghiệm, chúng ta sẽ sử dụng phương pháp tách bình phương. Với phương trình được cho:
\begin{align}
   x^2 + 2x + 4 &= 0 \nonumber\\ 
   \iff x^2 + 2x + 1 &= -3 \nonumber\\
   \iff (x + 1)^2 &= -3. \label{eq:ham_so_mot_bien:ham_da_thuc:gptdt6}
\end{align}
Một số thực nhân với chính nó sẽ ra một số không âm. Cho nên phương trình \ref{eq:ham_so_mot_bien:ham_da_thuc:gptdt6} không thể đúng. Vậy phương trình vô nghiệm trên tập số thực.

\stepcounter{subexercise}
\arabic{subexercise}. 
\begin{align*}
   &x^2 + 2x + 4 = 8 \\ 
   \iff &x^2 + 2x + 1 = 5 \\
   \iff &(x + 1)^2 = 5 \\
   \iff &\left[
      \begin{aligned}
         x + 1 &= \sqrt{5} \\
         x + 1 &= -\sqrt{5}
      \end{aligned}
   \right. \\
   \iff &\left[
      \begin{aligned}
         x &= \sqrt{5} - 1 \\
         x &= -\sqrt{5} - 1
      \end{aligned}
   \right..
\end{align*}
Vậy tập nghiệm của phương trình là $\displaystyle\left\{\sqrt{5} - 1; -\sqrt{5} - 1\right\}$.

\stepcounter{subexercise}
\arabic{subexercise}. Phần này tác giả làm khác so với phần 2. Chuyển đổi toàn bộ phương trình về một vế để đưa về dạng phương trình $f(x) = 0$:
\begin{align*}
   &5x^2 - 20x + 20 = x^2 - 4 \\
   \iff &4x^2 - 20x + 24 = 0 \\
   \iff &4\left(x^2 - 5x + 6\right) = 0 \\
   \iff &4\left(x^2 - 2x - 3x + 6\right) = 0 \\
   \iff &4\left(x(x - 2) - 3(x - 2)\right) = 0 \\
   \iff &4(x - 3)(x - 2) = 0 \\
   \iff &\left[
      \begin{aligned}
         x - 3 &= 0 \\
         x - 2 &= 0
      \end{aligned}
   \right. \iff x \in \left\{3; 2\right\}. 
\end{align*}
Vậy phương trình có tập nghiệm $\left\{3; 2\right\}$.

\stepcounter{subexercise}
\arabic{subexercise}. Nhân cả hai vế với $2$ để khử phân số trong phương trình:
\begin{align}
   &\frac{1}{2}kx^2 + \frac{1}{2}mv^2 = \frac{1}{2}kx_0^2 \nonumber \\
   \iff &kx^2 + mv^2 = kx_0^2. \label{eq:ham_so_mot_bien:ham_da_thuc:gptdt9}
\end{align}
Xong, thực hiện chuyển vế để giữ thừa số chứa $x^2$ ở một bên, phương trình \ref{eq:ham_so_mot_bien:ham_da_thuc:gptdt9} tương đương với
\begin{align*}
   (\ref{eq:ham_so_mot_bien:ham_da_thuc:gptdt9}) \iff &kx^2 = kx_0^2 - mv^2 \\
   \iff & x^2 = x_0^2 - \frac{mv^2}{k}.
\end{align*}

Với trường hợp $x_0^2 - \frac{mv^2}{k} < 0$ thì phương trình vô nghiệm do $x^2$ không thể âm. Trong trường hợp còn lại, lấy căn bậc hai hai vế để có $$x\in\left\{\sqrt{x_0^2 - \frac{mv^2}{k}}; -\sqrt{x_0^2 - \frac{mv^2}{k}}\right\}.$$ Tại giá trị đặc biệt mà khi $x_0^2 = \frac{mv^2}{k}$ thì tập nghiệm suy biến thành $\left\{0\right\}$.

Vậy, phương trình có nghiệm là 
$$
\begin{cases}
   \left\{\sqrt{x_0^2 - \frac{mv^2}{k}}; -\sqrt{x_0^2 - \frac{mv^2}{k}}\right\} &\text{ nếu } x_0^2 - \frac{mv^2}{k} \geq 0 \\
   \emptyset &\text{ nếu } x_0^2 - \frac{mv^2}{k} < 0
\end{cases}.
$$

\stepcounter{subexercise}
\arabic{subexercise}. Phân tích thừa số với để ý rằng $1$, $\frac{1}{2}$ và $\frac{1}{3}$ là nghiệm:
\begin{align*}
   &x^3 - \frac{11}{6}\cdot x^2 + x - \frac{1}{6} = 0 \\
   \iff &x^3 - x^2 - \frac{5}{6}x^2 + \frac{5}{6}x + \frac{1}{6}x - \frac{1}{6} = 0 \\
   \iff &x^2\left(x - 1\right) - \frac{5}{6}x\left(x - 1\right) + \frac{1}{6}\left(x - 1\right) = 0 \\
   \iff &\left(x - 1\right)\left(x^2 - \frac{5}{6}x + \frac{1}{6}\right) = 0 \\
   \iff &\left(x - 1\right)\left(x^2 - \frac{1}{2}x - \frac{1}{3}x + \frac{1}{6}\right) = 0 \\
   \iff &\left(x - 1\right)\left(x\left(x - \frac{1}{2}\right) - \frac{1}{3}\left(x - \frac{1}{2}\right)\right) = 0 \\
   \iff &\left(x - 1\right)\left(x - \frac{1}{2}\right)\left(x - \frac{1}{3}\right) = 0
\end{align*}
\begin{align*}
   \iff &\left[
      \begin{aligned}
         x - 1 &= 0 \\
         x - \frac{1}{2} &= 0 \\
         x - \frac{1}{3} &= 0
      \end{aligned}
   \right. \\
   \iff &\left[
      \begin{aligned}
         x &= 1 \\
         x &= \frac{1}{2} \\
         x &= \frac{1}{3}
      \end{aligned}
   \right..
\end{align*}
Cuối cùng, như chúng ta đã dự đoán, phương trình có nghiệm là $\displaystyle\left\{1; \frac{1}{2}; \frac{1}{3}\right\}$.

\stepcounter{subexercise}
\arabic{subexercise}. Có một cách là chuyển phương trình về một vế rồi nhẩm nghiệm. Dưới đây, tác giả sẽ trình bày một góc nhìn khác để giải bài toán này.
\begin{align}
   2x^3 - 2x^2 + 2x - 2 &= 6 + 6x^2 \nonumber\\
   \iff \left(2x^3 + 2x\right) - \left(2x^2 + 2\right) &= 6x^2 + 6 \nonumber\\
   \iff 2x\left(x^2 + 1\right) - 2\left(x^2 + 1\right) &= 6\left(x^2 + 1\right) \nonumber\\
   \iff \left(2x - 2\right)\left(x^2 + 1\right) &= 6\left(x^2 + 1\right). \label{eq:ham_so_mot_bien:ham_da_thuc:gptdt11}
\end{align}
Để ý rằng, do $x^2 \geq 0$ nên $x^2 + 1 \geq 1 > 0$. Chúng ta đã chỉ ra rằng $x^2 + 1 \neq 0$, và qua đó, chúng ta có thể an toàn chia hai vế của \ref{eq:ham_so_mot_bien:ham_da_thuc:gptdt11} cho $x^2 + 1$ để có:
\begin{align*}
   2x - 2 &= 6 \\
   \iff x &= 4.
\end{align*}
Vậy phương trình có nghiệm là $\displaystyle\left\{4\right\}$.

\stepcounter{subexercise}
\arabic{subexercise}. Dễ dàng thấy được có thể phân tích thừa số của đa thức được cho như sau:
\begin{align*}
   x^4 + 2x^3 - x^2 -2x &= 0 \\
   \iff x(x^3 + 2x^2 - x - 2) &= 0 \\
   \iff x(x + 2)(x^2 - 1) &= 0 \\
   \iff x(x + 2)(x - 1)(x + 1) &= 0.
\end{align*}

Chia làm $4$ trường hợp: $\left[
   \begin{aligned}
      x &= 0 \\
      x + 2&= 0 \\
      x - 1 &= 0 \\
      x + 1 &= 0
   \end{aligned}
\right.$ và giải từng trường hợp một để có $x \in \left\{0; -2; 1; -1\right\}$.

Vậy phương trình có bộ nghiệm là $\displaystyle\left\{0; -2; 1; -1\right\}$.

\def\varPhu {\textit{付}}

\stepcounter{subexercise}
\arabic{subexercise}. Đặt $\varPhu = x^2$ để đưa từ đa thức bậc bốn về đa thức bậc hai như sau:
\begin{align*}
   -x^4 - 3x^2 &= -5 \\
   \iff -\varPhu^2 - 3\varPhu &= -5 \\
   \iff \varPhu^2 + 3\varPhu - 5 &= 0.
\end{align*}

Giải phương trình bậc hai này bằng công thức nghiệm, chúng ta có:
$$
\left[
   \begin{aligned}
      \varPhu &= \frac{-3 + \sqrt{3^2 - 4 \cdot (-5)}}{2} = \frac{-3 + \sqrt{29}}{2} \\
      \varPhu &= \frac{-3 - \sqrt{3^2 - 4 \cdot (-5)}}{2} = \frac{-3 - \sqrt{29}}{2}
   \end{aligned}
\right..
$$ Tuy nhiên, do $\varPhu = x^2 \geq 0$ nên $\varPhu$ chỉ có thể nhận giá trị $\frac{-3 + \sqrt{29}}{2}$, và qua đó $x^2 = \frac{-3 + \sqrt{29}}{2}$. Vậy phương trình có nghiệm là $x \in \left\{\sqrt{\frac{-3 + \sqrt{29}}{2}}; -\sqrt{\frac{-3 + \sqrt{29}}{2}}\right\}$.

\stepcounter{subexercise}
\arabic{subexercise}. Bài tập này dành cho những bạn chuyên toán thuần hơn là về ứng dụng. Nhận thấy rằng $x = 0$ không là nghiệm của phương trình. Chia cả hai vế của phương trình cho $x^2 \neq 0$ để có 
\begin{equation}
   x^2 + \frac{1}{x^2} = 3x + 1 + \frac{3}{x}. 
   \label{eq:ham_so_mot_bien:da_thuc:gptdt14}
\end{equation}
Đặt $y = x + \frac{1}{x}$, bình phương hai vế để có $y^2 = \left(x + \frac{1}{x}\right)^2 = x^2 + \frac{1}{x^2} + 2$ hay $x^2 + \frac{1}{x^2} = y^2 - 2$. Thay vào \ref{eq:ham_so_mot_bien:da_thuc:gptdt14}, chúng ta có:
\begin{align*}
   y^2 - 2 &= 3y + 1 \\
   \iff y^2 - 3y - 3 &= 0.
\end{align*}
Giải phương trình này bằng công thức nghiệm:
\begin{equation}   
   \left[
      \begin{aligned}
         y &= \frac{3 + \sqrt{3^2 - 4 \cdot (-3)}}{2} = \frac{3 + \sqrt{21}}{2} \\
         y &= \frac{3 - \sqrt{3^2 - 4 \cdot (-3)}}{2} = \frac{3 - \sqrt{21}}{2}
      \end{aligned}
   \right..\label{eq:ham_so_mot_bien:da_thuc:gpt14y}
\end{equation}

Giải phương trình $x + \frac{1}{x} = y$. Nếu chúng ta nhân cả tử và mẫu với $x$, chúng ta sẽ có phương trình với đa thức bậc hai theo ẩn $x$: $$x^2 - yx + 1 = 0.$$ Cũng dùng công thức nghiệm để giải phương trình này để được:
$$
\left[
   \begin{aligned}
      x &= \frac{y + \sqrt{y^2 - 4}}{2} \\
      x &= \frac{y - \sqrt{y^2 - 4}}{2}
   \end{aligned}
\right..
$$

Do $x \in \mathbb{R}$ nên giá trị trong dấu khai căn $y^2 - 4$ phải không nhỏ hơn $0$. Kiểm tra hai giá trị $y$ tìm được từ \ref{eq:ham_so_mot_bien:da_thuc:gpt14y}, chúng ta thấy $y = \frac{3 + \sqrt{21}}{2}$ thỏa mãn điều kiện này. Thay thế trực tiếp để tìm được tập nghiệm của phương trình. Cuối cùng, chúng ta có được $x \in \displaystyle\left\{\frac{3+\sqrt{21}+\sqrt{14 + 6\sqrt{21}}}{2}; \frac{3+\sqrt{21}-\sqrt{14 + 6\sqrt{21}}}{2}\right\}$.

\exercise Giải các bất phương trình sau. Các bất phương trình đều có ẩn là $x \in \mathbb{R}$.

\begin{multicols}{2}
   \begin{enumerate}
      \item $4x + 7 < 0$;
      \item $-8x - 16 > 0$;
      \item $x - 2 \geq -2x + 5$;
      \item $x^2 + 6x + 10 \leq 0$;
      \item $x^2 - 11x + 30 > 0$;
      \item $-3x^2 + 15x - 12 > -2x^2 + 10x - 12$;
      \item $-x^4 + 18x^2 - 77 \geq 0$;
      \item $x^7 \geq x$.
   \end{enumerate}
\end{multicols}

\solution

{ 
   \begin{minipageindent}{0.55\textwidth}
      \setcounter{subexercise}{1}
      \arabic{subexercise}. 

      \begin{align*}
         4x + 7 &< 0 \\
         \iff 4x &< -7 \\
         \iff x &< -\frac{7}{4}.
      \end{align*}
      
      Tập nghiệm của bất phương trình là $\displaystyle\left(-\infty; -\frac{7}{4}\right)$.

      Để kiểm chứng lại kết quả, chúng ta sẽ sử dụng đồ thị của hàm số $f(x) = 4x + 7$ và kiểm tra xem những điểm nào trên đồ thị có giá trị nhỏ hơn $0$ như hình \ref{fig:ham_so_mot_bien:da_thuc:gbpt1}. Chúng ta không nhận giá trị tại nút nên tại điểm giao với trục hoành, chúng ta vẽ đường tròn rỗng thay vì hình tròn.
   \end{minipageindent}
   \hfill
   \begin{minipageindent}{0.4\textwidth}
      \begin{figure}[H]
         \centering
         \begin{tikzpicture}
            \draw[->] (-4, 0) -- (0, 0) node[right] {$x$};
            \draw[->] (0, -3) -- (0, 3) node[above] {$y$};
            \draw[graph thickness, color=colorEmphasisCyan, dashed, domain=-1.75:-1] plot (\x, {4*\x + 7});
            \filldraw[color=colorEmphasisCyan] (-1.5, 1) circle (\pointSize) node[right] {$\left(-\frac{3}{2}; 1\right)$};
            \draw[graph thickness, color=colorEmphasisCyan, domain=-2.5:-1.75] plot (\x, {4*\x + 7});
            \filldraw[color=colorEmphasisCyan] (-2, -1) circle (\pointSize) node[left] {$\left(-2; -1\right)$};
            \draw[color=colorEmphasisCyan, hollow point] (-1.75, 0) circle (\pointSize) node[below right] {$\left(-\frac{7}{4}; 0\right)$};
         \end{tikzpicture}
         \caption{Đồ thị biểu diễn $4x + 7$ và khoảng nhỏ hơn $0$}
         \label{fig:ham_so_mot_bien:da_thuc:gbpt1}
      \end{figure}
   \end{minipageindent}
}

{
   \begin{minipageindent}{0.55\textwidth}
      \stepcounter{subexercise}
      \arabic{subexercise}.

      \begin{align*}
         &-8x - 16 > 0 \\
         \iff &-8x > 16 \\
         \iff &x < -2.\qquad \text{\textcolor{colorEmphasis}{\parbox[c]{0.4\textwidth}{(Chia cho số âm thì đổi dấu bất phương trình.)}}}
      \end{align*}
      
      Tập nghiệm của bất phương trình là $\displaystyle\left(-\infty; -2\right)$.
   \end{minipageindent}
   \hfill
   \begin{minipageindent}{0.4\textwidth}
      \begin{figure}[H]
         \centering
         \begin{tikzpicture}
            \draw[->] (-3, 0) -- (3, 0) node[right] {$x$};
            \draw[->] (0, -4) -- (0, 1) node[above] {$y$};
            \draw[graph thickness, color=colorEmphasisCyan, domain=-3:-2] plot (\x, {-(\x) - 2});
            \draw[graph thickness, color=colorEmphasisCyan, dashed, domain=-2:2] plot (\x, {-(\x) - 2});
            \draw[color=colorEmphasisCyan, graph thickness, fill=white] (-2, 0) circle (\pointSize) node[above right] {$\left(-2; 0\right)$};
            \filldraw[color=colorEmphasisCyan] (0, -2) circle (\pointSize) node[above right] {$\left(0; -16\right)$};
         \end{tikzpicture}
         \label{fig:ham_so_mot_bien:da_thuc:gbpt2}
         \caption{Đồ thị biểu diễn $-8x - 16$ và khoảng lớn hơn $0$}
      \end{figure}
   \end{minipageindent}
}

{
   \begin{minipageindent}{0.55\textwidth}
      \stepcounter{subexercise}
      \arabic{subexercise}.
      
      \begin{align*}
         x - 2 &\geq -2x + 5 \\
         \iff x + 2x &\geq 5 + 2 \\
         \iff 3x &\geq 7 \\
         \iff x &\geq \frac{7}{3}.
      \end{align*}
      
      Tập nghiệm của bất phương trình là $\displaystyle\left[\frac{7}{3}; \infty\right)$.
   \end{minipageindent}
   \hfill
   \begin{minipageindent}{0.4\textwidth}
      \begin{figure}[H]
         \centering
         \begin{tikzpicture}
            \draw[->] (0, 0) -- (4, 0) node[right] {$x$};
            \draw[->] (0, -2) -- (0, 3) node[above] {$y$};
            \draw[graph thickness, color=colorEmphasis, domain=2.333:4] plot (\x, {((\x) - 2) / 2});
            \draw[graph thickness, color=colorEmphasis, dashed, domain=0:2.333] plot (\x, {((\x) - 2) / 2});
            \draw[graph thickness, color=colorEmphasisCyan, domain=2.333:4] plot (\x, {(-2 * (\x) + 5) / 2});
            \draw[graph thickness, color=colorEmphasisCyan, dashed, domain=0:2.333] plot (\x, {(-2 * (\x) + 5) / 2});
            \filldraw[color=colorEmphasisGreen] (2.333, {0.333 / 2}) circle (\pointSize);
            \node[right, color=colorEmphasisGreen] at (2.4, 0.1) {$\left(\frac{7}{3}; \frac{1}{3}\right)$};
            \filldraw[color=colorEmphasisCyan] (0, {5 / 2}) circle (\pointSize) node[right] {$\left(0; \frac{5}{2}\right)$};
            \filldraw[color=colorEmphasis] (0, -1) circle (\pointSize) node[below right] {$\left(0; -2\right)$};
         \end{tikzpicture}
         \caption{Đồ thị của $x - 2 \geq -2x + 5$}
         \label{fig:ham_so_mot_bien:da_thuc:gbpt3}
      \end{figure}
   \end{minipageindent}
}

\stepcounter{subexercise}
\arabic{subexercise}. Để ý rằng $x^2 + 6x + 10 = \left(x^2 + 6x + 9\right) + 1 = (x + 3)^2 + 1 \implies x^2 + 6x + 10 \geq 1$. Do đó, $x^2 + 6x + 10 \leq 0$ không có nghiệm.

{
   \begin{minipageindent}{0.55\textwidth}
      \stepcounter{subexercise}
      \arabic{subexercise}.

      \begin{align}
         x^2 - 11x + 30 &> 0 \nonumber\\
         \iff (x - 5)(x - 6) &> 0.\qquad \parbox[c]{0.35\textwidth}{\textcolor{colorEmphasis}{(Phân tích đa thức thành nhân tử.)}} \label{eq:ham_so_mot_bien:da_thuc:gbpt5}
      \end{align}

      Do \ref{eq:ham_so_mot_bien:da_thuc:gbpt5}, nên nghiệm của $(x - 5)(x - 6) = 0$ hay $x \in \left\{5; 6\right\}$ không là nghiệm của \ref{eq:ham_so_mot_bien:da_thuc:gbpt5}.

      Với $5 < x < 6$, $
      \begin{cases}
         x - 5 > 0 \\
         x - 6 < 0
      \end{cases}
      $ cho nên $(x - 5)(x - 6) < 0$. Trường hợp này không thỏa mãn \ref{eq:ham_so_mot_bien:da_thuc:gbpt5}.

      Với $x < 5$, $
      \begin{cases}
         x - 5 < 0 \\
         x - 6 < 0
      \end{cases}
      $ cho nên $(x-5)(x-6) > 0$ \textcolor{colorEmphasis}{(Tích hai số âm là một số dương.)} thỏa mãn \eqref{eq:ham_so_mot_bien:da_thuc:gbpt5}.

      Với $x > 6$, $
      \begin{cases}
         x - 5 > 0 \\
         x - 6 > 0
      \end{cases}
      \implies (x-5)(x-6) > 0$ thỏa mãn \eqref{eq:ham_so_mot_bien:da_thuc:gbpt5}.

      Vậy tập nghiệm của bất phương trình này là $\left(-\infty; 5\right) \cup \left(6; \infty\right)$.
   \end{minipageindent}
   \hfill
   \begin{minipageindent}{0.4\textwidth}
      \begin{figure}[H]
         \centering
         \begin{tikzpicture}
            \draw[->] (0, 0) -- (6, 0) node[right] {$x$};
            \draw[->] (0, -1) -- (0, 5) node[above] {$y$};
            \draw[snake it, name path=A] (0.25, -1) -- (0.25, 5);
            \draw[snake it, name path=B] (0.35, -1) -- (0.35, 5);
            \tikzfillbetween[of=A and B]{white}
            \draw[graph thickness, color=colorEmphasisCyan, domain=0.709:2.5] plot (\x, {(\x - 2.5) * (\x - 3.5)});
            \draw[graph thickness, color=colorEmphasisCyan, domain=3.5:5.291] plot (\x, {(\x - 2.5) * (\x - 3.5)});
            \draw[graph thickness, dashed, color=colorEmphasisCyan, domain=2.5:3.5] plot (\x, {(\x - 2.5) * (\x - 3.5)});
            \draw[color=colorEmphasisCyan, hollow point] (2.5, 0) circle (\pointSize) node[below left] {$\left(5; 0\right)$};
            \draw[color=colorEmphasisCyan, hollow point] (3.5, 0) circle (\pointSize) node[below right] {$\left(6; 0\right)$};
         \end{tikzpicture}
         \caption{Đồ thị của $x^2 - 11x +30 > 0$}
         \label{fig:ham_so_mot_bien:da_thuc:gbpt5}
      \end{figure}
   \end{minipageindent}
}

Khi bạn đọc đã quen xét trường hợp thì có thể viết ngắn gọn lại dưới dạng bảng xét dấu như bảng \ref{tab:ham_so_mot_bien:da_thuc:gbpt5}.

\begin{table}[H]
   \centering
   \begin{tabular}{|c|ccccccc|}
   \hline
   $x$          & $-\infty$ &     & $5$ &     & $6$ &   & $\infty$ \\
   \hline
   $x-5$        &           & $-$ &  0  &  +  &     & + &           \\
   \hline
   $x-6$        &           & $-$ &     & $-$ &  0  & + &           \\
   \hline
   $(x-5)(x-6)$ &           &  +  &  0  & $-$ &  0  & + &           \\
   \hline
   \end{tabular}
   \caption{Bảng xét dấu cho $(x - 5)(x - 6)$}
   \label{tab:ham_so_mot_bien:da_thuc:gbpt5}
\end{table}

{
   \begin{minipageindent}{0.55\textwidth}
      \stepcounter{subexercise}
      \arabic{subexercise}.
      
      \begin{align}
         -3x^2 + 15x - 12 &> -2x^2 + 10x - 12 \nonumber\\
         \iff -3x^2 + 2x^2 + 15x - 10x &> -12 + 12 \nonumber\\
         \iff -x^2 + 5x &> 0 \nonumber\\
         \iff x\left(-x + 5\right) &> 0 \nonumber\\
         \iff x\left(x - 5\right) &< 0. \label{eq:ham_so_mot_bien:da_thuc:gbpt6}
      \end{align}

      Kẻ bảng xét dấu:

      \begin{figure}[H]
         \centering
         \begin{tabular}{|c|ccccccc|}
            \hline
            $x$      & $-\infty$ &     & $0$ &     & $5$ &   & $\infty$ \\
            \hline
            $x$      &           & $-$ &  0  &  +  &     & + &           \\
            \hline
            $x-5$    &           & $-$ &     & $-$ &  0  & + &           \\
            \hline
            $x(x-5)$ &           &  +  &  0  & $-$ &  0  & + &           \\
            \hline
         \end{tabular}
         \caption{Bảng xét dấu cho $x(x-5)$}
         \label{tab:ham_so_mot_bien:da_thuc:gbpt6}
      \end{figure}

      Vậy nghiệm của \ref{eq:ham_so_mot_bien:da_thuc:gbpt6} là $x \in \left(0; 5\right)$.
   \end{minipageindent}
   \hfill
   \begin{minipageindent}{0.4\textwidth}
      \begin{figure}[H]
         \centering
         \begin{tikzpicture}
            \draw[->] (-1, 0) -- (5.5, 0) node[right] {$x$};
            \draw[->] (0, -5) -- (0, 3) node[above] {$y$};
            \draw[graph thickness, color=colorEmphasisCyan, dashed, domain=-0.193:0] plot (\x, {(-3*(\x)^2 + 15*(\x) - 12) / 3});
            \draw[graph thickness, color=colorEmphasisCyan, dashed, domain=5:5.193] plot (\x, {(-3*(\x)^2 + 15*(\x) - 12) / 3});
            \draw[graph thickness, color=colorEmphasisCyan, domain=0:5] plot (\x, {(-3*(\x)^2 + 15*(\x) - 12) / 3});
            \draw[graph thickness, color=colorEmphasis, dashed, domain=-0.284:0] plot (\x, {(-2*(\x)^2 + 10*(\x) - 12) / 3});
            \draw[graph thickness, color=colorEmphasis, dashed, domain=5:5.284] plot (\x, {(-2*(\x)^2 + 10*(\x) - 12) / 3});
            \draw[graph thickness, color=colorEmphasis, domain=0:5] plot (\x, {(-2*(\x)^2 + 10*(\x) - 12) / 3});

            \draw[color=colorEmphasisGreen, hollow point] (0, -4) circle (\pointSize) node[right] {$\left(0; -12\right)$};
            \draw[color=colorEmphasisGreen, hollow point] (5, -4) circle (\pointSize) node[left] {$\left(5; -12\right)$};
            \node[color=colorEmphasis, above] at (2.5, 0.1667) {$-2x^2 + 10x - 12$};
            \node[color=colorEmphasisCyan, above] at (2.5, 2.25) {$-3x^2 + 15x - 12$};
         \end{tikzpicture}
         \caption{Đồ thị của $-3x^2 + 15x - 12 > -2x^2 + 10x - 12$}
         \label{fig:ham_so_mot_bien:da_thuc:gbpt6}
      \end{figure}
   \end{minipageindent}
}

\stepcounter{subexercise}
\arabic{subexercise}. Có thể coi đa thức bậc $4$ này là một đa thức bậc $2$ với ẩn là $x^2$. Thực hiện phân tích nhân tử để có

\begin{align*}
   &-x^4 + 18x^2 - 77 \geq 0 \\
   \iff &x^4 - 18x^2 + 77 \leq 0 
   \qquad
   \parbox[c]{\widthof{(Nhân $-1$ ở cả hai vế.)}}{
      \begin{varwidth}{\linewidth}
         \textcolor{colorEmphasis}{(Nhân $-1$ ở cả hai vế.)}
      \end{varwidth}
   } \\
   \iff &\left(x^2 - 7\right)\left(x^2 - 11\right) \leq 0 \\
   \iff &\left(x - \sqrt{7}\right)\left(x + \sqrt{7}\right)\left(x - \sqrt{11}\right)\left(x + \sqrt{11}\right) \leq 0.
\end{align*}

Kẻ bảng xét dấu

\begin{figure}[H]
   \centering
   \begin{tabular}{|c|ccccccccccc|}
      \hline
      $x$                   & $-\infty$ &   & $-\sqrt{11}$ &     & $-\sqrt{7}$ &     & $\sqrt{7}$ &     & $\sqrt{11}$ &   & $\infty$ \\
      \hline
      $x-\sqrt{7}$          &           & $-$ &              & $-$ &             & $-$ &     0      &  +  &             & + &           \\
      \hline
      $x+\sqrt{7}$          &           & $-$ &              & $-$ &      0       & $+$ &          &  +  &             & + &           \\
      \hline
      $x-\sqrt{11}$          &           & $-$ &             & $-$ &             & $-$ &            & $-$ &      0      & + &           \\
      \hline
      $x+\sqrt{11}$          &           & $-$ &       0      & $+$ &             & $+$ &            & $+$ &            & + &           \\
      \hline
      $x^4 - 18x^2 + 77$ &           & + &      0       & $-$ &      0      &  +  &     0      & $-$ &      0      & + &           \\
      \hline
      \end{tabular}
   \caption{Bảng xét dấu cho $x^4 - 18x^2 + 77$}
   \label{tab:ham_so_mot_bien:da_thuc:gbpt6}
\end{figure}

Từ bảng, chúng ta có tập nghiệm của bất phương trình là $\left[-\sqrt{11}; -\sqrt{7}\right] \cup \left[\sqrt{7}; \sqrt{11}\right]$.

{
   \begin{minipageindent}{0.55\textwidth}
      \stepcounter{subexercise}
      \arabic{subexercise}.

      \begin{align}
         &x^7 \geq x \nonumber\\
         \iff &x^7 - x \geq 0  \nonumber\\
         \iff &x(x^6 - 1) \geq 0 \nonumber\\
         \iff &x(x^3 - 1)(x^3 + 1) \geq 0 \nonumber\\
         \iff &x(x-1)(x^2 + x + 1)(x+1)(x^2 + x + 1) \geq 0. \label{eq:ham_so_mot_bien:da_thuc:gbpt8}
      \end{align}

      Cần phải để ý rằng $x^2 + x + 1 = \left(x + \frac{1}{2}\right)^2 + \frac{3}{4} \geq \frac{3}{4}$ và $x^2 - x + 1 = \left(x - \frac{1}{2}\right)^2 + \frac{3}{4} \geq \frac{3}{4}$ đều là hai số dương. Chia cả hai vế của \ref{eq:ham_so_mot_bien:da_thuc:gbpt8} cho hai số dương, chúng ta có:

      \begin{equation}
         \text{(\ref{eq:ham_so_mot_bien:da_thuc:gbpt8})} \iff x(x-1)(x + 1) \geq 0. \label{eq:ham_so_mot_bien:da_thuc:gbpt8_2}
      \end{equation}
   \end{minipageindent}
\hfill
   \begin{minipageindent}{0.4\textwidth}
      \begin{figure}[H]
         \centering
         \begin{tikzpicture}
            \draw[->] (-3, 0) -- (3, 0) node[right] {$x$};
            \draw[->] (0, -3) -- (0, 3)  node[above] {$y$};

            \draw[graph thickness, color=colorEmphasisCyan, dashed, domain=-1.170:-1] plot (\x, {(\x)^7});
            \draw[graph thickness, color=colorEmphasis, dashed, domain=-3:-1] plot (\x, {(\x)});

            \draw[graph thickness, color=colorEmphasisCyan, domain=-1:0] plot (\x, {(\x)^7});
            \draw[graph thickness, color=colorEmphasis, domain=-1:0] plot (\x, {(\x)});

            \draw[graph thickness, color=colorEmphasisCyan, dashed, domain=0:1] plot (\x, {(\x)^7});
            \draw[graph thickness, color=colorEmphasis, dashed, domain=0:1] plot (\x, {(\x)});

            \draw[graph thickness, color=colorEmphasisCyan, domain=1:1.170] plot (\x, {(\x)^7});
            \draw[graph thickness, color=colorEmphasis, domain=1:3] plot (\x, {(\x)});
            \filldraw[color=colorEmphasisGreen] (-1, -1) circle (\pointSize) node[left] {$\left(-1; -1\right)$};
            \filldraw[color=colorEmphasisGreen] (0, 0) circle (\pointSize) node[above left] {$\left(0; 0\right)$};
            \filldraw[color=colorEmphasisGreen] (1, 1) circle (\pointSize) node[right] {$\left(1; 1\right)$};
         \end{tikzpicture}
         \caption{Đồ thị của $x^7 \geq x$}
         \label{fig:ham_so_mot_bien:da_thuc:gbpt8}
      \end{figure}
   \end{minipageindent}
}

Kẻ bảng xét dấu cho \ref{eq:ham_so_mot_bien:da_thuc:gbpt8_2}:
\begin{table}[H]
   \centering
   \begin{tabular}{|c|ccccccccc|}
      \hline
      $x$           & $-\infty$ &     & $-1$ &     & $0$ &     & $1$ &   & $\infty$ \\
      \hline
      $x$           &           & $-$ &      & $-$ &  0  &  +  &     & + &           \\
      \hline
      $x-1$         &           & $-$ &      & $-$ &     & $-$ &  0  & + &           \\
      \hline
      $x+1$         &           & $-$ &  0   &  +  &     &  +  &     & + &           \\
      \hline
      $x(x-1)(x+1)$ &           & $-$ &  0   &  +  &  0  & $-$ &  0  & + &           \\
      \hline
      \end{tabular}
\end{table}

Vậy tập nghiệm của bất phương trình là $\left(-\infty; -1\right] \cup \left[0; 1\right]$.

\exercise Xác định tập giá trị của những hàm sau:
\begin{multicols}{2}
   \begin{enumerate}
      \item $f(x) = 0$;
      \item $f(x) = 10x - 20$;
      \item $f(x) = x^2 + 2x + 3$;
      \item $f(x) = x^4 + 2x^2 + 3$.
   \end{enumerate}
\end{multicols}

\solution

\setcounter{subexercise}{1}
\arabic{subexercise}. Theo định nghĩa, do hàm chỉ trả về kết quả là $0$ nên tập giá trị của $f$ là $\left\{0\right\}$.

\stepcounter{subexercise}
\arabic{subexercise}. Nhận thấy mọi giá trị $y \in \mathbb{R}$ đều có thể là kết quả của $f$ do:
$$f\left(\frac{y}{10} + 2\right) = 10 \left(\frac{y}{10} + 2\right) - 20 = y.$$
Vậy tập giá trị của $f$ là $\mathbb{R}$.

\stepcounter{subexercise}
\arabic{subexercise}. Theo đồ thị \ref{fig:ham_so_mot_bien:da_thuc:x2_2x_3}, chúng ta thấy được $f$ nhận mọi giá trị trong khoảng $\left[2; \infty\right)$. Về mặt đại số, biến đổi $f$ để có:
$$f(x) = x^2 + 2x + 3 = (x + 1)^2 + 2 \geq 2.$$

Điều này khẳng định là nếu $y = f(x)$ thì $y \geq 2$. Tuy nhiên, nó chưa khẳng định là $y \geq 2$ là đủ để có $x$ thỏa mãn $y = f(x)$. Để làm được điểu này, chúng ta phải viết phương trình $y = f(x)$ và tìm một $x$ là nghiệm của phương trình đó. Với $y\geq 2$, chúng ta đặt $x = -1 + \sqrt{y - 2}$ và thực hiện tính $f(x)$:
\begin{align*}
   f\left(-1 + \sqrt{y - 2}\right) &= \left(-1 + \sqrt{y - 2}\right)^2 + 2\left(-1 + \sqrt{y - 2}\right) + 3 \\
   &= \left(1 - 2\sqrt{y - 2} + y - 2\right) + \left(- 2 + 2\sqrt{y - 2}\right) + 3 \\
   &= y.
\end{align*}
Qua đó, chúng ta kết luận với $y \geq 2$ thì tồn tại $x$ để $y = f(x)$.

Vậy tập giá trị của $f$ là $\left[2; \infty\right)$.

\stepcounter{subexercise}
\arabic{subexercise}. Bình phương của một số thì luôn không âm. Cho nên $x^2 \geq 0$ và $x^4 = x^2 \times x^2$ là tích của hai số không âm thì là một số không âm. Do đó $x^4 + 2x^2 + 3 \geq 3$.

Ngược lại, mọi số thực $y$ từ $3$ trở lên đều có thể có một giá trị $x$ sao cho $x^4 + 2x^2 + 3 = y$. Do $y \geq 3$ nên
\begin{align*}
   y - 2 &\geq 1 \\
   \iff \sqrt{y - 2} &\geq 1 \\
   \iff \sqrt{y - 2} - 1 &\geq 0.
\end{align*}
Qua đó, chúng ta có thể lấy khai căn và đặt $x = \sqrt{\sqrt{y - 2} - 1}$. Khi này, thực hiện tương tự như đã làm ở phần 3 để có $f(x) = y$. Vậy tập giá trị của $f$ là $\left[3; \infty\right)$.



% \input{\chapdir toan_hoc_nen_tang/ham_so_mot_bien/phep_tinh_ham.tex}
% \subsection{Hàm phân thức}

\ % Lùi đầu dòng

Hàm cộng, hàm trừ và hàm nhân của hai hàm đa thức là những hàm đa thức. Tuy nhiên, hàm thương lại không như vậy. Do khi chia hai đa thức có những tính chất đặc biệt, nên chúng ta xây dựng một khái niệm mới là hàm \defText{phân thức}. Một hàm $f$ được gọi là phân thức nếu $\defMath{f = 0}$, hoặc: $$\defMath{f = \left(\frac{p}{q}\right)}$$ với $p$ và $q$ là hai đa thức. Trong trường hợp $f \neq 0$, tập xác định của $f$ là tập hợp các giá trị $x$ sao cho $q(x) \neq 0$. 

Khái niệm về phân thức dẫn chúng ta một cách tự nhiên đến khái niệm về một dạng phân thức đặc biệt mang tên \defText{số mũ âm}. Khi mũ một số với số âm, chúng ta có thể viết lại là $$\defMath{x^{-n} = \frac{1}{x^n}}$$ với $x \in \mathbb{R} \setminus \{0\}$ và $n \in \mathbb{Z}^+$. Các tính chất liên quan đến số mũ không âm cũng được áp dụng cho số mũ âm.

\exercise Cho biết tập xác định, tập giá trị và phác thảo đồ thị của những hàm sau:
\begin{multicols}{3}
   \begin{enumerate}
      \item $\displaystyle f(x) = \frac{2}{x}$;
      \item $\displaystyle f(x) = \frac{1}{x^2 + 4x + 4}$;
      \item $\displaystyle f(x) = \frac{2x - 5}{x - 3}$;
      \item $\displaystyle f(x) = \frac{x^2 + 4x - 5}{x - 1}$;
      \item $\displaystyle f(x) = \frac{x - 1}{x^2 + 4x - 5}$;
      \item $\displaystyle f(x) = \frac{1}{x^2 + x + 1}$.
      \item $\displaystyle f(x) = \frac{x + 1}{2x^2 + 5x - 3}$;
      \item $\displaystyle f(x) = \frac{x^2 - 3x - 2}{x^2 + 2x + 1}$;
      \item $\displaystyle f(x) = \frac{2x^2 + 2}{x - 2}$;
   \end{enumerate}
\end{multicols}

\solution

{
   \begin{minipageindent}{0.44\textwidth}
      \setcounter{subexercise}{1}
      \arabic{subexercise}. Theo định nghĩa hàm phân thức, tập xác định của hàm $f(x) = \frac{2}{x}$ là $\mathbb{R} \setminus \left\{0\right\}$.
      
      Kết quả của $f(x)$ phải khác $0$ do nếu như vậy thì $f(x) = \frac{2}{x} = 0 \implies 2 = 0\times x = 0$, vô lí.
      
      Tuy nhiên, mọi số $y$ khác $0$ đều có thể là giá trị của $f(x)$ do $$f\left(\frac{2}{y}\right) = \frac{2}{\frac{2}{y}} = y.$$
      
      Vậy tập giá trị của $f(x)$ là $\mathbb{R} \setminus \left\{0\right\}$.
   \end{minipageindent}
   \hfill
   \begin{minipageindent}{0.55\textwidth}
      \begin{figure}[H]
         \centering
         \begin{tikzpicture}
            \draw[->] (-4, 0) -- (4, 0) node[right] {$x$};
            \draw[->] (0, -4) -- (0, 4) node[above] {$f(x)$};
            \draw[color=colorEmphasisCyan, graph thickness, smooth, samples=100] plot[domain=-4:-0.5] (\x, {2/\x});
            \draw[color=colorEmphasisCyan, graph thickness, smooth, samples=100] plot[domain=0.5:4] (\x, {2/\x});
            \foreach \x/\y/\pos in {1/2/right, -1/-2/left, -2/-1/above, 2/1/below} {
               \filldraw[color=colorEmphasisCyan] (\x, \y) circle (\pointSize) node[\pos] {$\left(\x; \y\right)$};
            }
         \end{tikzpicture}
         \caption{Đồ thị của hàm $f(x) = \frac{2}{x}$}
         \label{fig:ham_so_mot_bien:phan_thuc:2_x}
      \end{figure}
   \end{minipageindent}
}

{
   \begin{minipageindent}{0.44\textwidth}
      \stepcounter{subexercise}
\arabic{subexercise}. Để phân thức có nghĩa thì mẫu số của phân thức phải khác $0$. Viết và bất phương trình này:

      \begin{align*}
         x^2 + 4x + 4 &\neq 0\\
         \iff \left(x + 2\right)^2 &\neq 0\\
         \iff x + 2 &\neq 0 \\
         \iff x &\neq -2
      \end{align*}
      Vậy tập xác định của $f(x)$ là $\mathbb{R} \setminus \left\{-2\right\}$.

      Có mẫu số $x^2 + 4x + 4 = (x + 2)^2 \geq 0$, mà mẫu số phải khác $0$ nên có $x^2 + 4x + 4 > 0$. Chia hai số dương luôn được số dương, cho nên $f(x)$ chỉ nhận giá trị dương.
   \end{minipageindent}
   \hfill
   \begin{minipageindent}{0.55\textwidth}
      \begin{figure}[H]
         \centering
         \begin{tikzpicture}
            \draw[->] (-6, 0) -- (2, 0) node[right] {$x$};
            \draw[->] (0, 0) -- (0, 5)  node[above] {$f(x)$};
            \draw[graph thickness, samples=80, color=colorEmphasisCyan, domain=-6.000:-2.447] plot (\x, {1/((\x)^2 + 4*(\x) + 4)});
            \draw[graph thickness, samples=80, color=colorEmphasisCyan, domain=-1.553:2.000] plot (\x, {1/((\x)^2 + 4*(\x) + 4)});
            \filldraw[color=colorEmphasisCyan] (1, {1/9}) circle (\pointSize) node[below] {$\left(1; \frac{1}{9}\right)$};
            \filldraw[color=colorEmphasisCyan] (0, {1/4}) circle (\pointSize) node[below] {$\left(0; \frac{1}{4}\right)$};
            \filldraw[color=colorEmphasisCyan] (-1, 1) circle (\pointSize) node[right] {$\left(-1; 1\right)$};
            \filldraw[color=colorEmphasisCyan] (-3, 1) circle (\pointSize) node[left] {$\left(-3; 1\right)$};
            \filldraw[color=colorEmphasisCyan] (-{5 / 2}, 4) circle (\pointSize) node[left] {$\left(-\frac{5}{2}; 4\right)$};
         \end{tikzpicture}
         \caption{Đồ thị của hàm $f(x) = \frac{1}{x^2 + 4x + 4}$}
         \label{fig:ham_so_mot_bien:phan_thuc:1_x2_4x_4}
      \end{figure}
   \end{minipageindent}
}

Ngược lại, mọi giá trị dương $y$ đều có thể biểu diễn thông qua $f(x)$ do \begin{align*}
   &f\left(-2 + \frac{1}{\sqrt{y}}\right) = \frac{1}{\left(-2 + \frac{1}{\sqrt{y}}\right)^2 + 4\left(-2 + \frac{1}{\sqrt{y}}\right) + 4}\\
   =& \frac{1}{\left(\left(-2 + \frac{1}{\sqrt{y}}\right) + 2\right)^2} \\
   =&\frac{1}{\left(\frac{1}{\sqrt{y}}\right)^2} =\frac{1}{\frac{1}{y}} =y.
\end{align*}
Vậy tập giá trị của $f(x)$ là $\mathbb{R}^+$.

{
   \begin{minipageindent}{0.48\textwidth}
      \stepcounter{subexercise}
\arabic{subexercise}. Để $x$ thuộc tập xác định của hàm $f(x) = \frac{2x - 5}{x - 3}$ thì $x - 3 \neq 0 \implies x \neq 3$. Vậy tập xác định của $f(x)$ là $\mathbb{R} \setminus \left\{3\right\}$.

      Giả sử có $y$ sao cho $y = f(x)$. Khi này, chúng ta có \begin{align*}
         y &= \frac{2x - 5}{x - 3}\\
         \implies y(x - 3) &= 2x - 5\\
         \iff yx - 3y &= 2x - 5\\
         \iff yx - 2x &= 3y - 5\\
         \iff x(y - 2) &= 3y - 5.
      \end{align*}

      Nếu $y = 2$ thì chúng ta sẽ có $x(y - 2) = 3y - 5 \implies x(2 - 2) = 3\times 2 - 5 \implies 0 = 1$, vô lí.

      Nếu $y \neq 2$ thì $x = \frac{3y - 5}{y - 2}$. Thay ngược lại giá trị $x$ này: 
   \end{minipageindent}
   \hfill
   \begin{minipageindent}{0.5\textwidth}
      \begin{figure}[H]
         \centering
         \begin{tikzpicture}
            \draw[->] (-0.5, 0) -- (6.5, 0) node[right] {$x$};
            \draw[->] (0, -1) -- (0, 6)  node[above] {$f(x)$};
            \draw[graph thickness, samples=80, color=colorEmphasisCyan, domain=-0.500:2.667] plot (\x, {(2*(\x) - 5)/((\x) - 3)});
            \draw[graph thickness, samples=80, color=colorEmphasisCyan, domain=3.250:6.500] plot (\x, {(2*(\x) - 5)/((\x) - 3)});

            \filldraw[color=colorEmphasisCyan] (0, {5/3}) circle (\pointSize) node[above right] {$\left(0; \frac{5}{3}\right)$};
            \filldraw[color=colorEmphasisCyan] (1, {3/2}) circle (\pointSize) node[above right] {$\left(1; \frac{3}{2}\right)$};
            \filldraw[color=colorEmphasisCyan] ({5/2}, 0) circle (\pointSize) node[above right] {$\left(\frac{5}{2}; 0\right)$};

            \filldraw[color=colorEmphasisCyan] (4, 3) circle (\pointSize) node[below left] {$\left(4; 3\right)$};
            \filldraw[color=colorEmphasisCyan] (5, {5/2}) circle (\pointSize) node[below left] {$\left(5; \frac{5}{2}\right)$};

         \end{tikzpicture}
         \caption{Đồ thị của $f(x) = \frac{2 x - 5}{x - 3}$}
      \end{figure}
   \end{minipageindent}
}

\begin{equation*}
   f\left(\frac{3y - 5}{y - 2}\right) = \frac{2\left(\frac{3y - 5}{y - 2}\right) - 5}{\left(\frac{3y - 5}{y - 2}\right) - 3} = \frac{\frac{6y - 10 - 5y + 10}{y - 2}}{\frac{3y - 5 - 3y + 6}{y - 2}} = \frac{\frac{y}{y - 2}}{\frac{1}{y - 2}} = y.
\end{equation*}

Qua lập luận vừa rồi, chúng ta có kết luận rằng tập giá trị của $f(x)$ là $\mathbb{R} \setminus \left\{2\right\}$.

{
   \begin{minipageindent}{0.48\textwidth}
      \stepcounter{subexercise}
\arabic{subexercise}. Giải tập xác định:
      
      $$x - 1 \neq 0 \iff x \neq 1.$$
      
      Qua đó, tập xác định của $f(x)$ là $\mathbb{R} \setminus \left\{1\right\}$.
      
      Đặt $y = f(x)$, với giả thiết $x \neq 1$ thì 
      
      $$
      y = \frac{x^2 + 4x - 5}{x - 1} = \frac{(x + 5)(x - 1)}{x - 1} = x + 5.
      $$
      
      Nhận thấy rằng $y\neq 6$, do nếu ngược lại thì sẽ cần phải có $x = 1$, không thỏa mãn tập xác định của $f(x)$. Với mọi giá trị khác của $y$ đều có thể là đầu ra, do hiển nhiên rằng $f(y - 5) = y$ như biến đổi ở trên.
      
      Vậy tập giá trị của $f(x)$ là $\mathbb{R} \setminus \left\{6\right\}$.

      Tương tự khi giải bất phương trình, khi biểu diễn đồ thị có đứt đoạn, người ta thường vẽ đường tròn rỗng tại điểm bị đứt như đồ thị hình \ref{fig:ham_so_mot_bien:phan_thuc:145_1t1}.
   \end{minipageindent}
   \hfill
   \begin{minipageindent}{0.5\textwidth}
      \begin{figure}[H]
         \centering
         \begin{tikzpicture}
            \draw[->] (-3.5, 0) -- (3.5, 0) node[right] {$x$};
            \draw[->] (0, 0) -- (0, 8)  node[above] {$f(x)$};
            \draw[graph thickness, samples=80, color=colorEmphasisCyan, domain=-3.500:3.000] plot (\x, {(((\x)^2 + 4*(\x) - 5)/((\x) - 1)) / 1});
            \filldraw[color= colorEmphasisCyan] (-2, 3.0) circle (\pointSize) node[above left] {$\left(-2;3\right)$};
            \filldraw[color= colorEmphasisCyan] (-1, 4.0) circle (\pointSize) node[above left] {$\left(-1;4\right)$};
            \filldraw[color= colorEmphasisCyan] (0, 5.0) circle (\pointSize) node[above left] {$\left(0;5\right)$};
            \draw[color=colorEmphasisCyan, hollow point] (1, 6.0) circle (\pointSize) node[above left] {$\left(1;6\right)$};
            \filldraw[color= colorEmphasisCyan] (2, 7.0) circle (\pointSize) node[above left] {$\left(2;7\right)$};
         \end{tikzpicture}
         \caption{Đồ thị của $f(x) = \frac{x^{2} + 4 x - 5}{x - 1}$}
         \label{fig:ham_so_mot_bien:phan_thuc:145_1t1}
      \end{figure}
   \end{minipageindent}
}

\stepcounter{subexercise}
\arabic{subexercise}. Giải tập xác định, $f(x)$ xác định khi và chỉ khi

\begin{align*}
   x^2 + 4x - 5 &\neq 0\\
   \iff (x + 5)(x - 1) &\neq 0\\
   \iff x &\notin \left\{-5; 1\right\}.
\end{align*}

Tập xác định của $f(x)$ là $\mathbb{R} \setminus \left\{-5; 1\right\}$.

Đặt $y = f(x) = \frac{x - 1}{x^2 + 4x - 5}=\frac{x - 1}{(x + 5)(x - 1)} = \frac{1}{x + 5}$. Qua đó, $y$ không thể bằng $0$. Khi $y\neq 0$, biến đổi cho chúng ta được $x = \frac{1}{y} - 5$. Do điều kiện tập xác định lên $x$ nên $y\neq \frac{1}{6}$.

Kiểm chứng đại số cơ bản cho chúng ta được nếu $y\notin \left\{0; \frac{1}{6}\right\}$ thì có thể đặt $x = \frac{1}{y} - 5$ để có $f(x) = y$.

Vậy tập giá trị của $f(x)$ là $\mathbb{R} \setminus \left\{0; \frac{1}{6}\right\}$.

\begin{figure}[H]
	\centering
	\begin{tikzpicture}
		\draw[->] (-7, 0) -- (3, 0) node[right] {$x$};
		\draw[->] (0, -4) -- (0, 4)  node[above] {$f(x)$};
		\draw[graph thickness, samples=80, color=colorEmphasisCyan, domain=-7.000:-5.250] plot (\x, {(((\x) - 1)/((\x)^2 + 4*(\x) - 5)) / 1});
		\draw[graph thickness, samples=80, color=colorEmphasisCyan, domain=-4.750:3.000] plot (\x, {(((\x) - 1)/((\x)^2 + 4*(\x) - 5)) / 1});
		\filldraw[color= colorEmphasisCyan] (-3, 0.5) circle (\pointSize) node[above] {$\left(-3;\frac{1}{2}\right)$};
		\filldraw[color= colorEmphasisCyan] (-6, -1.0) circle (\pointSize) node[above right] {$\left(-6;-1\right)$};
		\filldraw[color= colorEmphasisCyan] (0, 0.2) circle (\pointSize) node[below] {$\left(0;\frac{1}{5}\right)$};
		\filldraw[color= colorEmphasisCyan] (2, 0.14285714285714285) circle (\pointSize) node[below] {$\left(2;\frac{1}{7}\right)$};
      \draw[color=colorEmphasisCyan, hollow point] (1, {1/6}) circle (\pointSize) node[above] {$\left(1;\frac{1}{6}\right)$};
	\end{tikzpicture}
	\caption{Đồ thị của $\frac{x - 1}{x^{2} + 4 x - 5}$}
   \label{fig:ham_so_mot_bien:phan_thuc:1t1_14t5}
\end{figure}

{
   \begin{minipageindent}{0.48\textwidth}
      \stepcounter{subexercise}
      \arabic{subexercise}. Để ý rằng $x^2 + x + 1 = \left(x + \frac{1}{2}\right)^2 + \frac{3}{4} \geq \frac{3}{4}$ với mọi giá trị thực của $x$. Cho nên $f(x) = \frac{1}{x^2 + x + 1}$ là hai số dương chia cho nhau luôn có nghĩa. Cho nên, tập xác định của $f(x)$ là $\mathbb{R}$.

      Cũng từ $x^2 + x + 1 \geq \frac{3}{4}$ mà chúng ta có $\frac{1}{x^2 + x + 1} \leq \frac{4}{3}$ \textcolor{colorEmphasis}{(Cùng chia cả hai vế cho số dương $\frac{3(x^2 + x + 1)}{4}$)}.
      
      Ngoài ra, do là phép chia hai số dương nên $\frac{1}{x^2 + x + 1} > 0$. Do đó, $0 < f(x) \leq \frac{4}{3}$.

      Ngược lại, mọi $y \in \left(0; \frac{4}{3}\right]$ đều có thể biểu diễn thông qua $f(x)$, do 
      \begin{align*}
         f\left(\sqrt{\frac{4-3y}{4y}} - \frac{1}{2} \right) &= \frac{1}{\left(\left(\sqrt{\frac{4-3y}{4y}} - \frac{1}{2}\right) + \frac{1}{2}\right)^2 + \frac{3}{4}} \\
         &= \frac{1}{\left(\sqrt{\frac{4-3y}{4y}}\right)^2 + \frac{3}{4}} \\
         &= \frac{1}{\frac{4-3y}{4y} + \frac{3}{4}} \\
         &= \frac{1}{\frac{4-3y + 3y}{4y}} \\
         &= \frac{1}{\frac{4}{4y}} \\
         &= y.
      \end{align*}

      Vậy tập giá trị của $f(x)$ là $\left(0; \frac{4}{3}\right]$.
   \end{minipageindent}
   \hfill
   \begin{minipageindent}{0.5\textwidth}
      \begin{figure}[H]
         \centering
         \begin{tikzpicture}
            \draw[->] (-4, 0) -- (3, 0) node[right] {$x$};
            \draw[->] (0, -2) -- (0, 4)  node[above] {$f(x)$};
            \draw[graph thickness, samples=80, color=colorEmphasisCyan, domain=-4.000:3.000] plot (\x, {(1/((\x)^2 + (\x) + 1)) / 0.5});
            \filldraw[color=colorEmphasisCyan] (0, 2.0) circle (\pointSize) node[right] {$\left(0;1\right)$};
            \filldraw[color=colorEmphasisCyan] (-0.5, 2.6666666666666665) circle (\pointSize) node[above] {$\left(- \frac{1}{2};\frac{4}{3}\right)$};
            \filldraw[color=colorEmphasisCyan] (1, 0.6666666666666666) circle (\pointSize) node[above right] {$\left(1;\frac{1}{3}\right)$};
            \filldraw[color=colorEmphasisCyan] (-2, 0.6666666666666666) circle (\pointSize) node[above left] {$\left(-2;\frac{1}{3}\right)$};
         \end{tikzpicture}
         \caption{Đồ thị của $f(x) = \frac{1}{x^{2} + x + 1}$}
         \label{fig:ham_so_mot_bien:phan_thuc:1_x2_1x_1}
      \end{figure}
   \end{minipageindent}
}

{
   \begin{minipageindent}{0.48\textwidth}
      \stepcounter{subexercise}
      \arabic{subexercise}. Để $f(x)$ có nghĩa thì mẫu số phải khác $0$. Có:
      \begin{align*}
         2x^2 + 5x - 3 &\neq 0 \\
         \iff (2x - 1)(x + 3) &\neq 0 \qquad \parbox[c]{0.36\textwidth}{\textcolor{colorEmphasis}{(Phân tích đa thức thành nhân tử.)}}\\
         \iff x &\notin \left\{\frac{1}{2}; -3\right\}.
      \end{align*}

      Qua đó, tập xác định của $f(x)$ là $\mathbb{R} \setminus \left\{\frac{1}{2}; -3\right\}$.

      Bây giờ, chúng ta cần tìm những giá trị $y$ sao cho tồn tại $x$ để $y = f(x)$. Với $y = 0$ thì có $f(-1) = 0$ từ đồ thị \ref{fig:ham_so_mot_bien:phan_thuc:1_x2_5x_3}.
      Với $y \neq 0$, đặt $$x = \frac{\sqrt{49y^2-2y+1}-5y+1}{4y}.$$ $x$ luôn nhận giá trị thực do mẫu số khác $0$ ($4y\neq 0$) và phần tử bên trong dấu khai căn $49y^2 - 2y + 1 = 48y^2 + y^2 - 2y + 1 = 48y^2 + (y - 1)^2$ luôn không âm. Thay giá trị $x$ này vào tử số của $f(x)$:
      \begin{align*}
         x + 1 &= \frac{\sqrt{49y^2-2y+1}-5y+1}{4y} + 1\\
         &= \frac{\sqrt{49y^2-2y+1}-y+1}{4y}.
      \end{align*}
   \end{minipageindent}
   \hfill
   \begin{minipageindent}{0.5\textwidth}
      \begin{figure}[H]
         \centering
         \begin{tikzpicture}
            \draw[->] (-5, 0) -- (2, 0) node[right] {$x$};
            \draw[->] (0, -4) -- (0, 4) node[above] {$f(x)$};
            \draw[graph thickness, samples=80, color=colorEmphasisCyan, domain=-5.000:-3.073] plot (\x, {((\x) + 1)/(2*(\x)^2 + 5*(\x) - 3)});
            \draw[graph thickness, samples=80, color=colorEmphasisCyan, domain=-2.930:0.448] plot (\x, {((\x) + 1)/(2*(\x)^2 + 5*(\x) - 3)});
            \draw[graph thickness, samples=80, color=colorEmphasisCyan, domain=0.555:2.000] plot (\x, {((\x) + 1)/(2*(\x)^2 + 5*(\x) - 3)});

            \filldraw[color= colorEmphasisCyan] (-1, 0.0) circle (\pointSize) node[below] {$\left(-1;0\right)$};
            \filldraw[color= colorEmphasisCyan] (0, -0.3333333333333333) circle (\pointSize) node[right] {$\left(0;- \frac{1}{3}\right)$};
            \filldraw[color= colorEmphasisCyan] (-2, 0.2) circle (\pointSize) node[above right] {$\left(-2;\frac{1}{5}\right)$};
            \filldraw[color= colorEmphasisCyan] (1.5, 0.2777777777777778) circle (\pointSize) node[above right] {$\left(\frac{3}{2};\frac{5}{18}\right)$};
            \filldraw[color= colorEmphasisCyan] (-5, -0.18181818181818182) circle (\pointSize) node[below] {$\left(-5;- \frac{2}{11}\right)$};
            \filldraw[color= colorEmphasisCyan] (-2.75, 1.0769230769230769) circle (\pointSize) node[above right] {$\left(- \frac{11}{4};\frac{14}{13}\right)$};
            \filldraw[color= colorEmphasisCyan] (-4, -0.3333333333333333) circle (\pointSize) node[right] {$\left(-4;- \frac{1}{3}\right)$};
         \end{tikzpicture}
         \caption{Đồ thị của $f(x) = \frac{x + 1}{2 x^{2} + 5 x - 3}$}
         \label{fig:ham_so_mot_bien:phan_thuc:1_x2_5x_3}
      \end{figure}
   \end{minipageindent}
}

Thay giá trị của $y$ vào mẫu:

\begin{align*}
   2x^2 + 5x - 3 &= (x + 3)(2x - 1)\\
   &= \left(\frac{\sqrt{49y^2-2y+1}-5y+1}{4y}+3\right)\left(2\cdot\frac{\sqrt{49y^2-2y+1}-5y+1}{4y}-1\right) \\
   &= \frac{\sqrt{49y^2-2y+1}+7y+1}{4y}\cdot\frac{\sqrt{49y^2-2y+1}-7y+1}{2y} \\
   \displaybreak[2]
   &= \frac{\left(\sqrt{49y^2-2y+1} + 1\right)^2 - (7y)^2}{8y^2}\\
   &= \frac{49y^2-2y+1 + 2\sqrt{49y^2-2y+1} + 1 - 49y^2}{8y^2}\\
   &= \frac{2\sqrt{49y^2-2y+1} - 2y + 2}{8y^2}\\
   &= \frac{\sqrt{49y^2-2y+1} - y + 1}{4y^2}.
\end{align*}

Mẫu số này khác $0$ do nếu bằng $0$ thì chúng ta sẽ có
\begin{align*}
   \sqrt{49y^2-2y+1} - y + 1 &= 0\\
   \iff \sqrt{49y^2-2y+1} &= y - 1\\
   \implies 49y^2 - 2y + 1 &= (y - 1)^2\\
   \iff 49y^2 - 2y + 1 &= y^2 - 2y + 1\\
   \iff 48y^2 &= 0\\
   \iff y &= 0
\end{align*} mâu thuẫn với giả thiết $y\neq 0$. Lấy tử số chia cho mẫu số và khử bỏ thừa số chúng để có
$$f\left(\frac{\sqrt{49y^2-2y+1}-5y+1}{4y}\right) = \frac{\frac{\sqrt{49y^2-2y+1}-y+1}{4y}}{\frac{\sqrt{49y^2-2y+1} - y + 1}{4y^2}} = y.$$

Chúng ta đã thể hiện rằng mọi số $y$ đều có thể biểu diễn thông qua $f(x)$. Vậy tập giá trị của $f(x)$ là $\mathbb{R}$.

\stepcounter{subexercise}
\arabic{subexercise}. Giải tập xác định:

\begin{align*}
   x^2 + 2x + 1 &\neq 0 \\
   \iff (x + 1)^2 &\neq 0 \\
   \iff x &\neq -1.
\end{align*}

Qua đó, chúng ta có tập xác định của $f(x)$ là $\mathbb{R} \setminus \left\{-1\right\}$.

Giải tập giá trị sẽ khó hơn. Gọi $y\in\mathbb{R}$ và giả sử $y = f(x)$. Khi này,

\begin{align}
   y &= \frac{x^2 - 3x - 2}{x^2 + 2x + 1} \nonumber\\
   \implies y(x^2 + 2x + 1) &= x^2 - 3x - 2 \nonumber\\
   \iff yx^2 + 2yx + y &= x^2 - 3x - 2 \nonumber\\
   \iff (y - 1)x^2 + (2y + 3)x + (y + 2) &= 0. \label{eq:ham_so_mot_bien:phan_thuc:p5}
\end{align}

Nếu $y = 1$ thì từ (\ref{eq:ham_so_mot_bien:phan_thuc:p5}), $5x + 3 = 0 \iff x = -\frac{3}{5}$. Vậy $1$ có thể là kết quả của $f(x)$.

Trong trường hợp còn lại, coi (\ref{eq:ham_so_mot_bien:phan_thuc:p5}) là phương trình bậc hai với $x$ là nghiệm. Để tồn tại nghiệm thì $\Delta \geq 0$, với $\Delta$ là 
\begin{align*}
   &= (2y + 3)^2 - 4(y - 1)(y + 2) \\
   &= 4y^2 + 12y + 9 - 4(y^2 + y - 2) \\
   &= 4y^2 + 12y + 9 - 4y^2 - 4y + 8 \\
   &= 8y + 17.
\end{align*}
Từ đó, để $\Delta \geq 0$ thì $8y + 17 \geq 0 \iff y \geq -\frac{17}{8}$.

Kiểm tra ngược tập giá trị, chúng ta đã biết $1$ thuộc tập giá trị này. Với mọi giá trị $y \geq -\frac{17}{8}$ khác $1$, đặt $x = \frac{2y + 3 - \sqrt{8y + 17}}{2(1 - y)}$, khi này

\begin{align*}
   f(x) &= \frac{x^2 - 3x - 2}{x^2 + 2x + 1} = \frac{x^2 - 3x - 2}{x^2 + 2x + 1} - y + y = \frac{x^2 -3x - 2 - yx^2 - 2yx - y}{\left(x + 1\right)^2} + y\\
   &= \frac{(1-y)x^2 - (2y+3)x - (2+y)}{\left(x + 1\right)^2} + y = \frac{x^2 - \left(\frac{2y+3}{1-y}\right)x - \frac{y+2}{1-y}}{\left(x + 1\right)^2} + y\\
   \displaybreak[2]
   &= \frac{x^2 - 2\cdot x\cdot \left(\frac{2y+3}{2(1-y)}\right) + \left(\frac{2y+3}{2(1 - y)}\right)^2 - \left(\frac{2y+3}{2(1-y)}\right)^2 - \frac{y+2}{1 - y} }{(x + 1)^2} + y \\
   &= \frac{\left(x - \frac{2y + 3}{2(1-y)}\right)^2-\frac{8y + 17}{4(1 - y)^2}}{(x + 1)^2} + y \\
   &= \frac{\left(\frac{2y + 3 - \sqrt{8y + 17}}{2(1 - y)} - \frac{2y + 3}{2(1-y)}\right)^2-\frac{8y + 17}{4(1 - y)^2}}{(x + 1)^2} + y\\
   &= \frac{\left(\frac{-\sqrt{8y+17}}{2(1-y)}\right)^2 - \frac{8y + 17}{4(1 - y)^2}}{(x + 1)^2} + y = \frac{\frac{8y + 17}{4(1 - y)^2} - \frac{8y + 17}{4(1 - y)^2}}{(x + 1)^2} + y = y.
\end{align*}

Vậy tập giá trị của $f(x)$ là $\left[-\frac{17}{8}; \infty\right)$. Đồ thị của $f(x) = \frac{x^2 - 3x - 2}{x^2 + 2x + 1}$ được thể hiện trong \ref{fig:ham_so_mot_bien:phan_thuc:1t3t2_121}.

\begin{figure}[H]
	\centering
	\begin{tikzpicture}
		\draw[->] (-6, 0) -- (6, 0) node[right] {$x$};
		\draw[->] (0, -2.5) -- (0, 5.5)  node[above] {$f(x)$};
		\draw[graph thickness, samples=80, color=colorEmphasisCyan, domain=-6.000:-2.423] plot (\x, {((\x)^2 - 3*(\x) - 2)/((\x)^2 + 2*(\x) + 1)});
		\draw[graph thickness, samples=80, color=colorEmphasisCyan, domain=-0.688:6.000] plot (\x, {((\x)^2 - 3*(\x) - 2)/((\x)^2 + 2*(\x) + 1)});
		\filldraw[color= colorEmphasisCyan] (-3.0, 4.0) circle (\pointSize) node[below right] {$\left(-3{,}0;4{,}0\right)$};
		\filldraw[color= colorEmphasisCyan] (-4.5, 2.5918367346938775) circle (\pointSize) node[below right] {$\left(-4{,}5;2{,}592\right)$};
		\filldraw[color= colorEmphasisCyan] (0.0, -2.0) circle (\pointSize) node[right] {$\left(0{,}0;-2{,}0\right)$};
		\filldraw[color= colorEmphasisCyan] (-0.2, -2.125) circle (\pointSize) node[below] {$\left(-\frac{1}{5};-\frac{17}{8}\right)$};
		\filldraw[color= colorEmphasisCyan] (1.0, -1.0) circle (\pointSize) node[below right] {$\left(1{,}0;-1{,}0\right)$};
		\filldraw[color= colorEmphasisCyan] (-0.5, -1.0) circle (\pointSize) node[left] {$\left(-0{,}5;-1{,}0\right)$};
		\filldraw[color= colorEmphasisCyan] (3.0, -0.125) circle (\pointSize) node[above left] {$\left(3{,}0;-0{,}125\right)$};
      \filldraw[color= colorEmphasisCyan] (-0.6, 1) circle (\pointSize) node[left] {$\left(-\frac{3}{5};1\right)$};
	\end{tikzpicture}
	\caption{Đồ thị của $f(x) = \frac{x^{2} - 3 x - 2}{x^{2} + 2 x + 1}$}
   \label{fig:ham_so_mot_bien:phan_thuc:1t3t2_121}
\end{figure}

\stepcounter{subexercise}
\arabic{subexercise}. Xét tập xác định của $f(x)$, cần phải có $x - 2 \neq 0 \iff x \neq 2$. Vậy tập xác định của $f(x)$ là $\mathbb{R} \setminus \left\{2\right\}$.

Đặt $y = f(x)$, chúng ta có:

\begin{align*}
   y &= \frac{2x^2 + 2}{x - 2} \\
   \displaybreak[2]
   \iff y(x - 2) &= 2x^2 + 2 \\
   \displaybreak[2]
   \iff yx - 2y &= 2x^2 + 2 \\
   \displaybreak[2]
   \iff 2x^2 - yx + 2y + 2 &= 0.
\end{align*}

Coi kết quả của biến đổi là phương trình bậc hai ẩn $x$. Để tồn tại $x$ thì cần phải có

\begin{align*}
   (-y)^2 - 4\cdot 2\cdot (2y + 2) & \geq 0\\
   \iff y^2 - 16y - 16 &\geq 0.
\end{align*}

Kẻ bảng xét dấu của $g(y) = y^2 - 16y - 16$:

\begin{table}[H]
   \centering
   \begin{tabular}{|c|ccccccc|}
   \hline
   $y$             & $-\infty$ &   & $8-4 \sqrt{5}$ &     & $8+4 \sqrt{5}$ &   & $\infty$ \\
   \hline
   $y^{2}-16y-16$  &           & + &        0        & $-$ &       0        & + &           \\
   \hline
   \end{tabular}
   \caption{Bảng xét dấu của $g(y) = y^2 - 16y - 16$}
   \label{tab:ham_so_mot_bien:phan_thuc:1t16t16}
\end{table}

Qua bảng, chúng ta có điều kiện của $y$ là $y \in \left(-\infty; 8 - 4\sqrt{5}\right] \cup \left[8 + 4\sqrt{5}; \infty\right)$. Chúng ta cũng có thể kiểm chứng bằng biến đổi đại số rằng với $y$ thuộc tập hợp này thì có $f\left(\frac{\sqrt{y^2 - 16y - 16} + y}{4}\right) = y$.

Vậy tập giá trị của $f(x)$ là $\left(-\infty; 8 - 4\sqrt{5}\right] \cup \left[8 + 4\sqrt{5}; \infty\right)$.

Do tính chất của đồ thị, trục tung của đồ thị trong lời giải của tác giả đã bị co lại $10$ lần, thể hiện ở hình \ref{fig:ham_so_mot_bien:phan_thuc:2x2_2_x_t2}.

\begin{figure}[H]
	\centering
	\begin{tikzpicture}
		\draw[->] (-3, 0) -- (7, 0) node[right] {$x$};
		\draw[->] (0, -4) -- (0, 4)  node[above] {$f(x)$};
		\draw[graph thickness, samples=80, color=colorEmphasisCyan, domain=-3.000:1.790] plot (\x, {((2*(\x)^2 + 2)/((\x) - 2)) / 10});
		\draw[graph thickness, samples=80, color=colorEmphasisCyan, domain=2.319:7.000] plot (\x, {((2*(\x)^2 + 2)/((\x) - 2)) / 10});
		\filldraw[color= colorEmphasisCyan] (-2, -0.25) circle (\pointSize) node[below] {$\left(-2;- \frac{5}{2}\right)$};
		\filldraw[color= colorEmphasisCyan] (1, -0.4) circle (\pointSize) node[below left] {$\left(1;-4\right)$};
		\filldraw[color= colorEmphasisCyan] (0, -0.1) circle (\pointSize) node[above] {$\left(0;-1\right)$};
		\filldraw[color= colorEmphasisCyan] (3, 2.0) circle (\pointSize) node[below left] {$\left(3;20\right)$};
		\filldraw[color= colorEmphasisCyan] (5, 1.7333333333333334) circle (\pointSize) node[above] {$\left(5;\frac{52}{3}\right)$};
	\end{tikzpicture}
	\caption{Đồ thị của $f(x) = \frac{2 x^{2} + 2}{x - 2}$}
   \label{fig:ham_so_mot_bien:phan_thuc:2x2_2_x_t2}
\end{figure}

\exercise Giải các phương trình sau với ẩn $x \in \mathbb{R}$.

\begin{multicols}{2}
   \begin{enumerate}
      \item $\displaystyle\frac{2x^2 - 5x + 2}{3x} = 0$;
      \item $\displaystyle \frac{4x + 2}{x^2 + x - 2} = 1$;
      \item $\displaystyle \frac{x^2 + 4x + 3}{x^3 + 3x^2 -x - 3} = \frac{1}{x - 1}$;
      \item $\displaystyle \frac{3x}{x + 2} - \frac{x}{x - 2} = \frac{8}{x^2 - 4}$;
      \item $\displaystyle A = \frac{h}{6x}\left(\frac{b_0}{x} + 4b_1 + b_2\right)$ với $A$, $b_0$, $b_1$, $b_2$, $h$ là những tham số thực dương;
      \item $\displaystyle \frac{3x}{x + 2} - \frac{x}{x - 2} = \frac{8}{4 - x^2}$;
      \item $\displaystyle \frac{24}{x + 2} + \frac{24}{x^2 - 5x + 6} = x^2$.
   \end{enumerate}
\end{multicols}

\solution

\setcounter{subexercise}{1}
\arabic{subexercise}. Không phải mọi giá trị của $x$ sẽ làm cho biểu thức được cho ở mỗi vế có nghĩa. Để $\frac{2x^2 - 5x + 2}{3x}$ có nghĩa thì $3x \neq 0 \iff x \neq 0$. Khi này:

\begin{align*}
   \frac{2x^2 - 5x + 2}{3x} &= 0 \\
   \implies 2x^2 - 5x + 2 &= 0 \\
   \iff (2x - 1)(x - 2) &= 0
\end{align*}
\begin{equation*}
   \iff \left[\begin{array}{l}
      2x - 1 = 0 \\
      x - 2 = 0
   \end{array}\right. \iff \left[\begin{array}{l}
      x = \frac{1}{2} \\
      x = 2
   \end{array}\right..
\end{equation*}

Kiểm tra trực tiếp, chúng ta thấy nghiệm thỏa mãn phương trình gốc. Vậy tập nghiệm của phương trình là $\left\{\frac{1}{2}; 2\right\}$.

\stepcounter{subexercise}
\arabic{subexercise}. Coi vế trái của phương trình được cho là một phân thức, chúng ta tìm tập xác định của nó:

\begin{align*}
   x^2 + x - 2 &\neq 0 \\
   \iff (x + 2)(x - 1) &\neq 0 \\
\end{align*}
\begin{equation*}
   \iff \begin{cases}
      x + 2 \neq 0 \\
      x - 1 \neq 0
   \end{cases} \iff \begin{cases}
      x \neq -2 \\
      x \neq 1
   \end{cases}.
\end{equation*}

Thực hiện biến đổi phương trình:

\begin{align*}
   \frac{4x + 2}{x^2 + x - 2} &= 1 \\
   \implies 4x + 2 &= x^2 + x - 2 \\
   \displaybreak[2]
   \iff 0 &= x^2 - 3x - 4 \\
   \iff 0 &= (x + 1)(x - 4) \\
   \iff x &\in \left\{-1; 4\right\}.
\end{align*}

Cả hai giá trị đều là nghiệm của phương trình bằng kiểm tra trực tiếp. Vậy phương trình có nghiệm là $\left\{-1; 4\right\}$.

\stepcounter{subexercise}
\arabic{subexercise}. Để cả vế trái và vế phải của phương trình xác định giá trị thì

\begin{equation*}
   \begin{cases}
      x^3 + 3x^2 - x - 3 \neq 0 \\
      x - 1 \neq 0
   \end{cases} \iff 
   \begin{cases}
      (x - 1)(x + 1)(x - 3) \neq 0 \\
      x - 1 \neq 0
   \end{cases}
   \iff x \notin \left\{-1; 1; 3\right\}.
\end{equation*}

Biến đổi phương trình:

\begin{align}
   \frac{x^2 + 4x + 3}{x^3 + 3x^2 -x - 3} &= \frac{1}{x - 1} \nonumber\\
   \iff \frac{(x + 1)(x + 3)}{(x - 1)(x + 1)(x - 3)} &= \frac{1}{x - 1} \nonumber\\
   \iff \frac{1}{x - 1} = \frac{1}{x - 1}. \label{eq:ham_so_mot_bien:ơhan_thuc:ptpt3}
\end{align}

Phương trình (\ref{eq:ham_so_mot_bien:ơhan_thuc:ptpt3}) luôn đúng với $x$ làm cho cả hai vế của phương trình xác định. Do đó, tập nghiệm của phương trình là $\mathbb{R} \setminus \left\{-1; 1; 3\right\}$.

\stepcounter{subexercise}
\arabic{subexercise}. Phương trình có tập xác định\footnote{Tập xác định chỉ có với hàm số. Ở đây, ý chúng ta muốn là những giá trị để cho cả hai vế có thể tính được.} là $\mathbb{R} \setminus \{-2; 2\}$.

\begin{align*}
   \frac{3x}{x + 2} - \frac{x}{x - 2} &= \frac{8}{x^2 - 4} \\
   \iff \frac{3x(x - 2)}{(x + 2)(x - 2)} - \frac{x(x + 2)}{(x - 2)(x + 2)} &= \frac{8}{(x - 2)(x + 2)} \\
   \implies \left(3x^2 - 6x\right) - \left(x^2 + 2x\right) &= 8 \\
   \iff 2x^2 - 8x - 8 &= 0 \\
   \iff x &\in \left\{2\left(1 + \sqrt{2}\right); 2(1 - \sqrt{2})\right\}.
\end{align*}

Kiểm tra lại, chúng ta có:

\begin{align*}
   &\frac{3\cdot 2(1+\sqrt{2})}{2(1+\sqrt{2}) + 2} - \frac{2(1+\sqrt{2})}{2(1+\sqrt{2}) - 2} \\
   = &\frac{6 + 6\sqrt{2}}{4 + 2\sqrt{2}} - \frac{2 + 2\sqrt{2}}{2\sqrt{2}} \\
   = &\frac{\left(6 + 6\sqrt{2}\right)2\sqrt{2} - \left(2 + 2\sqrt{2}\right)\left(4 + 2\sqrt{2}\right)}{\left(4 + 2\sqrt{2}\right)\left(2\sqrt{2}\right)} \\
   = &\frac{12\sqrt{2} + 24 - \left(16 + 12\sqrt{2}\right)}{\left(2\left(1 + \sqrt{2}\right)\right)^2 - 4} \\
   = &\frac{8}{\left(2\left(1 + \sqrt{2}\right)\right)^2 - 4}.
\end{align*}

Tương tự khi kiểm tra $x = 2\left(1 - \sqrt{2}\right)$. Vậy phương trình có nghiệm là $\left\{2\left(1 + \sqrt{2}\right); 2\left(1 - \sqrt{2}\right)\right\}$.

\stepcounter{subexercise}
\arabic{subexercise}. Phương trình xác định khi $x \neq 0$. Trên điều kiện này,

\begin{align}
   A &= \frac{h}{6x}\left(\frac{b_0}{x} + 4b_1 + b_2\right) \nonumber\\
   &= \frac{hb_0}{6x^2} + \frac{h\left(4b_1 + b_2\right)}{6x} \nonumber\\
   \iff 6x^2A &= hb_0 + h\left(4b_1 + b_2\right)x \nonumber\\
   \iff 6A\cdot x^2 - h\left(4b_1 + b_2\right)x - hb_0 &= 0. \label{eq:ham_so_mot_bien:ơhan_thuc:ptpt5}
\end{align}

Nhận thấy rằng nếu (\ref{eq:ham_so_mot_bien:ơhan_thuc:ptpt5}) có nghiệm thì nghiệm này phải khác $0$. Trái lại, nếu $0$ là nghiệm thì sẽ phải có $hb_0 = 0$. Nhưng từ giả thiết $b_0$ và $h$ đều dương, $hb_0 > 0$. Chúng ta cần phải có nhận định này để không cần phải kiểm tra lại điều kiện tập xác định khi giải ra nghiệm.

Xét biệt thức $\Delta = h^2(4b_1 + b_2)^2 + 24A\cdot hb_0$ của phương trình (\ref{eq:ham_so_mot_bien:ơhan_thuc:ptpt5}). Có các tham số đều là các giá trị dương nên $\Delta$ cũng là một giá trị dương. Cho nên, từ (\ref{eq:ham_so_mot_bien:ơhan_thuc:ptpt5}), chúng ta giải ra hai nghiệm
\begin{equation*}
   \left[\begin{array}{l}
      x = \frac{h\left(4b_1 + b_2\right) + \sqrt{\Delta}}{12A} \\
      x = \frac{h\left(4b_1 + b_2\right) - \sqrt{\Delta}}{12A} \\
   \end{array}\right..
\end{equation*}

Vậy tập nghiệm của phương trình là $\left\{\frac{h\left(4b_1 + b_2\right) + \sqrt{\Delta}}{12A}; \frac{h\left(4b_1 + b_2\right) - \sqrt{\Delta}}{12A}\right\}$.

\stepcounter{subexercise}
\arabic{subexercise}. Phương trình có tập xác định là $\mathbb{R} \setminus \{-2; 2\}$. Trong tập xác định này, 

\begin{align*}
   &\frac{3x}{x + 2} - \frac{x}{x - 2} = \frac{8}{4 - x^2} \\
   \iff &\frac{3x(x - 2)}{(x + 2)(x - 2)} - \frac{x(x + 2)}{(x - 2)(x + 2)} + \frac{8}{(x - 2)(x + 2)} = 0\\
   \iff &\frac{3x^2 - 6x - x^2 - 2x + 8}{(x + 2)(x - 2)} = 0\\
   \iff &\frac{2x^2 - 8x + 8}{(x + 2)(x - 2)} = 0\\
   \iff &2x^2 - 8x + 8 = 0\\
   \iff &x^2 - 4x + 4 = 0\\
   \iff &(x - 2)^2 = 0\\
   \iff &x = 2.
\end{align*}

Tuy nhiên, tập xác định yêu cầu không nhận giá trị $x$ này, cho nên phương trình này suy ra một điều mâu thuẫn. Vậy phương trình vô nghiệm.

\stepcounter{subexercise}
\arabic{subexercise}. Giải tập xác định của phương trình:

\begin{equation*}
   \begin{cases}
      x + 2 \neq 0 \\
      x^2 - 5x + 6 \neq 0
   \end{cases} \iff x \notin \{-2; 2; 3\}.
\end{equation*}

Giải phương trình:

\begin{align}
   &\frac{24}{x + 2} + \frac{24}{x^2 - 5x + 6} = x^2 \nonumber\\
   \iff &\frac{24}{x + 2} + \frac{24}{(x - 2)(x - 3)} - x^2 = 0 \nonumber\\
   \implies &24(x - 2)(x - 3) + 24(x + 2) - x^2(x - 2)(x - 3)(x + 2) = 0 \qquad \textcolor{colorEmphasis}{\begin{aligned}
      &\text{Nhân cả hai vế với}\\
      &\text{$(x + 2)(x - 2)(x - 3)$.}
   \end{aligned}} \nonumber\\
   \iff &\left(24x^2 - 120x + 144\right) + \left(24x + 48\right) - \left(x^5 - 3x^4 - 4x^3 + 12x^2\right) = 0 \nonumber\\
   \iff &-x^5 + 3x^4 + 4x^3 + 12x^2 - 96x + 144 = 0 \nonumber\\
   \iff &\left(4 - x\right)\left(x^4 + x^3 - 12x + 48\right) = 0 \label{eq:ham_so_mot_bien:phan_thuc:ptpt7}
\end{align}

Nhìn thấy ngay được, phương trình (\ref{eq:ham_so_mot_bien:phan_thuc:ptpt7}) có nghiệm $x = 4$. Xét trường hợp còn lại, đặt $f(x) = x^4 + x^3 - 12x + 48 = x(x - 2)\left(x^2 + 3x + 6\right) + 48$. Chúng ta sẽ chứng minh $f(x) > 0$ với mọi $x \in \mathbb{R}$. Chia làm hai trường hợp:

\textcolor{colorEmphasisCyan}{Trường hợp một --- $0 \leq x \leq 2$}: Chúng ta sẽ chặn giá trị của những thành phần sau:

\begin{itemize}
   \item $x(x - 2)$: \begin{align}
      x(x - 2) &= x^2 - 2x \nonumber \\
      &= \left(x - 1\right)^2 - 1 \nonumber \\
      \implies x(x - 2) &\geq -1. \label{eq:ham_so_mot_bien:phan_thuc:ptpt7_1}
   \end{align}
   \item $x^2 + 3x + 6$: \begin{align*}
      x^2 + 3x + 6 &= \left(x + \frac{3}{2}\right)^2 + \frac{15}{4} \\
      \implies x^2 + 3x + 6 &\geq \frac{15}{4} > 0.
   \end{align*}
\end{itemize}

Ngoài ra, theo giả thiết $0 \leq x \leq 2$,
\begin{equation}
   \begin{cases}
      x^2 \leq 4 \\
      3x \leq 6
   \end{cases} \implies x^2 + 3x + 6 \leq 16 \iff -\left(x^2 + 3x + 6\right) \geq -16. \label{eq:ham_so_mot_bien:phan_thuc:ptpt7_2}
\end{equation}

Kết hợp giữa \refeq{eq:ham_so_mot_bien:phan_thuc:ptpt7_1} và \refeq{eq:ham_so_mot_bien:phan_thuc:ptpt7_2} chúng ta có:

\begin{align*}
   x(x - 2) &\geq -1 \equationexplanation{Từ bất phương trình \refeq{eq:ham_so_mot_bien:phan_thuc:ptpt7_1}.}\\
   \iff x(x - 2)\left(x^2 + 3x + 6\right) &\geq -\left(x^2 + 3x + 6\right) \equationexplanation{Nhân cả hai vế với một số dương.}\\
   \iff x(x - 2)\left(x^2 + 3x + 6\right) &\geq -16 \equationexplanation{Từ bất phương trình ở \refeq{eq:ham_so_mot_bien:phan_thuc:ptpt7_2}.} \\
   \iff x\left(x - 2\right)\left(x^2 + 3x + 6\right) + 48 &\geq 32 \\
   \implies x^4 + x^3 - 12x + 48 > 0.
\end{align*}

Qua đó, chúng ta có được $f(x) = 0$ không có nghiệm trong đoạn $\left[0; 2\right]$.

\textcolor{colorEmphasis}{Trường hợp hai --- $x < 0$ hoặc $x > 2$}: Dễ dàng nhận thấy $x$ và $x - 2$ cùng dấu cho nên $x(x - 2) > 0$. Ngoài ra, đã có $x^2 + 3x + 6 > 0$ cho nên $x\left(x - 2\right)\left(x^2 + 3x + 6\right) > 0$. Suy ra, $f(x) > 0$ vói mọi $x \in \left)0; 2\right($.

Kết hợp cả hai trường hợp, chúng ta có $f(x) > 0$ với mọi $x \in \mathbb{R}$ như cần phải chứng minh.

Do đó, $\text{\refeq{eq:ham_so_mot_bien:phan_thuc:ptpt7}} \iff x = 4$. Kiểm tra trực tiếp chúng ta thấy nghiệm này thỏa mãn. Vậy phương trình có nghiệm duy nhất là $x = 4$.

\exercise Giải các bất phương trình sau trên ẩn $x$ thực.
\begin{multicols}{2}
   \begin{enumerate}
      \item $\frac{x-1}{x+2} > 0$;
      \item $\frac{x-2}{x^2+3x+2} \leq 0$;
      \item $\frac{x^2 - 3x + 2}{x^2 - 4} \geq 1$;
      \item $x \leq \frac{x^2 - 1}{x}$;
      \item $\frac {1}{x} + \frac{1}{x - 1} < \frac{2}{x - 2}$;
      \item $\frac{x^2 - 3x + 2}{x^2 - 4x + 3} \leq \frac{x^2 - 5x + 6}{x^2 - 6x + 8}$;
      \item $\frac{x^3-1}{x^2 - 1} > x - 1$;
      \item $x^2 + \frac{x^2}{1+x^2} > x + \frac{x}{1+x}$.
   \end{enumerate}
\end{multicols}

\solution

\setcounter{subexercise}{1}
\arabic{subexercise}. Để phân thức xác định thì $x + 2\neq 0$. Nhân cả hai vế với số dương $(x + 2)^2$, chúng ta có bất phương trình tương đương:
$$
   (x + 1)(x + 2) > 0.
$$
Từ đây, kiểm tra giá trị của $x$ trên các khoảng $\left(-\infty; -2\right)$, $\left(-2; -1\right)$, $\left(-1; \infty\right)$ ($x \notin \left\{-1; -2\right\}$ do bất phương trình không xảy ra dấu bằng), thấy được rằng nghiệm của bất phương trình là $\left(-\infty; -2\right) \cup \left(-1; \infty\right)$.

\stepcounter{subexercise}
\arabic{subexercise}.

\begin{align}
   &\frac{x-2}{x^2+3x+2} \leq 0 \nonumber\\
   \iff &\frac{x-2}{(x - 1)(x - 2)} \leq 0 \nonumber\\ 
   \iff &\begin{cases}
      \frac{1}{x - 1} \leq 0 \\
      x - 2 \neq 0
   \end{cases} \label{eq:toan_hoc_nen_tang:ham_so_mot_bien:phan_thuc:x-1x-2}
\end{align}

Xét $\frac{1}{x - 1} \leq 0$, giống như phần trước, có $x - 1 \neq 0$. Nhân cả hai vế với $(x - 1)^2$, kết hợp với $x - 1 \neq 0$, được $x - 1 > 0$ hay $x > 1$.

Từ đó, chúng ta có:
$$\refeq{eq:toan_hoc_nen_tang:ham_so_mot_bien:phan_thuc:x-1x-2} \iff \begin{cases}
   x > 1 \\
   x \neq 2
\end{cases}.$$

Vậy, tập nghiệm của bất phương trình là $\left(1; \infty\right) \setminus \{2\}$ hay $\left(1; 2\right) \cup \left(2; \infty\right)$.

\stepcounter{subexercise}
\arabic{subexercise}.

\begin{align*}
   &\frac{x^2 - 3x + 2}{x^2 - 4} \geq 1 \\
   \iff &\frac{(x - 1)(x - 2)}{(x + 2)(x - 2)} \geq 1 \\
   \iff &\begin{cases}
      \frac{x - 1}{x + 2} \geq 1 \\
      x - 2 \neq 0
   \end{cases} 
   \iff \begin{cases}
      \frac{x - 1}{x + 2} - 1 \geq 0 \\
      x \neq 2
   \end{cases} 
   \iff \begin{cases}
      \frac{-3}{x + 2} \geq 0 \\
      x \neq 2
   \end{cases} \\
   \iff &\begin{cases}
      -3(x + 2) \geq 0 \\
      x \neq 2
   \end{cases} 
   \iff \begin{cases}
      x \leq -2\\
      x \neq 2
   \end{cases} \iff x \leq -2.
\end{align*}

Tập nghiệm của bất phương trình là $\left(-\infty; -2\right)$.

Lỗi sai thường gặp: từ $\frac{x - 1}{x + 2} \geq 1$ suy ra $x - 1 \geq x + 2$ (nhân cả hai vế với $x + 2$). Điều này chỉ đúng khi $x + 2$ là số dương. Trong trường hợp $x + 2 < 0$, có thể thấy được rằng dấu đã đảo chiều:  $x - 1 \leq x + 2$.

\stepcounter{subexercise}
\arabic{subexercise}.

\begin{align*}
   x &\leq \frac{x^2 - 1}{x} \\
   \iff 0 &\leq \frac{x^2 - 1}{x} - x \\
   \iff 0 &\leq \frac{-1}{x} \\
   \iff x &< 0.
\end{align*}

Tập nghiệm của bất phương trình là $\left(-\infty; 0\right)$. Khả năng cao là các bạn đọc đã sử dụng phương pháp chia trường hợp và nhân cả hai vế với $x$ thì sẽ không bao giờ đọc lời giải này.

\stepcounter{subexercise}
\arabic{subexercise}. 

\begin{align*}
   &\frac{1}{x} + \frac{1}{x - 1} < \frac{2}{x - 2} \\
   \iff &\frac{1}{x} + \frac{1}{x - 1} - \frac{2}{x - 2} < 0 \\
   \iff &\frac{(x - 1)(x - 2) + x(x - 2) - 2x(x + 1)}{x(x - 1)(x - 2)} < 0 \\
   \iff &\frac{-3x + 2}{x(x - 1)(x - 2)} < 0 \\
   \iff &(2 - 3x)x(x - 1)(x - 2) < 0.
\end{align*}

Vẽ bảng xét dấu của $(2 - 3x)x(x - 1)(x - 2)$:
\begin{table}[H]
   \centering
   \begin{tabular}{|c|ccccccccccc|}
   \hline
   $x$                 & $-\infty$ &     & $0$ &     & $\frac{2}{3}$ &     & $1$ &     & $2$ &     & $\infty$ \\
   \hline
   $2-3 x$             &           &  +  &     &  +  &       0       & $-$ &     & $-$ &     & $-$ &           \\
   \hline
   $x$                 &           & $-$ &  0  &  +  &               &  +  &     &  +  &     &  +  &           \\
   \hline
   $x-1$               &           & $-$ &     & $-$ &               & $-$ &  0  &  +  &     &  +  &           \\
   \hline
   $x-2$               &           & $-$ &     & $-$ &               & $-$ &     & $-$ &  0  &  +  &           \\
   \hline
   $(2-3x)x(x-1)(x-2)$ &           & $-$ &  0  &  +  &       0       & $-$ &  0  &  +  &  0  & $-$ &           \\
   \hline
   \end{tabular}
   \caption{Bảng xét dấu của $(2 - 3x)x(x - 1)(x - 2)$}
\end{table}



\exercise Phác thảo đồ thị của những hàm sau:

\begin{multicols}{2}
   \begin{enumerate}
      \item $\displaystyle f(x) = \frac{2x}{x^2 + 1} + 1$;
      \item $\displaystyle f(x) = \frac{x^4 + 1}{3x^2} - x$;
      \item $\displaystyle f(x) = \frac{15x^3 + x^2 - 22x - 8}{3x^2 + 3x + 8}$;
      \item $\displaystyle f(x) = \frac{x}{x + 2} + \frac{1}{x - 2}$;
      \item $\displaystyle f(x) = \left(1 - \frac{2}{x + 4}\right) \left(1 + \frac{2}{x + 1}\right)$;
      \item $\displaystyle f(x) = \frac{\frac{x^3 + 3x^2 + 3x + 1}{x^4 + 4}}{\frac{2x^2 + 2}{3x^2 + 6x + 6}}$.
   \end{enumerate}
\end{multicols}

\solution

\setcounter{subexercise}{1}
\arabic{subexercise}.

\begin{figure}[H]
	\centering
	\begin{tikzpicture}
		\draw[->] (-6, 0) -- (6, 0) node[right] {$x$};
		\draw[->] (0, -1) -- (0, 5)  node[above] {$f(x)$};
		\draw[graph thickness, samples=80, color=colorEmphasisCyan, domain=-6.000:6.000] plot (\x, {((2 * (\x))/((\x)^2 + 1) + 1) / 0.5});
		\filldraw[color= colorEmphasisCyan] (-4.0, 1.0588235294117647) circle (\pointSize) node[above right] {$\left(-4{,}00;0{,}53\right)$};
		\filldraw[color= colorEmphasisCyan] (0.0, 2.0) circle (\pointSize) node[right] {$\left(0;1\right)$};
		\filldraw[color= colorEmphasisCyan] (-1.0, 0.0) circle (\pointSize) node[below] {$\left(-1;0\right)$};
		\filldraw[color= colorEmphasisCyan] (-2.0, 0.3999999999999999) circle (\pointSize) node[below left] {$\left(-2{,}00;0{,}20\right)$};
		\filldraw[color= colorEmphasisCyan] (4.0, 2.9411764705882355) circle (\pointSize) node[above right] {$\left(4{,}00;1{,}47\right)$};
		\filldraw[color= colorEmphasisCyan] (2.0, 3.6) circle (\pointSize) node[above right] {$\left(2{,}00;1{,}80\right)$};
	\end{tikzpicture}
	\caption{Đồ thị của $f(x) = \frac{2 x}{x^{2} + 1} + 1$}
\end{figure}

\stepcounter{subexercise}
\arabic{subexercise}.

\begin{figure}[H]
	\centering
	\begin{tikzpicture}
		\draw[->] (-6, 0) -- (6, 0) node[right] {$x$};
		\draw[->] (0, -2) -- (0, 5)  node[above] {$f(x)$};
		\draw[graph thickness, samples=80, color=colorEmphasisCyan, domain=-2.636:-0.266] plot (\x, {(((\x)^4 + 1)/(3*(\x)^2) - (\x)) / 1});
		\draw[graph thickness, samples=80, color=colorEmphasisCyan, domain=0.252:5.650] plot (\x, {(((\x)^4 + 1)/(3*(\x)^2) - (\x)) / 1});
		\filldraw[color=colorEmphasisCyan] (-0.762, 1.5296232222245185) circle (\pointSize) node[below] {$\left(-0{,}76;1{,}53\right)$};
		\filldraw[color=colorEmphasisCyan] (1.703, -0.6213294211232134) circle (\pointSize) node[below] {$\left(1{,}70;-0{,}62\right)$};
		\filldraw[color=colorEmphasisCyan] (0.765, -0.0003435000071200234) circle (\pointSize) node[above left] {$\left(0{,}77;0{,}00\right)$};
		\filldraw[color=colorEmphasisCyan] (2.962, 0.00047477505739657033) circle (\pointSize) node[below right] {$\left(2{,}96;0{,}00\right)$};
		\filldraw[color=colorEmphasisCyan] (4.0, 1.354166666666666) circle (\pointSize) node[below right] {$\left(4{,}00;1{,}35\right)$};
		\filldraw[color=colorEmphasisCyan] (-2.0, 3.4166666666666665) circle (\pointSize) node[below left] {$\left(-2{,}00;3{,}42\right)$};
		\filldraw[color=colorEmphasisCyan] (-0.3, 4.033703703703704) circle (\pointSize) node[above left] {$\left(-0{,}30;4{,}03\right)$};

	\end{tikzpicture}
	\caption{Đồ thị của $f(x) = \frac{x^{4} + 1}{3 x^{2}} - x$}
\end{figure}

\stepcounter{subexercise}
\arabic{subexercise}.

\begin{figure}[H]
	\centering
	\begin{tikzpicture}
		\draw[->] (-6, 0) -- (6, 0) node[right] {$x$};
		\draw[->] (0, -7) -- (0, 4)  node[above] {$f(x)$};
		\draw[graph thickness, samples=80, color=colorEmphasisCyan, domain=-6.000:6.000] plot (\x, {((15*((\x)/3)^3 + ((\x)/3)^2 - 22*((\x)/3) - 8)/(3*((\x)/3)^2 + 3*((\x)/3) + 8)) / 1});
		\filldraw[color=colorEmphasisCyan] (-3.0, 0.0) circle (\pointSize) node[above left] {$\left(-1;0\right)$};
		\filldraw[color=colorEmphasisCyan] (-1.2000000000000002, 0.0) circle (\pointSize) node[above right] {$\left(- \frac{2}{5};0\right)$};
		\filldraw[color=colorEmphasisCyan] (0.0, -1.0) circle (\pointSize) node[below left] {$\left(0;-1\right)$};
		\filldraw[color=colorEmphasisCyan] (4.0, 0.0) circle (\pointSize) node[above left] {$\left(\frac{4}{3};0\right)$};
      \filldraw[color=colorEmphasisCyan] (-2.142, 0.3733228632366404) circle (\pointSize) node[above] {$\left(-0{,}71;0{,}37\right)$};
		\filldraw[color=colorEmphasisCyan] (1.494, -1.6463553783683789) circle (\pointSize) node[below] {$\left(0{,}50;-1{,}65\right)$};
	\end{tikzpicture}
	\caption{Đồ thị của $f(x) = \frac{15 x^{3} + x^{2} - 22 x - 8}{3 x^{2} + 3 x + 8}$}
\end{figure}

\stepcounter{subexercise}
\arabic{subexercise}.

\begin{figure}[H]
	\centering
	\begin{tikzpicture}
		\draw[->] (-6, 0) -- (6, 0) node[right] {$x$};
		\draw[->] (0, -4) -- (0, 4)  node[above] {$f(x)$};
		\draw[graph thickness, samples=80, color=colorEmphasisCyan, domain=-6.000:-2.622] plot (\x, {(((\x)/1)/(((\x)/1) + 2) + 1/(((\x)/1) - 2)) / 1});
		\draw[graph thickness, samples=80, color=colorEmphasisCyan, domain=-1.576:1.776] plot (\x, {(((\x)/1)/(((\x)/1) + 2) + 1/(((\x)/1) - 2)) / 1});
		\draw[graph thickness, samples=80, color=colorEmphasisCyan, domain=2.288:6.000] plot (\x, {(((\x)/1)/(((\x)/1) + 2) + 1/(((\x)/1) - 2)) / 1});
		\filldraw[color=colorEmphasisCyan] (0.0, -0.5) circle (\pointSize) node[above] {$\left(0;- \frac{1}{2}\right)$};
		\filldraw[color=colorEmphasisCyan] (1.0, -0.6666666666666666) circle (\pointSize) node[above right] {$\left(1;- \frac{2}{3}\right)$};
		\filldraw[color=colorEmphasisCyan] (-1.0, -1.3333333333333333) circle (\pointSize) node[above left] {$\left(-1;- \frac{4}{3}\right)$};
		\filldraw[color=colorEmphasisCyan] (-3.0, 2.8) circle (\pointSize) node[below right] {$\left(-3;\frac{14}{5}\right)$};
		\filldraw[color=colorEmphasisCyan] (-4.0, 1.8333333333333333) circle (\pointSize) node[below right] {$\left(-4;\frac{11}{6}\right)$};
		\filldraw[color=colorEmphasisCyan] (5.0, 1.0476190476190477) circle (\pointSize) node[below] {$\left(5;\frac{22}{21}\right)$};
      \filldraw[color=colorEmphasisCyan] (1.75, -3.533333333333333) circle (\pointSize) node[right] {$\left(1{,}75;-3{,}53\right)$};
		\filldraw[color=colorEmphasisCyan] (2.5, 2.5555555555555554) circle (\pointSize) node[left] {$\left(2{,}50;2{,}56\right)$};
		\filldraw[color=colorEmphasisCyan] (3.5, 1.303030303030303) circle (\pointSize) node[below left] {$\left(3{,}50;1{,}30\right)$};
	\end{tikzpicture}
	\caption{Đồ thị của $f(x) = \frac{x}{x + 2} + \frac{1}{x - 2}$}
\end{figure}

\stepcounter{subexercise}
\arabic{subexercise}.

Để dễ nhận dạng hàm số này, thực hiện một số biến đổi như sau:

\begin{align*}
   f(x) &= \left(1 - \frac{2}{x + 4}\right) \left(1 + \frac{2}{x + 1}\right) \\
   &= \frac{x + 2}{x + 4}\cdot \frac{x + 3}{x + 1} \\
   &= \frac{(x + 2)(x + 3)}{(x + 4)(x + 1)}.
\end{align*}

\begin{figure}[H]
	\centering
	\begin{tikzpicture}
		\draw[->] (-8, 0) -- (4, 0) node[right] {$x$};
		\draw[->] (0, -4) -- (0, 4)  node[above] {$f(x)$};
		\draw[graph thickness, samples=80, color=colorEmphasisCyan, domain=-8.000:-4.208] plot (\x, {((1 - 2 / (((\x)/1) + 4)) * (1 + 2 / (((\x)/1) + 1))) / 1});
		\draw[graph thickness, samples=80, color=colorEmphasisCyan, domain=-3.860:-1.140] plot (\x, {((1 - 2 / (((\x)/1) + 4)) * (1 + 2 / (((\x)/1) + 1))) / 1});
		\draw[graph thickness, samples=80, color=colorEmphasisCyan, domain=-0.792:4.000] plot (\x, {((1 - 2 / (((\x)/1) + 4)) * (1 + 2 / (((\x)/1) + 1))) / 1});
		\filldraw[color=colorEmphasisCyan] ({-6.6}, { 1.1373626373626373 }) circle (\pointSize) node[below] {$\left(-6{,}60;1{,}14\right)$};
		\filldraw[color=colorEmphasisCyan] ({-5.5}, { 1.2962962962962963 }) circle (\pointSize) node[below right] {$\left(-5{,}50;1{,}30\right)$};
		\filldraw[color=colorEmphasisCyan] ({-4.4}, { 2.4705882352941164 }) circle (\pointSize) node[right] {$\left(-4{,}40;2{,}47\right)$};
		\filldraw[color=colorEmphasisCyan] ({-2.0}, { 0.0 }) circle (\pointSize) node[right, xshift=4pt, yshift=-3pt] {$\left(-2;0\right)$};
		\filldraw[color=colorEmphasisCyan] ({-2.5}, { 0.11111111111111106 }) circle (\pointSize) node[above] {$\left(-2{,}50;0{,}11\right)$};
		\filldraw[color=colorEmphasisCyan] ({-3.0}, { 0.0 }) circle (\pointSize) node[left, xshift=-4pt, yshift=-3pt] {$\left(-3;0\right)$};
		\filldraw[color=colorEmphasisCyan] ({0.0}, { 1.5 }) circle (\pointSize) node[above right] {$\left(0{,}00;1{,}50\right)$};
		\filldraw[color=colorEmphasisCyan] ({1.1}, { 1.1867413632119512 }) circle (\pointSize) node[below] {$\left(1{,}10;1{,}19\right)$};
		\filldraw[color=colorEmphasisCyan] ({2.2}, { 1.1008064516129032 }) circle (\pointSize) node[above] {$\left(2{,}20;1{,}10\right)$};
		\filldraw[color=colorEmphasisCyan] ({3.3}, { 1.0637145587766805 }) circle (\pointSize) node[below] {$\left(3{,}30;1{,}06\right)$};
	\end{tikzpicture}
	\caption{Đồ thị của $f(x) = \left(1 - \frac{2}{x + 4}\right) \left(1 + \frac{2}{x + 1}\right)$}
\end{figure}


\stepcounter{subexercise}
\arabic{subexercise}. Thực hiện một số biến đổi đơn giản:

\begin{align*}
   f(x) &= \frac{\frac{x^3 + 3x^2 + 3x + 1}{x^4 + 4}}{\frac{2x^2 + 2}{3x^2 + 6x + 6}} = \frac{\frac{(x + 1)^3}{(x^4 + 4x^2 + 4) - 4x^2}}{\frac{2\left(x^2 + 1\right)}{3\left(x^2 + 2x + 2\right)}}\\
   &= \frac{(x + 1)^3}{\left(x^2 + 2\right)^2 - (2x)^2}\cdot\frac{3\left(x^2 + 2x + 2\right)}{2\left(x^2 + 1\right)} \\
   &= \frac{(x + 1)^3}{\left(x^2 -2x + 2\right)\left(x^2 + 2x + 2\right)}\cdot\frac{3\left(x^2 + 2x + 2\right)}{2\left(x^2 + 1\right)} \\
   &= \frac{3(x + 1)^3}{2\left(x^2 + 1\right)\left(x^2 -2x + 2\right)}.
\end{align*}

\begin{figure}[H]
	\centering
	\begin{tikzpicture}
		\draw[->] (-8, 0) -- (3, 0) node[right] {$x$};
		\draw[->] (0, -1) -- (0, 6.5)  node[above] {$f(x)$};
		\draw[graph thickness, samples=80, color=colorEmphasisCyan, domain=-8.000:3.000] plot (\x, {((3*(((\x)/1) + 1)^3)/(2*(((\x)/1)^2+1)*(((\x)/1)^2 - 2*((\x)/1) + 2))) / 1});
		\filldraw[color=colorEmphasisCyan] (-5.9, -0.10137936276058916) circle (\pointSize) node[below] {$\left(-5{,}90;-0{,}10\right)$};
		\filldraw[color=colorEmphasisCyan] (-7.0, -0.0996923076923077) circle (\pointSize) node[above] {$\left(-7{,}00;-0{,}10\right)$};
		\filldraw[color=colorEmphasisCyan] (-1.0, 0.0) circle (\pointSize) node[below] {$\left(-1;0\right)$};
		\filldraw[color=colorEmphasisCyan] (0.0, 0.75) circle (\pointSize) node[below right] {$\left(0;\frac{3}{4}\right)$};
      \filldraw[color=colorEmphasisCyan] (-3.0, -0.07058823529411766) circle (\pointSize) node[above] {$\left(-3{,}00;-0{,}07\right)$};
		\filldraw[color=colorEmphasisCyan] (1.2, 6.294136191677176) circle (\pointSize) node[above] {$\left(1{,}20;6{,}29\right)$};
		\filldraw[color=colorEmphasisCyan] (2.4, 2.946385734847273) circle (\pointSize) node[above right] {$\left(2{,}40;2{,}95\right)$};
	\end{tikzpicture}
	\caption{Đồ thị của $f(x) = \frac{\frac{x^3 + 3x^2 + 3x + 1}{x^4 + 4}}{\frac{2x^2 + 2}{3x^2 + 6x + 6}}$}
\end{figure}

% \input{\chapdir toan_hoc_nen_tang/ham_so_mot_bien/phep_hop_ham.tex}
% \subsection{Hàm số xác định từng phần}

\ % Lùi đầu dòng

Không phải lúc nào hàm số trong đời sống có thể biểu diễn dưới dạng một biểu thức. Khi này, chúng ta sẽ chia nhỏ đồ thị của hàm số thành các phần nhỏ, và biểu diễn từng phần thông qua biểu thức. Đó cũng là lí do cho tên gọi \defText{hàm số xác định từng phần}.

Một ví dụ là hàm giá trị tuyệt đối. Khả năng cao bạn đọc đã biết rằng giá trị tuyệt đối của một số $x$\footnote{Nếu không biết thì bạn đọc đọc qua phần đồ thị vẫn hiểu chứ?} được xác định như sau,

\begin{equation*}
   \defMath{|x|= } \begin{cases}
      \defMath{x \defText{ nếu } x \geq 0 }\\
      \defMath{-x \defText{ nếu } x < 0 }
   \end{cases}.
\end{equation*}

Đây là hàm có tính thông dụng cao, và nếu thông dụng thì người ta sẽ tìm ra những tính chất quan trọng để sử dụng. Tác giả sẽ liệt kê ra một vài tính chất như sau: Với mọi số thực $x$ và $y$,
\begin{itemize}
   \item $|xy| = |x||y|$;
   \item $|x| + |y| \geq |x + y| \geq |x| - |y|$.
\end{itemize} 
Từ đây, chúng ta có một số mối quan hệ. Ví dụ:
$$
|-x| = \left|(-1)x\right| = |-1||x| = |x|.
$$

Hai hàm từng phần khác cũng quan trọng nhưng ít khi được đề cập đến là hàm sàn và hàm trần. \defText{Hàm sàn} (hay \defText{hàm phần nguyên}, \defText{hàm làm tròn xuống}) tác dụng lên $x$ sẽ làm tròn xuống $x$ đến số nguyên lớn nhất nhưng không nhỏ hơn $x$ với kí hiệu là $\defMath{\lfloor x \rfloor}$.
Tương tự, \defText{hàm trần} (hay \defText{hàm làm tròn lên}) của $x$ làm tròn lên $x$ đến số nguyên nhỏ nhất nhưng không lớn hơn $x$ với kí hiệu là $\defMath{\lceil x \rceil}$.

Ví dụ:
\begin{multicols}{3}
   \begin{itemize}
      \item $\lfloor 2{,}5 \rfloor = 2$;
      \item $\lceil 2{,}5 \rceil = 3$;
      \item $\lfloor -2{,}5 \rfloor = -3$;
      \item $\lceil -2{,}5 \rceil = -2$;
      \item $\lfloor 2 \rfloor = 2$;
      \item $\lceil 2 \rceil = 2$.
   \end{itemize}
\end{multicols}

Và rõ ràng rằng không phải mọi giá trị trong tự nhiên và xã hội đều biểu diễn tốt nhất dưới dạng số thập phân. Chúng ta không mấy khi cắt nửa quả táo để bán cho nhau, hay không có bãi đỗ xe nào nhận đỗ $0{,}2$ cái xe (hoặc ít nhất tác giả chưa thấy bãi nào như vậy).

Tương tự như hàm giá trị tuyệt đối, hàm sàn và hàm trần cũng có một số tính chất. Trong đó, phải kể đến tính chất hay được dùng nhất (theo góc nhìn chủ quan của tác giả) mà phục vụ việc xác định giá trị của hàm trần, hay hàm sàn là: Nếu $n$ là số nguyên và $x$ là số thực thì
\begin{equation*}
   \begin{cases}
      n \leq x < n + 1 \iff \lfloor x \rfloor = n \\
      n - 1 < x \leq n \iff \lceil x \rceil = n
   \end{cases}.
\end{equation*}

Đầu tiên, chúng ta sẽ chứng minh $n \leq x < n + 1 \iff \lfloor x \rfloor = n$. Chứng minh chiều xuôi, giả sử $n \leq x < n + 1$. Gọi $M_m$ là tập hợp các số nguyên nhỏ hơn hoặc bằng $x$. Rõ ràng rằng $M_m$ khác rỗng, do có $n$ làm phần tử. Ngoài ra, mọi số nguyên $a > n$ không làm phần tử của $M_m$ vì $a > n \iff a \geq n + 1 \implies a > x$. Kết hợp lại, mọi phần tử $b$ trong $M_m$ phải thỏa mãn $b \leq n$. Do đó, $n$ là số nguyên lớn nhất không vượt quá $x$, hay $\lfloor x \rfloor = n$.

Để chứng minh chiều ngược lại, giả sử $\lfloor x \rfloor = n$. Từ định nghĩa, $n \leq x$. Ngoài ra, nếu $n + 1 \leq x$, có $n + 1 > n$, cho nên $n + 1$ sẽ vi phạm vai trò của $n$, tạo nên mâu thuẫn. Do đó, $x < n + 1$. Kết hợp hai chiều, chúng ta có chứng minh đầu tiên.

Tiếp theo, cần phải chứng minh $n - 1 < x \leq n \iff \lceil x \rceil = n$. Chứng minh chiều xuôi, gọi $M_M$ là tập hợp các số nguyên lớn hơn hoặc bằng $x$. Có $n \in M_M$ nên $M_M$ khác rỗng. Ngoài ra, nếu $a < n$ thì $a \leq n - 1 \implies a < x$, cho nên $a \notin M_M$. Qua đó, $n$ là số nguyên nhỏ nhất không vượt quá $x$, hay $\lceil x \rceil = n$.

Chứng minh chiều ngược lại, giả sử $\lceil x \rceil = n$. Từ định nghĩa, $n \geq x$. Ngoài ra, nếu $n - 1 \geq x$, có $n - 1 < n$, cho nên $n - 1$ sẽ vi phạm vai trò của $n$, tạo nên mâu thuẫn. Do đó, $x > n - 1$. Kết hợp hai chiều, chúng ta có điều phải chứng minh.

Để giải những vấn đề với hàm xác định giá trị từng phần, tương tự như cái tên, chúng ta chủ yếu chia bài toán theo các trường hợp phù hợp với từng phần của hàm đầu vào.

\exercise Giải các phương trình và bất phương trình sau trên ẩn $x$ thực:

\begin{multicols}{2}
   \begin{enumerate}
      \item $|x + 4| = 9$;
      \item $|x - 3| = -9$;
      \item $|7 - 2x| < 9$;
      \item $|3 + 6x| \geq 9$;
      \item $2x + 3 + |3x + 4| > 0$;
      \item $|x + 4| = |7x - 12|$;
      \item $|6x + 9| > |6x - 3|$;
      \item $\left|2x + 2\right| + |x + 1| = 9$;
      \item $|3x + 3| + |3x - 4| \leq 7$;
      \item $\left|2(x - 1)^2 - 4\right|$ = 2;
      \item $\left|2x^2 - 2x - 2\right| = \left|3x^2 - 4x - 2\right|$;
      \item $\left|x^3 - 3x^2 + x\right| \leq |x|$.
   \end{enumerate}
\end{multicols}

\solution

\setcounter{subexercise}{1}
\arabic{subexercise}. Xét hai trường hợp:

\textcolor{colorEmphasisCyan}{Trường hợp một --- $x + 4 \geq 0$}. Khi này, phá dấu giá trị tuyệt đối để có $|x + 4| = x + 4$. Cho nên phương trình ban đầu sẽ tương đương với:

\begin{equation*}
   x + 4 = 9 \iff x = 5.
\end{equation*}

\textcolor{colorEmphasis}{Trường hợp hai --- $x + 4 < 0$}. Khi này,

\begin{align*}
   &\begin{cases}
      |x + 4| = 9 \\
      x + 4 < 0
   \end{cases} \\
   \iff &-(x + 4) = 9 \equationexplanation{Phá dấu giá trị tuyệt đối: $|x + 4| = -(x + 4)$.}\\
   \iff &x + 4 = -9 \iff x = -13.
\end{align*}

Kết hợp hai trường hợp, có được $x \in \{5; -13\}$. Thử lại trực tiếp thấy thỏa mãn.

Vậy tập nghiệm của phương trình là $\{5; -13\}$.

\stepcounter{subexercise}
\arabic{subexercise}. Đặt $f(x) = |x|$.

Nếu \textcolor{colorEmphasisCyan}{$x \geq 0$} thì $f(x) = |x| = x$ và hiển nhiên $f(x) \geq 0$. Nếu \textcolor{colorEmphasis}{$x < 0$} thì $f(x) = |x| = -x$. Có $x < 0 \iff -x > 0 \iff f(x) > 0$.

Kết hợp lại, chúng ta có $f(x) \geq 0$ với mọi $x \in \mathbb{R}$. Suy ra được rằng $f(x - 3) \geq 0$. Tuy nhiên, phương trình được cho có thể được viết lại là $f(x - 3) = -9$. Do đó, phương trình vô nghiệm.

Sai lầm thường gặp ở dạng bài này là có suy luận như sau:

\begin{equation*}
   |x - 3| = -9 \iff \left[\begin{array}{l}
      x - 3 = -9 \\
      x - 3 = 9
   \end{array}\right..
\end{equation*}

\stepcounter{subexercise}
\arabic{subexercise}. Một lần nữa, xét hai trường hợp:

\textcolor{colorEmphasisCyan}{Trường hợp một --- $7 - 2x \geq 0$}. Thực hiện biến đổi:

\begin{align*}
   &\begin{cases}
      |7 - 2x| < 9 \\
      7 - 2x \geq 0
   \end{cases} \\
   \iff &\begin{cases}
      7 - 2x < 9 \\
      7 - 2x \geq 0
   \end{cases} \\
   \iff &0 \leq 7 - 2x < 9 \\
   \iff &-7 \leq -2x < 2 \\
   \iff &\frac{7}{2} \geq x > -1.
\end{align*}

\textcolor{colorEmphasis}{Trường hợp hai --- $7 - 2x < 0$}:

\begin{align*}
   &\begin{cases}
      |7 - 2x| < 9 \\
      7 - 2x < 0
   \end{cases} \\
   \iff &\begin{cases}
      - (7 - 2x) < 9 \\
      7 - 2x < 0
   \end{cases} \\
   \iff & -9 < 7 - 2x < 0 \\
   \iff & -16 < -2x < -7 \\
   \iff & 8 > x > \frac{7}{2}.
\end{align*}

Kết hợp hai trường hợp, có được $x \in \left(-1; 8\right)$. Có biến đổi là tương đương trong tập xác định cho nên phương trình có nghiệm $x \in \left(-1; 8\right)$.

\stepcounter{subexercise}
\arabic{subexercise}. \textcolor{colorEmphasisCyan}{Trường hợp một --- $3 + 6x \geq 0$}:

\begin{align*}
   &\begin{cases}
      |3 + 6x| \geq 9 \\
      3 + 6x \geq 0
   \end{cases} \\
   \iff 3 + 6x \geq 9 \\
   \iff x \geq 1.
\end{align*}

\textcolor{colorEmphasis}{Trường hợp hai --- $3 + 6x < 0$}:

\begin{align*}
   &\begin{cases}
      |3 + 6x| \geq 9 \\
      3 + 6x < 0
   \end{cases} \\
   \iff &-(3 + 6x) \geq 9 \\
   \iff &3 + 6x \leq -9 \\
   \iff &x \leq -2.
\end{align*}

Do trong mỗi trường hợp, mọi biến đổi là tương đương, nên chúng ta có tập nghiệm của bất phương trình là $\left(-\infty; -2\right] \cup \left[1; \infty\right)$.

\stepcounter{subexercise}
\arabic{subexercise}. \textcolor{colorEmphasisCyan}{Trường hợp một --- $3x + 4 \geq 0$}. Khi này, 

\begin{equation*}
   \begin{cases}
      2x + 3 + |3x + 4| = 2x + 3 + (3x + 4) \\
      3x + 4 \geq 0
   \end{cases} \iff \begin{cases}
      2x + 3 + |3x + 4| = 5x + 7 \\
      x \geq -\frac{4}{3}
   \end{cases}
\end{equation*}

\begin{equation*}
   \implies 2x + 3 + |3x + 4| \geq 5 \left(-\frac{4}{3}\right) + 7 \iff 2x + 3 + |3x + 4| \geq \frac{1}{3} > 0.
\end{equation*}

\textcolor{colorEmphasis}{Trường hợp hai --- $3x + 4 < 0$}:

\begin{equation*}
   \begin{cases}
      2x + 3 + |3x + 4| = 2x + 3 - (3x + 4) \\
      3x + 4 < 0
   \end{cases} \iff \begin{cases}
      2x + 3 + |3x + 4| = -x - 1 \\
      x < -\frac{4}{3}
   \end{cases}
\end{equation*}

Từ đó, có:

\begin{equation*}
   -x > \frac{4}{3} \iff -x - 1 > \frac{1}{3} \implies 2x + 3 + |3x + 4| > 0.
\end{equation*}

Từ hai trường hợp, chúng ta có $2x + 3 + |3x + 4| > 0$ với mọi giá trị thực của $x$. Do đó, tập nghiệm của phương trình là $\mathbb{R}$.

\stepcounter{subexercise}
\arabic{subexercise}. Tóm tắt các trường hợp thông qua bảng xét dấu sau:

\begin{table}[H]
   \centering
   \begin{tabular}{|c|ccccccc|}
   \hline
   $x$          & $-\infty$ &     & $-4$ &     & $\frac{12}{7}$ &   & $\infty$ \\
   \hline
   $x+4$        &           & $-$ &  0  &  +  &     & + &           \\
   \hline
   $7x-12$        &           & $-$ &     & $-$ &  0  & + &           \\
   \hline
   \end{tabular}
   \caption{Bảng xét dấu cho $x+4$ và $7x-12$}
   \label{tab:toan_hoc_nen_tang:ham_so_mot_bien:ham_tung_phan:gpt7}
\end{table}

\textcolor{colorEmphasisCyan}{Trường hợp một --- $x < -4$}. Khi này, phương trình ban đầu trở thành

\begin{align*}
   |x + 4| &= |7x - 12| \\
   \iff -(x + 4) &= -(7x - 12) \\
   \iff x = \frac{8}{3}.
\end{align*}

Chúng ta không nhận nghiệm này trong trường hợp này do trái với giả thiết $x < -4$.

\textcolor{colorEmphasis}{Trường hợp hai --- $-4\leq x < \frac{12}{7} $}. Khi này, 

\begin{align*}
   |x + 4| &= |7x - 12| \\
   \iff x + 4 &= -(7x - 12) \\
   \iff x &= 1.
\end{align*}

\textcolor{colorEmphasisGreen}{Trường hợp ba --- $x \geq \frac{12}{7}$}. Từ đây,

\begin{align*}
   |x + 4| &= |7x - 12| \\
   \iff x + 4 &= 7x - 12 \\
   \iff x &= \frac{8}{3}.
\end{align*}

Kết hợp các trường hợp và kiểm tra lại các nghiệm, chúng ta có nghiệm của phương trình là $x \in \left\{1; \frac{8}{3}\right\}$.

\stepcounter{subexercise}
\arabic{subexercise}. Kẻ bảng xét dấu

\begin{table}[H]
   \centering
   \begin{tabular}{|c|ccccccc|}
   \hline
   $x$          & $-\infty$ &     & $-\frac{3}{2}$ &     & $\frac{1}{2}$ &   & $\infty$ \\
   \hline
   $6x + 9$        &           & $-$ &  0  &  +  &     & + &           \\
   \hline
   $6x - 3$        &           & $-$ &     & $-$ &  0  & + &           \\
   \hline
   \end{tabular}
   \caption{Bảng xét dấu cho $6x + 9$ và $6x - 3$}
   \label{tab:toan_hoc_nen_tang:ham_so_mot_bien:ham_tung_phan:gpt7}
\end{table}

\textcolor{colorEmphasisCyan}{Trường hợp một --- $x < -\frac{3}{2}$}. Khi này, bất phương trình ban đầu trở thành

\begin{align*}
   |6x + 9| &> |6x - 3| \\
   \iff -(6x + 9) &> -(6x - 3) \\
   \iff -9 &> 3.
\end{align*}

Bất phương trình sai với mọi $x$. Đối với trường hợp này, tập nghiệm là $\emptyset$.

\textcolor{colorEmphasis}{Trường hợp hai --- $-\frac{3}{2} \leq x < \frac{1}{2}$}. Khi này,

\begin{align*}
   |6x + 9| &> |6x - 3| \\
   \iff 6x + 9 &> -(6x - 3) \\
   \iff x &> -\frac{1}{2}.
\end{align*}

\textcolor{colorEmphasisGreen}{Trường hợp ba --- $x \geq \frac{1}{2}$}:

\begin{align*}
   |6x + 9| &> |6x - 3| \\
   \iff 6x + 9 &> 6x - 3 \\
   \iff 9 &> -3
\end{align*}
luôn đúng. Kết hợp với điều kiện, chúng ta có tập nghiệm $\left[\frac{1}{2}; \infty\right)$.

Hợp tập nghiệm của cả ba trường hợp, do mọi biến đổi trong mỗi trường hợp là tương đương cho nên bất phương trình có tập nghiệm là $\left(-\frac{1}{2}; \infty\right)$.

\stepcounter{subexercise}
\arabic{subexercise}. Biến đổi cơ bản để có

\begin{align}
   \left|2x + 2\right| + \left|x + 1\right| &= 9 \nonumber\\
   \iff \left|2(x + 1)\right| + \left|x + 1\right| &= 9. \label{eq:toan_hoc_nen_tang:ham_so_mot_bien:ham_tung_phan:pt9}
\end{align}

Với \textcolor{colorEmphasisCyan}{$x \geq -1$} thì $x+ 1 \geq 0 \iff \begin{cases}
   \left|2(x + 1)\right| = 2(x + 1) \\
   \left|x + 1\right| = x + 1
\end{cases}$. Cho nên 
\begin{align*}
   \text{\refeq{eq:toan_hoc_nen_tang:ham_so_mot_bien:ham_tung_phan:pt9}} \iff &2(x + 1) + (x + 1) = 9 \\
   \iff &x = 2.
\end{align*}

Với \textcolor{colorEmphasis}{$x < -1$} thì $x + 1 < 0 \iff \begin{cases}
   \left|2(x + 1)\right| = -\left(2(x + 1)\right) \\
   \left|x + 1\right| = -\left(x + 1\right)
\end{cases}$. Cho nên
\begin{align*}
   \text{\refeq{eq:toan_hoc_nen_tang:ham_so_mot_bien:ham_tung_phan:pt9}} \iff &-\left(2(x + 1)\right) -\left(x + 1\right) = 9 \\
   \iff &x = -4.
\end{align*}

Phương trình có nghiệm là $\left\{2; -4\right\}$.

\stepcounter{subexercise}
\arabic{subexercise}. Kẻ bảng xét dấu

\begin{table}[H]
   \centering
   \begin{tabular}{|c|ccccccc|}
   \hline
   $x$          & $-\infty$ &     & $-1$ &     & $\frac{4}{3}$ &   & $\infty$ \\
   \hline
   $3x + 3$        &           & $-$ &  0  &  +  &     & + &           \\
   \hline
   $3x - 4$        &           & $-$ &     & $-$ &  0  & + &           \\
   \hline
   \end{tabular}
   \caption{Bảng xét dấu cho $3x + 3$ và $3x - 4$}
   \label{tab:toan_hoc_nen_tang:ham_so_mot_bien:ham_tung_phan:gpt10}
\end{table}

\textcolor{colorEmphasisCyan}{Trường hợp một --- $x < -1$}. Từ đó,
\begin{align*}
   |3x + 3| + |3x - 4| &\leq 7 \\
   \iff -(3x + 3) - (3x - 4) &\leq 7 \\
   \iff -6x + 1 &\leq 7 \\
   \iff x &\geq -1.
\end{align*}

Điều này trái với điều kiện $x < -1$ của trường hợp này.

\textcolor{colorEmphasis}{Trường hợp hai --- $-1 \leq x < \frac{4}{3}$}. Với điều kiện này,
\begin{align*}
   |3x + 3| + |3x - 4| &\leq 7 \\
   \iff 3x + 3 - (3x - 4) &\leq 7 \\
   \iff 7 &\leq 7.
\end{align*}
Bất phương trình này là luôn đúng.

\textcolor{colorEmphasisGreen}{Trường hợp ba --- $x \geq \frac{4}{3}$}. Khi này,
\begin{align*}
   |3x + 3| + |3x - 4| &\leq 7 \\
   \iff 3x + 3 + 3x - 4 &\leq 7 \\
   \iff x &\leq \frac{4}{3}.
\end{align*}

Kết hợp với điều kiện, có được $x = \frac{4}{3}$.

Qua ba trường hợp, tập nghiệm của bất phương trình là $\left[-1; \frac{4}{3}\right]$.

\stepcounter{subexercise}
\arabic{subexercise}.
\begin{align}
   \left|2(x - 1)^2 - 4\right| &= 2 \nonumber\\
   \iff \left|2x^2 - 2x - 2\right| &= 2. \label{eq:toan_hoc_nen_tang:ham_so_mot_bien:ham_tung_phan:pt10}
\end{align}

Xét hai trường hợp. \textcolor{colorEmphasisCyan}{Trường hợp một --- $2x^2 - 2x - 2 \geq 0$}:

\begin{align*}
   \text{\refeq{eq:toan_hoc_nen_tang:ham_so_mot_bien:ham_tung_phan:pt10}} \iff 2x^2 - 2x - 2 &= 2 \\
   2x^2 - 2x - 4 &= 0 \\
   x^2 - x - 2 &= 0.
\end{align*}
Giải phương trình này để có $x\in\left\{-1; 2\right\}$, đều thỏa mãn điều kiện $2x^2 - 2x - 2 \geq 0$.

\textcolor{colorEmphasis}{Trường hợp hai --- $2x^2 - 2x - 2 < 0$}:

\begin{align*}
   \text{\refeq{eq:toan_hoc_nen_tang:ham_so_mot_bien:ham_tung_phan:pt10}} \iff -\left(2x^2 - 2x - 2\right) &= 2 \\
   \iff -2x^2 + 2x + 2 &= 2 \\
   \iff -2x^2 + 2x &= 0.
\end{align*}
Phương trình này có nghiệm $x = 0$ hoặc $x = 1$, đều thỏa mãn điều kiện $2x^2 - 2x - 2 < 0$.

Qua hai trường hợp, tập nghiệm của phương trình là $\left\{-1; 0; 1; 2\right\}$.

\stepcounter{subexercise}
\arabic{subexercise}. Vì cả hai đa thức bậc hai $2x^2 -2x - 2$ và $3x^2 - 4x - 2$ đều không có nghiệm đẹp, nên tác giả sẽ không vẽ bảng xét dấu cho bài này mà chia làm bốn trường hợp. Để rút gọn, tác giả sẽ gộp lại như sau:

\textcolor{colorEmphasisCyan}{Trường hợp một --- $
\begin{cases}
   2x^2 - 2x - 2 \geq 0 \\
   3x^2 - 4x - 2 \geq 0
\end{cases}$} và \textcolor{colorEmphasisCyan}{trường hợp hai --- $
\begin{cases}
   2x^2 - 2x - 2 < 0 \\
   3x^2 - 4x - 2 < 0
\end{cases}$}. Cả hai trường hợp này sau khi phá bỏ dấu giá trị tuyệt đối đều cho:
\begin{align*}
   2x^2 - 2x - 2 &= 3x^2 - 4x - 2 \\
   \iff 0 &= x^2 - 2x \\
   \iff x &\in \left\{0; 2\right\}.
\end{align*}

\textcolor{colorEmphasis}{Trường hợp ba --- $
\begin{cases}
   2x^2 - 2x - 2 \geq 0 \\
   3x^2 - 4x - 2 < 0
\end{cases}$} và \textcolor{colorEmphasis}{trường hợp bốn --- $
\begin{cases}
   2x^2 - 2x - 2 < 0 \\
   3x^2 - 4x - 2 \geq 0
\end{cases}$}. Cả hai trường hợp đều suy ra
\begin{align*}
   2x^2 - 2x - 2 &= -\left(3x^2 - 4x - 2\right) \\
   \iff 5x^2 - 6x - 4 &= 0 \\
   \iff x &\in \left\{\frac{3 + \sqrt{29}}{5}; \frac{3 - \sqrt{29}}{5}\right\}.
\end{align*}

Kết hợp các tập nghiệm và kiểm tra trực tiếp, chúng ta có tập nghiệm của phương trình là $$\left\{0; 2; \frac{3 + \sqrt{29}}{5}; \frac{3 - \sqrt{29}}{5}\right\}.$$

\stepcounter{subexercise}
\arabic{subexercise}. Xét \textcolor{colorEmphasisCyan}{trường hợp một --- $x < 0$}, có $x^3 - 3x^2 + x = x\left(x^2 - 3x + 1\right)$. Vì $x < 0$ nên $-3x > 0 \implies x^2 - 3x + 1 > 0$ $\implies x^3 - 3x^2 + x < 0$. Do đó,
$$
\begin{cases}
   |x^3 - 3x^2 + x| = -\left(x^3 - 3x^2 + x\right) \\
   |x| = -x
\end{cases}.
$$
Qua đó, bất phương trình ban đầu trở thành:
\begin{align*}
   \left|x^3 - 3x^2 + x\right| &\leq \left|x\right| \\
   \iff -\left(x^3 - 3x^2 + x\right) &\leq -x \\
   \iff -x^3 + 3x^2 &\leq 0 \\
   \iff x^2 (3 - x) &\leq 0 \\
   \iff 3 - x &\leq 0 \equationexplanation{$x^2 \geq 0$ với mọi $x \in \mathbb{R}$} \\
   \iff x &\geq 3.
\end{align*}
Kết quả này mâu thuẫn với điều kiện $x < 0$ nên không có nghiệm trong trường hợp này.

Xét $x \geq 0$, chúng ta chia làm hai trường hợp nhỏ. Cụ thể, \textcolor{colorEmphasis}{trường hợp hai --- $\begin{cases}
   x \geq 0 \\
   x^3 - 3x^2 + x \geq 0
\end{cases}$}. Khi này

\begin{align*}
   |x^3 - 3x^2 + x| &\leq |x| \\
   \iff x^3 - 3x^2 + x &\leq x \\
   \iff x^3 - 3x^2 &\leq 0 \\
   \iff x^2(x - 3) &\leq 0 \\
   \iff x - 3 &\leq 0 \equationexplanation{$x^2 \geq 0$ với mọi $x \in \mathbb{R}$} \\
   \iff x &\leq 3.
\end{align*}

Cần phải kết hợp với điều kiện để xác định nghiệm thỏa mãn. Có

\begin{align*}
   x^3 - 3x^2 + x &\geq 0 \\
   \iff x(x^2 - 3x + 1) &\geq 0 \\
   \iff x^2 - 3x + 1 &\geq 0 \equationexplanation{$x \geq 0$ theo điều kiện}.
\end{align*}

Kẻ bảng xét dấu

\begin{table}[H]
   \centering
   \begin{tabular}{|c|ccccccc|}
   \hline
   $x$           & $-\infty$ &   & $\frac{3 - \sqrt{5}}{2}$ &     & $\frac{3 + \sqrt{5}}{2}$ &   & $\infty$ \\
   \hline
   $x^{2}-3x+1$  &           & + &                 0                 & $-$ &                0                 & + &           \\
   \hline
   \end{tabular}
   \caption{Bảng xét dấu của $x^{2}-3x+1$}
   \label{tab:toan_hoc_nen_tang:ham_so_mot_bien:ham_tung_phan:bxd12_x2_t3x_1}
\end{table}

Qua đó, nghiệm của bất phương trình trong trường hợp này là $\left[0; \frac{3 - \sqrt{5}}{2}\right] \cup \left[\frac{3 + \sqrt{5}}{2}; 3\right]$.

\textcolor{colorEmphasisGreen}{Trường hợp ba --- $
\begin{cases}
   x \geq 0 \\
   x^3 - 3x^2 + x < 0
\end{cases}$} $\iff x^2 - 3x + 1 < 0$. Từ bảng xét dấu \ref{tab:toan_hoc_nen_tang:ham_so_mot_bien:ham_tung_phan:bxd12_x2_t3x_1}, $x$ phải nằm trong đoạn $\left[\frac{3 - \sqrt{5}}{2}; \frac{3 + \sqrt{5}}{2}\right]$. Ngoài ra, từ bất phương trình:
\begin{align*}
   |x^3 - 3x^2 + x| &\leq |x| \\
   \iff -\left(x^3 - 3x^2 + x\right) &\leq x \\
   \iff -x^3 + 3x^2 - 2x &\leq 0 \\
   \iff -x(x-1)(x-2) &\leq 0 \\
   \iff (x - 1)(x - 2) &\geq 0 \equationexplanation{$x < 0$ theo điều kiện}.
\end{align*}

Kẻ bảng xét dấu cho $(x - 1)(x - 2)$:
\begin{table}[H]
   \centering
   \begin{tabular}{|c|ccccccc|}
      \hline
      $x$          & $-\infty$ &     & $1$ &     & $2$ &   & $\infty$ \\
      \hline
      $x-1$        &           & $-$ &  0  &  +  &     & + &           \\
      \hline
      $x-2$        &           & $-$ &     & $-$ &  0  & + &           \\
      \hline
      $(x-1)(x-2)$ &           &  +  &  0  & $-$ &  0  & + &           \\
      \hline
   \end{tabular}
   \caption{Bảng xét dấu của $(x-1)(x-2)$}
\end{table}

Qua đó, nghiệm của bất phương trình trong trường hợp này là $\left[\frac{3 - \sqrt{5}}{2}; 1\right] \cup \left[2; \frac{3 + \sqrt{5}}{2}\right]$.

Qua ba trường hợp, chúng ta có tập nghiệm của bất phương trình: $\left[0; 1\right] \cup \left[2; 3\right]$.

\exercise Phác thảo đồ thị của những hàm sau:

\begin{multicols}{2}
   \begin{enumerate}
      \item $f(x) = \begin{cases}
         x + 1 \text{ nếu } x \leq 1 \\
         2 \text{ nếu } x > 1
      \end{cases}$;
      \item $f(x) = \begin{cases}
         x^3 + 4 \text{ nếu } x < 0 \\
         -x^2 + 1 \text{ nếu } x \geq 0
      \end{cases}$;
      \item $f(x) = \begin{cases}
         -\frac{4}{x^2} \text{ nếu } -2 > x \geq -3 \\
         \parbox{0.29\textwidth}{$\begin{array}{cl}
            -\frac{5}{x^2 + 1} &\text{nếu } x \geq -2 \text{ thực để} \\
            &\frac{5}{x^2 + 1}\text{ là số nguyên}
         \end{array}$}
      \end{cases}$;
      \item $f(x) = \begin{cases}
         \frac{2x - 1}{x - 1} \text{ nếu } -3 \leq x < 0 \\
         \left(x + 1\right)^2 - 3x \text{ nếu } 0 \leq x < 2 \\
         \frac{2x - 1}{x - 1} \text{ nếu } 2 \leq x \leq 3
      \end{cases}$;
      \item $f(x) = \begin{cases}
         x^3 + 3 \text{ nếu } x \leq 0 \\
         -2x + 2 \text{ nếu } 0 < x < 1 \\
         2 + x - x^2 \text{ nếu } x \geq 1
      \end{cases}$;
      \item $f(x) = |x|$;
      \item $f(x) = \left|2x^2 - 4x\right|$;
      \item $f(x) = \left|x^3 - 3x^2 + x\right| - \left|x\right|$;
   \end{enumerate}
\end{multicols}

\solution

\setcounter{subexercise}{1}
\arabic{subexercise}. Giống như tất cả các bài tập khác liên quan đến hàm xác định giá trị từng phần, chia bài toán thành các trường hợp phụ thuộc vào kết cấu của hàm số. Sau đó, bạn đọc có thể vẽ từng phần và chồng chúng lên nhau để tạo thành đồ thị cuối cùng.

\begin{figure}[H]
	\centering
	\begin{tikzpicture}
		\draw[->] (-4, 0) -- (4, 0) node[right] {$x$};
		\draw[->] (0, -4) -- (0, 4)  node[above] {$f(x)$};
		\draw[graph thickness, samples=80, color=colorEmphasisCyan, domain=-4.000:1] plot (\x, {(((\x)/1) + 1) / 1});
		\draw[graph thickness, samples=80, color=colorEmphasisCyan, domain=1:4] plot (\x, 2);
		\filldraw[color=colorEmphasisCyan] (-3.0, -2.0) circle (\pointSize) node[above left] {$\left(-3;-2\right)$};
		\filldraw[color=colorEmphasisCyan] (-1.0, 0.0) circle (\pointSize) node[above left] {$\left(-1;0\right)$};
		\filldraw[color=colorEmphasisCyan] (1.0, 2.0) circle (\pointSize) node[above] {$\left(1;2\right)$};
      \filldraw[color=colorEmphasisCyan] (3.0, 2.0) circle (\pointSize) node[above] {$\left(3;2\right)$};
	\end{tikzpicture}
	\caption{Đồ thị của $\begin{cases}
         x + 1 \text{ nếu } x \leq 1 \\
         2 \text{ nếu } x > 1
      \end{cases}$}
\end{figure}

\stepcounter{subexercise}
\arabic{subexercise}. Để ý rằng $f(x)$ đứt đoạn tại giá trị $x = 0$. Cụ thể, $f(x)$ không nhận giá trị $x^3 + 4$ khi $x = 0$. Tuy nhiên, không thể vẽ điểm ngay liền trước nó (không có số âm lớn nhất), nên người ta hay dùng đường tròn rỗng để biểu thị điểm đứt đoạn này.
\begin{figure}[H]
	\centering
	\begin{tikzpicture}
		\draw[->] (-4, 0) -- (4, 0) node[right] {$x$};
		\draw[->] (0, -4) -- (0, 5)  node[above] {$f(x)$};
      \draw[graph thickness, samples=80, color=colorEmphasisCyan, domain=-2.000:0.000] plot (\x, {(((\x)/1)^3 + 4) / 1});
      \filldraw[color=colorEmphasisCyan] (-2, -4) circle (\pointSize) node[above left] {$\left(-2;-4\right)$};
		\filldraw[color=colorEmphasisCyan] (-1.0, 3.0) circle (\pointSize) node[left] {$\left(-1;3\right)$};
		\draw[color=colorEmphasisCyan, hollow point] (0.0, 4.0) circle (\pointSize) node[right] {$\left(0;4\right)$};
      \draw[graph thickness, samples=80, color=colorEmphasisCyan, domain=0.000:2.236] plot (\x, {(-((\x)/1)^2 + 1) / 1});
		\filldraw[color=colorEmphasisCyan] (0.0, 1.0) circle (\pointSize) node[left] {$\left(0;1\right)$};
		\filldraw[color=colorEmphasisCyan] (1.0, 0.0) circle (\pointSize) node[above right] {$\left(1;0\right)$};
		\filldraw[color=colorEmphasisCyan] (2.0, -3.0) circle (\pointSize) node[left] {$\left(2;-3\right)$};
	\end{tikzpicture}
	\caption{Đồ thị của $\begin{cases}
         x^3 + 4 \text{ nếu } x < 0 \\
         -x^2 + 1 \text{ nếu } x \geq 0
      \end{cases}$}
\end{figure}

\stepcounter{subexercise}
\arabic{subexercise}. Trước hết, cần xác định các giá trị của $x \geq -2$ để $\frac{5}{x^2 + 1}$ là số nguyên. Do với mọi $x \in \mathbb{R}$ thì $$x^2 \geq 0 \iff x^2 + 1 \geq 1 > 0 \iff 5 \geq \frac{5}{x^2 + 1} > 0.$$ Mà cần phải để $\frac{5}{x^2 + 1} \in \mathbb{N}$ cho nên $\frac{5}{x^2 + 1} \in \left\{1; 2; 3; 4; 5\right\}$. Với để ý đến điều kiện $x \geq -2$, kẻ bảng để xác định các giá trị có thể của $x$:

\begin{table}[H]
   \centering
   \begin{tabular}{|c|c|c|c|c|c|}
   \hline
   $\displaystyle \frac{5}{x^2 + 1}$ & $1$ & $2$ & $3$ & $4$ & $5$ \\
   \hline
   $x^2 + 1$ & $5$ & $\displaystyle\frac{5}{2}$ & $\displaystyle\frac{5}{3}$ & $\displaystyle\frac{5}{4}$ & $1$ \\
   \hline
   $x^2$ & $4$ & $\displaystyle\frac{3}{2}$ & $\displaystyle\frac{2}{3}$ & $\displaystyle\frac{1}{4}$ & $1$ \\
   \hline
   $x$ & $\left\{-2; 2\right\}$ & $\left\{-\sqrt{\frac{3}{2}}; \sqrt{\frac{3}{2}}\right\}$ & $\left\{-\sqrt{\frac{2}{3}}; \sqrt{\frac{2}{3}}\right\}$ & $\left\{-\frac{1}{2}; \frac{1}{2}\right\}$ & $0$ \\
   \hline
   \end{tabular}
   \caption{Bảng giá trị của $\frac{5}{x^2 + 1}$ với $x$} 
\end{table}

Từ đây, có được đồ thị của $f(x)$:

\begin{figure}[H]
	\centering
	\begin{tikzpicture}
		\draw[->] (-4, 0) -- (3, 0) node[right] {$x$};
		\draw[->] (0, -6) -- (0, 1)  node[above] {$f(x)$};
		\draw[graph thickness, samples=80, color=colorEmphasisCyan, domain=-3.000:-2.000] plot (\x, {(-4 / (\x)^2)});
      \filldraw[color=colorEmphasisCyan] (-3.0, -0.4444444444444444) circle (\pointSize) node[left] {$\left(-3;- \frac{4}{9}\right)$};
		\filldraw[color=colorEmphasisCyan] (-2.0, -1.0) circle (\pointSize) node[above right] {$\left(-2;-1\right)$};

		\filldraw[color=colorEmphasisCyan] ({ 2.0 }, { -1.0 }) circle (\pointSize) node[above] {$\left({2};{-1}\right)$};
		\filldraw[color=colorEmphasisCyan] ({ 0.0 }, { -5.0 }) circle (\pointSize) node[below] {$\left({0};{-5}\right)$};
		\filldraw[color=colorEmphasisCyan] ({ -0.5*sqrt(6) }, { -2.0 }) circle (\pointSize) node[left] {$\left({- \frac{\sqrt{6}}{2}};{-2}\right)$};
		\filldraw[color=colorEmphasisCyan] ({ 0.5*sqrt(6) }, { -2.0 }) circle (\pointSize) node[right] {$\left({\frac{\sqrt{6}}{2}};{-2}\right)$};
		\filldraw[color=colorEmphasisCyan] ({ -0.333333333333333*sqrt(6) }, { -3.0 }) circle (\pointSize) node[left] {$\left({- \frac{\sqrt{6}}{3}};{-3}\right)$};
		\filldraw[color=colorEmphasisCyan] ({ 0.333333333333333*sqrt(6) }, { -3.0 }) circle (\pointSize) node[right] {$\left({\frac{\sqrt{6}}{3}};{-3}\right)$};
		\filldraw[color=colorEmphasisCyan] ({ -0.500000000000000 }, { -4.0 }) circle (\pointSize) node[left] {$\left({- \frac{1}{2}};{-4}\right)$};
		\filldraw[color=colorEmphasisCyan] ({ 0.500000000000000 }, { -4.0 }) circle (\pointSize) node[right] {$\left({\frac{1}{2}};{-4}\right)$};
	\end{tikzpicture}
	\caption{Đồ thị của $\begin{cases}
      -\frac{4}{x^2} \text{ nếu } -2 > x \geq -3 \\
      \parbox{0.29\textwidth}{$\begin{array}{cl}
         -\frac{5}{x^2 + 1} &\text{nếu } x \geq -2 \text{ thực để} \\
         &\frac{5}{x^2 + 1}\text{ là số nguyên}
      \end{array}$}
   \end{cases}$}
\end{figure}

\stepcounter{subexercise}
\arabic{subexercise}. 

\begin{figure}[H]
	\centering
	\begin{tikzpicture}
		\draw[->] (-4, 0) -- (4, 0) node[right] {$x$};
		\draw[->] (0, 0) -- (0, 4)  node[above] {$f(x)$};
		\draw[graph thickness, samples=80, color=colorEmphasisCyan, domain=-3:0] plot (\x, {((2*((\x)/1) - 1) / (((\x)/1) - 1)) / 1});
		\draw[graph thickness, samples=80, color=colorEmphasisCyan, domain=2:3] plot (\x, {((2*((\x)/1) - 1) / (((\x)/1) - 1)) / 1});
		\filldraw[color=colorEmphasisCyan] ({ -3.0 }, { 1.75 }) circle (\pointSize) node[above] {$\left({-3};{\frac{7}{4}}\right)$};
		\filldraw[color=colorEmphasisCyan] ({ 0.0 }, { 1.0 }) circle (\pointSize) node[above right] {$\left({0};{1}\right)$};
		\filldraw[color=colorEmphasisCyan] ({ 2.0 }, { 3.0 }) circle (\pointSize) node[above] {$\left({2};{3}\right)$};
		\filldraw[color=colorEmphasisCyan] ({ 3.0 }, { 2.5 }) circle (\pointSize) node[right] {$\left({3};{\frac{5}{2}}\right)$};
		\draw[graph thickness, samples=80, color=colorEmphasisCyan, domain=0:2] plot (\x, {((((\x)/1) + 1)^2 - 3 * ((\x)/1)) / 1});
	\end{tikzpicture}
	\caption{Đồ thị của $\begin{cases}
         \frac{2x - 1}{x - 1} \text{ nếu } -3 \leq x < 0 \\
         \left(x + 1\right)^2 - 3x \text{ nếu } 0 \leq x < 2 \\
         \frac{2x - 1}{x - 1} \text{ nếu } 2 \leq x \leq 3
      \end{cases}$}
\end{figure}

\stepcounter{subexercise}
\arabic{subexercise}.

\begin{figure}[H]
	\centering
	\begin{tikzpicture}
		\draw[->] (-4, 0) -- (4, 0) node[right] {$x$};
		\draw[->] (0, -4) -- (0, 4)  node[above] {$f(x)$};
		\draw[graph thickness, samples=80, color=colorEmphasisCyan, domain=-1.913:0] plot (\x, {(((\x)/1)^3 + 3) / 1});
		\filldraw[color=colorEmphasisCyan] ({ 0.0 }, { 3.0 }) circle (\pointSize) node[right] {$\left({0};{3}\right)$};
		\draw[graph thickness, samples=80, color=colorEmphasisCyan, domain=0.000:1.000] plot (\x, {(-2*(\x) + 2) / 1});
      \draw[color=colorEmphasisCyan, hollow point] (0, 2) circle (\pointSize) node[below left] {$\left({0};{2}\right)$};
      \draw[color=colorEmphasisCyan, hollow point] (1, 0) circle (\pointSize) node[below left] {$\left({1};{0}\right)$};
      \draw[graph thickness, samples=80, color=colorEmphasisCyan, domain=1.000:3.000] plot (\x, {(2 + ((\x)/1) - ((\x)/1)^2) / 1});
      \filldraw[color=colorEmphasisCyan] (1, 2) circle (\pointSize) node[above right] {$\left({1};{2}\right)$};
      \filldraw[color=colorEmphasisCyan] (2, 0) circle (\pointSize) node[above right] {$\left({2};{0}\right)$};
	\end{tikzpicture}
	\caption{Đồ thị của $\begin{cases}
      x^3 + 3 \text{ nếu } x \leq 0 \\
      -2x + 2 \text{ nếu } 0 < x < 1 \\
      2 + x - x^2 \text{ nếu } x \geq 1
   \end{cases}$}
\end{figure}

\stepcounter{subexercise}
\arabic{subexercise}.

\begin{figure}[H]
	\centering
	\begin{tikzpicture}
		\draw[->] (-4, 0) -- (4, 0) node[right] {$x$};
		\draw[->] (0, -1) -- (0, 4)  node[above] {$f(x)$};
		\draw[graph thickness, samples=80, color=colorEmphasisCyan, domain=0.000:4.000] plot (\x, {(((\x)/1)) / 1});
      \draw[graph thickness, samples=80, color=colorEmphasisCyan, domain=-4.000:0.000] plot (\x, {(-((\x)/1)) / 1});
		\filldraw[color=colorEmphasisCyan] ({ 0.0 }, { 0.0 }) circle (\pointSize) node[below] {$\left({0};{0}\right)$};
		\filldraw[color=colorEmphasisCyan] ({ -2.0 }, { 2.0 }) circle (\pointSize) node[below left] {$\left({-2};{2}\right)$};
		\filldraw[color=colorEmphasisCyan] ({ 2.0 }, { 2.0 }) circle (\pointSize) node[below right] {$\left({2};{2}\right)$};
	\end{tikzpicture}
	\caption{Đồ thị của $|x|$}
\end{figure}

\stepcounter{subexercise}
\arabic{subexercise}.

\begin{figure}[H]
	\centering
	\begin{tikzpicture}
		\draw[->] (-3, 0) -- (5, 0) node[right] {$x$};
		\draw[->] (0, -1) -- (0, 4)  node[above] {$f(x)$};
		\draw[graph thickness, samples=80, color=colorEmphasisCyan, domain=-0.732:0] plot (\x, {(2*((\x)/1)^2 - 4*((\x)/1)) / 1});
		\draw[graph thickness, samples=80, color=colorEmphasisCyan, domain=2:2.732] plot (\x, {(2*((\x)/1)^2 - 4*((\x)/1)) / 1});
		\filldraw[color=colorEmphasisCyan] ({ 0.0 }, { 0.0 }) circle (\pointSize) node[below left] {$\left({0};{0}\right)$};
		\filldraw[color=colorEmphasisCyan] ({ 2.0 }, { 0.0 }) circle (\pointSize) node[below right] {$\left({2};{0}\right)$};
      \filldraw[color=colorEmphasisCyan] ({ -0.581 }, { 3.0 }) circle (\pointSize) node[below left] {$\left(-0{,}58;3{,}00\right)$};
      \filldraw[color=colorEmphasisCyan] ({ 2.581 }, { 3.0 }) circle (\pointSize) node[below right] {$\left(2{,}58;3{,}00\right)$};
      \draw[graph thickness, samples=80, color=colorEmphasisCyan, domain=0.000:2.000] plot (\x, {(-2*((\x)/1)^2 + 4*((\x)/1)) / 1});
		\filldraw[color=colorEmphasisCyan] ({ 1.0 }, { 2.0 }) circle (\pointSize) node[above] {$\left({1};{2}\right)$};
	\end{tikzpicture}
	\caption{Đồ thị của $\left|2 x^{2} - 4 x\right|$}
\end{figure}

\stepcounter{subexercise}
\arabic{subexercise}.

\begin{figure}[H]
   \centering
   \begin{tikzpicture}
      \draw[->] (-3, 0) -- (6, 0) node[right] {$x$};
		\draw[->] (0, -4) -- (0, 4)  node[above] {$f(x)$};
      \draw[graph thickness, samples=80, color=colorEmphasisCyan, domain=-1.5:5.033] plot (\x, {abs((\x/1.5)^3 - 3*(\x/1.5)^2 + (\x/1.5)) - abs(\x/1.5)});
      \filldraw[color=colorEmphasisCyan] ({ 0.0 }, { 0.0 }) circle (\pointSize) node[below left] {$\left({0};{0}\right)$};
      \filldraw[color=colorEmphasisCyan] ({ 1.5 }, { 0.0 }) circle (\pointSize) node[below right] {$\left({1};{0}\right)$};
      \filldraw[color=colorEmphasisCyan] ({ 3.0 }, { 0.0 }) circle (\pointSize) node[above right] {$\left({2};{0}\right)$};
      \filldraw[color=colorEmphasisCyan] ({ 4.5 }, { 0.0 }) circle (\pointSize) node[below right] {$\left({3};{0}\right)$};
      \filldraw[color=colorEmphasisCyan] ({ 3.927 }, { -2.618 }) circle (\pointSize) node[below right] {$\left({2{,}62};{-2{,}62}\right)$};
      \filldraw[color=colorEmphasisCyan] ({ -1.5 }, { 4.0 }) circle (\pointSize) node[left] {$\left({-1};{4}\right)$};
   \end{tikzpicture}
\end{figure}

\exercise Cho $a$ và $b$ là hai số thực. Chứng minh rằng $|a||b| = |ab|$.

\solution

Để chứng minh $|a||b| = |ab|$ ngắn gọn, chúng ta thực hiện kẻ bảng:

\begin{table}[H]
   \centering
   \begin{tabular}{|c||c|c|c|c|}
      \hline
      Điều kiện & $\begin{cases}a\geq 0\\b\geq0\end{cases}$ & $\begin{cases}a\geq0\\b<0\end{cases}$ & $\begin{cases}a<0\\b\geq 0\end{cases}$ & $\begin{cases}a<0\\b<0\end{cases}$ \\
      \hline
      $|a|$, $|b|$ & $\begin{cases}|a| = a\\|b| = b\end{cases}$ & $\begin{cases}|a| = a\\|b| = -b\end{cases}$ & $\begin{cases}|a| = -a\\|b| = b\end{cases}$ & $\begin{cases}|a| = -a\\|b| = -b\end{cases}$ \\
      \hline
      $|a||b|$ & $ab$ & $(-a)b = -ab$ & $a(-b) = -ab$ & $(-a)(-b) = ab$ \\
      \hline
      Dấu của $ab$ & $\geq 0$ & $< 0$ & $<0$ & $\geq 0$ \\
      \hline
      $|ab|$ & $ab$ & $-ab$ & $-ab$ & $ab$ \\
      \hline
   \end{tabular}
   \caption{Bảng so sánh $|a||b|$ và $|ab|$}
\end{table}

Qua bảng, chúng ta luôn có $|a||b| = |ab|$. Chúng ta có điều phải chứng minh.

\exercise Cho sô thực $a$. Chứng minh rằng
\begin{multicols}{2}
   \begin{enumerate}
      \item $|a| \geq a$;
      \item $|a|^2 = a^2$.
   \end{enumerate}
\end{multicols}

\solution

\setcounter{subexercise}{1}
\arabic{subexercise}. Khi $a \geq 0$ thì $|a| = a$. Trong trường hợp còn lại, nếu $a < 0$, chúng ta có $-a > 0$. Do đó $|a| > a$.

Vậy $|a| \geq a$ với mọi $a$ thực.

\stepcounter{subexercise}
\arabic{subexercise}. 
\begin{align*}
   |a|^2 &= |a|\cdot |a| = |a \cdot a| = \left|a^2\right|\\
   &= a^2 \equationexplanation{$a^2$ thì luôn không âm}.
\end{align*}
Chúng ta qua đó có điều phải chứng minh.

\exercise Chứng minh rằng với $a$ và $b$ là hai số thực thì $|a| + |b| \geq |a + b|$.

\solution

Chúng ta có những đẳng thức và bất đẳng thức sau:
\begin{equation*}
   \begin{cases}
      |a|^2 = a^2 \\
      |b|^2 = b^2 \\
      |a||b| = |ab| \geq ab
   \end{cases}.
\end{equation*}
Qua đó, 
\begin{align}
   |a|^2 + 2|a||b| + |b|^2 &\geq a^2 + 2ab + b^2 \nonumber\\
   \iff \left(|a| + |b|\right)^2 &\geq (a + b)^2 \nonumber\\
   \iff \left(|a| + |b|\right)^2 - \left|a + b\right|^2 &\geq 0 \nonumber\\
   \iff \left(|a| + |b| - |a + b|\right)\left(|a| + |b| + |a + b|\right) &\geq 0. \label{eq:toan_hoc_nen_tang:ham_so_mot_bien:ham_tung_phan:bdt23}
\end{align}

Do $|a|$, $|b|$, $|a + b|$ đều không âm nên $|a| + |b| + |a + b| \geq 0$. Cho nên:

\begin{align*}
   \text{\refeq{eq:toan_hoc_nen_tang:ham_so_mot_bien:ham_tung_phan:bdt23}} \iff |a| + |b| - |a + b| &\geq 0 \\
   \iff |a| + |b| &\geq |a + b|.
\end{align*}
Đây là điều phải chứng minh.

\exercise Giải các phương trình sau trên ẩn $x$ thực:

\begin{multicols}{2}
   \begin{enumerate}
      \item $\left\lceil \frac{x}{4} \right\rceil = -2$;
      \item $2\left\lfloor -2x - 3 \right\rfloor - 1 = 1$;
      \item $3\lceil x \rceil^2 - 4\lceil x \rceil - 4 = 0$;
      \item $\lfloor x + 2 \rfloor^3 - \lfloor x \rfloor = 2$;
      \item $\left\lfloor \frac{x}{3} \right\rfloor + \left\lfloor \frac{x}{5} \right\rfloor = 7$. 
   \end{enumerate}
\end{multicols}

\solution

\setcounter{subexercise}{1}
\arabic{subexercise}. Sử dụng kết quả đã có: $\lceil x \rceil = a \iff a - 1 < x \leq a$ với $a \in \mathbb{N}$; chúng ta có
\begin{align*}
   &\left\lceil \frac{x}{4} \right\rceil = -2 \\
   \iff &-3 < \frac{x}{4} \leq -2 \\
   \iff &-12 < x \leq -8.
\end{align*}

Tập nghiệm của phương trình là $\left(-12; -8\right]$.

\stepcounter{subexercise}
\arabic{subexercise}. 

\begin{align*}
   2\lfloor -2x - 3 \rfloor - 1 &= 1 \\
   \iff \lfloor -2x - 3 \rfloor &= 1 \\
   \iff 1 \leq -2x - 3 &< 2 \\
   \iff -2 \geq x &> -\frac{5}{2}.
\end{align*}

Tập nghiệm của phương trình là $\left(-\frac{5}{2}; -2\right]$.

\stepcounter{subexercise}
\arabic{subexercise}.

\begin{align*}
   &3\lceil x \rceil^2 - 4\lceil x \rceil - 4 = 0 \\
   \iff &\left(3\lceil x \rceil + 2\right)\left(\lceil x \rceil - 2\right) = 0 \\
   \iff &\left[\begin{array}{l}
      \lceil x \rceil = -\frac{2}{3} \\
      \lceil x \rceil = 2
   \end{array}\right..
\end{align*}

Do kết quả của hàm trần luôn là số nguyên nên chỉ có một trường hợp
\begin{equation*}
   \lceil x \rceil = 2 \iff 1 < x \leq 2.
\end{equation*}
Vậy tập nghiệm của phương trình là $\left(1; 2\right]$.

\stepcounter{subexercise}
\arabic{subexercise}. Đặt $y = \lfloor x \rfloor$ với $y \in \mathbb{N}$. Theo định nghĩa, $y$ là số nguyên lớn nhất không vượt quá $x$. Theo một cách nói khác, $y$ là số nguyên duy nhất thỏa mãn $y \leq x < y + 1$. Công $2$ vào tất cả các vế, chúng ta có $y + 2 \leq x + 2 < y + 3$. Nhận thấy rằng, $z = y + 2$ là số nguyên duy nhất thỏa mãn $z \leq x + 2 < z + 1$, hay $z$ là số nguyên lớn nhất không quá $x + 2$. Do vậy, 
\begin{equation}
   \lfloor x + 2 \rfloor = y + 2 = \lfloor x \rfloor + 2.
\end{equation}

Sử dụng điều kiện này để biến đổi phương trình:
\begin{align*}
   &\lfloor x + 2 \rfloor^3 - \lfloor x \rfloor = 2\\
   \iff &\left(\lfloor x \rfloor + 2\right)^3 - \left(\lfloor x \rfloor + 2\right) = 0 \\
   \iff &\left(\lfloor x \rfloor + 2\right)\left(\left(\lfloor x \rfloor + 2\right)^2 - 1\right) = 0 \\
   \iff &\left(\lfloor x \rfloor + 2\right)\left(\lfloor x \rfloor + 1\right)\left(\lfloor x \rfloor + 3\right) = 0 \\
   \iff &\left[\begin{array}{l}
      \lfloor x \rfloor = -1 \\
      \lfloor x \rfloor = -2 \\
      \lfloor x \rfloor = -3
   \end{array}\right. \\
   \iff &\left[\begin{array}{l}
      -1 \leq x < 0 \\
      -2 \leq x < -1 \\
      -3 \leq x < -2
   \end{array}\right. \\
   \iff & -3 \leq x < 0.
\end{align*}

Qua đó, chúng ta có nghiệm $x \in \left[-3; 0\right)$.

\stepcounter{subexercise}
\arabic{subexercise}. Nếu $x \geq 15$, $$\begin{cases}
   \frac{x}{3} \geq 5 \\
   \frac{x}{5} \geq 3
\end{cases} \implies \begin{cases}
   \left\lfloor \frac{x}{3} \right\rfloor \geq 5 \\
   \left\lfloor \frac{x}{5} \right\rfloor \geq 3
\end{cases} \implies \left\lfloor \frac{x}{3} \right\rfloor + \left\lfloor \frac{x}{5} \right\rfloor \geq 8.$$

Nếu $x < 15$, một cách tương tự, chúng ta cũng có
\begin{equation*}
   \begin{cases}
      \frac{x}{3} < 5 \\
      \frac{x}{5} < 3
   \end{cases} \implies \begin{cases}
      \left\lfloor \frac{x}{3} \right\rfloor < 5 \\
      \left\lfloor \frac{x}{5} \right\rfloor < 3
   \end{cases}.
\end{equation*}
Do $\left\lfloor \frac{x}{3} \right\rfloor$ và $\left\lfloor \frac{x}{5} \right\rfloor$ đều là số nguyên cho nên
$$
\begin{cases}
   \left\lfloor \frac{x}{3} \right\rfloor \leq 4 \\
   \left\lfloor \frac{x}{5} \right\rfloor \leq 2
\end{cases} \implies \left\lfloor \frac{x}{3} \right\rfloor + \left\lfloor \frac{x}{5} \right\rfloor \leq 6.
$$

Qua hai trường hợp, chúng ta thấy không có $x$ để $\left\lfloor \frac{x}{3} \right\rfloor + \left\lfloor \frac{x}{5} \right\rfloor = 7$. Vậy phương trình vô nghiệm.
\subsection{Hàm căn thức}

\ % Lùi đầu dòng

\defText{Hàm căn thức} là một trong những hàm số cơ bản trong toán học, được định nghĩa dựa trên phép căn thức của số thực như sau:
$$\defMath{f(x) = \sqrt[n]{x}}$$
trong đó $x \in \mathbb{R}^+ \cup \{0\}$ và $n \in \mathbb{Z}^+$. Nếu $n = 2k + 1 (k \in \mathbb{N})$ là số lẻ, thì chúng ta có mở rộng của hàm căn thức trên toàn bộ tập số thực:
$$
\defMath{f(x) = }\begin{cases}
   \defMath{\sqrt[2k+1]{x} \defText{ nếu } x \geq 0} \\
   \defMath{-\sqrt[2k+1]{-x} \defText{ nếu } x < 0}
\end{cases}.
$$
Tối giản hóa định nghĩa trên, có thể viết $f(x) = \sqrt[2k+1]{x}$ trên toàn bộ $x$ thực.

Khi hợp hai hàm số $f \circ g$ mà $f$ là hàm căn thức, có $f\circ g(x) = \sqrt[n]{g(x)}$. Khi này, $g(x)$ có thể được gọi là \defText{biểu thức dưới dấu căn} hay \defText{biểu thức lấy căn}.

\exercise Giải các phương trình sau với ẩn $x$ thực

\begin{multicols}{2}
   \begin{enumerate}
      \item $\sqrt{x} = 2$;
      \item $\sqrt{x - 2} = -2$;
      \item $\sqrt[3]{x^5 + 1} = -2$;
      \item $\sqrt[4]{x^4 - 2x^2 + 8} = -x$;
      \item $\sqrt{x^3-3x+1} = \sqrt{x^3+2x-6}$;
      \item $\sqrt[4]{x^4 + 1} = \sqrt[4]{x^4-3x + 1}$;
      \item $2\sqrt{x^2 - 9} = (x + 5)\sqrt{\frac{x+3}{x-3}}$;
      \item $\sqrt{x + 4} + \sqrt{x + 9} = 5$;
      \item $\sqrt{x(x+1)} + \sqrt{x(x+2)} = \sqrt{x(x-3)}$;
      \item $\sqrt{x+3} = \sqrt[3]{5x+3}$;.
   \end{enumerate}
\end{multicols}

\solution

\setcounter{subexercise}{1}
\arabic{subexercise}. Tập xác định của phương trình là $\left[0; \infty\right)$. Theo định nghĩa của phép khai căn:
\begin{align*}
   \sqrt{x} &= 2 \\
   \iff x &= 2^2 = 4.
\end{align*}

Vậy $x = 4$ là nghiệm duy nhất của phương trình.

\stepcounter{subexercise}
\arabic{subexercise}. Theo định nghĩa của phép khai căn, với căn bậc chẵn, chúng ta có $\sqrt{x - 2} \geq 0$. Do đó, phương trình vô nghiệm.

\stepcounter{subexercise}
\arabic{subexercise}. Thực hiện biến đổi đại số:
\begin{align*}
   \sqrt[3]{x^5 + 1} &= -2 \\
   \iff x^5 + 1 &= (-2)^3 = -8 \\
   \iff x^5 &= -9 \\
   \iff x &= -\sqrt[5]{9}.
\end{align*}

Vậy tập nghiệm của phương trình là $\left\{-\sqrt[5]{9}\right\}$.

\stepcounter{subexercise}
\arabic{subexercise}.
\begin{align*}
   \sqrt[4]{x^4-2x^2+8} &= -x \\
   \implies x^4 - 2x^2 + 8 &= (-x)^4 = x^4 \\
   \implies 8 - 2x^2 &= 0 \\
   \implies x \in \{-2; 2\}.
\end{align*}

Kiểm tra trực tiếp, thấy $x = -2$ là nghiệm duy nhất thỏa mãn. Vậy tập nghiệm của phương trình là $\{2\}$.

\stepcounter{subexercise}
\arabic{subexercise}.
\begin{align*}
   \sqrt{x^3-3x+1} &= \sqrt{x^3+2x-6} \\
   \implies x^3-3x+1 &= x^3 + 2x - 6 \\
   \iff x &= \frac{7}{5}.
\end{align*}
Tuy nhiên, khi kiểm tra $x = \frac{7}{5}$ thì $x^3 - 3x + 1 = -\frac{57}{125}$ là một số âm, không thỏa mãn điều kiện xác định của $\sqrt{x^3-3x+1}$. Qua đó, chúng ta có tập nghiệm của phương trình là $\emptyset$.

\stepcounter{subexercise}
\arabic{subexercise}.
\begin{align*}
   \sqrt[4]{x^4 + 1} &= \sqrt[4]{x^4 - 3x + 1} \\
   \implies x^4 + 1 &= x^4 - 3x + 1 \\
   \iff x &= 0. 
\end{align*}

Kiểm tra lại, thấy cả hai vế đều bằng $1$ khi $x = 0$. Cho nên tập nghiệm của phương trình là $\{0\}$.

\stepcounter{subexercise}
\arabic{subexercise}. Giống như nhiều bài tập trước đó, bình phương lên hai vế để khử căn để được
\begin{align*}
   \left(2\sqrt{x^2 - 9}\right)^2 &= \left((x + 5)\sqrt{\frac{x+3}{x-3}}\right)^2 \\ 
   \implies 4\left(x^2-9\right) &= (x + 5)^2\frac{x+3}{x-3}\\
   \implies 4\left(x^2-9\right)(x - 3) &= (x + 5)^2(x + 3) \\
   \implies 4x^3 - 12x^2 - 36x + 108 &= x^3 + 13x^2 + 55x + 75 \\
   \iff 3x^3 - 25x^2 - 91x + 33 &= 0 \\
   \iff (x - 11)(x + 3)(3x - 1) &= 0 \\
   \iff x &\in \left\{11; -3; \frac{1}{3}\right\}.
\end{align*}

Thử lại, chúng ta kết luận được các nghiệm của phương trình là $x \in \left\{11; -3\right\}$.

\stepcounter{subexercise}
\arabic{subexercise}. Nếu $x$ thỏa mãn phương trình $\sqrt{x + 4} + \sqrt{x + 9} = 5$, thì cần phải có $\begin{cases}
   x + 4 \geq 0 \\
   x + 9 \geq 0
\end{cases}$. Ngoài ra, có $\begin{cases}
   \sqrt{x + 4} \geq 0 \\
   \sqrt{x + 9} \geq 0
\end{cases}$ nên $\sqrt{x + 4} + \sqrt{x + 9} \geq 0$. Do đó, bình phương hai vế để có phương trình tương đương
\begin{align*}
   \left(\sqrt{x + 4} + \sqrt{x + 9}\right)^2 &= 5^2 \\
   \iff x + 4 + 2\sqrt{x + 4}\sqrt{x + 9} + x + 9 &= 25 \\
   \iff 2\sqrt{(x + 4)(x + 9)} &= 12 \equationexplanation{\parbox{0.4\textwidth}{$\sqrt{x + 4}\sqrt{x + 9} = \sqrt{(x + 4)(x + 9)}$ do cả hai biểu thức dưới căn đều không âm.}} \\
   \displaybreak[2]
   \iff \sqrt{(x + 4)(x + 9)} &= 6 \\
   \iff (x + 4)(x + 9) &= 6^2 = 36 \\
   \iff x^2 + 13x + 36 &= 0 \\
   \iff x(x+13) &= 0 \\
   \iff x &= 0 \equationexplanation{$x+13 > x + 4 \geq 0$.}.
\end{align*}

Vậy tập nghiệm của phương trình là $\{0\}$.

\stepcounter{subexercise}
\arabic{subexercise}.
% \input{\chapdir toan_hoc_nen_tang/ham_so_mot_bien/nhi_thuc_niu_ton.tex}

% \section{Thuộc tính của hàm số}

\ % Lùi đầu dòng

Trước phần này, chúng ta mới chỉ xét nghiệm của hàm và hình dạng của hàm số thông qua đồ thị. Nhìn vào đồ thị, chúng ta có thể thấy được hàm số có nhiều thành phần đặc biệt. Ở trong phần này, chúng ta sẽ gọi tên và khảo sát những thành phần đặc biệt đó.

% \subsection{Hàm chẵn và hàm lẻ}

\ % Lùi đầu dòng

Phần tính chất đầu tiên mà chúng ta quan tâm đến là tính đối xứng của hàm số trên đồ thị. Nhắc lại một chút kiến thức hình học, một hình có thể có hai kiểu đối xứng là đối xứng trục và đối xứng điểm. Tạm thời, chúng ta chỉ quan tâm đến những trường hợp đối xứng cụ thể. Với đồ thị của một hàm số, một cách khá tự nhiên, chúng ta sẽ xem xét tính đối xứng trục tung hoặc qua điểm gốc tọa độ. 

Đầu tiên là đối xứng qua trục tung. Một hàm số có tính đối xứng như vậy được gọi là \defText{hàm chẵn}. Cụ thể, cho $f$ là một hàm số xác định trên $A$. $f$ là hàm chẵn nếu $\defMath{x \in A \implies -x \in A}$ và $$\defMath{f(-x) = f(x)}$$ với mọi $x \in A$. 

Tương tự, $f$ được gọi là \defText{hàm lẻ} nếu $\defMath{x \in A \implies -x \in A}$ và $$\defMath{f(-x) = -f(x)}$$ với mọi $x \in A$. Khi này, hàm sẽ đối xứng qua gốc tọa độ.

{
   \begin{minipageindent}{0.48\textwidth}
      \begin{figure}[H]
         \centering
         \begin{tikzpicture}
            \draw[->] (-4, 0) -- (4, 0) node[right] {$x$};
            \draw[->, color=colorEmphasis] (0, -4) -- (0, 4)  node[above] {$f(x)$};
            \draw[graph thickness, samples=80, color=colorEmphasisCyan, domain=-1.857:1.857] plot (\x, {(((\x)/1)^4 - 2*((\x)/1)^2 - 1) / 1});
            \filldraw[color=colorEmphasis] ({1.65}, { 0.967006249999999 }) circle (\pointSize) node[right] {$\left(x;f(x)\right)$};
            \filldraw[color=colorEmphasis] ({-1.65}, { 0.967006249999999 }) circle (\pointSize) node[left] {$\left(-x;f(x)\right)$};
            \draw[dashed, color=colorEmphasis] ({1.65}, { 0.967006249999999 }) -- ({-1.65}, { 0.967006249999999 });
         \end{tikzpicture}
         \caption{Đồ thị của một hàm chẵn}
      \end{figure}
   \end{minipageindent}
   \hfill
   \begin{minipageindent}{0.48\textwidth}
      \begin{figure}[H]
         \centering
         \begin{tikzpicture}
            \draw[->] (-4, 0) -- (4, 0) node[right] {$x$};
            \draw[->] (0, -4) -- (0, 4)  node[above] {$f(x)$};
            \draw[graph thickness, samples=80, color=colorEmphasisCyan, domain=-4.000:-1.133] plot (\x, {(((\x)/1) / (((\x)/1)^2 - 1)) / 1});
            \draw[graph thickness, samples=80, color=colorEmphasisCyan, domain=-0.883:0.883] plot (\x, {(((\x)/1) / (((\x)/1)^2 - 1)) / 1});
            \draw[graph thickness, samples=80, color=colorEmphasisCyan, domain=1.133:4.000] plot (\x, {(((\x)/1) / (((\x)/1)^2 - 1)) / 1});
            \filldraw[color=colorEmphasis] ({2.0}, { 0.6666666666666666 }) circle (\pointSize) node[above right] {$\left(x;f(x)\right)$};
            \filldraw[color=colorEmphasis] ({-2.0}, { -0.6666666666666666 }) circle (\pointSize) node[below left] {$\left(-x;f(x)\right)$};
            \filldraw[color=colorEmphasis] (0, 0) circle (\pointSize) node[below] {$\left(0;0\right)$};
            \draw[dashed, color=colorEmphasis] ({2.0}, { 0.6666666666666666 }) -- ({-2.0}, { -0.6666666666666666 });
         \end{tikzpicture}
         \caption{Đồ thị của một hàm lẻ}
      \end{figure}
      
   \end{minipageindent}
}

Bạn đọc, một cách rất tự nhiên, có thể đặt câu hỏi rằng liệu có hàm số nào đối xứng qua trục hoành không. Giả sử tồn tại hàm $f$ như vậy. Khi này, nếu một điểm có tọa độ $\left(x; y\right)$ trên đồ thị của $f$ thì đối xứng của nó qua trục hoành là $\left(x; -y\right)$. Tuy nhiên, do $f$ là hàm số nên chỉ tồn tại một điểm $\left(x; f(x)\right)$ trên đồ thị của $f$. Do đó, $f(x) = y = -y$ hay $f(x) = 0$ với mọi $x$ thuộc tập xác định. Vì chỉ tồn tại một hàm duy nhất đối xứng qua trục hoành và hàm này là hàm hằng cho nên chúng ta sẽ không đặt tên mới cho nó.

\exercise Xác định xem những hàm sau có phải là hàm chẵn, hàm lẻ hay không. Sau đó, vẽ đồ thị của chúng.
\begin{multicols}{2}
   \begin{enumerate}
      \item $f(x) = x^4 - 2x^2 - 3$;
      \item $f(x) = x^5 - x^3 + x$;
      \item $f(x) = \frac{x}{x^2 + 1}$;
      \item $f(x) = \frac{x^3 - \frac{1}{x^3}}{x + \frac{1}{x}}$;
      \item $f(x) = |x|^2 - \left|x^3\right| + 1$;
      \item $f(x) = \lceil x \rceil - \lfloor x \rfloor$.
   \end{enumerate}
\end{multicols}

\solution 

\setcounter{subexercise}{1}
\arabic{subexercise}. Tập xác định của hàm là $\mathbb{R}$. Với mọi $x \in \mathbb{R}$, có $-x \in \mathbb{R}$ và
\begin{align*}
   f(-x) &= (-x)^4 - 2(-x)^2 - 3\\
   &= x^4 - 2x^2 - 3\\
   &= f(x).
\end{align*}
Vậy $f(x)$ là hàm chẵn.

\stepcounter{subexercise}
\arabic{subexercise}. Tập xác định của hàm là $\mathbb{R}$. Với mọi $x \in \mathbb{R}$, có $-x \in \mathbb{R}$ và
\begin{align*}
   f(-x) &= (-x)^5 - (-x)^3 + (-x)\\
   &= -x^5 + x^3 - x\\
   &= -\left(x^5 - x^3 + x\right)\\
   &= -f(x).
\end{align*}
Vậy $f(x)$ là hàm lẻ.

{
   \begin{minipageindent}{0.48\textwidth}
      \begin{figure}[H]
         \centering
         \begin{tikzpicture}
            \draw[->] (-3, 0) -- (3, 0) node[right] {$x$};
            \draw[->] (0, -4) -- (0, 4)  node[above] {$f(x)$};
            \draw[graph thickness, samples=80, color=colorEmphasisCyan, domain=-2.040:2.040] plot (\x, {(((\x)/1)^4 - 2*((\x)/1)^2 - 3) / 1.5});
         \end{tikzpicture}
         \caption{Đồ thị của $x^{4} - 2 x^{2} - 3$}
      \end{figure}
   \end{minipageindent}
   \hfill
   \begin{minipageindent}{0.48\textwidth}
      \begin{figure}[H]
         \centering
         \begin{tikzpicture}
            \draw[->] (-3, 0) -- (3, 0) node[right] {$x$};
            \draw[->] (0, -4) -- (0, 4)  node[above] {$f(x)$};
            \draw[graph thickness, samples=80, color=colorEmphasisCyan, domain=-1.398:1.398] plot (\x, {(((\x)/1)^5 - ((\x)/1)^3 + ((\x)/1)) / 1});
         \end{tikzpicture}
         \caption{Đồ thị của $x^{5} - x^{3} + x$}
      \end{figure}
   \end{minipageindent}
}

\stepcounter{subexercise}
\arabic{subexercise}. Tập xác định của hàm là $\mathbb{R}$. Với mọi $x \in \mathbb{R}$, có $-x \in \mathbb{R}$ và
\begin{align*}
   f(-x) &= \frac{-x}{(-x)^2 + 1}\\
   &= \frac{-x}{x^2 + 1}\\
   &= -\frac{x}{x^2 + 1}\\
   &= -f(x).
\end{align*}
Vậy $f(x)$ là hàm lẻ.

\stepcounter{subexercise}
\arabic{subexercise}. Tìm tập xác định, $x$ làm cho $f(x)$ thỏa mãn khi và chỉ khi
\begin{equation*}
   \begin{cases}
      x \neq 0 \\
      x + \frac{1}{x} \neq 0
   \end{cases}.
\end{equation*}
Từ bất phương trình thứ hai, với điều kiện $x \neq 0$:
\begin{equation*}
   x + \frac{1}{x} \neq 0 \\
   \implies x^2 + 1 \neq 0
\end{equation*}
luôm đúng. Cho nên, tập xác định là $\mathbb{R} \setminus \left\{0\right\}$.

Để ý rằng, với $x$ thuộc tập xác định, thì có $x\neq 0 \iff -x \neq 0$. Cho nên $-x$ cũng thuộc tập xác định và
\begin{align*}
   f(-x) &= \frac{(-x)^3 - \frac{1}{(-x)^3}}{-x + \frac{1}{-x}} \\
         &= \frac{-x^3 - \frac{1}{-x^3}}{-x - \frac{1}{x}} \\
         \displaybreak[2]
         &= \frac{-\left(x^3 - \frac{1}{x^3}\right)}{-\left(x + \frac{1}{x}\right)} \\
         &= \frac{x^3 - \frac{1}{x^3}}{x + \frac{1}{x}} = f(x).
\end{align*}
Vậy $f(x)$ là hàm chẵn.

{
   \begin{minipageindent}{0.48\textwidth}
      \begin{figure}[H]
         \centering
         \begin{tikzpicture}
            \draw[->] (-3, 0) -- (3, 0) node[right] {$x$};
            \draw[->] (0, -4) -- (0, 4)  node[above] {$f(x)$};
            \draw[graph thickness, samples=80, color=colorEmphasisCyan, domain=-3.000:3.000] plot (\x, {(((\x)/1) / (((\x)/1)^2 + 1)) / 0.25});
         \end{tikzpicture}
         \caption{Đồ thị của $\frac{x}{x^{2} + 1}$}
      \end{figure}      
   \end{minipageindent}
   \hfill
   \begin{minipageindent}{0.48\textwidth}
      \begin{figure}[H]
         \centering
         \begin{tikzpicture}
            \draw[->] (-3, 0) -- (3, 0) node[right] {$x$};
            \draw[->] (0, -3) -- (0, 5)  node[above] {$f(x)$};
            \draw[graph thickness, samples=80, color=colorEmphasisCyan, domain=-2.425:-0.510] plot (\x, {((((\x)/1)^3 - 1/((\x)/1)^3) / (((\x)/1) + 1/((\x)/1)))});
            \draw[graph thickness, samples=80, color=colorEmphasisCyan, domain=0.510:2.425] plot (\x, {((((\x)/1)^3 - 1/((\x)/1)^3) / (((\x)/1) + 1/((\x)/1)))});
         \end{tikzpicture}
         \caption{Đồ thị của $\frac{x^{3} - \frac{1}{x^{3}}}{x + \frac{1}{x}}$}
      \end{figure}       
   \end{minipageindent}
}

\stepcounter{subexercise}
\arabic{subexercise}. Tập xác định của hàm là $\mathbb{R}$. Với mọi $x \in \mathbb{R}$, có $-x \in \mathbb{R}$ và
\begin{align*}
   f(-x) &= \left| -x \right|^2 - \left|(-x)^3\right| + 1 \\
         &= \left| x \right|^2 - \left|-x^3 \right| + 1 \\
         &= \left| x \right|^2 - \left| x^3 \right| + 1 \\
         &= f(x).
\end{align*}
Vậy $f(x)$ là hàm chẵn.

\stepcounter{subexercise}
\arabic{subexercise}. Tập xác định của hàm là $\mathbb{R}$.

Nếu \textcolor{colorEmphasisCyan}{$x \in \mathbb{Z}$} thì $\begin{cases}
   \left\lceil x \right\rceil = x \\ 
   \left\lfloor x \right\rfloor = x
\end{cases} \implies \left\lceil x \right\rceil - \left\lfloor x \right\rfloor = 0$.

Trong trường hợp còn lại, \textcolor{colorEmphasis}{$x \notin \mathbb{Z}$}. Đặt $\lfloor x \rfloor = n$ với $n$ nguyên. Điều này tương đương với $n \leq x < n + 1$. Do $x$ không phải là số nguyên nên $n \neq x$, cho nên $n < x < n + 1 \implies n < x \leq n + 1 \iff \left\lceil x \right\rceil = n + 1$. Do đó, $$\left\lceil x \right\rceil - \left\lfloor x \right\rfloor = (n + 1) - n = 1.$$

Cho nên, có thể viết lại hàm đã cho bằng

\begin{equation*}
   f(x) = \left\lceil x \right\rceil - \left\lfloor x \right\rfloor = \begin{cases}
      0 & \text{ nếu } x \in \mathbb{Z} \\
      1 & \text{ nếu } x \notin \mathbb{Z}
   \end{cases}.
\end{equation*}

Hiển nhiên rằng $f(x) = f(-x)$ là hàm chẵn.

{
   \begin{minipageindent}{0.48\textwidth}
      \begin{figure}[H]
         \centering
         \begin{tikzpicture}
            \draw[->] (-3, 0) -- (3, 0) node[right] {$x$};
            \draw[->] (0, -2) -- (0, 2)  node[above] {$f(x)$};
            \draw[graph thickness, samples=80, color=colorEmphasisCyan, domain=-1.864:1.864] plot (\x, {(((\x)/1)^2 - (((\x)/1)^6)^(1/2) + 1) / 1});
         \end{tikzpicture}
         \caption{Đồ thị của $|x|^2 - \left|x^3\right| + 1$}
      \end{figure}
   \end{minipageindent}
   \hfill
   \begin{minipageindent}{0.48\textwidth}
      \begin{figure}[H]
         \centering
         \begin{tikzpicture}
            \draw[->] (-3.5, 0) -- (3.5, 0) node[right] {$x$};
            \draw[->] (0, -1) -- (0, 3)  node[above] {$f(x)$};
            \draw[graph thickness, samples=80, color=colorEmphasisCyan, domain=-3.5:3.5] plot (\x, 1);
            \foreach \x in {-3,-2,-1,0,1,2,3} {
               \draw[color=colorEmphasisCyan, hollow point] (\x, 1) circle (\pointSize);
               \filldraw[color=colorEmphasisCyan] (\x, 0) circle (\pointSize);
            }
         \end{tikzpicture}
         \caption{Đồ thị của $\left\lceil x \right\rceil - \left\lfloor x \right\rfloor$}
      \end{figure}
   \end{minipageindent}
}

\exercise Tìm một hàm $f$ xác định trên tập số thực sao cho $f$ vừa mang tính chẵn, vừa mang tính lẻ. Chứng minh tại sao hàm đó là hàm duy nhất thỏa mãn điều kiện này.

\solution 

Hàm $f(x) = 0$ với mọi $x$ là hàm duy nhất thỏa mãn điều kiện này. Thật vậy, giả sử $f$ là một hàm có tính chẵn và lẻ thì với mọi $x$: 
\begin{align*}
   f(x) &= f(-x) = -f(x) \\
   \implies 2f(x) &= 0 \\
   \implies f(x) &= 0.
\end{align*}
Bằng sự quy chiếu đơn giản với định nghĩa, chúng ta có hàm $f$ này thỏa mãn. Vậy, chúng ta có hàm \begin{align*}
   f: \mathbb{R} &\to \{0\} \\
         x &\mapsto 0
\end{align*} thỏa mãn và đã được chứng minh là duy nhất.

\exercise Cho hàm $f$ xác định trên đoạn $[-a; a]$. Chứng minh rằng tồn tại duy nhất một bộ hai hàm số $\left(\chanF; \leF\right)$ sao cho $\chanF$ là hàm chẵn, $\leF$ là hàm lẻ và $f(x) = \chanF(x) + \leF(x)$ với mọi $x \in [-a; a]$.

\solution

Giả sử bộ hai hàm số này tồn tại, khi này
$$
f(-x) = \chanF(-x) + \leF(-x) = \chanF(x) - \leF(x).
$$
Do đó, kết hợp với $f(x) = \chanF(x) + \leF(x)$, thực hiện một số biến đổi đại số để có \begin{equation*}
   \begin{cases}
      \chanF(x) = \frac{f(x) + f(-x)}{2} \\
      \leF(x) = \frac{f(x) - f(-x)}{2}
   \end{cases}.
\end{equation*}

Vậy, nếu bộ hai hàm số này tồn tại thì chỉ có thể nhận giá trị là

{
   \begin{minipageindent}{0.45\textwidth}
      \begin{equation*}
         \begin{array}{rccc}
            \chanF: &[-a; a] &\to &\mathbb{R} \\
            &x &\mapsto &\frac{f(x) + f(-x)}{2}
         \end{array};
      \end{equation*}
   \end{minipageindent}
   và 
   \begin{minipageindent}{0.45\textwidth}
      \begin{equation*}
         \begin{array}{rccc}
            \leF: &[-a; a] &\to &\mathbb{R} \\
            &x &\mapsto &\frac{f(x) - f(-x)}{2}
         \end{array}.
      \end{equation*}
   \end{minipageindent}
}

Để hoàn thành chứng minh, chúng ta sẽ khẳng định rằng hai hàm này thỏa mãn. Hiển nhiên rằng, nếu $x \in [-a; a]$ thì $-x \in [-a; a]$. Ngoài ra, với mọi $x \in [-a; a]$,

\begin{equation*}
   \begin{cases}
      \chanF(-x) &= \frac{f(-x) + f\left(-(-x)\right)}{2} = \frac{f(-x) + f(x)}{2} = \chanF(x) \\
      \leF(-x) &= \frac{f(-x) - f\left(-(-x)\right)}{2} = \frac{f(-x) - f(x)}{2} = -\leF(x)
   \end{cases}.
\end{equation*}
Qua đó, chúng ta có điều phải chứng minh.
\subsection{Hàm đồng biến và nghịch biến}

\ % Lùi đầu dòng

Thông qua biểu diễn hình học của một hàm số, người ta sẽ thấy hàm số tăng và giảm theo giá trị đầu vào. Từ đó, xây dựng được hai khái niệm là hàm đồng biến và hàm nghịch biến. Cụ thể, $f$ được gọi là \defText{đồng biến} trên tập $D$ nếu với mọi $x_1, x_2 \in D$, có $$\defMath{x_1 < x_2 \implies f\left(x_1\right) < f\left(x_2\right)}.$$ Bằng một định nghĩa tương tự, $f$ được gọi là \defText{nghịch biến} trên tập $D$ nếu với mọi $x_1, x_2 \in D$, có $$\defMath{x_1 < x_2 \implies f\left(x_1\right) > f\left(x_2\right)}.$$ 

{
   \begin{minipageindent}{0.48\textwidth}
      \begin{figure}[H]
         \centering
         \begin{tikzpicture}
            \draw[->] (-2, 0) -- (4, 0) node[right] {$x$};
            \draw[->] (0, -1) -- (0, 4)  node[above] {$f(x)$};
            \foreach \x/\y in {-1/0.707106781, 3/2.838138064} {
               \draw[dashed] (\x,\y) -- (\x,0);
               \draw[dashed] (\x,\y) -- (0,\y);
               \filldraw[color=colorEmphasisCyan] (\x, \y) circle (\pointSize);
               }
            \draw[graph thickness, samples=80, color=colorEmphasisCyan, domain=-2.000:4.000] plot (\x, {(2^((\x)/2)) / 1});
            \node[right] at (0,0.707106781) {$y_1$};
            \node[left] at (0,2.838138064) {$y_2$};
            \node[below] at (-1,0) {$x_1$};
            \node[below] at (3,0) {$x_2$};
      \end{tikzpicture}
      \caption{Ví dụ hàm $f$ đồng biến}
   \end{figure}
   \end{minipageindent}
   \hfill
   \begin{minipageindent}{0.48\textwidth}
      \begin{figure}[H]
         \centering
         \begin{tikzpicture}
            \draw[->] (-2, 0) -- (4, 0) node[right] {$x$};
            \draw[->] (0, -1) -- (0, 4)  node[above] {$f(x)$};
            \draw[dashed] (0.75, 0) -- (0.75, 1.155);
            \draw[dashed] (3, 0) -- (3, 0.5773502691896257);
            \draw[dashed] (0, 1.155) -- (0.75, 1.155);
            \draw[dashed] (0, 0.5773502691896257) -- (3, 0.5773502691896257);
            \filldraw[color=colorEmphasis] ({0.75}, { 1.155 }) circle (\pointSize);
            \filldraw[color=colorEmphasis] ({3.0}, { 0.5773502691896257 }) circle (\pointSize);
            \draw[graph thickness, samples=80, color=colorEmphasis, domain=0.062:4.000] plot (\x, {(1/(((\x)/1)^(1/2))) / 1});
            \node[left] at (0,1.155) {$y_1$};
            \node[left] at (0,0.5773502691896257) {$y_2$};
            \node[below] at (0.75,0) {$x_1$};
            \node[below] at (3,0) {$x_2$};
         \end{tikzpicture}
         \caption{Ví dụ hàm $f$ nghịch biến}
      \end{figure}      
   \end{minipageindent}
}

Bạn đọc hoàn toàn có thể thu hẹp định nghĩa này từ tập $D$ thành một khoảng $\left(a; b\right)$. Lí do để chỉ xét trong một khoảng như vậy là từ ứng dụng trong thực tiễn, ít khi nào người ta xét sự tăng giảm của hàm số trên nhiều khoảng tách biệt với nhau.

\exercise Chứng minh rằng
\begin{enumerate}
   \item $2x$ đồng biến trên $\mathbb{R}$;
   \item $(x-1)^2$ nghịch biến trên $\left(-\infty; 1\right)$;
   \item $|x|\cdot \left|\left|x - 1\right| - 1\right|$ đồng biến trên $(0; 1)$ và nghịch biến trên $(1; 2)$;
   \item $x^2 - \left(x - \lfloor x \rfloor\right)$ đồng biến trên mỗi khoảng $(n; n + 1)$ với $n$ là số nguyên dương;
   \item $x^2 - \left(x - \lfloor x \rfloor\right)$ đồng biến trên $\left(1; \infty\right)$.
\end{enumerate}

\solution

\setcounter{subexercise}{1}
\arabic{subexercise}. Xét hai giá trị $x_1, x_2 \in \mathbb{R}$ sao cho $x_1 < x_2$. Khi này, hiển nhiên có được $2x_1 < 2x_2$. Do đó, kết luận được $2x$ là đồng biến trên $\mathbb{R}$.

\stepcounter{subexercise}
\arabic{subexercise}. Xét hai giá trị $x_1, x_2 \in \left(-\infty; 1\right)$ sao cho $x_1 < x_2$. Thực hiện một số biến đổi:
\begin{align*}
   &x_1 - 1 < x_2 - 1 < 0 \\
   \iff &x_1 - 1 > x_2 - 1 > 0 \\
   \implies &\begin{cases}
      \left(x_1 - 1\right)^2 > \left(x_1 - 1\right)\left(x_2 - 1\right) \equationexplanation{cùng nhân hai vế với $x_1 - 1$ dương}\\
      \left(x_1 - 1\right)\left(x_2 - 1\right) > \left(x_2 - 1\right)^2 \equationexplanation{cùng nhân hai vế với $x_2 - 1$ dương}
   \end{cases} \\
   \implies &\left(x_1 - 1\right)^2 > \left(x_2 - 1\right)^2.
\end{align*}
Chúng ta dễ dàng thấy điều phải chứng minh.

\stepcounter{subexercise}
\arabic{subexercise}. Trên khoảng $(0; 1)$, $|x| = x$ và $\left|\left|x - 1\right| - 1\right| = \left|1 - x - 1\right| = \left|-x\right| = x$. Do đó, $|x|\cdot \left|\left|x - 1\right| - 1\right| = x^2$. Khi này, với $0 < x_1 < x_2 < 1$, dễ dàng có được $x_1^2 < x_2^2$. Do đó, $|x|\cdot \left|\left|x - 1\right| - 1\right|$ đồng biến trên $(0; 1)$.

\stepcounter{subexercise}
\arabic{subexercise}. 

% \input{\chapdir toan_hoc_nen_tang/ham_luong_giac.tex}
