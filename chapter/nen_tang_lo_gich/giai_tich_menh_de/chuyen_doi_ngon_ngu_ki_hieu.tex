\section{Chuyển đổi mệnh đề lô-gích từ dạng ngôn ngữ sang dạng kí hiệu}

\ % Lùi đầu dòng

Ngôn ngữ là kết quả của sự phát triển ý thức con người tới một trình độ nhất định. Nó cũng là một công cụ vô cùng cần thiết để các cá nhân, tập thể có thể trao đổi với nhau để từ đó xã hội có thể hoạt động. Ngôn ngữ, mặc dù là một sản phẩm mang trình độ cao, vẫn tồn tại trong nó nhiều thiếu sót. Đó là khi, với ngôn ngữ, chúng ta cũng có thể tạo ra những lỗ hổng về lập luận hay những cạm bẫy về tư duy lô-gích. Và với lí do như vậy, để thực hiện phân tích lô-gích, mọi câu không chỉ cần được biểu diễn dưới dạng mệnh đề dưới dạng ngôn ngữ, mà còn cần phải biểu diễn dưới dạng kí hiệu.

Để thực hiện việc chuyển đổi từ các mệnh đề sử dụng ngôn ngữ tự nhiên sang các mệnh đề kí hiệu một cách chính xác, từ một mệnh đề phức, cần phải xác định các mệnh đề nguyên tử mà không thể tách ra thành những mệnh đề con nữa. Từ đó, xác định các phép nối mệnh đề giữa chúng và thực hiện phân tích cấu trúc của mệnh đề ban đầu để có thể xây dựng được chính xác mệnh đề dưới dạng kí hiệu. Lấy ví dụ, cần chuyển đổi mệnh đề
\begin{center}
    ``Nếu nền kinh tế tăng trưởng và lạm phát thấp thì đời sống nhân dân được cải thiện.''    
\end{center}
sang dạng kí hiệu. Các mệnh đề nguyên tử bao gồm:
\begin{itemize}
    \item $P$: ``Nền kinh tế tăng trưởng.'';
    \item $Q$: ``Lạm phát thấp'';
    \item $R$: ``Đời sống nhân dân được cải thiện''.
\end{itemize}
Thêm vào đó, chúng ta có những phép nối:
\begin{itemize}
    \item $\land$ qua từ nối ``và'';
    \item $\implies$ qua cặp từ nối ``Nếu\dots\ thì\dots''.
\end{itemize}
Qua đó, mệnh đề ban đầu đã được chuyển đổi thành
$$
P \land Q \implies R.
$$

Có thể bạn đọc nghĩ rằng chúng ta đã hoàn thành quá trình chuyển đổi. Tuy nhiên, có một vấn đề cần phải đề cập tới, đó là cần phải hiểu mệnh đề này như thế nào? Hiểu như
``$
(P \land Q) \implies R
$''
hay
``$
P \land (Q \implies R)
$''? Với mệnh đề đơn giản như thế này, bạn đọc có thể nhìn ra ngay rằng cách hiểu đầu tiên sẽ là hợp lí nhất. Nó vừa tuân theo thứ tự các phép nối, vừa tuân theo cách hiểu tự nhiên. Mặc dù vậy, khi các mệnh đề trở nên phức tạp với nhiều điều kiện chồng chéo, ngôn ngữ tự nhiên thường bị "quá tải" trong việc phân định phạm vi của các từ nối. Chẳng hạn, trích quy chế của một trường đại học có thể là:
\begin{center}
    ``Để nhận học bổng, sinh viên phải có điểm trung bình từ $8,0$ \\và là hộ nghèo hoặc có bài báo nghiên cứu khoa học.''
\end{center}
Để phân tích câu này, đặt
\begin{itemize}
    \item $G$: ``Sinh viên có điểm trung bình từ $8,0$'';
    \item $P$: ``Sinh viên thuộc diện hộ nghèo'';
    \item $S$: ``Sinh viên có nghiên cứu khoa học''.
\end{itemize}
Khi này, mệnh đề trở thành $G \land P \lor S$. Cân nhắc lại rằng, ngôn ngữ tự nhiên không áp dụng quy tắc về thứ tự các phép nối. Hai cách kết hợp từ ``và'' và ``hoặc'' tạo ra hai cách hiểu khác nhau hoàn toàn về quyền lợi của một sinh viên $A$ có điểm số chưa tốt nhưng có bài báo khoa học.
\begin{enumerate}
    \item \textcolor{colorEmphasisCyan}{Trường hợp 1 --- $(G \land P) \lor S$}: $A$ thỏa mãn $S$, cho nên toàn bộ mệnh đề này đúng. $A$ được xét nhận học bổng;
    \item \textcolor{colorEmphasis}{Trường hợp 2 --- $G \land (P \lor S)$}: Mặc dù $A$ có thành tích $S$, $A$ chưa đạt điều kiện tiên quyết là $G$, cho nên $A$ không được xét học bổng.
\end{enumerate}
Từ ví dụ này, chúng ta có thể đưa ra được một phát hiện rằng việc cần phải loại bỏ toàn bộ những hoài nghi và cách hiểu tối nghĩa về câu cũng là lí do tại sao những tờ hợp đồng, bộ luật đều dùng những ngôn từ được phức tạp hóa mà tuy ẩn sâu trong đó là những ý tưởng đơn giản.

\exercise Phiên dịch các mệnh đề sau thành dưới dạng kí hiệu. Nếu rút gọn mệnh đề thành kí hiệu chữ cái thì cần phải chú thích rõ mỗi chữ cái ứng với mệnh đề cấu thành tương ứng nào.
\begin{enumerate}
    \item ``Bạn đọc không thể vừa đi làm đúng giờ vừa ngủ nướng.'';
    \item ``Nếu nước trong bể bơi được làm sạch, thì hoặc A-bi-cô có thể nhìn thấy đáy của bể bơi, hoặc mắt của A-bi-cô có vấn đề.'';
    \item ``Da-đê-en sẽ đi dã ngoại, trừ khi ngày mai trời mưa.'';
    \item ``Không đúng là tôi không không thích em.'';
    \item ``Nếu không phải là bạn đọc nộp bài đúng hạn và bài tập làm là chính xác, thì nếu giáo viên không châm trước, bạn đọc sẽ bị trượt.'';
    \item ``Chỉ khi cô bé quàng khăn đỏ không quàng khăn màu đỏ và Cám là một người em tốt của Tấm, thì lô-gích mới ngừng là một môn phức tạp và bộ não Ê-mô-ri-ô mới được giải phóng.'';
    \item ``Đế quốc Mĩ đã rất nhiều lần lên kế hoạch ngăn chặn Việt Nam thống nhất, nhưng, bởi vì dân tộc ta có một tinh thần đoàn kết vững mạnh, quân đội được xây dựng thành lực lượng hùng mạnh, và sự lãnh đạo của Đảng đầy sáng suốt, đất nước chúng ta đã có thống nhất, độc lập, tự do như ngày hôm nay.''.
\end{enumerate}

\solution

\setcounter{subexercise}{1}
\arabic{subexercise}. 
Kí hiệu các mệnh đề con:
\begin{itemize}
    \item $O$: ``Bạn đọc đi làm đúng giờ.'';
    \item $S$: ``Bạn đọc ngủ nướng.''.
\end{itemize}
Phép nối lô-gích: $\uparrow$, thông qua ``không thể\dots\ vừa\dots\ vừa\dots''. Vậy có mệnh đề dưới dạng kí hiệu là $O \uparrow S$.

\stepcounter{subexercise}
\arabic{subexercise}. Đặt biến:
\begin{itemize}
    \item $C$: ``Nước trong bể bơi được làm sạch.'';
    \item $B$: ``A-bi-cô có thể nhìn thấy đáy của bể bơi.'';
    \item $P$: ``Mắt của A-bi-cô có vấn đề.''.
\end{itemize}
Mệnh đề dưới dạng kí hiệu: $C \implies B \lor P$ hoặc $C \implies B \oplus P$ nếu hiểu ``hoặc\dots\ hoặc'' theo nghĩa loại trừ.

\stepcounter{subexercise}
\arabic{subexercise}. Viết lại câu với nghĩa không đổi\footnote{Không chỉ ở mỗi bài tập môn tiếng Anh mới phải làm điều này.}: 
\begin{center}
    ``Nếu ngày mai trời mưa thì Da-đê-en không đi dã ngoại.''\footnote{Đây không phải là cách hiểu duy nhất. Nhắc lại, ngôn ngữ chỉ là phương tiện, không phải luật lệ chính xác.}.
\end{center}
Kí hiệu các mệnh đề con:
\begin{itemize}
    \item $P$: ``Ngày mai trời mưa.'';
    \item $D$: ``Da-đê-en đi dã ngoại.''.
\end{itemize}
Mệnh đề dưới dạng kí hiệu: $P \implies \neg D$.

\stepcounter{subexercise}
\arabic{subexercise}. Đặt biến:
\begin{itemize}
    \item $A$: ``Tôi thích em.''.
\end{itemize}
Phân tích:
\begin{itemize}
    \item ``Tôi không không thích em.'': $\neg \neg A$;
    \item ``Không phải là\dots'': phủ định bao trùm.
\end{itemize}
Mệnh đề dưới dạng kí hiệu: $\overline{\neg \neg A}$.

\stepcounter{subexercise}
\arabic{subexercise}. Đặt biến:
\begin{itemize}
    \item $T$: ``Bạn đọc nộp bài đúng hạn.'';
    \item $C$: ``Bài tập làm là chính xác.'';
    \item $A$: ``Giáo viên châm trước.'';
    \item $F$: ``Bạn đọc sẽ bị trượt.''.
\end{itemize}
Mệnh đề dưới dạng kí hiệu: $T \uparrow C \implies (\neg A \implies F)$.

\stepcounter{subexercise}
\arabic{subexercise}. Đặt biến:
\begin{itemize}
    \item $R$: ``Cô bé quàng khăn đỏ quàng khăn màu đỏ.'';
    \item $S$: ``Cám là một người em tốt của Tấm.'';
    \item $C$: ``Lô-gích ngừng là một môn phức tạp.'';
    \item $E$: ``Bộ não của Ê-mô-ri-ô mới được giải phóng.''.
\end{itemize}
Mệnh đề dưới dạng kí hiệu: $\neg R \land S \implies C \land E$. Từ ``mới'' là một từ mang tính nhấn mạnh, không có giá trị về mặt lô-gích.


\stepcounter{subexercise}
\arabic{subexercise}. Kí hiệu các mệnh đề con:
\begin{itemize}
    \item $A$: ``Đế quốc Mĩ đã rất nhiều lần lên kế hoạch ngăn chặn Việt Nam thống nhất.'';
    \item $B$: ``Dân tộc ta có một tinh thần đoàn kết vững mạnh.'';
    \item $C$: ``Quân đội được xây dựng thành lực lượng hùng mạnh.'';
    \item $D$: ``Sự lãnh đạo của Đảng đầy sáng suốt.'';
    \item $E$: ``Đất nước chúng ta đã có thống nhất, độc lập, tự do như ngày hôm nay.''.
\end{itemize}

Từ ``nhưng'' không chỉ biểu diễn hai sự việc xảy ra, mà còn thể hiện một sự việc có tác động trái ngược đến kết quả của sự việc khác. Tuy nhiên, trong lô-gích diễn dịch, không có mối quan hệ phủ định lẫn nhau, cho nên chúng ta biểu diễn nhưng bằng phép nối $\land$.

Xét cấu trúc ``bởi vì $X$ nên $Y$'', theo nghĩa thông thường, không chỉ biểu hiện mối quan hệ nhân quả giữa $X$ và $Y$, mà nó còn biểu hiện là $X$ và $Y$ đã xảy ra. Cho nên, cấu trúc này tương đương với ``có $X$ và $Y$, và ngoài ra $X$ là nguyên nhân của $Y$''.

Vậy, mệnh đề dưới dạng kí hiệu sẽ là
$$A \land (B \land C \land D \land E \land (B \land C \land D \implies E)).$$
