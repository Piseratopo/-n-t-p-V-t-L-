\subsection{Số thực}

\ % Lùi đầu dòng

Các ngành toán học đều có nhiều khái niệm, định lí, chứng minh trên các vật thể khác nhau, nhưng tổng quát trong đấy vẫn có nhiều điểm chung. Một cái chung như vậy là việc sử dụng \defText{dấu bằng}, $\defMath{=}$, để biểu diễn quan hệ giống nhau. Ở trong khuôn khổ cuốn sách này, chúng ta sẽ hiểu một cách nôm na rằng hai vế sẽ bằng nhau khi và chỉ khi hai vế có giá trị bằng nhau.

Phần đầu tiên này đề cập các yếu tố đại số cơ bản của \defText{số thực}, cụ thể là những hệ thức mà trong đó số thực tương tác với một số hữu hạn các \defText{phép cộng} và \defText{phép nhân}. 

Gọi $\defMath{\mathbb{R}}$ là tập hợp số thực. Nếu $a, b, c$ đều thuộc $\mathbb{R}$, với phép cộng và phép nhân mang ý nghĩa thông thường, có:
\begin{itemize}
   \item $a + b$ và $a\cdot b$ (hay $a \times b$, $ab$) đều thuộc $\mathbb{R}$;
   \item $a+b=b+a$ và $a\cdot b=b\cdot a$ (\defText{tính giao hoán});
   \item $a+(b+c)=(a+b)+c$ và $a\cdot (b\cdot c)=(a\cdot b)\cdot c$ (\defText{tính kết hợp});
   \item $a\cdot (b+c)=a\cdot b+a\cdot c$ (\defText{tính phân phối});
   \item $a\cdot 1 = a$ (\defText{đơn vị});
   \item $a + 0 = a$ và $a\cdot 0 = 0$ (\defText{số không});
   \item $a + c = b + c \implies a = b$ (\defText{tính giản ước được});
   \item Nếu $c \neq 0$, $a\cdot c = b\cdot c \implies a = b$ (\defText{tính giản ước được}).
\end{itemize}
Tính chất giao hoán cho phép viết $a + b + c$ và $a\cdot b\cdot c$ mà không phần phải quan tâm đến thứ tự tính toán của các phép tính trong hai biểu thức này. Một điều cần lưu ý là không phải mọi thực thể trong toán học đều đơn giản như vậy. Lấy ví dụ, phép nhân có hướng hai véc-tơ, một phép tính thường xuyên được sử dụng trong vật lí, vừa không có tính giao hoán, vừa không có tính kết hợp.

Mỗi $a$ chỉ tồn tại một \defText{số đối} $-a$ duy nhất sao cho $a + (-a) = 0$ và nếu $a\neq 0$, tồn tại một \defText{số nghịch đảo} $\frac{1}{a}$ duy nhất sao cho $a\cdot \frac{1}{a} = 1$. Từ đó, chúng ta có được định nghĩa của hai phép tính cơ bản còn lại. \defText{Phép trừ} được định nghĩa là phép cộng với số đối: $$\defMath{a-b = a + (-b)};$$ và \defText{phép chia} được định nghĩa như phép nhân với số nghịch đảo $$\defMath{\frac{a}{b} = a\cdot \frac{1}{b}}.$$ Trên tập số thực, không có nghịch đảo của $0$.

Khi nhân hai số mà có liên quan đến số đối, chúng ta thực hiện có kết quả dựa vào bảng như sau:

\begin{table}[H]
   \centering
   \caption{Bảng nhân hai số có sự tồn tại của số đối}
   \begin{tblr}{
      colspec={|c|c|c|},
      hlines,
      vlines
   }
      $\cdot$ & $m$   & $-m$  \\
      $n$     & $mn$  & $-mn$ \\
      $-n$    & $-mn$ & $mn$  \\
   \end{tblr}
\end{table}


