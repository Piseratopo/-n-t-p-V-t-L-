\chapter{Giải tích vị từ bậc nhất}

\ % Lùi đầu dòng

Ở chương về giải tích mệnh đề, chúng ta đã làm quen với mệnh đề, các phép nối mệnh đề và làm một số chuyển đổi từ ngôn ngữ thông thường sang dạng lô-gích kí hiệu. Một chút suy ngẫm sẽ đưa chúng ta đến một kết luận hiển nhiên rằng cấu trúc ``ngữ pháp'' cơ bản như thế này không thể nào đủ cho các lập luận thông thường trong cuộc sống vật lí. Lập luận lô-gích bằng giải tích mệnh đề không mạnh bằng ngữ pháp tiếng Việt thông thường ở những khoản như:
\begin{itemize}
    \item Phân biệt các từ loại --- danh từ, động từ, tính từ,\dots;
    \item Phân tích cấu trúc nội tại của câu để xác định mối quan hệ giữa chủ thể và hành động (chủ ngữ và vị ngữ);
    \item Diễn đạt các khái niệm về số lượng và phạm vi của đối tượng;
    \item Xác định các mối liên hệ phức tạp giữa nhiều thực thể khác nhau trong cùng một mệnh đề (ví dụ: quan hệ sở hữu, quan hệ huyết thống);
    \item Thể hiện các sắc thái về thời gian (đã, đang, sẽ) và các trạng thái tình thái (có lẽ, chắc chắn, bắt buộc).
\end{itemize}
Lô-gích không được dùng để thay thế ngôn ngữ, nhưng cần phải đủ để có thể biểu diễn các sự vật, sự việc một cách khách quan và có hệ thống. Điều này yêu cầu mở rộng giới hạn của giải tích mệnh đề. Để làm như vậy, chúng ta cần một vài định nghĩa mới.

\section{Thành tố cấu thành mệnh đề}

\ % Lùi đầu dòng

Trong giải tích mệnh đề, mệnh đề đơn là thành tố nhỏ nhất và không thể chia nhỏ ra được. Ở trong giải tích vị từ, mệnh đề còn có thể chia nhỏ thành các bộ phận, giống như một câu có thể được chia ra thành các từ và cụm từ cấu thành vậy.

\subsection{Hạng thức}