\subsection{Phép hợp hai hàm số}

\ % Lùi đầu dòng

Khi áp dụng hàm $f$ lên kết quả của một hàm $g$ khác, chúng ta có \defText{phép hợp} hai hàm số $f$ và $g$ và kí hiệu hàm hợp này như sau: $$\defMath{\left(f\circ g\right)(x) = f\left(g(x)\right)}.$$

Để ý rằng mặc dù $f$ được viết trước $g$, $g$ được thực hiện trước $f$.

Nếu mà hàm $f$ được hợp với chinh nó, thì có kí hiệu rút gọn viết giống như dưới dạng lũy thừa như sau:
$$\defMath{f\circ f = f^2}.$$
Nếu lặp lại $f$ với $n$ lần, thì có:
$$\defMath{\underbrace{f\circ f\circ \cdots \circ f}_{n\defText{ lần}}  = f^n}.$$
Thật không may là cách viết này lại thông thường mang nghĩa lũy thừa thông thường. Ví dụ, người ta thường viết $\sin^2(x) = \left(\sin (x)\right)^2$ thay vì $\sin^2(x) = \sin \left(\sin (x)\right)$. Tác giả sẽ dùng cách viết lũy thừa hàm để biểu thị hàm hợp. Nếu cần lũy thừa của kết quả hàm, tác giả sẽ viết dưới dạng $\left(f (x)\right)^n$.

\exercise Cho các hàm $f$ và $g$ được định nghĩa thông qua đồ thị như sau:

{
   \begin{minipage}{0.48\textwidth}
      \begin{figure}[H]
         \centering
         \begin{tikzpicture}
            \draw[->] (-2.6666666666666665, 0) -- (2.6666666666666665, 0) node[right] {$x$};
            \draw[->] (0, -2.6666666666666665) -- (0, 2.6666666666666665)  node[above] {$f(x)$};
            \filldraw[color=colorEmphasisCyan] (-2.0, -1.3333333333333333) circle (\pointSize) node[above] {$\left(-3;-2\right)$};
            \filldraw[color=colorEmphasisCyan] (-1.3333333333333333, -0.6666666666666666) circle (\pointSize) node[above] {$\left(-2;-1\right)$};
            \filldraw[color=colorEmphasisCyan] (-0.6666666666666666, 0.0) circle (\pointSize) node[above] {$\left(-1;0\right)$};
            \filldraw[color=colorEmphasisCyan] (0.0, -0.6666666666666666) circle (\pointSize) node[above] {$\left(0;-1\right)$};
            \filldraw[color=colorEmphasisCyan] (0.6666666666666666, -1.3333333333333333) circle (\pointSize) node[above] {$\left(1;-2\right)$};
            \filldraw[color=colorEmphasisCyan] (1.3333333333333333, -0.6666666666666666) circle (\pointSize) node[above] {$\left(2;-1\right)$};
            \filldraw[color=colorEmphasisCyan] (2.0, -1.3333333333333333) circle (\pointSize) node[above] {$\left(3;-2\right)$};
         \end{tikzpicture}
         \caption{Đồ thị của $f$}
      \end{figure}
   \end{minipage}
   \hfill
   \begin{minipage}{0.48\textwidth}
      \begin{figure}[H]
         \centering
         \begin{tikzpicture}
            \draw[->] (-2.6666666666666665, 0) -- (2.6666666666666665, 0) node[right] {$x$};
            \draw[->] (0, -2.6666666666666665) -- (0, 2.6666666666666665)  node[above] {$g(x)$};
            \filldraw[color=colorEmphasis] (-2.00000000000000, 2.00000000000000) circle (\pointSize) node[above] {$\left(-3;3\right)$};
            \filldraw[color=colorEmphasis] (-1.33333333333333, 0.666666666666667) circle (\pointSize) node[above] {$\left(-2;1\right)$};
            \filldraw[color=colorEmphasis] (-0.666666666666667, -0.666666666666667) circle (\pointSize) node[above] {$\left(-1;-1\right)$};
            \filldraw[color=colorEmphasis] (0, -2.00000000000000) circle (\pointSize) node[above] {$\left(0;-3\right)$};
            \filldraw[color=colorEmphasis] (0.666666666666667, -0.666666666666667) circle (\pointSize) node[above] {$\left(1;-1\right)$};
            \filldraw[color=colorEmphasis] (1.33333333333333, 0.666666666666667) circle (\pointSize) node[above] {$\left(2;1\right)$};
            \filldraw[color=colorEmphasis] (2.00000000000000, 2.00000000000000) circle (\pointSize) node[above] {$\left(3;3\right)$};
         \end{tikzpicture}
         \caption{Đồ thị của $g$}
      \end{figure}
   \end{minipage}
}

Xác định tập xác định, tập giá trị của các hàm hợp $f\circ g$ và $g\circ f$. Sau đó, vẽ đồ thị của hai hàm này.

\solution

Tập xác định của $f$ và $g$ đều là $\{-3; -2; -1; 0; 1; 2; 3\}$. Ngoài ra, tập giá trị của $f$ là $\{-2;-1;0\}$ và tập giá trị của $g$ là $\{-3; -1; 1; 3\}$.

Có $\left(f\circ g\right)(x) = f\left(g(x)\right)$ với mọi $x$ làm cho $f\circ g$ có nghĩa. Để ngắn gọn, chúng ta thực hiện kẻ bảng:

\begin{table}[H]
   \centering
   \begin{tabular}{|c|c|c|c|c|c|c|c|}
      \hline
      $x$ & $-3$ & $-2$ & $-1$ & $0$ & $1$ & $2$ & $3$\\
      \hline
      $g(x)$ & $3$ & $1$ & $-1$ & $-3$ & $-1$ & $1$ & $3$\\
      \hline
      $f\left(g(x)\right)$ & $-2$ & $0$ & $-2$ & $-2$ & $-2$ & $0$ & $-2$\\
      \hline
   \end{tabular}
   \caption{Bảng giá trị của $f\circ g$}
\end{table}

Vậy tập xác định và tập giá trị của $f\circ g$ lần lượt là $\{-3; -2; -1; 0; 1; 2; 3\}$ và $\{-2; 0\}$.

Tương tự, chúng ta có $\left(g\circ f\right)(x) = g\left(f(x)\right)$. Kẻ bảng giá trị:

\begin{table}[H]
   \centering
   \begin{tabular}{|c|c|c|c|c|c|c|c|}
      \hline
      $x$ & $-3$ & $-2$ & $-1$ & $0$ & $1$ & $2$ & $3$\\
      \hline
      $f(x)$ & $-2$ & $-1$ & $0$ & $-1$ & $-2$ & $-1$ & $-2$\\
      \hline
      $g\left(f(x)\right)$ & $1$ & $-1$ & $-3$ & $-1$ & $1$ & $-1$ & $1$\\
      \hline
   \end{tabular}
   \caption{Bảng giá trị của $g\circ f$}
\end{table}

Từ bảng, chúng ta có $g\circ f$ có tập xác định là $\{-3; -2; -1; 0; 1; 2; 3\}$ và tập giá trị là $\{-3; -1; 1\}$.

Cuối cùng, chúng ta có đồ thị của hai hàm:

{
   \begin{minipage}{0.48\textwidth}
      \begin{figure}[H]
         \centering
         \begin{tikzpicture}
            \draw[->] (-2.6666666666666665, 0) -- (2.6666666666666665, 0) node[right] {$x$};
            \draw[->] (0, -2.6666666666666665) -- (0, 2.6666666666666665)  node[above] {$\left(f\circ g\right)(x)$};
            \filldraw[color=colorEmphasisCyan] (-2.00000000000000, -1.33333333333333) circle (\pointSize) node[above] {$\left(-3;-2\right)$};
            \filldraw[color=colorEmphasisCyan] (-1.33333333333333, 0) circle (\pointSize) node[above] {$\left(-2;0\right)$};
            \filldraw[color=colorEmphasisCyan] (-0.666666666666667, -1.33333333333333) circle (\pointSize) node[above] {$\left(-1;-2\right)$};
            \filldraw[color=colorEmphasisCyan] (0, -1.33333333333333) circle (\pointSize) node[below] {$\left(0;-2\right)$};
            \filldraw[color=colorEmphasisCyan] (0.666666666666667, -1.33333333333333) circle (\pointSize) node[above] {$\left(1;-2\right)$};
            \filldraw[color=colorEmphasisCyan] (1.33333333333333, 0) circle (\pointSize) node[above] {$\left(2;0\right)$};
            \filldraw[color=colorEmphasisCyan] (2.00000000000000, -1.33333333333333) circle (\pointSize) node[above] {$\left(3;-2\right)$};
         \end{tikzpicture}
         \caption{Đồ thị của $f\circ g$}
      \end{figure}
   \end{minipage}
   \hfill
   \begin{minipage}{0.48\textwidth}
      \begin{figure}[H]
         \centering
         \begin{tikzpicture}
            \draw[->] (-2.6666666666666665, 0) -- (2.6666666666666665, 0) node[right] {$x$};
            \draw[->] (0, -2.6666666666666665) -- (0, 2.6666666666666665)  node[above] {$\left(g\circ f\right)(x)$};
            \filldraw[color=colorEmphasis] (-2.00000000000000, 0.666666666666667) circle (\pointSize) node[above] {$\left(-3;1\right)$};
            \filldraw[color=colorEmphasis] (-1.33333333333333, -0.666666666666667) circle (\pointSize) node[above] {$\left(-2;-1\right)$};
            \filldraw[color=colorEmphasis] (-0.666666666666667, -2.00000000000000) circle (\pointSize) node[above] {$\left(-1;-3\right)$};
            \filldraw[color=colorEmphasis] (0, -0.666666666666667) circle (\pointSize) node[above] {$\left(0;-1\right)$};
            \filldraw[color=colorEmphasis] (0.666666666666667, 0.666666666666667) circle (\pointSize) node[above] {$\left(1;1\right)$};
            \filldraw[color=colorEmphasis] (1.33333333333333, -0.666666666666667) circle (\pointSize) node[above] {$\left(2;-1\right)$};
            \filldraw[color=colorEmphasis] (2.00000000000000, 0.666666666666667) circle (\pointSize) node[above] {$\left(3;1\right)$};
         \end{tikzpicture}
         \caption{Đồ thị của $g\circ f$}
      \end{figure}
   \end{minipage}
}

\exercise Có tồn tại hàm $\chungF$ sao cho với mọi $f$: $\chungF \circ f = f \circ \chungF = f$?

\solution

Có, đặt $\chungF(x) = x$ với mọi $x$. Khi này, hiển nhiên rằng, với mọi $x$ thuộc tập xác định của $f$:
$$\left(\chungF \circ f\right)(x) = \chungF\left(f(x)\right) = f(x)~\text{và}$$
$$\left(f \circ \chungF\right)(x) = f\left(\chungF(x)\right) = f(x).$$

Tên gọi của $\chungF$ là \defText{hàm đồng nhất}.