\section{Thành tố cấu tạo của giải tích vị từ}

\ % Lùi đầu dòng

Trong giải tích mệnh đề, mệnh đề đơn là thành tố nhỏ nhất và không thể chia nhỏ ra được. Ở trong giải tích vị từ, mệnh đề còn có thể chia nhỏ thành các bộ phận, giống như một câu có thể được chia ra thành các từ và cụm từ cấu thành vậy.

\subsection{Hạng thức}

\ % Lùi đầu dòng

Nếu nói là ``biến'' hay ``ẩn'' thì chắc sẽ quen hơn so với từ ``hạng thức''. Bạn đọc đã làm toán rồi thì sẽ quen với việc biến số được dùng để thay thế các số chưa biết hay các số không xác định nhằm thể hiện một số tính chất không phụ thuộc vào biến số. Trong lô-gích, hạng thức cũng có vai trò tương tự. Lấy ví dụ.
\begin{center}
    ``Đã rất nhiều lần anh đi học muộn rồi!''
\end{center}
là tương đương với
\begin{center}
    ``Đây là lần thứ $N$ anh đi học muộn rồi!''
\end{center}
và trong ví dụ này, $N$ được thay thế cho số lần đi học muộn trong câu đầu tiên. Hạng thức không cần phải biểu diễn bất cứ sự vật hay hiện tượng cụ thể gì cả, như ``$x$'' trong
\begin{center}
    ``$x > 3$''
\end{center}
nhưng việc hiểu rằng $x$ đại diện cho một số nào đó vẫn có một giá trị nhất định. Qua đó, chúng ta có thể thấy \defText{hạng thức} là các thực thể đại diện cho các đối tượng.

\subsection{Vị từ}

\ % Lùi đầu dòng

Xét các mệnh đề sau:
\begin{itemize}
    \item ``Quả táo có màu đỏ.'';
    \item ``Hoa hồng có màu đỏ.'';
    \item ``Cờ Việt Nam có màu đỏ.''.
\end{itemize}
Các mệnh đề đều có cụm ``có màu đỏ'', cho nên chúng ta có thể đặt một hạng thức $x$ và đưa các mệnh đề về dạng ``$x$ có màu đỏ.''. Tương tự, với các mệnh đề
\begin{itemize}
    \item ``Định luật bảo toàn năng lượng là định luật vật lí có tính ứng dụng cao.'',
    \item ``Định luật I nhiệt động lực học là định luật vật lí có tính ứng dụng cao.'',
    \item ``Định luật Ohm là định luật vật lí có tính ứng dụng cao.'',
\end{itemize}
chúng ta có thể viết chúng dưới dạng chung là ``$y$ là định luật vật lí có tính ứng dụng cao.''. Các bộ phận ``có màu đỏ'' và ``là định luật vật lí có tính ứng dụng cao'' là các khẳng định về thuộc tính của một đối tượng hoặc mối quan hệ giữa các đối tượng, và qua đó được gọi là \defText{vị từ}. Vị từ thường có vai trò như các \emph{vị ngữ} trong câu, tuy nhiên, vị trí của vị từ có thể thay đổi tùy thuộc vào cấu trúc ngữ pháp của mệnh đề.

Quay trở lại về ví dụ vị từ ``có màu đỏ''. Chúng ta có thể viết dưới dạng kí hiệu dạng chung của các mệnh đề là
$$R(x).$$
Khi này, $R$ đã thay thế cho phần vị từ đã nhắc đến. Xét một ví dụ khác:
\begin{center}
    ``Việt Nam nằm gần biển hơn so với Lào.''.
\end{center}
Chúng ta hoàn toàn có thể viết mệnh đề này dưới dạng $S(q)$ với $S$ có nghĩa là ``nằm gần biển hơn so với Lào'' hay $T(q)$ với $T$ có nghĩa là ``Việt Nam nằm gần biển hơn so với (quốc gia này)'', nhưng sẽ tổng quát hơn nếu chúng ta thực hiện thay thế cả hai quốc gia bởi hạng thức:
\begin{center}
    ``$v$ nằm gần biển hơn so với $q$.''
\end{center}
và kí hiệu bởi $B(v, q)$. Nếu như vị từ không có hạng thức thì sao? Nó sẽ trở thành một trạng thái của tồn tại hay một tính chất bao trùm lên mọi khái niệm, và có kí hiệu là $N()$ hoặc $N$ giống như một mệnh đề.

\subsection{Lượng từ}

\ % Lùi đầu dòng

Có mệnh đề sau với hạng thức $x$:
\begin{center}
    ``$x$ là một thủ đô.''.
\end{center}
Đây không phải là mệnh đề, do chúng ta chưa xác định $x$ là nơi nào, hay $x$ còn có phải là nơi chốn hay không. Có thể thay thế $x$ bởi một đại từ cụ thể, nhưng chúng ta không cần phải làm thế bởi lô-gích cho phép rút ra các mệnh đề đúng hoặc sai bằng cách mở rộng mệnh đề về thủ đô đó. Chúng ta có thể thêm ``với mọi'':
\begin{center}
    ``Với mọi $x$, $x$ là một thủ đô.''
\end{center}
để thành một mệnh đề sai, hay có thể thêm ``tồn tại'':
\begin{center}
    ``Tồn tại $x$, $x$ là một thủ đô.''
\end{center}
để trở thành một mệnh đề đúng. Từ đó, bạn đọc có thể thấy hai kiểu lượng từ. \defText{Lượng từ phổ quát} có phạm vi ảnh hưởng toàn cục, được kí hiệu bởi $\defMath{\forall}$. Quay trở lại ví dụ, gọi vị từ ``là một thủ đô'' là $T()$, kí hiệu được mệnh đề là
\begin{equation*}
    \forall x, T(x)
\end{equation*}
hoặc
\begin{equation*}
    (\forall x)(T(x)). 
\end{equation*}
\defText{Lượng từ tồn tại} có phạm vi ảnh hưởng cụ thể, kí hiệu là $\defMath{\exists}$. Mệnh đề ``tồn tại'' được viết lại là:
\begin{center}
    $\exists x, T(x)$.
\end{center}
Có thể kết hợp nhiều lượng từ trong một mệnh đề:
\begin{equation*}
    \exists x, \forall y, x\text{ yêu }y.    
\end{equation*}

\subsection{Quy tắc xây dựng công thức}

\ 

Việc mở rộng thêm các bộ phận mới cho phép định nghĩa lại mệnh đề theo một cách chặt chẽ hơn. Nhưng trước khi đi đến với phương pháp xây dựng mệnh đề, chúng ta sẽ tìm hiểu một khái niệm yếu hơn --- \emph{công thức}.

Một \defText{công thức nguyên tử} là một vị từ đi kèm với đầy đủ những hạng thức cần thiết. Do vậy, ``$R(x)$'', ``$a + b = c$'', ``bông hoa $h$ có màu đỏ'' đều là các công thức nguyên tử. Từ đó, chúng ta có định nghĩa \defText{công thức} theo các quy tắc truy hồi như sau:
\begin{enumerate}
    \item Mọi công thức nguyên tử đều là công thức;
    \item Với mọi công thức $P$, $\neg (P)$ đều là công thức;
    \item Với mọi công thức $P$ và $Q$, $(P) \land (Q)$, $(P) \lor (Q)$, $(P) \implies (Q)$, $(P) \iff (Q)$, $(P) \uparrow (Q)$, $(P) \downarrow (Q)$, $(P) \impliedby (Q)$, $(P) \nRightarrow (Q)$, $(P) \nLeftarrow (Q)$, $(P) \nLeftrightarrow (Q)$, $(P) \oplus (Q)$, $(P) \odot (Q)$ đều là công thức;
    \item Với mọi công thức $P$ và hạng thức $x$, $\forall x, (P)$ và $\exists x, (P)$ đều là công thức;
    \item Những biểu thức khác không thể được lập từ sự kết hợp các quy tắc ở trên không là công thức.
\end{enumerate}
Định nghĩa này đã cho phép chuẩn hóa các cảm nhận mà chúng ta đã coi như là hiển nhiên từ trước\footnote{Định nghĩa này được viết theo kiểu định nghĩa hay được đưa ra ở trong nhiều tài liệu. Tuy nhiên, không có một nhóm các quy tắc nào có thể thực sự chặt chẽ định nghĩa thế nào là công thức, bởi vì cần phải có một tập hợp các vị từ và hạng thức sơ cấp đầy đủ phản ánh toàn bộ yêu cầu phản ánh ngôn ngữ, nhưng tập này là vô cùng. Hơn nữa, trong một ngôn ngữ lập luận thật sự chặt chẽ, chúng ta sẽ cần phải định nghĩa các phép toán mới như trong $x+y$.}. Và qua định nghĩa, chúng ta có thể phân biệt giữa những mệnh đề có nghĩa và những mệnh đề vô nghĩa như ``$X \land \impliedby Y Z\odot$''.

\subsection{Tự do hay phụ thuộc?}

\ 

Như quy tắc thực hiện thứ tự các phép nối mệnh đề, chúng ta có một số quy tắc để giúp việc giải giá trị chân lí của các mệnh đề trong giải tích vị từ dễ dàng hơn.

\exercise Phiên dịch các mệnh đề sau thành dưới dạng kí hiệu. Trình bày rõ các kí hiệu dành cho các hạng thức, vị từ và mệnh đề con nếu có.
\begin{enumerate}
    \item ``Tất cả những người hay khoe mẽ thì không đẹp.'';
    \item ``Một số sinh viên nam ở trường đại học Trăm khoa đồng tính nam.'';
    \item ``Có sinh viên năm nhất và năm tư đều phải học lại môn Giải tích.'';
    \item ``Không có việc gì khó, chỉ sợ lòng không bền.'';
    \item ``Đào núi và lấp biển, quyết chí ắt làm nên.''.
\end{enumerate}

\solution

Nhắc lại, cách hiểu của tác giả có thể khác với bạn đọc. Lời giải được đưa ra mang tính chất tham khảo.

\setcounter{subexercise}{1}
\arabic{subexercise}. Gọi $n$ kí hiệu cho hạng thức ``người''. Kí hiệu các vị từ:
\begin{itemize}
    \item $K()$: ``(người) hay khoe mẽ'',
    \item $D()$: ``(người) đẹp''.
\end{itemize}
Để phục vụ việc chuyển đổi sang dạng kí hiệu, viết lại câu thành
\begin{center}
    ``Với tất cả mọi người, nếu người đó hay khoe mẽ thì người đó không đẹp.''.
\end{center}
Kí hiệu hóa:
\begin{equation*}
    \forall n, (K(n) \implies \neg D(n)).
\end{equation*}

\stepcounter{subexercise}
\arabic{subexercise}. Gọi $l$ kí hiệu cho hạng thức ``sinh viên nam ở trường đại học Trăm khoa''. Chỉ có một vị từ là $N()$: ``(\dots) đồng tính nam''.

Cụm từ ``một số'' tương đương với từ ``tồn tại'', do vậy, có kí hiệu hóa mệnh đề như sau:
\begin{equation*}
    \exists l, N(l).
\end{equation*}

\stepcounter{subexercise}
\arabic{subexercise}. Gọi $m$ và $b$ kí hiệu cho ``sinh viên năm nhất'' và ``sinh viên năm tư''. Vị từ ``phải học lại môn Giải tích'' kí hiệu là $G()$. Viết lại câu:
\begin{center}
    ``Tồn tại sinh viên năm nhất và tồn tại sinh viên năm tư sao cho sinh viên năm nhất và sinh viên năm tư này đều phải học lại môn Giải tích.''.
\end{center}
Kí hiệu hóa:
\begin{equation*}
    \exists m, \exists b, (G(m) \land G(b)).
\end{equation*}

\stepcounter{subexercise}
\arabic{subexercise}. Viết lại câu:
\begin{center}
    ``Nếu có lòng bền bỉ, thì không tồn tại bất kỳ việc gì là khó cả.''.
\end{center}
Kí hiệu:
\begin{itemize}
    \item Mệnh đề ``lòng bền bỉ'': $B$,
    \item Hạng thức ``việc'': $v$,
    \item Vị từ ``(việc) khó'': $K()$.
\end{itemize}
Kí hiệu hóa là
\begin{equation*}
    B \implies \neg \big(\exists v, K(v)\big).
\end{equation*}

\stepcounter{subexercise}
\arabic{subexercise}. Viết lại câu:
\begin{center}
    ``Với mọi việc, nếu quyết chí thì làm nên việc đó.''.
\end{center}
Kí hiệu:
\begin{itemize}
    \item Mệnh đề ``quyết chí'': $Q$,
    \item Hạng thức ``việc'': $v$,
    \item Vị từ ``làm nên (việc)'': $L()$.
\end{itemize}
Kí hiệu hóa mệnh đề ban đầu là
\begin{equation*}
    \forall v, (Q \implies L(v)).
\end{equation*}