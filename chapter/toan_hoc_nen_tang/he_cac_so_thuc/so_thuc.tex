\subsection{Số thực}

\ % Lùi đầu dòng

Các ngành toán học đều có nhiều khái niệm, định lí, chứng minh trên các vật thể khác nhau, nhưng tổng quát trong đấy vẫn có nhiều điểm chung. Một cái chung như vậy là việc sử dụng \defText{dấu bằng}, $\defMath{=}$, để biểu diễn quan hệ giống nhau. Ở trong khuôn khổ cuốn sách này, chúng ta sẽ hiểu một cách nôm na rằng hai vế sẽ bằng nhau khi và chỉ khi hai vế có giá trị bằng nhau.

Phần này đề cập các yếu tố đại số cơ bản của \defText{số thực}, cụ thể là những hệ thức mà trong đó số thực tương tác với một số hữu hạn các \defText{phép cộng} và \defText{phép nhân}. 

Gọi $\defMath{\mathbb{R}}$ là tập hợp số thực. Nếu $a, b, c$ đều thuộc $\mathbb{R}$, với phép cộng và phép nhân mang ý nghĩa thông thường, có:
\begin{itemize}
   \item $a + b$ và $a\times b$ (hay $a\cdot b$, $ab$) đều thuộc $\mathbb{R}$;
   \item $a+b=b+a$ và $ab=ba$ (\defText{tính giao hoán});
   \item $a+(b+c)=(a+b)+c$ và $a(bc)=(ab)c$ (\defText{tính kết hợp});
   \item $a(b+c)=ab+ac$ (\defText{tính phân phối});
   \item $a\times 1 = a$ (\defText{đơn vị});
   \item $a + 0 = a$ và $a\times 0 = 0$ (\defText{số không});
   \item $a + c = b + c \implies a = b$ (\defText{tính giản ước được});
   \item Nếu $c \neq 0$, $ac = bc \implies a = b$ (\defText{tính giản ước được}).
\end{itemize}

Mỗi $a$ chỉ tồn tại một \defText{số đối} $-a$ duy nhất sao cho $a + (-a) = 0$ và nếu $a\neq 0$, tồn tại một \defText{số nghịch đảo} $\frac{1}{a}$ duy nhất sao cho $a\times \frac{1}{a} = 1$. \defText{Phép trừ} được định nghĩa là $$\defMath{a-b = a + (-b)}$$ và \defText{phép chia} được định nghĩa là $$\defMath{\frac{a}{b} = a\times \frac{1}{b}}.$$ Trên tập số thực, không có nghịch đảo của $0$.

