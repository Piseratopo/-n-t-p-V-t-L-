\subsection{Khai căn}

\ % Lùi đầu dòng

Nhắc tới số vô tỉ $\sqrt{2}$, chúng ta cần phải đề cập tới \defText{phép khai căn}. Có câu nói rằng ngược của phép cộng là phép trừ, ngược của phép nhân là phép chia, ngược của phép lũy thừa là phép khai căn. 

Cho $x$ là một số thực không âm, và $n$ là một số nguyên dương. Chúng ta sẽ thống nhất với nhau rằng tồn tại duy nhất một số thực không âm $y$ thỏa mãn $y^n = x$. Từ đây, chúng ta có định nghĩa phép khai căn như sau:
$$\defMath{ \sqrt[n]{x} = y \iff y^n = x\qquad \left(y \geq 0\right)}.$$
$n$ được gọi là \defText{bậc} của phép khai căn. Nếu $n = 2$, người ta thường viết tắt $\defMath{\sqrt{x}}$ thay vì $\defMath{\sqrt[2]{x}}$.

Với $m,n\in \mathbb{Z}_+$ và $x, y \in \mathbb{R}_+ \cup \{0\}$ bất kì, có những tính chất như sau:
$$
\begin{array}{rcl}
   \defMath{\sqrt[m]{\sqrt[n]{x}}} & \defMath{=} & \defMath{\sqrt[mn]{x}}; \\ 
   \defMath{\sqrt[n]{xy}} & \defMath{=} & \defMath{\sqrt[n]{x} \sqrt[n]{y}}; \\ 
   \defMath{\sqrt[n]{\frac{x}{y}}} & \defMath{=} & \defMath{\frac{\sqrt[n]{x}}{\sqrt[n]{y}}\qquad\left(y\neq 0\right)}.
\end{array}
$$



Giống như với số thực, chúng ta sẽ khẳng định tính chính xác của phép khai căn này nếu có cơ hội.