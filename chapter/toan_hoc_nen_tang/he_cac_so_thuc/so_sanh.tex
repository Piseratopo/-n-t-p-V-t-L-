\subsection{So sánh các số thực}

\ % Lùi đầu dòng

Nếu chúng ta muốn biểu diễn $a$ không bằng $b$, chúng ta có thể kí hiệu $\defMath{a \neq b}$.

Trong $\mathbb{R}$ cũng tồn tại quan hệ thứ tự toàn phần $\defMath{\leq}$, \defText{nhỏ hơn hoặc bằng}\footnote{Còn những kí hiệu khác cho dấu nhỏ hơn hoặc bằng là $\leqq$, $\leqslant$.}, thỏa mãn các tính chất sau:
\begin{itemize}
   \item $\leq$ có tính bắc cầu: với mọi $a; b; c \in \mathbb{R}$, nếu $a \leq b$ và $b \leq c$ thì $a \leq c$;
   \item $\leq$ có tính phản xạ: với mọi $a \in \mathbb{R}$ thì $a \leq a$;
   \item $\le $ có tính phản đối xứng: với mọi $a; b \in \mathbb{R}$, nếu $\begin{cases}
      a \leq b\\
      b \leq a
   \end{cases}$ thì $a = b$;
   \item $\leq$ có tính toàn phần: với mọi $a; b \in \mathbb{R}$ thì $a \leq b$ hoặc $b \leq a$.
\end{itemize}

Từ $\le$, chúng ta có thể định nghĩa các kí hiệu so sánh khác, bao gồm $\defMath{<}$ (\defText{nhỏ hơn}), $\defMath{\leq}$ (\defText{nhỏ hơn hoặc bằng})\footnote{Còn những kí hiệu khác cho dấu lơn hơn hoặc bằng là $\geqq$, $\geqslant$.}, $\defMath{>}$ (\defText{lớn hơn})\footnote{Ngoài những dấu được kể, còn những dấu mang tính chất so sánh như $\nless$ (không nhỏ hơn), $\ngtr$ (không lớn hơn), $\nleq$, $\not \leqq$ hay $\nleqslant$ (không nhỏ hơn hoặc bằng), $\ngeq$, $\not \geqq$ hay $\ngeqslant$ (không lớn hơn hoặc bằng), và những dấu bị nguyền rủa $\lessgtr$ (nhỏ hơn hoặc lớn hơn), $\lesseqgtr$ hay $\lesseqqgtr$ (nhỏ hơn, lớn hơn hoặc bằng).} như sau:
\begin{itemize}
   \item $\defMath{a < b \iff} \begin{cases}
      \defMath{a \leq b}\\
      \defMath{a \neq b}
   \end{cases}$;
   \item $\defMath{a \geq b \iff b \leq a}$;
   \item $\defMath{a > b \iff b < a}$.
\end{itemize}