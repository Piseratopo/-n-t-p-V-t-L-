\subsection{So sánh các số thực}

\ % Lùi đầu dòng

Nếu chúng ta muốn biểu diễn $a$ không bằng $b$, chúng ta có thể kí hiệu $\defMath{a \neq b}$.

Trong $\mathbb{R}$ cũng tồn tại quan hệ thứ tự toàn phần $\defMath{\leq}$, \defText{nhỏ hơn hoặc bằng}\footnote{Còn những kí hiệu khác cho dấu nhỏ hơn hoặc bằng là $\leqq$, $\leqslant$.}, thỏa mãn các tính chất sau:
\begin{itemize}
   \item $\leq$ có tính bắc cầu: với mọi $a; b; c \in \mathbb{R}$, nếu $a \leq b$ và $b \leq c$ thì $a \leq c$;
   \item $\leq$ có tính phản xạ: với mọi $a \in \mathbb{R}$ thì $a \leq a$;
   \item $\le $ có tính phản đối xứng: với mọi $a; b \in \mathbb{R}$, nếu $\begin{cases}
      a \leq b\\
      b \leq a
   \end{cases}$ thì $a = b$;
   \item $\leq$ có tính toàn phần: với mọi $a; b \in \mathbb{R}$ thì $a \leq b$ hoặc $b \leq a$.
\end{itemize}

Từ $\le$, chúng ta có thể định nghĩa các kí hiệu so sánh khác, bao gồm $\defMath{<}$ (\defText{nhỏ hơn}), $\defMath{\leq}$ (\defText{nhỏ hơn hoặc bằng})\footnote{Còn những kí hiệu khác cho dấu lơn hơn hoặc bằng là $\geqq$, $\geqslant$.}, $\defMath{>}$ (\defText{lớn hơn})\footnote{Ngoài những dấu được kể, còn những dấu mang tính chất so sánh như $\nless$ (không nhỏ hơn), $\ngtr$ (không lớn hơn), $\nleq$, $\not \leqq$ hay $\nleqslant$ (không nhỏ hơn hoặc bằng), $\ngeq$, $\not \geqq$ hay $\ngeqslant$ (không lớn hơn hoặc bằng), và những dấu bị nguyền rủa $\lessgtr$ (nhỏ hơn hoặc lớn hơn), $\lesseqgtr$ hay $\lesseqqgtr$ (nhỏ hơn, lớn hơn hoặc bằng).} như sau:
\begin{itemize}
   \item $\defMath{a < b \iff} \begin{cases}
      \defMath{a \leq b}\\
      \defMath{a \neq b}
   \end{cases}$;
   \item $\defMath{a \geq b \iff b \leq a}$;
   \item $\defMath{a > b \iff b < a}$.
\end{itemize}

Ngoài sự so sánh đơn lẻ, chúng ta còn có thể so sánh một cách toàn cục. Với $A$ là một tập con của $\mathbb{R}$ và $x$ là một phần tử trong $\mathbb{R}$, định nghĩa:
\begin{itemize}
   \item $x$ là \defText{chặn trên} của $A$ nếu $y \leq x$ với mọi $y$ trong $A$;
   \item $x$ là \defText{chặn dưới} của $A$ nếu $x \leq y$ với mọi $y$ trong $A$.
\end{itemize}
Trong cả hai định nghĩa đó, nếu $x$ thuộc $A$ thì $x$ sẽ là \defText{phần tử lớn nhất} và \defText{phần tử nhỏ nhất} tương ứng. Chúng ta có thể kí hiệu phần tử lớn nhất là $\defMath{\ptln{(A)}}$, $\defMath{\Max{(A)}}$ hay $\defMath{\max{(A)}}$ và phần tử nhỏ nhất là $\defMath{\ptnn{(A)}}$, $\defMath{\min{(A)}}$ và $\defMath{\Min{(A)}}$. Có thể nhận thấy rằng nếu $A$ có phần tử lớn nhất thì nó chỉ có đúng một phần tử như vậy. Thực vậy, nếu $x$ và $y$ cùng là phần tử lớn nhất của $A$, thì theo định nghĩa phần tử lớn nhất của $x$ và $y \in A$ nên $y \leq x$. Tương tự, có $x \leq y$. Theo tính phản xạ của $\le$, suy ra được $x = y$.

Các số trong $\mathbb{R}$ làm chặn trên của $A$ tạo thành một tập hợp. Nếu tập hợp đó có phần tử nhỏ nhất thì phần tử đó có tên là \defText{biên trên} (hay \defText{chặn trên đúng}) và được kí hiệu là $\defMath{\sup{(A)}}$, $\defMath{\Sup{(A)}}$ hoặc $\defMath{\sup_{\mathbb{R}}{(A)}}$, $\defMath{\Sup_{\mathbb{R}}{(A)}}$. Tương tự, số lớn nhất trong các chặn dưới của $A$ thì gọi là \defText{biên dưới}, hay \defText{chặn dưới đúng} và kí hiệu là $\defMath{\inf{(A)}}$, $\defMath{\Inf{(A)}}$ hoặc $\defMath{\inf_{\mathbb{R}}{(A)}}$, $\defMath{\Inf_{\mathbb{R}}{(A)}}$\footnote{Phần lớn các kí hiệu hàm trong toán không phân biệt hoa thường. Tuy nhiên, có một vài tài liệu có sự phân biệt trong một số hàm đặc biệt như $\arg$ và $\Arg$. Cho nên, đừng lười và đừng lẫn lộn hoa thường!}.

Những phép so sánh không bằng cũng có những tính chất đại số như những phép bằng. Cụ thể, với $x, y, z$ là các số thực,
\begin{itemize}
   \item $\defMath{x \le y \iff x + z \le y + z}$;
   \item $\begin{cases}
      \defMath{x \le y }\\
      \defMath{z \ge 0}
   \end{cases} \defMath{\iff x\cdot z \le y\cdot z}$.
\end{itemize}

\exercise Cho $w, x, y, z \in \mathbb{R}$ thỏa mãn $\begin{cases}
   w \le y \\
   x \le z
\end{cases}$. Chứng minh rằng $w + x \le y + z$.

\solution

Có $w \le y$ nên $w + x \le y + x$. Tương tự, cũng có $x \le z \iff x + y \le z + y$. Theo tính chất bắc cầu, chúng ta có:
$$w + x \le x + y \le y + z.$$
Qua đó, chúng ta có điều phải chứng minh.

\exercise Cho $w, x, y, z \in \mathbb{R}$ thỏa mãn $\begin{cases}
   0 \le w \le y \\
   0\le x \le z
\end{cases}$. Chứng minh rằng $w \cdot x \le y\cdot z$.

\solution 

Tương tự như bài trước, có 
$$\begin{cases}
   \begin{cases}
      w \le y \\
      x \geq 0
   \end{cases} \\
   \begin{cases}
      x \le z \\
      w \geq 0
   \end{cases}
\end{cases}\implies \begin{cases}
   w\cdot x \leq y\cdot x \\
   x\cdot y \leq z\cdot y
\end{cases} \implies w\cdot x \le y\cdot z.$$ Chúng ta có điều phải chứng minh.

\exercise Cho $w, x, y, z \in \mathbb{R}$ thỏa mãn $\begin{cases}
   w \le y \\
   x < z
\end{cases}$. Chứng minh rằng $w + x < y + z$.

\solution

Có $x < z \implies x \le z$, cho nên $w + x \le y + z$.

Giả sử $w + x = y + z$. Có $w \le y$ nên $w + x \le x + y$. Từ đây, suy ra được $y + z \le x + y \iff z \le x$. 

Tuy nhiên, nếu kết hợp kết luận này với điều kiện đã có $x < z$ hay $x \le z$ và $x \ne z$ thì sẽ xảy ra mâu thuẫn. Do đó, điều giả sử là sai. Theo chứng minh phản chứng, có $w + x < y + z$, điều phải chứng minh.

\exercise Cho $x, y$ là số thực và $z$ là một số thực âm. Chứng minh nếu $x \leq y$ thì $x\cdot z \geq y\cdot z$.

\solution

Do $z$ là một số thực âm, cho nên:
\begin{align}
   z &< 0 \nonumber\\
   \implies z &\leq 0 \nonumber\\
   \iff z + (-z) &\leq -z \nonumber\\
   \iff 0 &\leq -z. \label{eq:toan_hoc_nen_tang:he_cac_so_thuc:so_sanh:-z}
\end{align}

Giả sử $x \leq y$. Với $-z$ là số không âm theo \refeq{eq:toan_hoc_nen_tang:he_cac_so_thuc:so_sanh:-z}, nhân cả hai vế của $x \leq y$ với $-z$, chúng ta có:
\begin{align*}
   x\cdot (-z) &\leq y\cdot(-z) \\
   \implies -x \cdot z &\leq -y\cdot z \\
   \iff -x \cdot z + x\cdot z + y \cdot z &\leq -y\cdot z + \cdot z + y \cdot z \\
   \iff y \cdot z &\leq x \cdot z.
\end{align*}

Chúng ta có điều phải chứng minh.
