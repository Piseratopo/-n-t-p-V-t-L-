\subsection{Số nguyên và số hữu tỉ}

\ % Lùi đầu dòng

Từ tập số tự nhiên, chúng ta có thể xây dựng tập \defText{số nguyên} bằng việc kết hợp các dạng số, $n$ và $-n$ với $n$ là số tự nhiên nào đó. Kí hiệu tập số nguyên là $\defMath{\mathbb{Z}}$. Trong một vài trường hợp, chúng ta sẽ chỉ quan tâm đến só dương, khi này, có tập số nguyên dương $\defMath{\mathbb{Z}^+}$ hay $\defMath{\mathbb{N}^*}$.

Mở rộng tập số nguyên, các số có dạng $\frac{a}{b}$ với $a$ là số nguyên và $b$ là số nguyên khác $0$ tạo thành tập \defText{số hữu tỉ} kí hiệu là $\defMath{\mathbb{Q}}$.

Để xây dựng số thực thì sẽ cần những khái niệm cao cấp hơn. Một số thực có thể được định nghĩa là giới hạn của một dãy số hữu tỉ. Các số thực không phải số hữu tỉ thì là \defText{số vô tỉ}. Tuy rằng hiện tại chúng ta chưa đề cập đến định nghĩa toán học chặt chẽ của số thực, nhưng khả năng cao là bạn đọc đã có làm quen với nhiều số thực như $\sqrt{2}$ hay $\pi$. Do đó, tác giả sẽ thừa nhận các tính chất của số thực, và sẽ xây dựng lại định nghĩa khi điều kiện cho phép.