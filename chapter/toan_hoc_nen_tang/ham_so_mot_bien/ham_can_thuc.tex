\subsection{Hàm căn thức}

\ % Lùi đầu dòng

\defText{Hàm căn thức} là một trong những hàm số cơ bản trong toán học, được định nghĩa dựa trên phép căn thức của số thực như sau:
$$\defMath{f(x) = \sqrt[n]{x}}$$
trong đó $x \in \mathbb{R}^+ \cup \{0\}$ và $n \in \mathbb{Z}^+$. Nếu $n = 2k + 1 (k \in \mathbb{N})$ là số lẻ, thì chúng ta có mở rộng của hàm căn thức trên toàn bộ tập số thực:
$$
\defMath{f(x) = }\begin{cases}
   \defMath{\sqrt[2k+1]{x} \defText{ nếu } x \geq 0} \\
   \defMath{-\sqrt[2k+1]{-x} \defText{ nếu } x < 0}
\end{cases}.
$$
Tối giản hóa định nghĩa trên, có thể viết $f(x) = \sqrt[2k+1]{x}$ trên toàn bộ $x$ thực.

Khi hợp hai hàm số $f \circ g$ mà $f$ là hàm căn thức, có $f\circ g(x) = \sqrt[n]{g(x)}$. Khi này, $g(x)$ có thể được gọi là \defText{biểu thức dưới dấu căn} hay \defText{biểu thức lấy căn}.

\exercise Giải các phương trình sau với ẩn $x$ thực

\begin{multicols}{2}
   \begin{enumerate}
      \item $\sqrt{x} = 2$;
      \item $\sqrt{x - 2} = -2$;
      \item $\sqrt[3]{x^5 + 1} = -2$;
      \item $\sqrt[4]{x^4 - 2x^2 + 8} = -x$;
      \item $\sqrt{x^3-3x+1} = \sqrt{x^3+2x-6}$;
      \item $\sqrt[4]{x^4 + 1} = \sqrt[4]{x^4-3x + 1}$;
      \item $\sqrt{x+3} = \sqrt[3]{5x+3}$;
      \item $2\sqrt{x^2 - 9} = (x + 5)\sqrt{\frac{x+3}{x-3}}$;
      \item $\sqrt{x + 4} + \sqrt{x + 9} = 5$;
      \item $\sqrt{x(x+1)} + \sqrt{x(x+2)} = \sqrt{x(x+4)}$.
   \end{enumerate}
\end{multicols}

\solution

\setcounter{subexercise}{1}
\arabic{subexercise}. Tập xác định của phương trình là $\left[0; \infty\right)$. Theo định nghĩa của phép khai căn:
\begin{align*}
   \sqrt{x} &= 2 \\
   \iff x &= 2^2 = 4.
\end{align*}

Vậy $x = 4$ là nghiệm duy nhất của phương trình.

\stepcounter{subexercise}
\arabic{subexercise}. Theo định nghĩa của phép khai căn, với căn bậc chẵn, chúng ta có $\sqrt{x - 2} \geq 0$. Do đó, phương trình vô nghiệm.

\stepcounter{subexercise}
\arabic{subexercise}. Thực hiện biến đổi đại số:
\begin{align*}
   \sqrt[3]{x^5 + 1} &= -2 \\
   \iff x^5 + 1 &= (-2)^3 = -8 \\
   \iff x^5 &= -9 \\
   \iff x &= -\sqrt[5]{9}.
\end{align*}

Vậy tập nghiệm của phương trình là $\left\{-\sqrt[5]{9}\right\}$.

\stepcounter{subexercise}
\arabic{subexercise}.
\begin{align*}
   \sqrt[4]{x^4-2x^2+8} &= -x \\
   \implies x^4 - 2x^2 + 8 &= (-x)^4 = x^4 \\
   \implies 8 - 2x^2 &= 0 \\
   \implies x \in \{-2; 2\}.
\end{align*}

Kiểm tra trực tiếp, thấy $x = -2$ là nghiệm duy nhất thỏa mãn. Vậy tập nghiệm của phương trình là $\{2\}$.

\stepcounter{subexercise}
\arabic{subexercise}.
\begin{align*}
   \sqrt{x^3-3x+1} &= \sqrt{x^3+2x-6} \\
   \implies x^3-3x+1 &= x^3 + 2x - 6 \\
   \iff x &= \frac{7}{5}.
\end{align*}

