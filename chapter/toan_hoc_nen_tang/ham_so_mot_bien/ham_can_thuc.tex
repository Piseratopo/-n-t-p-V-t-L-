\subsection{Hàm căn thức}

\ % Lùi đầu dòng

\defText{Hàm căn thức} là một trong những hàm số cơ bản trong toán học, được định nghĩa dựa trên phép căn thức của số thực như sau:
$$\defMath{f(x) = \sqrt[n]{x}}$$
trong đó $x \in \mathbb{R}^+ \cup \{0\}$ và $n \in \mathbb{Z}^+$. Nếu $n = 2k + 1 (k \in \mathbb{N})$ là số lẻ, thì chúng ta có mở rộng của hàm căn thức trên toàn bộ tập số thực:
$$
\defMath{f(x) = }\begin{cases}
   \defMath{\sqrt[2k+1]{x} \defText{ nếu } x \geq 0} \\
   \defMath{-\sqrt[2k+1]{-x} \defText{ nếu } x < 0}
\end{cases}.
$$
Tối giản hóa định nghĩa trên, có thể viết $f(x) = \sqrt[2k+1]{x}$ trên toàn bộ $x$ thực.

Khi hợp hai hàm số $f \circ g$ mà $f$ là hàm căn thức, có $f\circ g(x) = \sqrt[n]{g(x)}$. Khi này, $g(x)$ có thể được gọi là \defText{biểu thức dưới dấu căn} hay \defText{biểu thức lấy căn}.

\exercise Giải các phương trình sau với ẩn $x$ thực

\begin{multicols}{2}
   \begin{enumerate}
      \item $\sqrt{x} = 2$;
      \item $\sqrt{x - 2} = -2$;
      \item $\sqrt[3]{x^5 + 1} = -2$;
      \item $\sqrt[4]{x^4 - 2x^2 + 8} = -x$;
      \item $\sqrt{x^3-3x+1} = \sqrt{x^3+2x-6}$;
      \item $\sqrt[4]{x^4 + 1} = \sqrt[4]{x^4-3x + 1}$;
      \item $2\sqrt{x^2 - 9} = (x + 5)\sqrt{\frac{x+3}{x-3}}$;
      \item $\sqrt{x + 4} + \sqrt{x + 9} = 5$;
      \item $\sqrt{x(x+1)} + \sqrt{x(x+2)} = \sqrt{x(x-3)}$;
      \item $\sqrt{x+3} = \sqrt[3]{5x+3}$;.
   \end{enumerate}
\end{multicols}

\solution

\setcounter{subexercise}{1}
\arabic{subexercise}. Tập xác định của phương trình là $\left[0; \infty\right)$. Theo định nghĩa của phép khai căn:
\begin{align*}
   \sqrt{x} &= 2 \\
   \iff x &= 2^2 = 4.
\end{align*}

Vậy $x = 4$ là nghiệm duy nhất của phương trình.

\stepcounter{subexercise}
\arabic{subexercise}. Theo định nghĩa của phép khai căn, với căn bậc chẵn, chúng ta có $\sqrt{x - 2} \geq 0$. Do đó, phương trình vô nghiệm.

\stepcounter{subexercise}
\arabic{subexercise}. Thực hiện biến đổi đại số:
\begin{align*}
   \sqrt[3]{x^5 + 1} &= -2 \\
   \iff x^5 + 1 &= (-2)^3 = -8 \\
   \iff x^5 &= -9 \\
   \iff x &= -\sqrt[5]{9}.
\end{align*}

Vậy tập nghiệm của phương trình là $\left\{-\sqrt[5]{9}\right\}$.

\stepcounter{subexercise}
\arabic{subexercise}.
\begin{align*}
   \sqrt[4]{x^4-2x^2+8} &= -x \\
   \implies x^4 - 2x^2 + 8 &= (-x)^4 = x^4 \\
   \implies 8 - 2x^2 &= 0 \\
   \implies x \in \{-2; 2\}.
\end{align*}

Kiểm tra trực tiếp, thấy $x = -2$ là nghiệm duy nhất thỏa mãn (có $x^4-2x^2+8 = 16$ trong cả hai trường hợp của nghiệm, cho nên $\sqrt[4]{x^4-2x^2+8} = 2 = -(-2)$). Vậy tập nghiệm của phương trình là $\{2\}$.

\stepcounter{subexercise}
\arabic{subexercise}.
\begin{align*}
   \sqrt{x^3-3x+1} &= \sqrt{x^3+2x-6} \\
   \implies x^3-3x+1 &= x^3 + 2x - 6 \\
   \iff x &= \frac{7}{5}.
\end{align*}
Tuy nhiên, khi kiểm tra $x = \frac{7}{5}$ thì $x^3 - 3x + 1 = -\frac{57}{125}$ là một số âm, không thỏa mãn điều kiện xác định của $\sqrt{x^3-3x+1}$. Qua đó, chúng ta có tập nghiệm của phương trình là $\emptyset$.

\stepcounter{subexercise}
\arabic{subexercise}.
\begin{align*}
   \sqrt[4]{x^4 + 1} &= \sqrt[4]{x^4 - 3x + 1} \\
   \implies x^4 + 1 &= x^4 - 3x + 1 \\
   \iff x &= 0. 
\end{align*}

Kiểm tra lại, thấy cả hai vế đều bằng $1$ khi $x = 0$. Cho nên tập nghiệm của phương trình là $\{0\}$.

\stepcounter{subexercise}
\arabic{subexercise}. Giống như nhiều bài tập trước đó, bình phương lên hai vế để khử căn để được
\begin{align*}
   \left(2\sqrt{x^2 - 9}\right)^2 &= \left((x + 5)\sqrt{\frac{x+3}{x-3}}\right)^2 \\ 
   \implies 4\left(x^2-9\right) &= (x + 5)^2\frac{x+3}{x-3}\\
   \implies 4\left(x^2-9\right)(x - 3) &= (x + 5)^2(x + 3) \\
   \implies 4x^3 - 12x^2 - 36x + 108 &= x^3 + 13x^2 + 55x + 75 \\
   \iff 3x^3 - 25x^2 - 91x + 33 &= 0 \\
   \iff (x - 11)(x + 3)(3x - 1) &= 0 \\
   \iff x &\in \left\{11; -3; \frac{1}{3}\right\}.
\end{align*}

Thử lại, chúng ta kết luận được các nghiệm của phương trình là $x \in \left\{11; -3\right\}$.

\stepcounter{subexercise}
\arabic{subexercise}. Nếu $x$ thỏa mãn phương trình $\sqrt{x + 4} + \sqrt{x + 9} = 5$, thì cần phải có $\begin{cases}
   x + 4 \geq 0 \\
   x + 9 \geq 0
\end{cases}$. Ngoài ra, có $\begin{cases}
   \sqrt{x + 4} \geq 0 \\
   \sqrt{x + 9} \geq 0
\end{cases}$ nên $\sqrt{x + 4} + \sqrt{x + 9} \geq 0$. Do đó, bình phương hai vế để có phương trình tương đương
\begin{align*}
   \left(\sqrt{x + 4} + \sqrt{x + 9}\right)^2 &= 5^2 \\
   \iff x + 4 + 2\sqrt{x + 4}\sqrt{x + 9} + x + 9 &= 25 \\
   \iff 2\sqrt{(x + 4)(x + 9)} &= 12 \equationexplanation{\parbox{0.4\textwidth}{$\sqrt{x + 4}\sqrt{x + 9} = \sqrt{(x + 4)(x + 9)}$ do cả hai biểu thức dưới căn đều không âm.}} \\
   \displaybreak[2]
   \iff \sqrt{(x + 4)(x + 9)} &= 6 \\
   \iff (x + 4)(x + 9) &= 6^2 = 36 \\
   \iff x^2 + 13x + 36 &= 0 \\
   \iff x(x+13) &= 0 \\
   \iff x &= 0 \equationexplanation{$x+13 > x + 4 \geq 0$.}.
\end{align*}

Vậy tập nghiệm của phương trình là $\{0\}$.

\stepcounter{subexercise}
\arabic{subexercise}.
\begin{align*}
   \sqrt{x(x+1)} + \sqrt{x(x+2)} &= \sqrt{x(x-3)} \\
   \implies x(x + 1) + 2\sqrt{x(x+1)}\sqrt{x(x+2)} + x(x+2) &= x(x-3) \equationexplanation{Bình phương hai vế.}\\
   \iff 2\sqrt{x^2(x + 1)(x + 2)} &= -x^2 - 6x \\
   \displaybreak[2]
   \implies 4x^2(x + 1)(x + 2) &= (-x^2 - 6x)^2 \\
   \iff 4x^4 + 12x^3 + 8x^2 &= x^4 + 12x^3 + 36x^2 \\
   \iff 3x^4 - 28x^2 &= 0 \\
   \iff x^2(3x^2 - 28) &= 0.
\end{align*}

Ngoài nghiệm hiển nhiên $x = 0$, chúng ta cũng có thể có nghiệm $x$ thỏa mãn $3x^2 - 28 = 0$. Giải phương trình bậc hai này cho $x \in \left\{\frac{-2\sqrt{21}}{3};\frac{2\sqrt{21}}{3}\right\}$. Tuy nhiên, sau kiểm tra, chúng ta chỉ còn các nghiệm $0$ và $\frac{-2\sqrt{21}}{3}$. Vậy, tập nghiệm của phương trình là $\left\{0; \frac{-2\sqrt{21}}{3}\right\}$. 

\stepcounter{subexercise}
\arabic{subexercise}. Đặt $\begin{cases}
y = \sqrt{x + 3} \\ 
z = \sqrt[3]{5x + 3}
\end{cases}$. Có ngay $y = z$ theo phương trình đã cho. Ngoài ra, viết lại $x$ theo $y$ và $z$ để có $$
   x = y^2 - 3 = \frac{z^3 - 3}{5}.
$$ Từ đây, có 
\begin{align*}
   y^2 - 3 &= \frac{y^3 - 3}{5} \\
   0 &= y^3 - 5y^2 + 12 \\
   0 &= (y - 2)(y^2 - 3y - 6).
\end{align*}
Giải phương trình cho các giá trị của $y$: $y \in \left\{2; \frac{3 - \sqrt{33}}{2}; \frac{3 + \sqrt{33}}{2}\right\}$. Để ý rằng $y \geq 0$ theo định nghĩa của căn bậc hai, cho nên $y\in \left\{2; \frac{3 + \sqrt{33}}{2}\right\}$. Kết hợp với $x = y^2 - 3$ để có $x \in \left\{1; \frac{15+3\sqrt{33}}{2}\right\}$. Thử lại để thấy cả hai nghiệm đều thỏa mãn.