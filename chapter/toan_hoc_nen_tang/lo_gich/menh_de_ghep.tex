\section{Mệnh đề ghép và các phép nối mệnh đề}

\ % Lùi đầu dòng

Hãy xem xét câu sau:
\begin{center}
  ''Hôm nay trời mua \emph{và} hội thao đã phải lùi lịch.''.
\end{center}
Đây rõ ràng là một mệnh đề do chúng ta có thể dễ dàng xác định tính đúng sai của nó. Câu hỏi quan trọng hơn là chúng ta xác định tính chính xác của câu này như thế nào. Một cách tự nhiên, chúng ta sẽ xem xét từng phần ``hôm nay trời mưa'' và ``hội thao đã phải lùi lịch''. Từ tính đúng sai của hai vế, tính đúng sai của mệnh đề ban đầu được xác định. Đây là một ví dụ của \defText{mệnh đề phức}, một mệnh đề được cấu tạo từ những mệnh đề con và các \defText{phép nối mệnh đề}.

Chỉ có $5$ phép nối mệnh đề với tên tương ứng như sau:
\begin{itemize}
  \item $\defMath{\neg}$: phép \defText{đối},
  \item $\defMath{\land}$: phép \defText{và} hoặc phép \defText{hội},
  \item $\defMath{\lor}$: phép \defText{hoặc} hoặc phép \defText{tuyển},
  \item $\defMath{\implies}$: \defText{kéo theo},
  \item $\defMath{\iff}$: phép \defText{tương đương}.
\end{itemize}
Kết hợp với chúng là hai dấu ngoặc, ngoặc đơn đóng $)$ và ngoặc đơn mở $($, để xác định thứ tự giải giá trị lô-gích của mệnh đề phức.

\subsection{Phép đối}

\ %

Thông thường, để phủ định một câu khẳng định, chúng ta hay dùng từ ``không'' hay những từ gần nghĩa như ``chưa'' hay ``chẳng''. Ví dụ, có thể phủ định câu ``Cơm hôm nay ngon.'' thành ``Cơm hôm nay \emph{không} ngon.''. Tuy nhiên, với những câu phức tạp hơn như ví dụ về hội thao ở trước đó thì việc thêm các chữ ``không'' như
\begin{center}
  ``Hôm nay trời \emph{không} mưa \emph{\textcolor{colorEmphasis}{và}} hội thao đã \emph{không} phải lùi lịch.''
\end{center}
là không thỏa đáng. Cách viết đúng sẽ khá dài dòng:
\begin{center}
  ``\emph{Không phải trường hợp rằng} hôm nay trời mưa và hội thao đã phải lùi lịch.''.
\end{center}
Sử dụng kí hiệu thì sẽ dễ dàng hơn. Kiểu như:
\begin{center}
  ``$\neg \left(\text{cơm hôm nay ngon}\right)$''
\end{center}
hay
\begin{center}
  ``$\neg \left(\text{hôm nay trời mưa và hội thao đã phải lùi lịch}\right)$''.
\end{center}
Nhìn chung, nếu $P$ là một mệnh đề thì phủ định của nó sẽ là $\neg P$.

\subsection{Bảng giá trị chân lí}

\ %

Trước khi đi đến những phép nối phức tạp hơn, chúng ta sẽ đề cập đến khái niệm bảng giá trị chân lí