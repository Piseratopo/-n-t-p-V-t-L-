\section{Mệnh đề ghép và các phép nối mệnh đề}

\ % Lùi đầu dòng

Hãy xem xét câu sau:
\begin{center}
   ''Hôm nay trời mua \emph{và} hội thao đã phải lùi lịch.''.
\end{center}
Đây rõ ràng là một mệnh đề do chúng ta có thể dễ dàng xác định tính đúng sai của nó. Câu hỏi quan trọng hơn là chúng ta xác định tính chính xác của câu này như thế nào. Một cách tự nhiên, chúng ta sẽ xem xét từng phần ``hôm nay trời mưa'' và ``hội thao đã phải lùi lịch''. Từ tính đúng sai của hai vế, tính đúng sai của mệnh đề ban đầu được xác định. Đây là một ví dụ của \defText{mệnh đề phức hợp} (hay \defText{mệnh đề phức}), một mệnh đề được cấu tạo từ một hoặc một số \defText{mệnh đề thành phần} và các \defText{phép nối mệnh đề}.

Trong toán học, chúng ta hay sử dụng $6$ phép nối mệnh đề\footnote{``Tưởng là có $5$ phép nối mệnh đề thôi?''. Tác giả đã thêm phép nối mệnh đề $\impliedby$, trong trường hợp một vài bạn đọc đọc sách từ phải qua trái.} (lần đầu được đề xuất bởi Phrây-gơ\footnote{Friedrich Ludwig Gottlob Frege (1848 -- 1925)}):
\begin{itemize}
   \item Phép \defText{đối} --- $\defMath{\neg}$,
   \item Phép \defText{hội} hoặc phép \defText{và} --- $\defMath{\land}$ hoặc $\defMath{\&}$,
   \item Phép \defText{tuyển} hoặc phép \defText{hoặc} --- $\defMath{\lor}$,
   \item Phép \defText{kéo theo} --- $\defMath{\implies}$ hoặc $\defMath{\Rightarrow}$, $\defMath{\rightarrow}$,
   \item Phép \defText{hệ quả} --- $\defMath{\impliedby}$ hoặc $\defMath{\Leftarrow}$, $\defMath{\leftarrow}$,
   \item Phép \defText{tương đương} --- $\defMath{\iff}$ hoặc $\defMath{\Leftrightarrow}$, $\defMath{\leftrightarrow}$.
\end{itemize}
Kết hợp với chúng là hai dấu ngoặc, ngoặc đơn đóng --- $)$ --- và ngoặc đơn mở --- $($\footnote{Có thể dùng thêm ngoặc vuông --- $[]$ --- hay ngoặc nhọn --- $\{\}$ --- nếu cần tăng khả năng nhận diện của mệnh đề phức hợp.} --- để xác định thứ tự giải giá trị lô-gích của mệnh đề phức hợp.

\subsection{Phép đối}

\ %

Thông thường, để phủ định một câu khẳng định, chúng ta hay dùng từ ``không'' hay những từ gần nghĩa như ``chưa'' hay ``chẳng''. Ví dụ, có thể phủ định câu ``Cơm hôm nay ngon.'' thành ``Cơm hôm nay \emph{không} ngon.''. Tuy nhiên, với những câu phức tạp hơn như ví dụ về hội thao ở trước đó thì việc thêm các chữ ``không'' như
\begin{center}
   ``Hôm nay trời \emph{không} mưa \emph{\textcolor{colorEmphasis}{và}} hội thao đã \emph{không} phải lùi lịch.''
\end{center}
là không thỏa đáng. Cách viết đúng sẽ khá dài dòng:
\begin{center}
   ``\emph{Không phải trường hợp rằng} hôm nay trời mưa và hội thao đã phải lùi lịch.''.
\end{center}
Sử dụng kí hiệu thì sẽ dễ dàng hơn, tuy nhìn hơi kì, kiểu như:
\begin{center}
   ``$\neg \left(\text{cơm hôm nay ngon}\right)$''
\end{center}
hay
\begin{center}
   ``$\neg \left(\text{hôm nay trời mưa và hội thao đã phải lùi lịch}\right)$''.
\end{center}
Nhìn chung, nếu $P$ là một mệnh đề thì phủ định của nó sẽ có kí hiệu là $\defMath{\neg P}$ hoặc $\defMath{\overline{P}}$. Nếu $P$ đúng thì $\neg P$ sai và ngược lại, nếu $P$ sai thì $\neg P$ đúng.

\subsection{Bảng giá trị chân lí}

\ %

Trước khi đi đến những phép nối phức tạp hơn, chúng ta sẽ đề cập đến khái niệm bảng giá trị chân lí. Khi sử dụng các phép toán lô-gích để tạo ra mệnh đề phức hợp thì chúng ta cần phải xem xét các trường hợp có thể của các \defText{giá trị chân lí}, một cách nói văn hoa hơn cho cụm từ ``tính đúng sai'', của từng mệnh đề thành phần. Khi mà số mệnh đề thành phần lớn lên thì số lượng trường hợp cũng tăng theo theo cấp số nhân. Để tránh việc phải viết nhiều, \defText{bảng giá trị chân lí} đã được khai sinh\footnote{Đây có vẻ là một khái niệm đơn giản, bởi vì lập bảng là một thao tác đã được thực hiện thường xuyên xuyên suốt lịch sử loài người, tuy nhiên, không có quá nhiều tài liệu lịch sử nói về bảng giá trị chân lí. Tài liệu sớm nhất mà tác giả tìm được cho thấy sự sử dụng của kiểu bảng này xuất phát từ thế kỉ XIX\cite{anellis2012peirce}. Có thể, trường hợp thứ nhất, người xưa thấy việc viết (hay nói, biết chữ là một thứ xa xỉ) các mệnh đề lô-gích phức hợp là bình thường, hoặc, trường hợp thứ hai với khả năng xảy ra cao hơn, kiến thức lịch sử của tác giả còn hạn hẹp.}.

Chúng ta sẽ lấy ví dụ ngay trên phép nối mệnh đề chúng ta vừa được tiếp cận. Khi xây dựng bảng giá trị chân lí, tác giả sẽ viết tắt ``Đ'' và ``S'' lần lượt cho mệnh đề có giá trị chân lí đúng và sai.
\begin{table}[H]
   \centering
   \caption{Bảng giá trị chân lí của mệnh đề với phép đối}
   \begin{tabular}{|c!{\vrule width 1.5pt}c|}
      \hline 
      $P$ & $\neg P$ \\
      \headerDivider
      Đ & S \\
      S & Đ \\
      \hline
   \end{tabular}
\end{table}

\subsection{Phép hội}

\ %

Cho hai mệnh đề $P$ và $Q$. Mệnh đề phức ``$P$ và $Q$'' là \defText{hội} của $P$ và $Q$ và có kí hiệu là $\defMath{P \land Q}$. Mệnh đề này chỉ đúng khi cả hai mệnh đề thành phần $P$ và $Q$ đều đúng, thể hiện dưới dạng bảng giá trị chân lí \ref{tab:toan_hoc_nen_tang:lo_gich:menh_de_phuc_hop:dn_hoi}.
\begin{table}[H]
   \centering
   \caption{Bảng giá trị chân lí của mệnh đề với phép hội}
   \label{tab:toan_hoc_nen_tang:lo_gich:menh_de_phuc_hop:dn_hoi}
   \begin{tabular}{|c|c!{\vrule width 1.5pt}c|}
      \hline 
      $P$ & $Q$ & $P \land Q$ \\
      \headerDivider
      Đ & Đ & Đ \\
      Đ & S & S \\
      S & Đ & S \\
      S & S & S \\
      \hline
   \end{tabular}
\end{table}

\subsection{Phép tuyển}

\ % Lùi đầu dòng

Mệnh đề phức ``$P$ hoặc $Q$'' là \defText{tuyển} của hai mệnh đề $P$ và $Q$ và có kí hiệu $\defMath{P \lor Q}$. Mệnh đề này đúng khi tối thiểu một trong hai mệnh đề đầu vào đúng, và chỉ sai khi cả hai mệnh đề đầu vào đều sai. Bảng \ref{tab:toan_hoc_nen_tang:lo_gich:menh_de_phuc_hop:dn_tuyen} cho giá trị chân lí của mệnh đề tuyển.
\begin{table}[H]
   \centering
   \caption{Bảng giá trị chân lí của mệnh đề với phép tuyển}
   \label{tab:toan_hoc_nen_tang:lo_gich:menh_de_phuc_hop:dn_tuyen}
   \begin{tabular}{|c|c!{\vrule width 1.5pt}c|}
      \hline
      $P$ & $Q$ & $P \lor Q$ \\
      \headerDivider
      Đ & Đ & Đ \\
      Đ & S & Đ \\ 
      S & Đ & Đ \\
      S & S & S \\
      \hline
   \end{tabular}
\end{table}

Ý nghĩa của từ ``hoặc'' trong toán học hơi khác với ý nghĩa thông thường. Khi người ta nói ``hoặc'', người ta hay ám chỉ một trong số các trường hợp liệt kê ra là đúng. Trong lô-gích, cả hai mệnh đề đúng vẫn làm cho mệnh đề phức hợp đúng.

\subsection{Phép kéo theo và phép hệ quả}

\ % Lùi đầu dòng

Đây lại là một phép nối nữa mà ý nghĩa của nó (có thể) khác với ý nghĩa thông thường. Đây là phép cũng gây nhiều lỗi lập luận lô-gích nhất. Mệnh đề với phép \defText{kéo theo} $\defMath{P \implies Q}$ chỉ sai khi $P$ không suy ra $Q$, điều này tương đương, và chỉ tương đương, với có $P$ mà không có $Q$. Bạn đọc nên để ý kĩ hai dòng cuối cùng của bảng giá trị chân lí \ref{tab:toan_hoc_nen_tang:lo_gich:menh_de_phuc_hop:dn_keo_theo}.
\begin{table}[H]
   \centering
   \caption{Bảng giá trị chân lí của mệnh đề với phép kéo theo và phép hệ quả}
   \label{tab:toan_hoc_nen_tang:lo_gich:menh_de_phuc_hop:dn_keo_theo}
   \begin{tabular}{|c|c!{\vrule width 1.5pt}c!{\vrule width 1.5pt}c|}
      \hline
      $P$ & $Q$ & $P \implies Q$ & $Q \impliedby P$ \\
      \headerDivider
      Đ & Đ & Đ & Đ \\
      Đ & S & S & S \\
      \emphcolor{S} & \emphcolor{Đ} & \emphcolor{Đ} & \emphcolor{Đ} \\
      \emphcolor{S} & \emphcolor{S} & \emphcolor{Đ} & \emphcolor{Đ} \\
      \hline
   \end{tabular}
\end{table}

Bảng \ref{tab:toan_hoc_nen_tang:lo_gich:menh_de_phuc_hop:dn_keo_theo} cũng đề cập đến một phép lô-gích nữa, chính là phép \defText{hệ quả}. Khi nói ``$P$ suy ra $Q$'', cũng có thể nói ``$Q$ là hệ quả của $P$'' trong trường hợp đó. Bảng \ref{tab:toan_hoc_nen_tang:lo_gich:menh_de_phuc_hop:dn_he_qua} được đưa ra trong trường hợp bạn đọc muốn nhìn thấy phép hệ quả theo thứ tự bảng chữ cái.
\begin{table}[H]
   \centering
   \caption{Bảng giá trị chân lí của mệnh đề với phép hệ quả}
   \label{tab:toan_hoc_nen_tang:lo_gich:menh_de_phuc_hop:dn_he_qua}
   \begin{tabular}{|c|c!{\vrule width 1.5pt}c|}
      \hline
      $P$ & $Q$ & $P \impliedby Q$ \\
      \headerDivider
      Đ & Đ & Đ \\
      Đ & S & Đ \\
      S & Đ & S \\
      S & S & Đ \\
      \hline
   \end{tabular}
\end{table}

\subsection{Phép tương đương}

\ % Lùi đầu dòng

Thông thường, hai sự vật ``tương đương'' có nghĩa là hai sự vật có giá trị ngang nhau và có thể thay thế được cho nhau. Tương tự, trong lô-gích, hai mệnh đề \defText{tương đương} khi và chỉ khi cả hai luôn cùng đúng hoặc cùng sai.
\begin{table}[H]
   \centering
   \caption{Bảng giá trị chân lí của mệnh đề với phép tương đương}
   \label{tab:toan_hoc_nen_tang:lo_gich:menh_de_phuc_hop:dn_tuong_duong}
   \begin{tabular}{|c|c!{\vrule width 1.5pt}c|}
      \hline
      $P$ & $Q$ & $P \iff Q$ \\
      \headerDivider 
      Đ & Đ & Đ \\
      Đ & S & S \\
      S & Đ & S \\
      S & S & Đ \\
      \hline
   \end{tabular}
\end{table}

\subsection{Thứ tự giải giá trị chân lí của mệnh đề của các phép nối}

\ % Lùi đầu dòng

Giống như thứ tự các phép tính số học\footnote{[\dots], thứ mà con người phát minh ra [\dots]}để tính giá trị các biểu thức số học, để xác định giá trị chân lí của mệnh đề có nhiều phép nối, các phép nối mệnh đề cũng có sắp xếp thứ tự\footnote{[\dots]và vẫn chưa có sự thống nhất[\dots]} từ mức ưu tiên cao đến ưu tiên thấp như sau:
\begin{enumerate}
   \item Phép đối --- $\neg$ (mức ưu tiên cao nhất);
   \item Phép và --- $\land$;
   \item Phép hoặc --- $\lor$;
   \item Phép kéo theo --- $\implies$, phép hệ quả --- $\impliedby$ --- và phép tương đương --- $\iff$ (mức ưu tiên thấp nhất).
\end{enumerate}
Với các phép ở trên cùng một mức, có thể quy ước xử lí theo nhiều cách khác nhau: thứ tự xuất hiện (từ trái qua phải), kết hợp tính toán từ trái qua phải với thứ tự phép kéo theo đi trước phép tương đương, hoặc yêu cầu sử dụng các dấu ngoặc để sắp xếp thứ tự. Trong cuốn sách này, tác giả sẽ sử dụng quy ước \emph{từ trái qua phải} là chủ yếu, kèm với việc sử dấu ngoặc để giúp tăng khả năng đọc khi cần thiết.

\exercise[ex:toan_hoc_nen_tang:lo_gich:menh_de_phuc_hop:vd_bang_gia_tri_chan_li] Xây dựng bảng giá trị chân lí của các mệnh đề sau:
\begin{multicols}{2}
   \begin{enumerate}
      \item $\neg P \implies Q$;
      \item $(P \iff Q) \lor \neg Q$;
      \item $P \land (Q \implies R)$;
      \item $P \land Q \lor Q \land \neg R$;
      \item $Q \implies R \land R \lor \neg P$;
      \item $(P \implies Q) \implies (R \implies Q \implies P)$.
   \end{enumerate}
\end{multicols}

\solution[ex:toan_hoc_nen_tang:lo_gich:menh_de_phuc_hop:vd_bang_gia_tri_chan_li]

\setcounter{subexercise}{1}
\arabic{subexercise}. 
Để ý đến thứ tự các phép nối, dấu $\neg$ sẽ được thực hiện trước $\implies$. Thực hiện xây dựng như ở bảng sau.
\begin{table}[H]
   \centering
   \caption{Bảng giá trị chân lí cho bài \ref{ex:toan_hoc_nen_tang:lo_gich:menh_de_phuc_hop:vd_bang_gia_tri_chan_li} phần \arabic{subexercise}}
   \begin{tabular}{|c|c|c!{\vrule width 1.5pt}c|}
      \hline
      $P$ & $Q$ & $\neg P$ & $\neg P \implies Q$ \\
      \headerDivider
      Đ & Đ & S & Đ \\
      Đ & S & S & Đ \\
      S & Đ & S & Đ \\
      S & S & Đ & S \\
      \hline
   \end{tabular}
\end{table}

{
\begin{minipage}[c]{0.3\linewidth}
   \raggedright
   Có thể thực hiện viết bảng giá trị chân lí dưới dạng rút gọn như ở bảng 
   \ref{tab:toan_hoc_nen_tang:lo_gich:menh_de_phuc_hop:vd_bang_gia_tri_chan_li_1}. 
   Sau khi thực hiện xong một phép nối, có thể viết ở ngay dưới mệnh đề cần tìm 
   giá trị chân lí thay vì tách thành cột riêng để tiết kiệm giấy nếu như mệnh đề dài.
\end{minipage}%
\hfill
\begin{minipage}[c]{0.68\linewidth}
   \begin{table}[H]
      \centering
      \caption{Bảng giá trị chân lí rút gọn cho phần \arabic{subexercise} bài 
      \ref{ex:toan_hoc_nen_tang:lo_gich:menh_de_phuc_hop:vd_bang_gia_tri_chan_li}}
      \label{tab:toan_hoc_nen_tang:lo_gich:menh_de_phuc_hop:vd_bang_gia_tri_chan_li_1}
      \begin{tabular}{|c|c|cccc!{\vrule width 1.5pt}c|}
         \hline
         $P$ & $Q$ & $\neg$ & $P$ & $\implies$ & $Q$ & $\neg P \implies Q$ \\
         \headerDivider 
         Đ & Đ & S & Đ & \emphcolor{Đ} & Đ & Đ \\
         Đ & S & S & Đ & \emphcolor{Đ} & Đ & Đ \\
         S & Đ & Đ & S & \emphcolor{Đ} & Đ & Đ \\
         S & S & Đ & S & \emphcolor{S} & S & S \\
         \hline
      \end{tabular}
   \end{table}
\end{minipage}
}

\stepcounter{subexercise}
\arabic{subexercise}. 
\begin{table}[H]
   \centering
   \caption{Bảng giá trị chân lí rút gọn cho phần \arabic{subexercise} bài 
   \ref{ex:toan_hoc_nen_tang:lo_gich:menh_de_phuc_hop:vd_bang_gia_tri_chan_li}}
   \begin{tabular}{|c|c|cccccc!{\vrule width 1.5pt}c|}
      \hline
      $P$ & $Q$ & $(P$ & $\iff$ & $Q)$ & $\lor$ & $\neg$ & $Q$ & $(P \iff Q) \lor \neg Q$ \\
      \headerDivider
      Đ & Đ & Đ & Đ & Đ & \emphcolor{Đ} & S & Đ & Đ \\
      Đ & S & Đ & S & S & \emphcolor{Đ} & Đ & S & Đ \\
      S & Đ & S & S & Đ & \emphcolor{S} & S & Đ & S \\
      S & S & S & S & S & \emphcolor{Đ} & Đ & S & Đ \\
      \hline
   \end{tabular}
\end{table}

\stepcounter{subexercise}
\arabic{subexercise}. Khi một mệnh đề phức hợp cho cùng giá trị chân lí ở nhiều trường hợp khác nhau, và trong các trường hợp đó một số mệnh đề thành phần có cùng giá trị chân lí, chúng ta có thể rút gọn bảng bằng cách:
\begin{itemize}
   \item Giữ nguyên những mệnh đề thành phần có giá trị giống nhau;
   \item Thay những mệnh đề thành phần thay đổi bằng ký hiệu $X$.
\end{itemize}
Ví dụ như ở bảng \ref{tab:toan_hoc_nen_tang:lo_gich:menh_de_phuc_hop:vd_bang_gia_tri_chan_li_3}, nhận thấy rằng khi $P$ sai thì mệnh đề phức hợp luôn sai, nên các cột $Q$ và $R$ và những cột có phần mệnh đề liên quan đến hai mệnh đề này ở hàng thứ năm được thay bởi những chữ $X$.
\begin{table}[H]
   \centering
   \caption{Bảng giá trị chân lí rút gọn cho phần \arabic{subexercise} bài 
   \ref{ex:toan_hoc_nen_tang:lo_gich:menh_de_phuc_hop:vd_bang_gia_tri_chan_li}}
   \label{tab:toan_hoc_nen_tang:lo_gich:menh_de_phuc_hop:vd_bang_gia_tri_chan_li_3}
   \begin{tabular}{|c|c|c|ccccc!{\vrule width 1.5pt}c|}
      \hline
      $P$ & $Q$ & $R$ & $P$ & $\land$ & $(Q$ & $\implies$ & $R)$ & $P \land (Q \implies R)$ \\
      \headerDivider
      Đ & Đ & Đ & Đ & \emphcolor{Đ} & Đ & Đ & Đ & Đ \\
      Đ & Đ & S & Đ & \emphcolor{S} & Đ & S & S & S \\
      Đ & S & Đ & Đ & \emphcolor{Đ} & S & Đ & Đ & Đ \\
      Đ & S & S & Đ & \emphcolor{Đ} & S & Đ & S & Đ \\
      S & X & X & S & \emphcolor{S} & X & X & X & S \\
      \hline
   \end{tabular}
\end{table}

\stepcounter{subexercise}
\arabic{subexercise}. 
\begin{table}[H]
   \centering
   \caption{Bảng giá trị chân lí rút gọn cho phần \arabic{subexercise} bài 
   \ref{ex:toan_hoc_nen_tang:lo_gich:menh_de_phuc_hop:vd_bang_gia_tri_chan_li}}
   \begin{tabular}{|c|c|c|cccccccc!{\vrule width 1.5pt}c|}
      \hline
      $P$ & $Q$ & $R$ & $P$ & $\land$ & $Q$ & $\lor$ & $Q$ & $\land$ & $\neg$ & $R$ & $P \land Q \lor Q \land \neg R$ \\
      \headerDivider
      Đ & Đ & Đ & Đ & Đ & Đ & \emphcolor{Đ} & Đ & S & S & Đ & Đ \\
      Đ & Đ & S & Đ & Đ & Đ & \emphcolor{Đ} & Đ & Đ & Đ & Đ & Đ \\
      S & Đ & Đ & S & S & Đ & \emphcolor{S} & Đ & S & S & Đ & S \\
      S & Đ & S & S & S & Đ & \emphcolor{Đ} & Đ & Đ & Đ & S & Đ \\
      X & S & X & X & S & S & \emphcolor{S} & S & S & X & X & S \\
      \hline
   \end{tabular}
\end{table}

\stepcounter{subexercise}
\arabic{subexercise}. 
\begin{table}[H]
   \centering
   \caption{Bảng giá trị chân lí rút gọn cho phần \arabic{subexercise} bài 
   \ref{ex:toan_hoc_nen_tang:lo_gich:menh_de_phuc_hop:vd_bang_gia_tri_chan_li}}
   \begin{tabular}{|c|c|c|cccccccc!{\vrule width 1.5pt}c|}
      \hline
      $P$ & $Q$ & $R$ & $Q$ & $\implies$ & $R$ & $\land$ & $R$ & $\lor$ & $\neg$ & $P$ & $Q \implies R \land R \lor \neg P$ \\
      \headerDivider
      Đ & Đ & Đ & Đ & \emphcolor{Đ} & Đ & Đ & Đ & Đ & S & Đ & Đ \\
      Đ & Đ & S & Đ & \emphcolor{S} & S & S & S & S & S & Đ & S \\
      S & Đ & Đ & Đ & \emphcolor{Đ} & Đ & Đ & Đ & Đ & Đ & S & Đ \\
      S & Đ & S & Đ & \emphcolor{Đ} & S & S & S & Đ & Đ & S & Đ \\
      X & S & X & S & \emphcolor{Đ} & X & X & X & X & X & X & Đ \\
      \hline
   \end{tabular}
\end{table}

\stepcounter{subexercise}
\arabic{subexercise}. Do giới hạn của khổ giấy, bảng giá trị chân lí sẽ được tách làm hai phần.
\begin{table}[H]
   \centering
   \caption{Bảng giá trị chân lí rút gọn cho phần \arabic{subexercise} bài 
   \ref{ex:toan_hoc_nen_tang:lo_gich:menh_de_phuc_hop:vd_bang_gia_tri_chan_li}}
   \label{tab:toan_hoc_nen_tang:lo_gich:menh_de_phuc_hop:vd_bang_gia_tri_chan_li_6}
   \begin{tabular}{|c|c|c|ccccccccc|}
      \hline
      $P$ & $Q$ & $R$ & $(P$ & $\implies$ & $Q)$ & $\implies$ & $(R$ & $\implies$ & $Q$ & $\implies$ & $P)$ \\
      \headerDivider
      Đ & X & X & Đ & X & X & \emphcolor{Đ} & X & X & X & Đ & Đ \\
      S & Đ & X & S & Đ & Đ & \emphcolor{S} & X & Đ & Đ & S & S \\
      S & S & Đ & S & Đ & S & \emphcolor{Đ} & Đ & S & S & Đ & S \\
      S & S & S & S & Đ & S & \emphcolor{S} & S & Đ & S & S & S \\
      \hline
   \end{tabular}
   \begin{tabular}{|c|c|c!{\vrule width 1.5pt}c|}
      \hline
      $P$ & $Q$ & $R$ & $(P \implies Q) \implies (R \implies Q \implies R) $ \\
      \headerDivider
      Đ & X & X & Đ \\
      S & Đ & X & S \\
      S & S & Đ & Đ \\
      S & S & S & S \\
      \hline
   \end{tabular}
\end{table}
   
\exercise Gọi $V$ là một mệnh đề đúng và $M$ là một mệnh đề sai nào đó. Sử dụng bảng giá trị chân lí, chứng minh các mệnh đề sau luôn đúng\footnote{Mệnh đề luôn đúng còn được gọi là \defText{mệnh đề hằng đúng}.} với mọi giá trị chân lí của $P$, $Q$ và $R$.
\begin{itemize}
   \item $P \lor V$ (tính chất thống trị với phép tuyển);
   \item $\overline{P \land M}$ (tính chất thống trị với phép hội);
   \item $P \land V \iff P$ (tính chất\footnote{Các tính chất còn được gọi là \defText{luật}.} đồng nhất với phép hội);
   \item $P \lor M \iff P$ (tính chất đồng nhất với phép tuyển);
   \item $P \lor \neg P$ (tính chất loại trừ trung gian\footnote{Tính chất này chỉ rằng một mệnh đề hoặc đúng, hoặc phủ định của nó đúng; không tồn tại khả năng trung gian.});
   \item $\overline{P \land \neg P}$ (tính chất không mâu thuẫn\footnote{Tính chất này khẳng định Không thể có một mệnh đề vừa đúng vừa sai cùng lúc và theo cùng một nghĩa.});
   \item $\overline{\neg P} \iff P$ (tính chất phủ định kép);
   \item $P \land P \iff P$ (tính chất lũy đẳng với phép hội);
   \item $P \lor P \iff P$ (tính chất lũy đẳng với phép tuyển);
   \item $P \land Q \iff Q \land P$ (tính giao hoán với phép hội);
   \item $P \lor Q \iff Q \lor P$ (tính giao hoán với phép tuyển);
   \item $\overline{P\lor Q} \iff \neg P \land \neg Q$ (định luật Đờ Moóc-gơn\footnote{Augustus De Morgan (1806 -- 1871)}, phần 1);
   \item $\overline{P\land Q} \iff \neg P \lor \neg Q$ (định luật Đờ Moóc-gơn, phần 2);
   \item $P \implies Q \iff \neg P \lor Q$ (định nghĩa phép kéo theo);
   \item $\left(P \implies Q\right) \iff \left(\neg P \impliedby \neg Q\right)$ (tính chất phản đảo);
   \item $P \implies P \lor Q$ (tính chất cộng);
   \item $P \land Q \implies P$ (tính chất rút gọn);
   \item $\left(P \implies Q\right) \land P \implies Q$ (quy tắc khẳng định\footnote{Modus ponens.});
   \item $\left(P \implies Q\right) \land \neg Q \implies \neg P$ (quy tắc phủ định\footnote{Modus tollens.});
   \item $\left(P \lor Q\right) \land \neg P \implies Q$ (tam đoạn luận tuyển);
   \item $\left(P \iff Q\right) \iff \left(P \implies Q\right) \land \left(P \impliedby Q\right)$ (định nghĩa phép tương đương);
   \item $\left(P \land Q\right) \land R \iff P \land \left(Q \land R\right)$ (tính chất kết hợp với phép hội);
   \item $\left(P \lor Q\right) \lor R \iff P \lor \left(Q \lor R\right)$ (tính chất kết hợp với phép tuyển);
   \item $\left(P \lor Q\right) \land \left(P \lor R\right) \iff P \land (Q \lor R)$ (tính chất phân phối, phần 1);
   \item $\left(P \land Q\right) \lor \left(P \land R\right) \iff P \lor (Q \land R)$ (tính chất phân phối, phần 2);
   \item $\left(P \implies Q\right) \land \left(Q \implies R\right) \implies \left(P \implies R\right)$ (tính chất bắc cầu);
   \item $\left(P \iff Q\right) \land \left(Q \iff R\right) \iff $
\end{itemize}

\solution

\begin{table}[H]
   \centering
   \caption{Bảng giá trị chân lí của $P \lor V$ và $\overline{P \land M}$}
   \begin{tabular}{|c|ccc!{\vrule width 1.5pt}c|cccc!{\vrule width 1.5pt}c|}
      \hline
      $P$ & $P$ & $\lor$ & $V$ & $P \lor V$ & $\neg$ & $(P$ & $\land$ & $M)$ & $\overline{P \land M}$ \\
      \headerDivider
      X & X & \emphcolor{Đ} & Đ & Đ & \emphcolor{Đ} & X & S & S & Đ \\
      \hline
   \end{tabular}
\end{table}

\begin{table}[H]
   \centering
   \caption{Bảng giá trị chân lí của $P \land V \iff P$ và $P \lor M \iff P$}
   \begin{tabular}{|c|ccccc!{\vrule width 1.5pt}c|ccccc!{\vrule width 1.5pt}c|}
      \hline
      $P$ & $P$ & $\land$ & $V$ & $\iff$ & $P$ & $P \land V \iff P$ & $P$ & $\lor$ & $M$ & $\iff$ & $P$ & $P \lor M \iff P$ \\
      \headerDivider
      Đ & Đ & Đ & Đ & \emphcolor{Đ} & Đ & Đ & Đ & Đ & S & \emphcolor{Đ} & Đ & Đ \\
      S & S & S & Đ & \emphcolor{Đ} & S & Đ & S & S & S & \emphcolor{Đ} & S & Đ \\
      \hline
   \end{tabular}
\end{table}

\begin{table}[H]
   \centering
   \caption{Bảng giá trị chân lí của $P \lor \neg P$ và $\overline{P \land \neg P}$}
   \begin{tabular}{|c|cccc!{\vrule width 1.5pt}c|ccccc!{\vrule width 1.5pt}c|}
      \hline
      $P$ & $P$ & $\lor$ & $\neg$ & $P$ & $P \lor \neg P$ & $\neg$ & $(P$ & $\land$ & $\neg$ & $P)$ & $\overline{P \land \neg P}$ \\
      \headerDivider
      Đ & Đ & \emphcolor{Đ} & S & Đ & Đ & \emphcolor{Đ} & Đ & S & S & Đ & Đ \\
      S & S & \emphcolor{Đ} & Đ & S & Đ & \emphcolor{Đ} & S & S & Đ & S & Đ \\
      \hline
   \end{tabular}
\end{table}

\begin{table}[H]
   \centering
   \caption{Bảng giá trị chân lí của $\overline{\neg P} \iff P$}
   \begin{tabular}{|c|ccccc!{\vrule width 1.5pt}c|}
      \hline
      $P$ & $\neg$ & $(\neg$ & $P)$ & $\iff$ & $P$ & $\overline{\neg P} \iff P$ \\
      \headerDivider
      Đ & Đ & S & Đ & \emphcolor{Đ} & Đ & Đ \\
      S & S & Đ & S & \emphcolor{Đ} & S & Đ \\
      \hline
   \end{tabular}
\end{table}

\begin{table}[H]
   \centering
   \caption{Bảng giá trị chân lí của $P \land P \iff P$ và $P \lor P \iff P$}
   \begin{tabular}{|c|ccccc!{\vrule width 1.5pt}c|ccccc!{\vrule width 1.5pt}c|}
      \hline
      $P$ & $P$ & $\land$ & $P$ & $\iff$ & $P$ & $P \land P \iff P$ & $P$ & $\lor$ & $P$ & $\iff$ & $P$ & $P \lor P \iff P$ \\
      \headerDivider
      Đ & Đ & Đ & Đ & \emphcolor{Đ} & Đ & Đ & Đ & Đ & Đ & \emphcolor{Đ} & Đ & Đ \\
      S & S & S & S & \emphcolor{Đ} & S & Đ & S & S & S & \emphcolor{Đ} & S & Đ \\
      \hline
   \end{tabular}
\end{table}

\begin{table}[H]
   \centering
   \caption{Bảng giá trị chân lí của $P \land Q \iff Q \land P$}
   \begin{tabular}{|c|c|ccccccc!{\vrule width 1.5pt}c|}
      \hline
      $P$ & $Q$ & $P$ & $\land$ & $Q$ & $\iff$ & $Q$ & $\land$ & $P$ & $P \land Q \iff Q \land P$ \\
      \headerDivider
      Đ & Đ & Đ & Đ & Đ & \emphcolor{Đ} & Đ & Đ & Đ & Đ \\
      Đ & S & Đ & S & S & \emphcolor{Đ} & S & S & Đ & Đ \\
      S & Đ & S & S & Đ & \emphcolor{Đ} & Đ & S & S & Đ \\
      S & S & S & S & S & \emphcolor{Đ} & S & S & S & Đ \\
      \hline
   \end{tabular}
\end{table}

\begin{table}[H]
   \centering
   \caption{Bảng giá trị chân lí của $P \lor Q \iff Q \lor P$}
   \begin{tabular}{|c|c|ccccccc!{\vrule width 1.5pt}c|}
      \hline
      $P$ & $Q$ & $P$ & $\lor$ & $Q$ & $\iff$ & $Q$ & $\lor$ & $P$ & $P \lor Q \iff Q \lor P$ \\
      \headerDivider
      Đ & Đ & Đ & Đ & Đ & \emphcolor{Đ} & Đ & Đ & Đ & Đ \\
      Đ & S & Đ & Đ & S & \emphcolor{Đ} & S & Đ & Đ & Đ \\
      S & Đ & S & Đ & Đ & \emphcolor{Đ} & Đ & Đ & S & Đ \\
      S & S & S & S & S & \emphcolor{Đ} & S & S & S & Đ \\
      \hline
   \end{tabular}
\end{table}

\begin{table}[H]
   \centering
   \caption{Bảng giá trị chân lí của $\overline{P \lor Q} \iff \neg P \land \neg Q$}
   \begin{tabular}{|c|c|cccccccccc!{\vrule width 1.5pt}c|}
      \hline
      $P$ & $Q$ & $\neg$ & $(P$ & $\lor$ & $Q)$ & $\iff$ & $\neg$ & $P$ & $\land$ & $\neg$ & $Q$ & $\overline{P \lor Q} \iff \neg P \land \neg Q$ \\
      \headerDivider
      Đ & Đ & S & Đ & Đ & Đ & \emphcolor{Đ} & S & Đ & S & S & Đ & Đ \\
      Đ & S & S & Đ & Đ & S & \emphcolor{Đ} & S & Đ & S & Đ & S & Đ \\
      S & Đ & S & S & Đ & Đ & \emphcolor{Đ} & Đ & S & S & S & Đ & Đ \\
      S & S & Đ & S & S & S & \emphcolor{Đ} & Đ & S & Đ & Đ & S & Đ \\
      \hline
   \end{tabular}
\end{table}

\begin{table}[H]
   \centering
   \caption{Bảng giá trị chân lí của $\overline{P \land Q} \iff \neg P \lor \neg Q$}
   \begin{tabular}{|c|c|cccccccccc!{\vrule width 1.5pt}c|}
      \hline
      $P$ & $Q$ & $\neg$ & $(P$ & $\land$ & $Q)$ & $\iff$ & $\neg$ & $P$ & $\lor$ & $\neg$ & $Q$ & $\overline{P \land Q} \iff \neg P \lor \neg Q$ \\
      \headerDivider
      Đ & Đ & S & Đ & Đ & Đ & \emphcolor{Đ} & S & Đ & S & S & Đ & Đ \\
      Đ & S & Đ & Đ & S & S & \emphcolor{Đ} & S & Đ & Đ & Đ & S & Đ \\
      S & Đ & Đ & S & S & Đ & \emphcolor{Đ} & Đ & S & Đ & S & Đ & Đ \\
      S & S & Đ & S & S & S & \emphcolor{Đ} & Đ & S & Đ & Đ & S & Đ \\
      \hline
   \end{tabular}
\end{table}

\begin{table}[H]
   \centering
   \caption{Bảng giá trị chân lí của $P \implies Q \iff \neg P \lor Q$}
   \begin{tabular}{|c|c|cccccccc!{\vrule width 1.5pt}c|}
      \hline
      $P$ & $Q$ & $P$ & $\implies$ & $Q$ & $\iff$ & $\neg$ & $P$ & $\lor$ & $Q$ & $P \implies Q \iff \neg P \lor Q$ \\
      \headerDivider
      Đ & Đ & Đ & Đ & Đ & \emphcolor{Đ} & S & Đ & Đ & Đ & Đ \\
      Đ & S & Đ & S & S & \emphcolor{Đ} & S & Đ & S & S & Đ \\
      S & X & S & Đ & X & \emphcolor{Đ} & Đ & S & Đ & X & Đ \\
      \hline
   \end{tabular}
\end{table}

\begin{table}[H]
   \centering
   \caption{Bảng giá trị chân lí của $\left(P \implies Q\right) \iff \left(\neg P \impliedby \neg Q\right)$}
   \begin{tabular}{|c|c|ccccccccc!{\vrule width 1.5pt}c|}
      \hline
      $P$ & $Q$ & $(P$ & $\implies$ & $Q)$ & $\iff$ & $(\neg$ & $P$ & $\impliedby$ & $\neg$ & $Q)$ & $\left(P \implies Q\right) \iff \left(\neg P \impliedby \neg Q\right)$ \\
      \headerDivider
      Đ & Đ & Đ & Đ & Đ & \emphcolor{Đ} & S & Đ & Đ & S & Đ & Đ \\
      Đ & S & Đ & S & S & \emphcolor{Đ} & S & Đ & S & Đ & S & Đ \\
      S & X & S & Đ & X & \emphcolor{Đ} & Đ & S & Đ & X & X & Đ \\
      \hline
   \end{tabular}
\end{table}

\begin{table}[H]
   \centering
   \caption{Bảng giá trị chân lí của $P \implies P \lor Q$}
   \begin{tabular}{|c|c|ccccc!{\vrule width 1.5pt}c|}
      \hline
      $P$ & $Q$ & $P$ & $\implies$ & $P$ & $\lor$ & $Q$ & $P \implies P \lor Q$ \\
      \headerDivider
      Đ & X & Đ & \emphcolor{Đ} & Đ & Đ & X & Đ \\
      S & X & S & \emphcolor{Đ} & S & X & X & Đ \\
      \hline 
   \end{tabular}
\end{table}

\begin{table}[H]
   \centering
   \caption{Bảng giá trị chân lí của $P \land Q \implies P$}
   \begin{tabular}{|c|c|ccccc!{\vrule width 1.5pt}c|}
      \hline
      $P$ & $Q$ & $P$ & $\land$ & $Q$ & $\implies$ & $P$ & $P \land Q \implies P$ \\
      \headerDivider
      Đ & X & Đ & X & X & \emphcolor{Đ} & Đ & Đ \\
      S & X & S & S & X & \emphcolor{Đ} & S & Đ \\
      \hline 
   \end{tabular}
\end{table}

\begin{table}[H]
   \centering
   \caption{Bảng giá trị chân lí của $\left(P \implies Q\right) \land P \implies Q$}
   \begin{tabular}{|c|c|ccccccc!{\vrule width 1.5pt}c|}
      \hline
      $P$ & $Q$ & $(P$ & $\implies$ & $Q)$ & $\land$ & $P$ & $\implies$ & $Q$ & $\left(P \implies Q\right) \land P \implies Q$ \\
      \headerDivider
      Đ & S & Đ & S & S & S & Đ & \emphcolor{Đ} & S & Đ \\
      S & S & S & Đ & Đ & S & S & \emphcolor{Đ} & S & Đ \\
      X & Đ & X & Đ & Đ & X & X & \emphcolor{Đ} & Đ & Đ \\
      \hline 
   \end{tabular}
\end{table}

\begin{table}[H]
   \centering
   \caption{Bảng giá trị chân lí của $\left(P \implies Q\right) \land \neg Q \implies \neg P$}
   \begin{tabular}{|c|c|ccccccccc!{\vrule width 1.5pt}c|}
      \hline
      $P$ & $Q$ & $(P$ & $\implies$ & $Q)$ & $\land$ & $\neg$ & $Q$ & $\implies$ & $\neg$ & $P$ & $\left(P \implies Q\right) \land \neg Q \implies \neg P$ \\
      \headerDivider
      Đ & Đ & Đ & Đ & Đ & S & S & Đ & \emphcolor{Đ} & S & Đ & Đ \\
      Đ & S & Đ & S & S & S & Đ & S & \emphcolor{Đ} & S & Đ & Đ \\
      S & X & S & Đ & X & X & X & X & \emphcolor{Đ} & Đ & S & Đ \\
      \hline 
   \end{tabular}
\end{table}

\begin{table}[H]
   \centering
   \caption{Bảng giá trị chân lí của $\left(P \iff Q\right) \iff \left(P \implies Q\right) \land \left(P \impliedby Q\right)$}
   \begin{tabular}{|c|c|ccccccccccc|}
      \hline
      $P$ & $Q$ & $(P$ & $\iff$ & $Q)$ & $\iff$ & $(P$ & $\implies$ & $Q)$ & $\land$ & $(P$ & $\impliedby$ & $Q)$ \\
      \headerDivider 
      Đ & Đ & Đ & Đ & Đ & \emphcolor{Đ} & Đ & Đ & Đ & Đ & Đ & Đ & Đ \\
      Đ & S & Đ & S & S & \emphcolor{Đ} & Đ & S & S & S & Đ & Đ & S \\
      S & Đ & S & Đ & Đ & \emphcolor{Đ} & S & Đ & Đ & S & Đ & S & S \\
      S & S & S & Đ & S & \emphcolor{Đ} & S & Đ & S & Đ & S & Đ & S \\
      \hline
   \end{tabular}
   \begin{tabular}{|c|c!{\vrule width 1.5pt}c|}
      \hline
      $P$ & $Q$ & $\left(P \iff Q\right) \iff \left(P \implies Q\right) \land \left(P \impliedby Q\right)$ \\
      \headerDivider
      Đ & Đ & \multirow{4}{*}{\huge \vspace{-12pt} Đ} \\
      Đ & S & \\
      S & Đ & \\
      S & S & \\
      \hline
   \end{tabular}
\end{table}