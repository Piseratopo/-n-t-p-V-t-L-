\section{Mệnh đề ghép và các phép nối mệnh đề}

\ % Lùi đầu dòng

Hãy xem xét câu sau:
\begin{center}
   ''Hôm nay trời mua \emph{và} hội thao đã phải lùi lịch.''.
\end{center}
Đây rõ ràng là một mệnh đề do chúng ta có thể dễ dàng xác định được tính đúng sai của nó. Câu hỏi quan trọng hơn cần được đặt ra là chúng ta đã xác định tính chính xác của câu này như thế nào. Một cách tự nhiên, chúng ta sẽ xem xét từng phần ``hôm nay trời mưa'' và ``hội thao đã phải lùi lịch''. Từ tính đúng sai của hai vế, tính đúng sai của mệnh đề ban đầu được xác định. Đây là một ví dụ của \defText{mệnh đề phức hợp} (hay \defText{mệnh đề phức}), một mệnh đề được cấu tạo từ một hoặc một số \defText{mệnh đề thành phần} và các \defText{phép nối mệnh đề}.

Trong toán học, chúng ta hay sử dụng $5$ phép nối mệnh đề (lần đầu được đề xuất bởi Phrây-gơ\footnote{Friedrich Ludwig Gottlob Frege (1848 -- 1925)}):
\begin{itemize}
   \item Phép \defText{đối} --- $\defMath{\neg}$, $\defMath{'}$, $\defMath{!}$;
   \item Phép \defText{hội} hoặc phép \defText{và} --- $\defMath{\land}$ hoặc $\defMath{\&}$, $\defMath{\cdot}$;
   \item Phép \defText{tuyển} hoặc phép \defText{hoặc} --- $\defMath{\lor}$ hoặc $\defMath{\parallel}$, $\defMath{+}$;
   \item Phép \defText{kéo theo} --- $\defMath{\implies}$ hoặc $\defMath{\Rightarrow}$, $\defMath{\rightarrow}$, $\defMath{\supset}$;
   \item Phép \defText{đẳng giá} hoặc phép \defText{tương đương lô-gích} --- $\defMath{\iff}$ hoặc $\defMath{\Leftrightarrow}$, $\defMath{\leftrightarrow}$, $\defMath{\leftrightarrows}$, $\defMath{\equiv}$.
\end{itemize}

Đi kèm với những phép nối này, với sự phát triển của điện tử số, người ta đề xuất thêm những phép nối mệnh đề khác:
\begin{itemize}
   \item Phép \defText{phủ định hội} -- $\defMath{\uparrow}$ hoặc $\defMath{\overline{\land}}$;
   \item Phép \defText{phủ định tuyển} -- $\defMath{\downarrow}$ hoặc $\defMath{\overline{\lor}}$; 
   \item Phép \defText{hệ quả} -- $\defMath{\impliedby}$ hoặc $\defMath{\Leftarrow}$, $\defMath{\leftarrow}$, $\defMath{\subset}$;
   \item Phép \defText{phủ định kéo theo} -- $\defMath{\nRightarrow}$ hoặc $\defMath{\nrightarrow}$, $\defMath{\text{\hspace{0.6em}$\not$\hspace{-0.75em}$\implies$}}$, \hspace{0.1em}$\defMath{\not}$\hspace{-0.1em}$\defMath{\rightarrow}$, $\defMath{\not\supset}$;
   \item Phép \defText{phủ định hệ quả} -- $\defMath{\nLeftarrow}$ hoặc $\defMath{\nleftarrow}$, $\defMath{\defMath{\text{\hspace{0.8em}$\not$\hspace{-0.94em}$\impliedby$}}}$, \hspace{0.2em}$\defMath{\not}$\hspace{-0.2em}$\defMath{\leftarrow}$, $\defMath{\not\subset}$;
   \item Phép \defText{phủ định đẳng giá} -- $\defMath{\nLeftrightarrow}$ hoặc $\defMath{\nleftrightarrow}$, $\defMath{\text{\hspace{0.8em}$\not$\hspace{-0.94em}$\iff$}}$, \hspace{0.13em}$\defMath{\not}$\hspace{-0.13em}$\defMath{\leftrightarrow}$; 
   \item Phép \defText{tuyển chọn} hoặc phép \defText{tuyển lại trừ} -- $\defMath{\oplus}$ hoặc $\defMath{\veebar}$, \hspace{0.13em}$\defMath{\not}$\hspace{-0.13em}$\defMath{\leftrightarrow}$\footnote{Kí hiệu giống với phủ định đẳng giá không phải là do trùng hợp.}, $\defMath{\not\equiv}$;
   \item Phép \defText{phủ định tuyển chọn} -- $\defMath{\odot}$.
\end{itemize}

Kết hợp với chúng là hai dấu ngoặc, ngoặc đơn đóng --- $)$ --- và ngoặc đơn mở --- $($\footnote{Có thể dùng thêm ngoặc vuông --- $[]$ --- hay ngoặc nhọn --- $\{\}$ --- nếu cần tăng khả năng nhận diện của mệnh đề phức hợp.} --- để xác định thứ tự giải giá trị lô-gích của mệnh đề phức hợp.

\subsection{Phép đối}

\ %

Thông thường, để phủ định một câu khẳng định, chúng ta hay dùng từ ``không'' hay những từ gần nghĩa như ``chưa'' hay ``chẳng''. Ví dụ, có thể phủ định câu ``Cơm hôm nay ngon.'' thành ``Cơm hôm nay \emph{không} ngon.''. Cần phải để ý rằng, với những câu phức tạp hơn như ví dụ về hội thao ở trước đó thì việc thêm các chữ ``không'' như
\begin{center}
   ``Hôm nay trời \emph{không} mưa \emph{\textcolor{colorEmphasis}{và}} hội thao đã \emph{không} phải lùi lịch.''
\end{center}
là không thỏa đáng. Cách viết đúng cần phải phủ định toàn bộ câu chứ không chỉ một vế. Có thể viết mệnh đề phủ định như sau, tuy sẽ khá dài dòng:
\begin{center}
   ``\emph{Không phải trường hợp rằng} hôm nay trời mưa và hội thao đã phải lùi lịch.''.
\end{center}
Để tăng tính khái quát hóa và rút gọn mệnh đề, chúng ta có thể sử dụng kí hiệu, tuy rằng câu sau đó nhìn sẽ hơi kì, kiểu như:
\begin{center}
   ``$\neg \left(\text{cơm hôm nay ngon}\right)$''
\end{center}
hay
\begin{center}
   ``$\neg \left(\text{hôm nay trời mưa và hội thao đã phải lùi lịch}\right)$''.
\end{center}
Nhìn chung, nếu $P$ là một mệnh đề thì phủ định của nó sẽ có kí hiệu là $\defMath{\neg P}$ hoặc $\defMath{\overline{P}}$. Nếu $P$ đúng thì $\neg P$ sai và ngược lại, nếu $P$ sai thì $\neg P$ đúng.

\subsection{Bảng giá trị chân lí}

\ %

Trước khi đi đến những phép nối phức tạp hơn, chúng ta sẽ đề cập đến khái niệm bảng giá trị chân lí. Khi sử dụng các phép toán lô-gích để tạo ra mệnh đề phức hợp thì chúng ta cần phải xem xét các trường hợp có thể của các \defText{giá trị chân lí}, một cách nói văn hoa hơn cho cụm từ ``tính đúng sai'', của từng mệnh đề thành phần. Khi mà số mệnh đề thành phần lớn lên thì số lượng trường hợp cũng tăng theo theo cấp số nhân. Để tránh việc phải viết nhiều, \defText{bảng giá trị chân lí} đã được khai sinh\footnote{Đây có vẻ là một khái niệm đơn giản, bởi vì lập bảng là một thao tác đã được thực hiện thường xuyên xuyên suốt lịch sử loài người, tuy nhiên, không có quá nhiều tài liệu lịch sử nói về bảng giá trị chân lí. Tài liệu sớm nhất mà tác giả tìm được cho thấy sự sử dụng của kiểu bảng này xuất phát từ thế kỉ XIX\cite{anellis2012peirce}. Có thể, trường hợp thứ nhất, người xưa thấy việc viết (hay nói, biết chữ là một thứ xa xỉ) các mệnh đề lô-gích phức hợp là bình thường, hoặc, trường hợp thứ hai với khả năng xảy ra cao hơn, kiến thức lịch sử của tác giả còn hạn hẹp.}.

Chúng ta sẽ lấy ví dụ ngay trên phép nối mệnh đề chúng ta vừa được tiếp cận. Khi xây dựng bảng giá trị chân lí, tác giả sẽ viết tắt ``Đ'' và ``S'' lần lượt cho mệnh đề có giá trị chân lí đúng và sai.
\begin{table}[H]
   \centering
   \caption{Bảng giá trị chân lí của mệnh đề với phép đối}
   \begin{tabular}{|c!{\vrule width 1.5pt}c|}
      \hline 
      $P$ & $\neg P$ \\
      \headerDivider
      Đ & S \\
      S & Đ \\
      \hline
   \end{tabular}
\end{table}

\subsection{Phép hội}

\ %

Trong hệ thống an ninh ngân hàng, thông thường, để đăng nhập vào tài khoản, có thể bạn đọc sẽ cần $P$: ``nhập đúng mật khẩu'' và $Q$: ``nhập đúng mã OTP\footnote{One Time Password --- Mật khẩu dùng một lần.}''. Chỉ khi \emphcolor{$P$ và $Q$} đúng thì bạn đọc mới được phép sử dụng các sản phẩm của ngân hàng.

Trong ngôn ngữ tự nhiên, chúng ta thường dùng từ ``và'' để nối hai ý tưởng lại với nhau. Trong logic học, việc kết hợp hai mệnh đề $P$ và $Q$ để tạo thành một mệnh đề mới được gọi là phép hội. Mệnh đề này được biểu diễn bằng lời là ``$P$ và $Q$'', kí hiệu là
$$\defMath{P \land Q}.$$

Mệnh đề này có tính "khắt khe": nó chỉ đúng khi cả $P$ và $Q$ đồng thời cùng đúng. Nếu ít nhất một trong hai mệnh đề thành phần bị sai, kết quả của phép hội sẽ là sai, thể hiện dưới dạng bảng giá trị chân lí \ref{tab:toan_hoc_nen_tang:lo_gich:menh_de_phuc_hop:dn_hoi}.
\begin{table}[H]
   \centering
   \caption{Bảng giá trị chân lí của mệnh đề với phép hội}
   \label{tab:toan_hoc_nen_tang:lo_gich:menh_de_phuc_hop:dn_hoi}
   \begin{tabular}{|c|c!{\vrule width 1.5pt}c|}
      \hline 
      $P$ & $Q$ & $P \land Q$ \\
      \headerDivider
      Đ & Đ & Đ \\
      Đ & S & S \\
      S & Đ & S \\
      S & S & S \\
      \hline
   \end{tabular}
\end{table}

Giả sử rằng bạn đọc thấy bữa ăn buổi hôm nay rất ngon, do sự nhiệt tình của bạn bồi bàn. Bạn đọc đã vào được phần mềm ngân hàng của mình và định cho bo cho bồ bàn một chút tiền. Tuy nhiên, bạn đọc lại nhìn thấy bảng ghi chính sách của quán: "Khách hàng không được phép đưa tiền bo cho nhân viên của quán". Nếu bạn đọc vừa $P$: là khách hàng của quán, và $Q$: đưa tiền bo cho nhân viên, thì bạn đọc đã vi phạm chính sách rằng \emphcolor{không phải rằng $P$ và $Q$}. Bạn đọc đành đưa ra đề nghị với nhà hàng là bạn muốn được phục vụ lại bởi bồi bàn này trong những lần sau.

Nếu phép hội đại diện cho sự ``đồng thuận tuyệt đối'', thì phép phủ định hội (còn gọi là phép toán Sép-phơ\footnote{Henry Maurice Sheffer (1882 -- 1964)}) lại đại diện cho sự phủ nhận của sự đồng thuận đó. Phủ định hội của $P$ và $Q$ được kí hiệu là $$\defMath{P \uparrow Q}.$$ Mệnh đề này chỉ sai trong trường hợp duy nhất là cả $P$ và $Q$ đều đúng.
\begin{table}[H]
   \centering
   \caption{Bảng giá trị chân lí của mệnh đề với phép phủ định hội}
   \label{tab:toan_hoc_nen_tang:lo_gich:menh_de_phuc_hop:dn_phu_dinh_hoi}
   \begin{tabular}{|c|c!{\vrule width 1.5pt}c|}
      \hline 
      $P$ & $Q$ & $P \uparrow Q$ \\
      \headerDivider
      Đ & Đ & S \\
      Đ & S & Đ \\
      S & Đ & Đ \\
      S & S & Đ \\
      \hline
   \end{tabular}
\end{table}

\subsection{Phép tuyển}

\ % Lùi đầu dòng

Ê-mô-ri muốn được nhập học tại trường U-ni-véc-xi-ta-tô chẳng hạn, đương nhiên Ê-mô-ri cần phải tìm hiểu điều kiện xét tuyển của trường này. Với ngành học mà Ê-mô-ri mong muốn, có yêu cầu rằng Ê-mô-ri cần phải có $P$: tổng điểm thi trung học phổ thông quốc gia từ $24$ điểm trở lên, hoặc $Q$: có giải thành phố từ giải ba trở lên. Do đó, Ê-mô-ri muốn được tuyển vào trường theo ngành này cần phải có thành tích \emphcolor{$P$ hoặc $Q$}. 

Mệnh đề phức ``$P$ tuyển $Q$'' có kí hiệu $\defMath{P \lor Q}$. Mệnh đề này đúng khi tối thiểu một trong hai mệnh đề đầu vào đúng, và chỉ sai khi cả hai mệnh đề đầu vào đều sai. Bảng \ref{tab:toan_hoc_nen_tang:lo_gich:menh_de_phuc_hop:dn_tuyen} cho giá trị chân lí của mệnh đề tuyển.
\begin{table}[H]
   \centering
   \caption{Bảng giá trị chân lí của mệnh đề với phép tuyển}
   \label{tab:toan_hoc_nen_tang:lo_gich:menh_de_phuc_hop:dn_tuyen}
   \begin{tabular}{|c|c!{\vrule width 1.5pt}c|}
      \hline
      $P$ & $Q$ & $P \lor Q$ \\
      \headerDivider
      Đ & Đ & Đ \\
      Đ & S & Đ \\ 
      S & Đ & Đ \\
      S & S & S \\
      \hline
   \end{tabular}
\end{table}

Ý nghĩa của từ ``hoặc'' trong toán học hơi khác với ý nghĩa thông thường. Khi người ta nói ``hoặc'', người ta hay ám chỉ một trong số các trường hợp liệt kê ra là đúng. Trong lô-gích, cả hai mệnh đề đúng vẫn làm cho mệnh đề phức hợp đúng. Như trong ví dụ ở trên, nếu Ê-mô-ri được cả hai thành tích thì càng tăng khả năng vào trường, đúng không?

Trường U-ni-véc-xi-ta-tô là một trường đại học, tuy không phải là giàu có nhưng cũng có đủ điều kiện cơ sở vật chất để sắm $2$ máy chủ. Trường khá yên tâm với hệ thống của mình do chỉ khi không phải trường hợp rằng $P$: máy chủ thứ nhất hoạt động hoặc $Q$: máy chủ thứ hai hoạt động thì khi đó hệ thống mới sập hoàn toàn --- \emphcolor{không phải rằng P hoặc Q}. 

Tương tự với phép tuyển, phép hội cũng có ``anh em'' phủ định tương ứng. Mệnh đề \defText{tuyển phủ định} $\defMath{P \downarrow Q}$ sai khi tồn tại một mệnh đề sai trong hai mệnh đề $P$ và $Q$.
\begin{table}[H]
   \centering
   \caption{Bảng giá trị chân lí của mệnh đề với phép phủ định tuyển}
   \label{tab:toan_hoc_nen_tang:lo_gich:menh_de_phuc_hop:dn_phu_dinh_tuyen}
   \begin{tabular}{|c|c!{\vrule width 1.5pt}c|}
      \hline
      $P$ & $Q$ & $P \downarrow Q$ \\
      \headerDivider
      Đ & Đ & S \\
      Đ & S & S \\ 
      S & Đ & S \\
      S & S & Đ \\
      \hline
   \end{tabular}
\end{table}

\subsection{Phép kéo theo và phép hệ quả}

\ % Lùi đầu dòng

Đây lại là một phép nối nữa mà ý nghĩa của nó (có thể) khác với ý nghĩa thông thường. Đây là phép cũng gây nhiều lỗi lập luận lô-gích nhất. Mệnh đề với phép kéo theo $\defMath{P \implies Q}$ chỉ sai khi $P$ không suy ra $Q$, điều này tương đương, và chỉ tương đương, với có $P$ mà không có $Q$. Bạn đọc nên để ý kĩ hai dòng cuối cùng của bảng giá trị chân lí \ref{tab:toan_hoc_nen_tang:lo_gich:menh_de_phuc_hop:dn_keo_theo}.
\begin{table}[H]
   \centering
   \caption{Bảng giá trị chân lí của mệnh đề với phép kéo theo}
   \label{tab:toan_hoc_nen_tang:lo_gich:menh_de_phuc_hop:dn_keo_theo}
   \begin{tabular}{|c|c!{\vrule width 1.5pt}c|}
      \hline
      $P$ & $Q$ & $P \implies Q$ \\
      \headerDivider
      Đ & Đ & Đ \\
      Đ & S & S \\
      \emphcolor{S} & \emphcolor{Đ} & \emphcolor{Đ} \\
      \emphcolor{S} & \emphcolor{S} & \emphcolor{Đ} \\
      \hline
   \end{tabular}
\end{table}

Hãy xem xét một ví dụ minh họa. Không biết rằng bạn đọc đã từng được bố mẹ hứa hẹn thưởng cho một món theo sở thích khi được điểm 10 bao giờ chưa, còn đối với tác giả là có rồi. Nếu chúng ta xét mệnh đề: ``Nếu $P$: con đạt điểm 10, thì $Q$: bố sẽ tặng con một cuốn sách.'', chúng ta sẽ có $3$ trường hợp xảy ra:
\begin{itemize}
   \item Trường hợp Đ $\implies$ Đ: Con được 10 điểm và bố tặng sách. Lời hứa được thực hiện và mệnh đề này là đúng;
   \item Trường hợp Đ $\implies$ S: Con được 10 điểm nhưng bố không tặng sách. Lời hứa không được thực hiện và mệnh đề này là sai;
   \item Trường hợp S $\implies$ Đ/S: Con không được 10 điểm. Trong trường hợp này, dù người bố có tặng sách hay không thì cũng \emph{không vi phạm lời hứa} ban đầu. Cho nên, mệnh đề là đúng.
\end{itemize}
Vậy trường hợp duy nhất mà lời hứa bị phá vỡ --- \emphcolor{P kéo theo Q} sai --- là khi người con không được tặng sách mặc dù được điểm 10.

Phép kết nối xét theo chiều ngược lại của phép kéo theo chính là phép hệ quả. Tương tự như phép kéo theo, mệnh đề hệ quả $\defMath{P \impliedby Q}$ chỉ sai khi $P$ không được kéo theo từ $Q$ như được thể hiện ở bảng \ref{tab:toan_hoc_nen_tang:lo_gich:menh_de_phuc_hop:dn_he_qua}.
\begin{table}[H]
   \centering
   \caption{Bảng giá trị chân lí của mệnh đề với phép hệ quả}
   \label{tab:toan_hoc_nen_tang:lo_gich:menh_de_phuc_hop:dn_he_qua}
   \begin{tabular}{|c|c!{\vrule width 1.5pt}c|}
      \hline
      $P$ & $Q$ & $P \impliedby Q$ \\
      \headerDivider
      Đ & Đ & Đ \\
      Đ & S & Đ \\
      S & Đ & S \\
      S & S & Đ \\
      \hline
   \end{tabular}
\end{table}

Không phải lúc nào trong cuộc sống mà cũng có mối quan hệ kéo theo -- hệ quả rõ ràng. Ví dụ, nhiều người thường quan niệm rằng $P$: thành công là do $Q$: đọc sách tự cải thiện bản thân\footnote{Là sách self-help đó. Làm ơn đừng đọc.}. Tuy nhiên, Bác Hồ đã tự mình bôn ba năm châu bốn biển và sau đó thành công trong việc tìm xác định được đường lối phù hợp với đất nước Việt Nam lúc bấy giờ và dẫn dắt được đất nước qua quãng thời gian bế tắc. Rõ ràng là do Bác đọc những tác phẩm mang tính cách mạng cao (mà không phải sách tự cải thiện bản thân), cho nên có thể nói rằng Bác Hồ đã làm cho phản mệnh đề $P \impliedby Q$ đúng, hay nói cách khác, là ví dụ cho sự \emphcolor{phủ định hệ quả $P$ do $Q$}.

Đảo của phép kéo theo và phép hệ quả, phủ định kéo theo $\defMath{P \nRightarrow Q}$ và \defText{phủ định hệ quả} $\defMath{P \nLeftarrow Q}$ có bảng giá trị chân lí như ở bảng \ref{tab:toan_hoc_nen_tang:lo_gich:menh_de_phuc_hop:dn_phu_dinh_keo_theo_he_qua}.
\begin{table}[H]
   \centering
   \caption{Bảng giá trị chân lí của mệnh đề với phép phủ định kéo theo và phủ định hệ quả}
   \label{tab:toan_hoc_nen_tang:lo_gich:menh_de_phuc_hop:dn_phu_dinh_keo_theo_he_qua}
   \begin{tabular}{|c|c!{\vrule width 1.5pt}c!{\vrule width 1.5pt}c|}
      \hline
      $P$ & $Q$ & $P \nRightarrow Q$ & $P \nLeftarrow Q$ \\
      \headerDivider
      Đ & Đ & S & S \\
      Đ & S & Đ & S \\
      S & Đ & S & Đ \\
      S & S & S & S \\
      \hline
   \end{tabular}
\end{table}

\subsection{Phép đẳng giá}

\ % Lùi đầu dòng

Giống như các trường đại học khác, U-ni-véc-xi-ta-tô cũng xét tốt nghiệp sinh viên theo số tín chỉ tích lũy và các môn điều kiện. Sinh viên $P$: được công nhận tốt nghiệp khi và chỉ khi $Q$: Sinh viên tích lũy đủ số tín chỉ và đã qua các môn điều kiện. Nếu như có sinh viên xong các điều kiện môn mà không được xét bằng, hoặc chưa học xong mà đã có bằng, thì \emphcolor{P tương đương Q} sai, và có một sự nhầm lẫn/gian lận nào đó ở đây. 

Phép này đại diện cho sự ``cùng tiến, cùng lùi''. Mệnh đề có chứa phép đẳng giá $\defMath{P \iff Q}$ đúng khi và chỉ khi cả hai mệnh đề con cấu thành $P$ và $Q$ cúng đúng hoặc cùng sai.
\begin{table}[H]
   \centering
   \caption{Bảng giá trị chân lí của mệnh đề với phép đẳng giá}
   \label{tab:toan_hoc_nen_tang:lo_gich:menh_de_phuc_hop:dn_tuong_duong}
   \begin{tabular}{|c|c!{\vrule width 1.5pt}c|}
      \hline
      $P$ & $Q$ & $P \iff Q$ \\
      \headerDivider 
      Đ & Đ & Đ \\
      Đ & S & S \\
      S & Đ & S \\
      S & S & Đ \\
      \hline
   \end{tabular}
\end{table}

Lấy một ví dụ tương tự, Ê-mô-ri cần phải lập trình thông báo lỗi của một hệ thống cơ bản với quy định: ``$P$: Đèn tín hiệu xanh, khi và chỉ khi $Q$: hệ thống đang chạy.''. Hệ thống có lỗi, hay vi phạm quy định, đồng nghĩa với \emphcolor{$P$ không đẳng giá với $Q$}, đèn xanh mà hệ thống không chạy, hoặc đèn tắt mà hệ thống vẫn chạy. Đây là hệ thống có vai trò vô cùng quan trọng, do nó cho cái nhìn tổng quát về tổng thể hoạt động hệ thống và giúp nhanh chóng phát hiện sự cố.

Phủ định đẳng giá $\defMath{P \nLeftrightarrow Q}$ thể hiện việc bác bỏ khẳng định rằng hai mệnh đề luôn cùng đúng hoặc cùng sai.

\begin{table}[H]
   \centering
   \caption{Bảng giá trị chân lí của mệnh đề với phép đẳng giá}
   \label{tab:toan_hoc_nen_tang:lo_gich:menh_de_phuc_hop:dn_phu_dinh_dang_gia}
   \begin{tabular}{|c|c!{\vrule width 1.5pt}c|}
      \hline
      $P$ & $Q$ & $P \nLeftrightarrow Q$ \\
      \headerDivider 
      Đ & Đ & S \\
      Đ & S & Đ \\
      S & Đ & Đ \\
      S & S & S \\
      \hline
   \end{tabular}
\end{table}

\subsection{Phép tuyển chọn}

\ % Lùi đầu dòng

Ê-mô-ri đi ăn cùng với một nhóm bạn trong một cửa hàng đồ ăn nhanh. Khi chọn một bộ các món ăn và uống, nhóm bạn được phép $P$: sử dụng nước cam, hoặc $Q$: sử dụng nước táo, tuy nhiên nhà hàng không cho phép chọn cả hai loại nước. Nhóm bạn của Ê-mô-ri, khi này, phải thực hiện \emphcolor{tuyển chọn $P$ hoặc $Q$}.

Khác với phép tuyển (hoặc sao cũng được), phép tuyển chọn $\defMath{P \oplus Q}$ bắt buộc phải chọn duy nhất một trong hai khả năng. Nếu chọn cả hai hoặc bỏ cả hai, mệnh đề $P \oplus Q$ sẽ sai.

\begin{table}[H]
   \centering
   \caption{Bảng giá trị chân lí của mệnh đề với phép tuyển chọn}
   \label{tab:toan_hoc_nen_tang:lo_gich:menh_de_phuc_hop:dn_tuyen_chon}
   \begin{tabular}{|c|c!{\vrule width 1.5pt}c|}
      \hline
      $P$ & $Q$ & $P \oplus Q$ \\
      \headerDivider 
      Đ & Đ & S \\
      Đ & S & Đ \\
      S & Đ & Đ \\
      S & S & S \\
      \hline
   \end{tabular}
\end{table}

Phép đối của phép tuyển chọn có tên là phép phủ định tuyển chọn. Khi chúng ta cần biểu diễn không phải là chọn một trong hai thì chúng ta sự dụng phép nối lô-gích này. Hai phép tuyển chọn và phủ định tuyển chọn thường được ghép chung vào cùng một nhóm, cho nên chúng cũng có kí hiệu gần giống nhau. Nếu như phép tuyển chọn được kí hiệu bằng một vòng tròn bao một dấu cộng, phép phủ định tuyển chọn được kí hiệu là một vòng tròn nhưng bao dấu nhân chấm: $\defMath{P \odot Q}$.

\begin{table}[H]
   \centering
   \caption{Bảng giá trị chân lí của mệnh đề với phép phủ định tuyển chọn}
   \label{tab:toan_hoc_nen_tang:lo_gich:menh_de_phuc_hop:dn_phu_dinh_tuyen_chon}
   \begin{tabular}{|c|c!{\vrule width 1.5pt}c|}
      \hline
      $P$ & $Q$ & $P \odot Q$ \\
      \headerDivider 
      Đ & Đ & Đ \\
      Đ & S & S \\
      S & Đ & S \\
      S & S & Đ \\
      \hline
   \end{tabular}
\end{table}

\subsection{Thứ tự giải giá trị chân lí của mệnh đề của các phép nối}

\ % Lùi đầu dòng

Giống như thứ tự các phép tính số học\footnote{[\dots], thứ mà con người phát minh ra [\dots]}để tính giá trị các biểu thức số học, để xác định giá trị chân lí của mệnh đề có nhiều phép nối, các phép nối mệnh đề cũng có sắp xếp thứ tự\footnote{[\dots]và vẫn chưa có sự thống nhất[\dots]} từ mức ưu tiên cao đến ưu tiên thấp như sau:
\begin{enumerate}[start=0]
   \item Thực hiện các phép bên trong cặp dấu ngoặc --- $()$ --- trước, ngoài ngoặc sau
   \item Phủ định --- $\neg$ (mức ưu tiên cao nhất);
   \item Nhóm hội --- Phép hội --- $\land$ --- và phép phủ định hội --- $\uparrow$;
   \item Nhóm tuyển --- Phép tuyển --- $\lor$ --- và phép phủ định tuyển --- $\downarrow$;
   \item Nhóm tuyển chọn --- Phép tuyển chọn --- $\oplus$ --- và phép phủ định tuyển chọn --- $\odot$;
   \item Nhóm suy luận --- Phép kéo theo --- $\implies$, phép hệ quả --- $\impliedby$, phép phủ định kéo theo --- $\nRightarrow$, phép phủ định hệ quả --- $\nLeftarrow$, phép đẳng giá --- $\iff$ --- và phép phủ định đẳng giá --- $\nLeftrightarrow$ (mức ưu tiên thấp nhất).
\end{enumerate}
Để xác định giá trị chân lí của một mệnh đề phức hợp gồm nhiều phép nối, thực hiện các phép từ mức ưu tiên cao nhất đến những phép có mức yêu tiên thấp hơn. Nếu các phép có cùng một mức độ ưu tiên thì thực hiện theo quy tắc \defText{kết hợp trái} --- thực hiện từ trái qua phải\footnote{}.

Cần phải để ý rằng trong nhiều tài liệu, người ta hay lạm dụng kí hiệu (hay ngôn ngữ) khi viết $P \implies Q \implies R$ và $P \iff Q \iff R$ với ngầm hiểu rằng hai mệnh đề này lần lượt là $(P \implies Q) \land (Q \implies R)$ và $(P \iff Q) \land (Q \iff R)$ chứ không phải là $(P \implies Q) \implies R$ và $(P \iff Q) \iff R$ .

\exercise[ex:toan_hoc_nen_tang:lo_gich:menh_de_phuc_hop:vd_bang_gia_tri_chan_li] Xây dựng bảng giá trị chân lí của các mệnh đề sau:
\begin{multicols}{2}
   \begin{enumerate}
      \item $\neg P \implies Q$;
      \item $(P \iff Q) \lor \neg Q$;
      \item $P \land (Q \implies R)$;
      \item $P \land Q \lor Q \land \neg R$;
      \item $Q \implies R \land R \lor \neg P$;
      \item $(P \implies Q) \implies (R \implies Q \implies P)$.
   \end{enumerate}
\end{multicols}

\solution[ex:toan_hoc_nen_tang:lo_gich:menh_de_phuc_hop:vd_bang_gia_tri_chan_li]

\setcounter{subexercise}{1}
\arabic{subexercise}. 
Để ý đến thứ tự các phép nối, dấu $\neg$ sẽ được thực hiện trước $\implies$. Thực hiện xây dựng như ở bảng sau.
\begin{table}[H]
   \centering
   \caption{Bảng giá trị chân lí cho bài \ref{ex:toan_hoc_nen_tang:lo_gich:menh_de_phuc_hop:vd_bang_gia_tri_chan_li} phần \arabic{subexercise}}
   \begin{tabular}{|c|c|c!{\vrule width 1.5pt}c|}
      \hline
      $P$ & $Q$ & $\neg P$ & $\neg P \implies Q$ \\
      \headerDivider
      Đ & Đ & S & Đ \\
      Đ & S & S & Đ \\
      S & Đ & S & Đ \\
      S & S & Đ & S \\
      \hline
   \end{tabular}
\end{table}

{
\begin{minipage}[c]{0.3\linewidth}
   \raggedright
   Có thể thực hiện viết bảng giá trị chân lí dưới dạng rút gọn như ở bảng 
   \ref{tab:toan_hoc_nen_tang:lo_gich:menh_de_phuc_hop:vd_bang_gia_tri_chan_li_1}. 
   Sau khi thực hiện xong một phép nối, có thể viết ở ngay dưới mệnh đề cần tìm 
   giá trị chân lí thay vì tách thành cột riêng để tiết kiệm giấy nếu như mệnh đề dài.
\end{minipage}%
\hfill
\begin{minipage}[c]{0.68\linewidth}
   \begin{table}[H]
      \centering
      \caption{Bảng giá trị chân lí rút gọn cho phần \arabic{subexercise} bài 
      \ref{ex:toan_hoc_nen_tang:lo_gich:menh_de_phuc_hop:vd_bang_gia_tri_chan_li}}
      \label{tab:toan_hoc_nen_tang:lo_gich:menh_de_phuc_hop:vd_bang_gia_tri_chan_li_1}
      \begin{tabular}{|c|c|cccc!{\vrule width 1.5pt}c|}
         \hline
         $P$ & $Q$ & $\neg$ & $P$ & $\implies$ & $Q$ & $\neg P \implies Q$ \\
         \headerDivider 
         Đ & Đ & S & Đ & \emphcolor{Đ} & Đ & Đ \\
         Đ & S & S & Đ & \emphcolor{Đ} & Đ & Đ \\
         S & Đ & Đ & S & \emphcolor{Đ} & Đ & Đ \\
         S & S & Đ & S & \emphcolor{S} & S & S \\
         \hline
      \end{tabular}
   \end{table}
\end{minipage}
}

\stepcounter{subexercise}
\arabic{subexercise}. 
\begin{table}[H]
   \centering
   \caption{Bảng giá trị chân lí rút gọn cho phần \arabic{subexercise} bài 
   \ref{ex:toan_hoc_nen_tang:lo_gich:menh_de_phuc_hop:vd_bang_gia_tri_chan_li}}
   \begin{tabular}{|c|c|cccccc!{\vrule width 1.5pt}c|}
      \hline
      $P$ & $Q$ & $(P$ & $\iff$ & $Q)$ & $\lor$ & $\neg$ & $Q$ & $(P \iff Q) \lor \neg Q$ \\
      \headerDivider
      Đ & Đ & Đ & Đ & Đ & \emphcolor{Đ} & S & Đ & Đ \\
      Đ & S & Đ & S & S & \emphcolor{Đ} & Đ & S & Đ \\
      S & Đ & S & S & Đ & \emphcolor{S} & S & Đ & S \\
      S & S & S & S & S & \emphcolor{Đ} & Đ & S & Đ \\
      \hline
   \end{tabular}
\end{table}

\stepcounter{subexercise}
\arabic{subexercise}. Khi một mệnh đề phức hợp cho cùng giá trị chân lí ở nhiều trường hợp khác nhau, và trong các trường hợp đó một số mệnh đề thành phần có cùng giá trị chân lí, chúng ta có thể rút gọn bảng bằng cách:
\begin{itemize}
   \item Giữ nguyên những mệnh đề thành phần có giá trị giống nhau;
   \item Thay những mệnh đề thành phần thay đổi bằng ký hiệu $X$.
\end{itemize}
Ví dụ như ở bảng \ref{tab:toan_hoc_nen_tang:lo_gich:menh_de_phuc_hop:vd_bang_gia_tri_chan_li_3}, nhận thấy rằng khi $P$ sai thì mệnh đề phức hợp luôn sai, nên các cột $Q$ và $R$ và những cột có phần mệnh đề liên quan đến hai mệnh đề này ở hàng thứ năm được thay bởi những chữ $X$.
\begin{table}[H]
   \centering
   \caption{Bảng giá trị chân lí rút gọn cho phần \arabic{subexercise} bài 
   \ref{ex:toan_hoc_nen_tang:lo_gich:menh_de_phuc_hop:vd_bang_gia_tri_chan_li}}
   \label{tab:toan_hoc_nen_tang:lo_gich:menh_de_phuc_hop:vd_bang_gia_tri_chan_li_3}
   \begin{tabular}{|c|c|c|ccccc!{\vrule width 1.5pt}c|}
      \hline
      $P$ & $Q$ & $R$ & $P$ & $\land$ & $(Q$ & $\implies$ & $R)$ & $P \land (Q \implies R)$ \\
      \headerDivider
      Đ & Đ & Đ & Đ & \emphcolor{Đ} & Đ & Đ & Đ & Đ \\
      Đ & Đ & S & Đ & \emphcolor{S} & Đ & S & S & S \\
      Đ & S & Đ & Đ & \emphcolor{Đ} & S & Đ & Đ & Đ \\
      Đ & S & S & Đ & \emphcolor{Đ} & S & Đ & S & Đ \\
      S & X & X & S & \emphcolor{S} & X & X & X & S \\
      \hline
   \end{tabular}
\end{table}

\stepcounter{subexercise}
\arabic{subexercise}. 
\begin{table}[H]
   \centering
   \caption{Bảng giá trị chân lí rút gọn cho phần \arabic{subexercise} bài 
   \ref{ex:toan_hoc_nen_tang:lo_gich:menh_de_phuc_hop:vd_bang_gia_tri_chan_li}}
   \begin{tabular}{|c|c|c|cccccccc!{\vrule width 1.5pt}c|}
      \hline
      $P$ & $Q$ & $R$ & $P$ & $\land$ & $Q$ & $\lor$ & $Q$ & $\land$ & $\neg$ & $R$ & $P \land Q \lor Q \land \neg R$ \\
      \headerDivider
      Đ & Đ & Đ & Đ & Đ & Đ & \emphcolor{Đ} & Đ & S & S & Đ & Đ \\
      Đ & Đ & S & Đ & Đ & Đ & \emphcolor{Đ} & Đ & Đ & Đ & Đ & Đ \\
      S & Đ & Đ & S & S & Đ & \emphcolor{S} & Đ & S & S & Đ & S \\
      S & Đ & S & S & S & Đ & \emphcolor{Đ} & Đ & Đ & Đ & S & Đ \\
      X & S & X & X & S & S & \emphcolor{S} & S & S & X & X & S \\
      \hline
   \end{tabular}
\end{table}

\stepcounter{subexercise}
\arabic{subexercise}. 
\begin{table}[H]
   \centering
   \caption{Bảng giá trị chân lí rút gọn cho phần \arabic{subexercise} bài 
   \ref{ex:toan_hoc_nen_tang:lo_gich:menh_de_phuc_hop:vd_bang_gia_tri_chan_li}}
   \begin{tabular}{|c|c|c|cccccccc!{\vrule width 1.5pt}c|}
      \hline
      $P$ & $Q$ & $R$ & $Q$ & $\implies$ & $R$ & $\land$ & $R$ & $\lor$ & $\neg$ & $P$ & $Q \implies R \land R \lor \neg P$ \\
      \headerDivider
      Đ & Đ & Đ & Đ & \emphcolor{Đ} & Đ & Đ & Đ & Đ & S & Đ & Đ \\
      Đ & Đ & S & Đ & \emphcolor{S} & S & S & S & S & S & Đ & S \\
      S & Đ & Đ & Đ & \emphcolor{Đ} & Đ & Đ & Đ & Đ & Đ & S & Đ \\
      S & Đ & S & Đ & \emphcolor{Đ} & S & S & S & Đ & Đ & S & Đ \\
      X & S & X & S & \emphcolor{Đ} & X & X & X & X & X & X & Đ \\
      \hline
   \end{tabular}
\end{table}

\stepcounter{subexercise}
\arabic{subexercise}. Do giới hạn của khổ giấy, bảng giá trị chân lí sẽ được tách làm hai phần.
\begin{table}[H]
   \centering
   \caption{Bảng giá trị chân lí rút gọn cho phần \arabic{subexercise} bài 
   \ref{ex:toan_hoc_nen_tang:lo_gich:menh_de_phuc_hop:vd_bang_gia_tri_chan_li}}
   \label{tab:toan_hoc_nen_tang:lo_gich:menh_de_phuc_hop:vd_bang_gia_tri_chan_li_6}
   \begin{tabular}{|c|c|c|ccccccccc|}
      \hline
      $P$ & $Q$ & $R$ & $(P$ & $\implies$ & $Q)$ & $\implies$ & $(R$ & $\implies$ & $Q$ & $\implies$ & $P)$ \\
      \headerDivider
      Đ & X & X & Đ & X & X & \emphcolor{Đ} & X & X & X & Đ & Đ \\
      S & Đ & X & S & Đ & Đ & \emphcolor{S} & X & Đ & Đ & S & S \\
      S & S & Đ & S & Đ & S & \emphcolor{Đ} & Đ & S & S & Đ & S \\
      S & S & S & S & Đ & S & \emphcolor{S} & S & Đ & S & S & S \\
      \hline
   \end{tabular}
   \begin{tabular}{|c|c|c!{\vrule width 1.5pt}c|}
      \hline
      $P$ & $Q$ & $R$ & $(P \implies Q) \implies (R \implies Q \implies R) $ \\
      \headerDivider
      Đ & X & X & Đ \\
      S & Đ & X & S \\
      S & S & Đ & Đ \\
      S & S & S & S \\
      \hline
   \end{tabular}
\end{table}
   
\exercise Gọi $V$ là một mệnh đề đúng và $M$ là một mệnh đề sai nào đó. Sử dụng bảng giá trị chân lí, chứng minh các mệnh đề sau luôn đúng\footnote{Mệnh đề luôn đúng còn được gọi là \defText{mệnh đề hằng đúng}.} với mọi giá trị chân lí của $P$, $Q$ và $R$.
\begin{itemize}
   \item $P \lor V$ (tính chất thống trị với phép tuyển);
   \item $\overline{P \land M}$ (tính chất thống trị với phép hội);
   \item $P \land V \iff P$ (tính chất\footnote{Các tính chất còn được gọi là \defText{luật}.} đồng nhất với phép hội);
   \item $P \lor M \iff P$ (tính chất đồng nhất với phép tuyển);
   \item $P \lor \neg P$ (tính chất loại trừ trung gian\footnote{Tính chất này chỉ rằng một mệnh đề hoặc đúng, hoặc phủ định của nó đúng; không tồn tại khả năng trung gian.});
   \item $\overline{P \land \neg P}$ (tính chất không mâu thuẫn\footnote{Tính chất này khẳng định Không thể có một mệnh đề vừa đúng vừa sai cùng lúc và theo cùng một nghĩa.});
   \item $\overline{\neg P} \iff P$ (tính chất phủ định kép);
   \item $P \land P \iff P$ (tính chất lũy đẳng với phép hội);
   \item $P \lor P \iff P$ (tính chất lũy đẳng với phép tuyển);
   \item $P \land Q \iff Q \land P$ (tính giao hoán với phép hội);
   \item $P \lor Q \iff Q \lor P$ (tính giao hoán với phép tuyển);
   \item $\overline{P\lor Q} \iff \neg P \land \neg Q$ (định luật Đờ Moóc-gơn\footnote{Augustus De Morgan (1806 -- 1871)}, phần 1);
   \item $\overline{P\land Q} \iff \neg P \lor \neg Q$ (định luật Đờ Moóc-gơn, phần 2);
   \item $P \implies P \lor Q$ (tính chất cộng);
   \item $P \land Q \implies P$ (tính chất rút gọn);
   \item $P \implies Q \iff \neg P \lor Q$ (định nghĩa phép kéo theo);
   \item $\left(P \implies Q\right) \iff \left(\neg P \impliedby \neg Q\right)$ (tính chất phản đảo);
   \item $\left(P \implies Q\right) \land P \implies Q$ (quy tắc khẳng định\footnote{Modus ponens.});
   \item $\left(P \implies Q\right) \land \neg Q \implies \neg P$ (quy tắc phủ định\footnote{Modus tollens.});
   \item $\left(P \lor Q\right) \land \neg P \implies Q$ (tam đoạn luận tuyển);
   \item $P \iff Q \iff P \land Q \lor \neg P \land \neg Q$ (định nghĩa phép tương đương, phần 1);
   \item $\left(P \iff Q\right) \iff \left(P \implies Q\right) \land \left(P \impliedby Q\right)$ (định nghĩa phép tương đương, phần 2);
   \item $\left(P \iff Q\right) \iff \left(Q \iff P\right)$ (tính chất giao hoán với phép tương đương);
   \item $\left(P \land Q\right) \land R \iff P \land \left(Q \land R\right)$ (tính chất kết hợp với phép hội);
   \item $\left(P \lor Q\right) \lor R \iff P \lor \left(Q \lor R\right)$ (tính chất kết hợp với phép tuyển);
   \item $\left(\left(P \iff Q\right) \iff R \right) \iff \left(P \iff \left(Q \iff R\right)\right)$ (tính chất kết hợp với phép tương đương);
   \item $\left(P \lor Q\right) \land \left(P \lor R\right) \iff P \lor (Q \land R)$ (tính chất phân phối, phần 1);
   \item $\left(P \land Q\right) \lor \left(P \land R\right) \iff P \land (Q \lor R)$ (tính chất phân phối, phần 2);
   \item $\left(P \implies Q\right) \land \left(P \implies R\right) \iff (P \implies Q \land R)$ (tính chất phân phối, phần 3);
   \item $\left(P \implies Q\right) \lor \left(P \implies R\right) \iff (P \implies Q \lor R)$ (tính chất phân phối, phần 4);
   \item $\left(P \implies Q\right) \land \left(Q \implies R\right) \implies \left(P \implies R\right)$ (tính chất bắc cầu).
\end{itemize}

\solution

\begin{table}[H]
   \centering
   \caption{Bảng giá trị chân lí của $P \lor V$ và $\overline{P \land M}$}
   \begin{tabular}{|c|ccc!{\vrule width 1.5pt}c|cccc!{\vrule width 1.5pt}c|}
      \hline
      $P$ & $P$ & $\lor$ & $V$ & $P \lor V$ & $\neg$ & $(P$ & $\land$ & $M)$ & $\overline{P \land M}$ \\
      \headerDivider
      X & X & \emphcolor{Đ} & Đ & Đ & \emphcolor{Đ} & X & S & S & Đ \\
      \hline
   \end{tabular}
\end{table}

\begin{table}[H]
   \centering
   \caption{Bảng giá trị chân lí của $P \land V \iff P$ và $P \lor M \iff P$}
   \begin{tabular}{|c|ccccc!{\vrule width 1.5pt}c|ccccc!{\vrule width 1.5pt}c|}
      \hline
      $P$ & $P$ & $\land$ & $V$ & $\iff$ & $P$ & $P \land V \iff P$ & $P$ & $\lor$ & $M$ & $\iff$ & $P$ & $P \lor M \iff P$ \\
      \headerDivider
      Đ & Đ & Đ & Đ & \emphcolor{Đ} & Đ & Đ & Đ & Đ & S & \emphcolor{Đ} & Đ & Đ \\
      S & S & S & Đ & \emphcolor{Đ} & S & Đ & S & S & S & \emphcolor{Đ} & S & Đ \\
      \hline
   \end{tabular}
\end{table}

\begin{table}[H]
   \centering
   \caption{Bảng giá trị chân lí của $P \lor \neg P$ và $\overline{P \land \neg P}$}
   \begin{tabular}{|c|cccc!{\vrule width 1.5pt}c|ccccc!{\vrule width 1.5pt}c|}
      \hline
      $P$ & $P$ & $\lor$ & $\neg$ & $P$ & $P \lor \neg P$ & $\neg$ & $(P$ & $\land$ & $\neg$ & $P)$ & $\overline{P \land \neg P}$ \\
      \headerDivider
      Đ & Đ & \emphcolor{Đ} & S & Đ & Đ & \emphcolor{Đ} & Đ & S & S & Đ & Đ \\
      S & S & \emphcolor{Đ} & Đ & S & Đ & \emphcolor{Đ} & S & S & Đ & S & Đ \\
      \hline
   \end{tabular}
\end{table}

\begin{table}[H]
   \centering
   \caption{Bảng giá trị chân lí của $\overline{\neg P} \iff P$}
   \begin{tabular}{|c|ccccc!{\vrule width 1.5pt}c|}
      \hline
      $P$ & $\neg$ & $(\neg$ & $P)$ & $\iff$ & $P$ & $\overline{\neg P} \iff P$ \\
      \headerDivider
      Đ & Đ & S & Đ & \emphcolor{Đ} & Đ & Đ \\
      S & S & Đ & S & \emphcolor{Đ} & S & Đ \\
      \hline
   \end{tabular}
\end{table}

\begin{table}[H]
   \centering
   \caption{Bảng giá trị chân lí của $P \land P \iff P$ và $P \lor P \iff P$}
   \begin{tabular}{|c|ccccc!{\vrule width 1.5pt}c|ccccc!{\vrule width 1.5pt}c|}
      \hline
      $P$ & $P$ & $\land$ & $P$ & $\iff$ & $P$ & $P \land P \iff P$ & $P$ & $\lor$ & $P$ & $\iff$ & $P$ & $P \lor P \iff P$ \\
      \headerDivider
      Đ & Đ & Đ & Đ & \emphcolor{Đ} & Đ & Đ & Đ & Đ & Đ & \emphcolor{Đ} & Đ & Đ \\
      S & S & S & S & \emphcolor{Đ} & S & Đ & S & S & S & \emphcolor{Đ} & S & Đ \\
      \hline
   \end{tabular}
\end{table}

\begin{table}[H]
   \centering
   \caption{Bảng giá trị chân lí của $P \land Q \iff Q \land P$}
   \begin{tabular}{|c|c|ccccccc!{\vrule width 1.5pt}c|}
      \hline
      $P$ & $Q$ & $P$ & $\land$ & $Q$ & $\iff$ & $Q$ & $\land$ & $P$ & $P \land Q \iff Q \land P$ \\
      \headerDivider
      Đ & Đ & Đ & Đ & Đ & \emphcolor{Đ} & Đ & Đ & Đ & Đ \\
      Đ & S & Đ & S & S & \emphcolor{Đ} & S & S & Đ & Đ \\
      S & Đ & S & S & Đ & \emphcolor{Đ} & Đ & S & S & Đ \\
      S & S & S & S & S & \emphcolor{Đ} & S & S & S & Đ \\
      \hline
   \end{tabular}
\end{table}

\begin{table}[H]
   \centering
   \caption{Bảng giá trị chân lí của $P \lor Q \iff Q \lor P$}
   \begin{tabular}{|c|c|ccccccc!{\vrule width 1.5pt}c|}
      \hline
      $P$ & $Q$ & $P$ & $\lor$ & $Q$ & $\iff$ & $Q$ & $\lor$ & $P$ & $P \lor Q \iff Q \lor P$ \\
      \headerDivider
      Đ & Đ & Đ & Đ & Đ & \emphcolor{Đ} & Đ & Đ & Đ & Đ \\
      Đ & S & Đ & Đ & S & \emphcolor{Đ} & S & Đ & Đ & Đ \\
      S & Đ & S & Đ & Đ & \emphcolor{Đ} & Đ & Đ & S & Đ \\
      S & S & S & S & S & \emphcolor{Đ} & S & S & S & Đ \\
      \hline
   \end{tabular}
\end{table}

\begin{table}[H]
   \centering
   \caption{Bảng giá trị chân lí của $\overline{P \lor Q} \iff \neg P \land \neg Q$}
   \begin{tabular}{|c|c|cccccccccc!{\vrule width 1.5pt}c|}
      \hline
      $P$ & $Q$ & $\neg$ & $(P$ & $\lor$ & $Q)$ & $\iff$ & $\neg$ & $P$ & $\land$ & $\neg$ & $Q$ & $\overline{P \lor Q} \iff \neg P \land \neg Q$ \\
      \headerDivider
      Đ & Đ & S & Đ & Đ & Đ & \emphcolor{Đ} & S & Đ & S & S & Đ & Đ \\
      Đ & S & S & Đ & Đ & S & \emphcolor{Đ} & S & Đ & S & Đ & S & Đ \\
      S & Đ & S & S & Đ & Đ & \emphcolor{Đ} & Đ & S & S & S & Đ & Đ \\
      S & S & Đ & S & S & S & \emphcolor{Đ} & Đ & S & Đ & Đ & S & Đ \\
      \hline
   \end{tabular}
\end{table}

\begin{table}[H]
   \centering
   \caption{Bảng giá trị chân lí của $\overline{P \land Q} \iff \neg P \lor \neg Q$}
   \begin{tabular}{|c|c|cccccccccc!{\vrule width 1.5pt}c|}
      \hline
      $P$ & $Q$ & $\neg$ & $(P$ & $\land$ & $Q)$ & $\iff$ & $\neg$ & $P$ & $\lor$ & $\neg$ & $Q$ & $\overline{P \land Q} \iff \neg P \lor \neg Q$ \\
      \headerDivider
      Đ & Đ & S & Đ & Đ & Đ & \emphcolor{Đ} & S & Đ & S & S & Đ & Đ \\
      Đ & S & Đ & Đ & S & S & \emphcolor{Đ} & S & Đ & Đ & Đ & S & Đ \\
      S & Đ & Đ & S & S & Đ & \emphcolor{Đ} & Đ & S & Đ & S & Đ & Đ \\
      S & S & Đ & S & S & S & \emphcolor{Đ} & Đ & S & Đ & Đ & S & Đ \\
      \hline
   \end{tabular}
\end{table}

\begin{table}[H]
   \centering
   \caption{Bảng giá trị chân lí của $P \implies P \lor Q$}
   \begin{tabular}{|c|c|ccccc!{\vrule width 1.5pt}c|}
      \hline
      $P$ & $Q$ & $P$ & $\implies$ & $P$ & $\lor$ & $Q$ & $P \implies P \lor Q$ \\
      \headerDivider
      Đ & X & Đ & \emphcolor{Đ} & Đ & Đ & X & Đ \\
      S & X & S & \emphcolor{Đ} & S & X & X & Đ \\
      \hline 
   \end{tabular}
\end{table}

\begin{table}[H]
   \centering
   \caption{Bảng giá trị chân lí của $P \land Q \implies P$}
   \begin{tabular}{|c|c|ccccc!{\vrule width 1.5pt}c|}
      \hline
      $P$ & $Q$ & $P$ & $\land$ & $Q$ & $\implies$ & $P$ & $P \land Q \implies P$ \\
      \headerDivider
      Đ & X & Đ & X & X & \emphcolor{Đ} & Đ & Đ \\
      S & X & S & S & X & \emphcolor{Đ} & S & Đ \\
      \hline 
   \end{tabular}
\end{table}

\begin{table}[H]
   \centering
   \caption{Bảng giá trị chân lí của $P \implies Q \iff \neg P \lor Q$}
   \begin{tabular}{|c|c|cccccccc!{\vrule width 1.5pt}c|}
      \hline
      $P$ & $Q$ & $P$ & $\implies$ & $Q$ & $\iff$ & $\neg$ & $P$ & $\lor$ & $Q$ & $P \implies Q \iff \neg P \lor Q$ \\
      \headerDivider
      Đ & Đ & Đ & Đ & Đ & \emphcolor{Đ} & S & Đ & Đ & Đ & Đ \\
      Đ & S & Đ & S & S & \emphcolor{Đ} & S & Đ & S & S & Đ \\
      S & X & S & Đ & X & \emphcolor{Đ} & Đ & S & Đ & X & Đ \\
      \hline
   \end{tabular}
\end{table}

\begin{table}[H]
   \centering
   \caption{Bảng giá trị chân lí của $\left(P \implies Q\right) \iff \left(\neg P \impliedby \neg Q\right)$}
   \begin{tabular}{|c|c|ccccccccc!{\vrule width 1.5pt}c|}
      \hline
      $P$ & $Q$ & $(P$ & $\implies$ & $Q)$ & $\iff$ & $(\neg$ & $P$ & $\impliedby$ & $\neg$ & $Q)$ & $\left(P \implies Q\right) \iff \left(\neg P \impliedby \neg Q\right)$ \\
      \headerDivider
      Đ & Đ & Đ & Đ & Đ & \emphcolor{Đ} & S & Đ & Đ & S & Đ & Đ \\
      Đ & S & Đ & S & S & \emphcolor{Đ} & S & Đ & S & Đ & S & Đ \\
      S & X & S & Đ & X & \emphcolor{Đ} & Đ & S & Đ & X & X & Đ \\
      \hline
   \end{tabular}
\end{table}

\begin{table}[H]
   \centering
   \caption{Bảng giá trị chân lí của $\left(P \implies Q\right) \land P \implies Q$}
   \begin{tabular}{|c|c|ccccccc!{\vrule width 1.5pt}c|}
      \hline
      $P$ & $Q$ & $(P$ & $\implies$ & $Q)$ & $\land$ & $P$ & $\implies$ & $Q$ & $\left(P \implies Q\right) \land P \implies Q$ \\
      \headerDivider
      Đ & S & Đ & S & S & S & Đ & \emphcolor{Đ} & S & Đ \\
      S & S & S & Đ & Đ & S & S & \emphcolor{Đ} & S & Đ \\
      X & Đ & X & Đ & Đ & X & X & \emphcolor{Đ} & Đ & Đ \\
      \hline 
   \end{tabular}
\end{table}

\begin{table}[H]
   \centering
   \caption{Bảng giá trị chân lí của $\left(P \implies Q\right) \land \neg Q \implies \neg P$}
   \begin{tabular}{|c|c|ccccccccc!{\vrule width 1.5pt}c|}
      \hline
      $P$ & $Q$ & $(P$ & $\implies$ & $Q)$ & $\land$ & $\neg$ & $Q$ & $\implies$ & $\neg$ & $P$ & $\left(P \implies Q\right) \land \neg Q \implies \neg P$ \\
      \headerDivider
      Đ & Đ & Đ & Đ & Đ & S & S & Đ & \emphcolor{Đ} & S & Đ & Đ \\
      Đ & S & Đ & S & S & S & Đ & S & \emphcolor{Đ} & S & Đ & Đ \\
      S & X & S & Đ & X & X & X & X & \emphcolor{Đ} & Đ & S & Đ \\
      \hline 
   \end{tabular}
\end{table}

\begin{table}[H]
   \centering
   \caption{Bảng giá trị chân lí của $\left(P \lor Q\right) \land \neg P \implies Q$}
   \begin{tabular}{|c|c|cccccccc!{\vrule width 1.5pt}c|}
      \hline
      $P$ & $Q$ & $(P$ & $\lor$ & $Q)$ & $\land$ & $\neg$ & $P$ & $\implies$ & $Q$ & $\left(P \lor Q\right) \land \neg P \implies Q$ \\
      \headerDivider
      Đ & S & Đ & Đ & S & S & S & Đ & \emphcolor{Đ} & S & Đ \\
      S & S & S & S & S & S & Đ & S & \emphcolor{Đ} & S & Đ \\
      X & Đ & X & Đ & Đ & X & X & X & \emphcolor{Đ} & Đ & Đ \\
      \hline
   \end{tabular}
\end{table}

\begin{table}[H]
   \centering
   \caption{Bảng giá trị chân lí của $P \iff Q \iff P \land Q \lor \neg P \land \neg Q$}
   \begin{tabular}{|c|c|ccccccccccccc|}
      \hline
      $P$ & $Q$ & $P$ & $\iff$ & $Q$ & $\iff$ & $P$ & $\land$ & $Q$ & $\lor$ & $\neg$ & $P$ & $\land$ & $\neg$ & $Q$ \\
      \headerDivider 
      Đ & Đ & Đ & Đ & Đ & \emphcolor{Đ} & Đ & Đ & Đ & Đ & S & Đ & S & S & Đ \\
      Đ & S & Đ & S & S & \emphcolor{Đ} & Đ & S & S & S & S & Đ & S & Đ & S \\
      S & Đ & S & S & Đ & \emphcolor{Đ} & S & S & Đ & S & Đ & S & S & S & Đ \\
      S & S & S & Đ & S & \emphcolor{Đ} & S & S & S & Đ & S & Đ & Đ & S & Đ \\
      \hline
   \end{tabular}
   \begin{tabular}{|c|c!{\vrule width 1.5pt}c|}
      \hline
      $P$ & $Q$ & $P \iff Q \iff P \land Q \lor \neg P \land \neg Q$ \\
      \headerDivider
      Đ & Đ & \multirow{4}{*}{\huge \vspace{-12pt} Đ} \\
      Đ & S & \\
      S & Đ & \\
      S & S & \\
      \hline
   \end{tabular}
\end{table}

\begin{table}[H]
   \centering
   \caption{Bảng giá trị chân lí của $\left(P \iff Q\right) \iff \left(P \implies Q\right) \land \left(P \impliedby Q\right)$}
   \begin{tabular}{|c|c|ccccccccccc|}
      \hline
      $P$ & $Q$ & $(P$ & $\iff$ & $Q)$ & $\iff$ & $(P$ & $\implies$ & $Q)$ & $\land$ & $(P$ & $\impliedby$ & $Q)$ \\
      \headerDivider 
      Đ & Đ & Đ & Đ & Đ & \emphcolor{Đ} & Đ & Đ & Đ & Đ & Đ & Đ & Đ \\
      Đ & S & Đ & S & S & \emphcolor{Đ} & Đ & S & S & S & Đ & Đ & S \\
      S & Đ & S & S & Đ & \emphcolor{Đ} & S & Đ & Đ & S & Đ & S & S \\
      S & S & S & Đ & S & \emphcolor{Đ} & S & Đ & S & Đ & S & Đ & S \\
      \hline
   \end{tabular}
   \begin{tabular}{|c|c!{\vrule width 1.5pt}c|}
      \hline
      $P$ & $Q$ & $\left(P \iff Q\right) \iff \left(P \implies Q\right) \land \left(P \impliedby Q\right)$ \\
      \headerDivider
      Đ & Đ & \multirow{4}{*}{\huge \vspace{-12pt} Đ} \\
      Đ & S & \\
      S & Đ & \\
      S & S & \\
      \hline
   \end{tabular}
\end{table}

\begin{table}[H]
   \centering
   \caption{Bảng giá trị chân lí của $\left(P \iff Q\right) \iff \left(Q \iff P\right)$}
   \begin{tabular}{|c|c|ccccccc!{\vrule width 1.5pt}c|}
      \hline
      $P$ & $Q$ & $(P$ & $\iff$ & $Q)$ & $\iff$ & $(Q$ & $\iff$ & $P)$ & $\left(P \iff Q\right) \iff \left(Q \iff P\right)$ \\
      \headerDivider
      Đ & Đ & Đ & Đ & Đ & \emphcolor{Đ} & Đ & Đ & Đ & Đ \\
      Đ & S & Đ & S & S & \emphcolor{Đ} & S & S & Đ & Đ \\
      S & Đ & S & S & Đ & \emphcolor{Đ} & Đ & S & S & Đ \\
      S & S & S & Đ & S & \emphcolor{Đ} & S & Đ & S & Đ \\
      \hline
   \end{tabular}
\end{table}

\begin{table}[H]
   \centering
   \caption{Bảng giá trị chân lí của $\left(P \land Q\right) \land R \iff P \land (Q \land R)$}
   \begin{tabular}{|c|c|c|ccccccccccc!{\vrule width 1.5pt}c|}
      \hline
      $P$ & $Q$ & $R$ & $(P$ & $\land$ & $Q)$ & $\land$ & $R$ & $\iff$ & $P$ & $\land$ & $(Q$ & $\land$ & $R)$ & $\left(P \land Q\right) \land R \iff P \land (Q \land R)$\\
      \headerDivider
      Đ & Đ & Đ & Đ & Đ & Đ & Đ & Đ & \emphcolor{Đ} & Đ & Đ & Đ & Đ & Đ & Đ \\
      S & X & X & S & S & X & S & X & \emphcolor{Đ} & S & S & X & X & X & Đ \\
      X & S & X & X & S & S & S & X & \emphcolor{Đ} & X & S & S & S & X & Đ \\
      X & X & S & X & X & X & S & S & \emphcolor{Đ} & X & S & X & S & S & Đ \\
      \hline
   \end{tabular}
\end{table}

\begin{table}[H]
   \centering
   \caption{Bảng giá trị chân lí của $\left(P \lor Q\right) \lor R \iff P \lor (Q \lor R)$}
   \begin{tabular}{|c|c|c|ccccccccccc!{\vrule width 1.5pt}c|}
      \hline
      $P$ & $Q$ & $R$ & $(P$ & $\lor$ & $Q)$ & $\lor$ & $R$ & $\iff$ & $P$ & $\lor$ & $(Q$ & $\lor$ & $R)$ & $\left(P \lor Q\right) \lor R \iff P \lor (Q \lor R)$\\
      \headerDivider
      S & S & S & S & S & S & S & S & \emphcolor{Đ} & S & S & S & S & S & Đ \\
      Đ & X & X & Đ & Đ & X & Đ & X & \emphcolor{Đ} & Đ & Đ & X & X & X & Đ \\
      X & Đ & X & X & Đ & Đ & Đ & X & \emphcolor{Đ} & X & Đ & Đ & Đ & X & Đ \\
      X & X & Đ & X & X & X & Đ & Đ & \emphcolor{Đ} & X & Đ & X & Đ & Đ & Đ \\
      \hline
   \end{tabular}
\end{table}

\begin{table}[H]
	\centering
	\caption{Bảng giá trị chân lí của $( ( P \iff Q ) \iff R ) \iff ( P \iff ( Q \iff R ) )$}
	\begin{tabular}{|c|c|c|ccccccccccc|}
		\hline
		P & Q & R & $((P$ & $\iff$ & $Q)$ & $\iff$ & $R)$ & $\iff$ & $(P$ & $\iff$ & $(Q$ & $\iff$ & $R))$ \\
		\headerDivider
		Đ & Đ & Đ & Đ & Đ & Đ & Đ & Đ & \emphcolor{Đ} & Đ & Đ & Đ & Đ & Đ \\
		Đ & Đ & S & Đ & Đ & Đ & S & S & \emphcolor{Đ} & Đ & S & Đ & S & S \\
		Đ & S & Đ & Đ & S & S & S & Đ & \emphcolor{Đ} & Đ & S & S & S & Đ \\
		Đ & S & S & Đ & S & S & Đ & S & \emphcolor{Đ} & Đ & Đ & S & Đ & S \\
		S & Đ & Đ & S & S & Đ & S & Đ & \emphcolor{Đ} & S & S & Đ & Đ & Đ \\
		S & Đ & S & S & S & Đ & Đ & S & \emphcolor{Đ} & S & Đ & Đ & S & S \\
		S & S & Đ & S & Đ & S & Đ & Đ & \emphcolor{Đ} & S & Đ & S & S & Đ \\
		S & S & S & S & Đ & S & S & S & \emphcolor{Đ} & S & S & S & Đ & S \\
		\hline
	\end{tabular}
   \begin{tabular}{|c|c|c!{\vrule width 1.5pt}c|}
		\hline
		P & Q & R & $( ( P \iff Q ) \iff R ) \iff ( P \iff ( Q \iff R ) )$ \\
		\headerDivider
		Đ & Đ & Đ & \multirow{8}{*}{\huge \vspace{-24pt} Đ}\\
		Đ & Đ & S & \\
		Đ & S & Đ & \\
		Đ & S & S & \\
		S & Đ & Đ & \\
		S & Đ & S & \\
		S & S & Đ & \\
		S & S & S & \\
		\hline
	\end{tabular}
\end{table}

\begin{table}[H]
	\centering
	\caption{Bảng giá trị chân lí của $( P \lor Q ) \land ( P \lor R ) \iff P \lor ( Q \land R )$}
	\begin{tabular}{|c|c|c|ccccccccccccc|}
		\hline
		P & Q & R & $(P$ & $\lor$ & $Q)$ & $\land$ & $(P$ & $\lor$ & $R)$ & $\iff$ & $P$ & $\lor$ & $(Q$ & $\land$ & $R)$ \\
		\headerDivider
		Đ & X & X & Đ & Đ & X & Đ & Đ & Đ & X & \emphcolor{Đ} & Đ & Đ & X & X & X \\
		S & S & X & S & S & S & S & S & X & X & \emphcolor{Đ} & S & S & S & S & X \\
		S & Đ & Đ & S & Đ & Đ & Đ & S & Đ & Đ & \emphcolor{Đ} & S & Đ & Đ & Đ & Đ \\
		S & Đ & S & S & Đ & Đ & S & S & S & S & \emphcolor{Đ} & S & S & Đ & S & S \\
		\hline
	\end{tabular}
   \begin{tabular}{|c|c|c!{\vrule width 1.5pt}c|}
		\hline
		P & Q & R & $( P \lor Q ) \land ( P \lor R ) \iff P \lor ( Q \land R )$ \\
		\headerDivider
		Đ & X & X & \multirow{4}{*}{\huge \vspace{-12pt} Đ} \\
		S & S & X & \\
		S & Đ & Đ & \\
		S & Đ & S & \\
		\hline
	\end{tabular}
\end{table}

\begin{table}[H]
	\centering
	\caption{Bảng giá trị chân lí của $( P \land Q ) \lor ( P \land R ) \iff P \land ( Q \lor R )$}
	\begin{tabular}{|c|c|c|ccccccccccccc|}
		\hline
		P & Q & R & $(P$ & $\land$ & $Q)$ & $\lor$ & $(P$ & $\land$ & $R)$ & $\iff$ & $P$ & $\land$ & $(Q$ & $\lor$ & $R)$ \\
		\headerDivider
		S & X & X & S & S & X & S & S & S & X & \emphcolor{Đ} & S & S & X & X & X \\
		Đ & Đ & X & Đ & Đ & Đ & Đ & Đ & X & X & \emphcolor{Đ} & Đ & Đ & Đ & Đ & X \\
		Đ & S & Đ & Đ & S & S & Đ & Đ & Đ & Đ & \emphcolor{Đ} & Đ & Đ & S & Đ & Đ \\
		Đ & S & S & Đ & S & S & S & Đ & S & S & \emphcolor{Đ} & Đ & S & S & S & S \\
		\hline
	\end{tabular}
   \begin{tabular}{|c|c|c!{\vrule width 1.5pt}c|}
		\hline
		P & Q & R & $( P \land Q ) \lor ( P \land R ) \iff P \land ( Q \lor R )$ \\
		\headerDivider
		S & X & X & \multirow{4}{*}{\huge \vspace{-12pt} Đ} \\
		Đ & Đ & X & \\
		Đ & S & Đ & \\
		Đ & S & S & \\
		\hline
	\end{tabular}
\end{table}

\begin{table}[H]
	\centering
	\caption{Bảng giá trị chân lí của $( P \implies Q ) \land ( P \implies R ) \iff ( P \implies  Q \land R  )$}
	\begin{tabular}{|c|c|c|ccccccccccccc|}
		\hline
		P & Q & R & $(P$ & $\implies$ & $Q)$ & $\land$ & $(P$ & $\implies$ & $R)$ & $\iff$ & $(P$ & $\implies$ & $Q$ & $\land$ & $R)$ \\
		\headerDivider
		S & X & X & S & Đ & X & Đ & S & Đ & X & \emphcolor{Đ} & S & Đ & X & X & X \\
		Đ & S & X & Đ & S & S & S & Đ & X & X & \emphcolor{Đ} & Đ & S & S & S & X \\
		Đ & Đ & Đ & Đ & Đ & Đ & Đ & Đ & Đ & Đ & \emphcolor{Đ} & Đ & Đ & Đ & Đ & Đ \\
		Đ & Đ & S & Đ & Đ & Đ & S & Đ & S & S & \emphcolor{Đ} & Đ & S & Đ & S & S \\
		\hline
	\end{tabular}
   \begin{tabular}{|c|c|c!{\vrule width 1.5pt}c|}
		\hline
		P & Q & R & $( P \implies Q ) \land ( P \implies R ) \iff ( P \implies  Q \land R  )$ \\
		\headerDivider
		Đ & X & X & \multirow{4}{*}{\huge \vspace{-12pt} Đ} \\
		Đ & S & X & \\
		Đ & Đ & Đ & \\
		Đ & Đ & S & \\
		\hline
	\end{tabular}
\end{table}

\begin{table}[H]
	\centering
	\caption{Bảng giá trị chân lí của $( P \implies Q ) \lor ( P \implies R ) \iff ( P \implies Q \lor R )$}
	\begin{tabular}{|c|c|c|ccccccccccccc|}
		\hline
		P & Q & R & $(P$ & $\implies$ & $Q)$ & $\lor$ & $(P$ & $\implies$ & $R)$ & $\iff$ & $(P$ & $\implies$ & $Q$ & $\lor$ & $R)$ \\
		\headerDivider
		S & X & X & S & Đ & X & Đ & S & Đ & X & \emphcolor{Đ} & S & Đ & X & X & X \\
		Đ & Đ & X & Đ & Đ & Đ & Đ & Đ & X & X & \emphcolor{Đ} & Đ & Đ & Đ & Đ & X \\
		Đ & S & Đ & Đ & S & S & Đ & Đ & Đ & Đ & \emphcolor{Đ} & Đ & Đ & S & Đ & Đ \\
		Đ & S & S & Đ & S & S & S & Đ & S & S & \emphcolor{Đ} & Đ & S & S & S & S \\
		\hline
	\end{tabular}
   \begin{tabular}{|c|c|c!{\vrule width 1.5pt}c|}
		\hline
		P & Q & R & $ ( P \implies Q ) \lor ( P \implies R )  \iff ( P \implies Q \lor R )$ \\
		\headerDivider
		S & X & X & \multirow{4}{*}{\huge \vspace{-12pt} Đ} \\
		Đ & Đ & X & \\
		Đ & S & Đ & \\
		Đ & S & S & \\
		\hline
	\end{tabular}
\end{table}

\begin{table}[H]
	\centering
	\caption{Bảng giá trị chân lí của $( P \implies Q ) \land ( P \implies R ) \implies ( P \implies R )$}
	\begin{tabular}{|c|c|c|ccccccccccc|}
		\hline
		P & Q & R & $(P$ & $\implies$ & $Q)$ & $\land$ & $(Q$ & $\implies$ & $R)$ & $\implies$ & $(P$ & $\implies$ & $R)$ \\
		\headerDivider
		X & X & Đ & X & X & X & Đ & X & Đ & Đ & \emphcolor{Đ} & X & Đ & Đ \\
		S & X & S & S & Đ & X & X & X & X & S & \emphcolor{Đ} & S & Đ & S \\
		Đ & Đ & S & Đ & Đ & Đ & S & Đ & S & S & \emphcolor{Đ} & Đ & S & S \\
		Đ & S & S & Đ & S & S & S & S & Đ & S & \emphcolor{Đ} & Đ & S & S \\
		\hline
	\end{tabular}
\end{table}
