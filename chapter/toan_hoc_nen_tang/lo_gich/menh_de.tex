\section{Mệnh đề}

\ % Lùi đầu dòng

Lô-gích là ngành nghiên cứu về sự lập luận, là sự giao thoa giữa toán học và triết học. Không có một ngành nào của toán học hay khoa học nói chung mà không có sự tồn tại của các lập luận, chứng minh hay phản biện. Chúng ta sẽ bắt đầu với định nghĩa của mệnh đề lô-gích và những phép toán cho phép chúng ta kết hợp các mệnh đề.

\defText{Mệnh đề lô-gích} (gọi tắt là \defText{mệnh đề}) là một phát biểu hoặc đúng, hoặc sai, nhưng không thể cả hai cùng lúc. Một vài ví dụ của mệnh đề như sau:
\begin{itemize}
  \item $2 + 3 = 5$  \hfill (Mệnh đề đúng);
  \item $4 < 1$ \hfill (Mệnh đề sai);
  \item ``Hà Nội là thủ đô của Việt Nam.'' \hfill (Mệnh đề đúng);
  \item ``Thực dân Pháp nổ súng vào bán đảo Sơn Trà vào ngày 1/1/1858.'' \hfill (Mệnh đề sai).
\end{itemize}
Còn những ví dụ sau không phải là mệnh đề:
\begin{itemize}
  \item ``Bạn có khỏe không?'' \hfill (Câu hỏi, không phải mệnh đề);
  \item ``Hãy học tập chăm chỉ.'' \hfill (Câu cầu khiến, không phải mệnh đề);
  \item ``Ôi đẹp quá!'' \hfill (Câu cảm thán, không phải mệnh đề);
  \item $x + 2 = 5$ \hfill (Biểu thức chứa biến, chưa xác định giá trị đúng/sai).
\end{itemize}

Có nhiểu kiểu lô-gích khác nhau, nhưng chủ yếu đều thuộc hai loại. Loại thứ nhất là \defText{lô-gích quy nạp} mà ở đó nếu như lập luận là tốt thì kết luận cũng 