\section{Giới thiệu về lô-gích}

\subsection{Điều kiện tồn tại của toán học}

\ % Lùi đầu dòng

Đa phần các nhà toán học và các nhà khoa học đều thừa nhận rằng toán học không phải là một ngành khoa học. Tuy nhiên, toán học cũng không hoàn toàn thuần túy trong lí tưởng và tách biệt khỏi thực tế như nhiều người quan niệm. Không có một sản phẩm của con người nào lại bắt nguồn từ hư vô. Toán học, cũng như vậy, được xây dựng và phát triển dựa trên những quan sát của con người và sự phản ánh của họ lên thế giới vật lí xung quanh. 

Lấy ví dụ, số $1$. Khái niệm về số $1$ bắt nguồn từ việc con người quan sát thấy rằng trong tự nhiên, có những vật thể riêng lẻ, tách biệt với nhau. Chúng ta có thể thực hiện các sự biến đổi lên số $1$ như $1 + 2 = 3$ dựa trên giả thiết rằng vật thể sẽ không tự nhiên biến mất, xuất hiện thêm, hay chuyển hóa thành một dạng vật thể khác trong quá trình biến đổi. Với những đại lượng liên tục, khái niệm ``một'' là không tồn tại. Chẳng hạn, chúng ta không nói ``một sữa cộng một sữa'' mà cần phải có trợ từ đi kèm (như ``một lít sữa'') để xác định đại lượng. 

Cho nên, trước khi làm toán, cần phải có một môi trường với những luật lệ nhất định để có thể thực hiện toán. Nhưng một thông lệ chung, luật lệ này được nhiều người chấp nhận là những quy tắc \defText{lô-gích}.

\subsection{Đối tượng của lô-gích - Mệnh đề}

\ % Lùi đầu dòng

Lô-gích là ngành nghiên cứu về sự lập luận, là sự giao thoa giữa toán học và triết học. Không có một ngành nào của toán học hay khoa học nói chung mà không có sự tồn tại của các lập luận, chứng minh hay phản biện; mà khi lập luận thì cần phải có đối tượng để có thể thực hiện lập luận trên, chúng được gọi là các mệnh đề.

\defText{Mệnh đề lô-gích} (gọi tắt là \defText{mệnh đề}) là một phát biểu hoặc đúng, hoặc sai, nhưng không thể cả hai cùng lúc, và cũng không thể vừa không đúng vừa không sai. Một vài ví dụ của mệnh đề như sau:
\begin{itemize}
   \item $2 + 3 = 5$  \hfill (Mệnh đề đúng);
   \item $4 < 1$ \hfill (Mệnh đề sai);
   \item ``Hà Nội là thủ đô của Việt Nam.'' \hfill (Mệnh đề đúng);
   \item ``Thực dân Pháp nổ súng vào bán đảo Sơn Trà vào ngày 1/1/1858.'' \hfill (Mệnh đề sai).
\end{itemize}
Còn những ví dụ sau không phải là mệnh đề:
\begin{itemize}
   \item ``Bạn có khỏe không?'' \hfill (Câu hỏi, không phải mệnh đề);
   \item ``Hãy học tập chăm chỉ.'' \hfill (Câu cầu khiến, không phải mệnh đề);
   \item ``Ôi đẹp quá!'' \hfill (Câu cảm thán, không phải mệnh đề);
   \item $x + 2 = 5$ \hfill (Biểu thức chứa biến, chưa xác định giá trị đúng/sai).
\end{itemize}

Trong phần sau của cuốn sách này, sẽ có nhiều chỗ yêu cầu đề cập đến tính chất của các mệnh đề theo một cách tổng quát nhất. Khi này, các mệnh đề thường được kí hiệu bởi một chữ cái in hoa.

\subsection{Các loại lập luận lô-gích}

\ % Lùi đầu dòng

Có nhiểu kiểu lập luận lô-gích khác nhau, nhưng chủ yếu đều thuộc hai loại. Loại thứ nhất là lập luận theo \defText{lô-gích quy nạp} (gọi tắt là \defText{lập luận quy nạp} mà ở đó kết luận khái quát được đưa ra từ việc quan sát nhiều trường hợp cụ thể. Từ đó, chúng ta có cấu trúc của một lập luận bằng lô-gích quy nạp gồm hai phần: các mệnh đề \defText{quan sát}, liệt kê ra những dẫn chứng đã được thực nghiệm, và các mệnh đề \defText{kết luận} được đưa ra từ những quan sát. Một vài ví dụ cho lô-gích quy nạp được đưa ra ở bảng \ref{tab:toan_hoc_nen_tang:lo_gich:menh_de:vd_logic_quy_nap}.

\begin{table}[H]
   \centering
   \caption{Ví dụ cho lô-gích quy nạp}
   \label{tab:toan_hoc_nen_tang:lo_gich:menh_de:vd_logic_quy_nap}
   \begin{tabular}{|c|c|}
      \hline
      Quan sát & Kết luận \\
      \headerDivider
      Mỗi sáng mặt trời đều mọc ở hướng Đông. & Mặt trời luôn mọc ở hướng Đông. \\
      \hline
      Nhiều kim loại (sắt, đồng, nhôm) đều nở ra khi nóng lên. & Kim loại nói chung sẽ nở ra khi nóng. \\
      \hline
      Một nhóm học sinh chăm chỉ đạt điểm cao. & Học sinh chăm chỉ thường sẽ đạt kết quả tốt. \\
      \hline
   \end{tabular}
\end{table}

Các lập luận này có lí hay không dựa vào sức thuyết phục của quá trình quan sát. Đây là nền tảng quan trọng trong khoa học, triết học, và nghiên cứu thực nghiệm. Tuy nhiên, lô-gích quy nạp không đảm bảo tuyệt đối đúng. Việc dựa vào một số lượng quan sát hạn chế có thể dẫn đến những kết luận sai lầm hoặc phiến diện, đặc biệt khi các quan sát đó không đại diện cho toàn bộ hiện tượng. Ví dụ, chúng ta không thể đưa ra kết luận ``mọi học sinh đều 6 tuổi'' bằng sự quan sát của một lớp học hay của một khối lớp\footnote{Việc chỉ lấy những quan sát thuận lợi cho mình mà bỏ qua những dẫn chứng bất lợi còn được gọi là ``chọn lọc thiên vị''.}.

Loại lập luận theo lô-gích quy nạp theo xu hướng suy luận từ cái riêng ra cái chung. Nếu suy luận theo hướng ngược lại, từ cái chung ra cái riêng, thì loại lập luận này được gọi là \defText{lô-gích diễn dịch}. Ở đây, kết luận sẽ là chính xác hoàn toàn nếu như không có mắc lỗi trong lập luận. Thành tố của một lập luận diễn dịch bao gồm những mệnh đề \defText{giả thiết} (thường gọi tắt là \defText{tiên đề}), đưa ra bối cảnh và các sự vật cho lập luận, và những mệnh đề \defText{kết luận}, những tính chất của sự vật suy ra từ bối cảnh đó. Bảng \ref{tab:toan_hoc_nen_tang:lo_gich:menh_de:vd_logic_dien_dich} cho một vài ví dụ về loại lập luận này.

\begin{table}[H]
   \centering
   \caption{Ví dụ cho lô-gích diễn dịch}
   \label{tab:toan_hoc_nen_tang:lo_gich:menh_de:vd_logic_dien_dich}
   \begin{tabular}{|l|c|}
      \hline 
      \multicolumn{1}{|c|}{Giả thiết} & Kết luận \\
      \headerDivider
      $\begin{cases}
         \text{Tất cả các số chia hết cho $2$ đều là số chẵn;} \\
         \text{$8$ là một số chia hết cho $2$.}
      \end{cases}$ & $8$ là một số chắn. \\ 
      \hline
      $\begin{cases}
         \text{Mọi hành vi trộm cắp đều vi phạm pháp luật;} \\
         \text{$A$ đã thực hiện hành vi trộm cắp.}
      \end{cases}$ & Vậy $A$ đã vi phạm pháp luật.\\ 
      \hline
      $\begin{cases}
         \text{Nếu một người là bác sĩ, thì người đó đã học y khoa;} \\
         \text{$L$ là bác sĩ.}
      \end{cases}$ & Do đó, $L$ đã học y khoa.\\  
      \hline
   \end{tabular}
\end{table}

\exercise Trong văn học, có hai kiểu văn nghị luận. Kiểu thứ nhất là văn nghị luận chứng minh, kiểu bài viết nhằm khẳng định tính đúng đắn của một luận điểm bằng cách đưa ra các dẫn chứng cụ thể, xác thực và lập luận chặt chẽ. Kiểu còn lại, văn nghị luận giải thích, là kiểu bài viết nhằm làm rõ một tư tưởng, một hiện tượng, một khái niệm, giúp người đọc hiểu sâu sắc và đúng đắn hơn về vấn đề được nêu ra. Hãy chỉ rõ mối liên hệ giữa các kiểu văn nghị luận và các loại lập luận lô-gích. Từ đó, chỉ ra sự khác nhau giữa chứng minh toán học và chứng minh trong các môn khoa học khác.

\solution

Trong văn nghị luận chứng minh, người viết thường sử dụng các dẫn chứng thực tế, sự kiện, nhân vật, hiện tượng để chứng minh cho một luận điểm đã nêu. Đây là cách lập luận \emph{quy nạp}: từ những trường hợp riêng lẻ, người viết rút ra kết luận chung, nhằm thuyết phục người đọc về tính đúng đắn của luận điểm.

Trong văn nghị luận giải thích, người viết thường xuất phát từ một tư tưởng, đạo lý, hiện tượng, v.v rồi dùng lập luận, phân tích, so sánh, đối chiếu để làm sáng tỏ vấn đề. Đây là cách lập luận \emph{diễn dịch}: từ một tiền đề chung, người viết suy luận ra các biểu hiện cụ thể, giúp người đọc hiểu sâu sắc hơn về bản chất của vấn đề.

Khác với văn chứng minh trong các bài văn khoa học, chứng minh trong toán học sử dụng lô-gích diễn dịch mà ở đó kết luận được suy ra từ các tiên đề, định nghĩa và định lý đã được công nhận.


