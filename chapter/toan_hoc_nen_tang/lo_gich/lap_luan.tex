\section{Phương pháp lập luận lô-gích}

\ % Lùi đầu dòng

\subsection{Điều kiện cho lập luận lô-gích}

\ % Lùi đầu dòng

Quá trình lập luận lô-gích được kiểm soát bởi những bộ quy luật hết sức đơn giản, đơn giản tới mức mà cả tác giả và bạn đọc đều có thể và đã từng làm theo. Có rất nhiều yêu cầu về mặt kĩ thuật cần phải quan tâm khi xây dựng một bộ quy luật lập luận mà khuôn khổ cuốn sách này không có khả năng đề cập. Tác giả sẽ chỉ đưa ra các điều kiện cơ bản nhưng quan trọng nhất khi lập luận lô-gích diễn dịch:
\begin{enumerate}
    \item Điều kiện về \defText{tính ổn định}: Bộ quy luật lập luận không thể vừa chưng minh một mệnh đề và đồng thời là phủ định của nó. Hiểu theo cách khác, bộ quy luật không thể chứng minh đồng thời $P$ và $\neg P$ cùng đúng;
    \item Điều kiện về \defText{tính hoàn thiện}: Bộ quy luật lập luận cần phải cho phép chứng minh \emph{tất cả} các kết luận có thể từ một nhóm các giả thiết nhất định.
\end{enumerate}

\subsection{Định nghĩa chứng minh}

\ % Lùi đầu dòng

Từ các điều kiện, chúng ta lại đặt câu hỏi, thế nào là chứng minh? Để tránh định nghĩa lặp ``chứng minh là một quá trình lập luận'', tác giả sẽ đưa ra một định nghĩa đơn giản như sau: Nếu như cho biết $W$, mệnh đề $X$ \defText{chứng minh} mệnh đề $Y$ khi và chỉ khi $\defMath{X \implies Y}$ (dưới điều kiện $W$)\footnote{Từ giờ trở đi, mỗi mệnh đề được đưa ra sẽ được mặc định là đúng.}.

Nếu có nhiều mệnh đề cho giả thiết $X_1$, $X_2$,\dots, $X_n$, thì khi này mệnh đề giả thiết tổng là hợp của tất cả các giả thiết con
$$X = X_1 \land X_2 \land \cdots \land X_n = \bigwedge_{i=1}^{n} X_i.$$
Một cách tương tự, chúng ta cũng có thể định nghĩa điều kiện tổng $W$ và kết luận tổng $Y$.
\begin{align*}
    W &= W_1 \land W_2 \land \cdots \land W_n = \bigwedge_{i=1}^{n} W_i; \\
    Y &= Y_1 \land Y_2 \land \cdots \land Y_n = \bigwedge_{i=1}^{n} Y_i.
\end{align*}

Do chúng ta chưa xây dựng hệ thống số cho nên việc kí hiệu bằng dấu $\cdots$ hay dấu $\bigwedge$ lớn sẽ là hơi lạm dụng. Do đó, với trường hợp xác định được số lượng các mệnh đề, nên viết đầy đủ và tường minh toàn bộ biểu thức thay vì viết tắt.

Ngoài ra, bạn đọc cũng có thể đã để ý rằng tác giả tách một phần gọi là ``điều kiện'' mặc dù hoàn toàn có thể đưa điều kiện vào trong phần giả thiết. Sở dĩ tác giả làm như vậy là do thực tế, khi chứng minh phần lớn các bài toán, chúng ta cũng sẽ ngầm hiểu một số điều kiện nhất định --- định nghĩa, tính chất, hay kết quả đã có. Ví dụ, kể đến về một định lí quen thuộc:
\begin{center}
    ``Nếu $x$ là một số thực thì $x^2 \geq 0$.''
\end{center} 
Một học sinh khá của cấp trung học cơ sở hoàn toàn có thể hiểu và lí giải được điều này sử dụng các tính chất của số thực được cung cấp. Học sinh muốn chứng minh định lí đó sẽ không cần phải chép lại cả quyển sách giáo khoa và đưa toàn bộ lí thuyết vào phần giả thiết, mà sẽ chỉ tập trung vào ý tưởng cốt lõi nhất và sự hiểu biết rằng người đọc (chủ yếu là giáo viên) sẽ biết về các tính chất của số thực được sử dụng trong chứng minh.

Xét một ví dụ cơ bản cho việc chứng minh, giả sử có giả thiết như sau:
\begin{enumerate}
    \item ``Nếu trời mưa thì sân vận động sẽ ướt.'';
    \item ``Trời đang mưa.''
\end{enumerate}
và chứng minh kết luận ``Sân vận động thì ướt''. Để thực hiện chứng minh, chuyển sang dưới dạng kí hiệu
\begin{itemize}
    \item $P$: ``Trời mưa.''; 
    \item $Q$: ``Sân vận động ướt.''.
\end{itemize}
Qua đó, chúng ta có thể viết lại lập luận này dưới dạng như sau:
\begin{center}
    \begin{tabular}{r|l}
        Giả thiết   & $P \implies Q$ \\
                    & $P$ \\
        \headerDivider
        Kết luận & $Q$
    \end{tabular}.
\end{center}
Mệnh đề mà chúng ta cần khẳng định đúng để hoàn thành chứng minh là $(P \implies Q) \land P \implies Q$. Thực vậy, có bảng \ref{tab:toan_hoc_nen_tang:lo_gich:lap_luan:vd_chung_minh}.
    
\begin{table}[H]
    \centering
    \caption{Bảng giá trị chân lí của $( P \implies Q ) \land P \implies Q$}
    \label{tab:toan_hoc_nen_tang:lo_gich:lap_luan:vd_chung_minh}
    \begin{tabular}{|c|c|ccccccc|}
        \hline
        $P$ & $Q$ & $(P$ & $\implies$ & $Q)$ & $\land$ & $P$ & $\implies$ & $Q$ \\
        \headerDivider
        Đ & Đ & Đ & Đ & Đ & Đ & Đ & \emphcolor{Đ} & Đ \\
        Đ & S & Đ & S & S & S & Đ & \emphcolor{Đ} & S \\
        S & Đ & S & Đ & Đ & S & S & \emphcolor{Đ} & Đ \\
        S & S & S & Đ & S & S & S & \emphcolor{Đ} & S \\
        \hline
    \end{tabular}
\end{table}
Vậy, từ giả thiết, chúng ta có kết được kết luận. Điều phải chứng minh.

Từ định nghĩa chứng minh, chúng ta cũng có định nghĩa của \defText{không chứng minh}: $X$ không chứng minh $Y$ nếu tồn tại giá trị chân lí của $X$ và $Y$ để $X \implies Y$ sai.

Ví dụ, một phép ngụy biện thường xuyên được sử dụng là ngụy biện khẳng định hệ quả:
\begin{center}
    \begin{tabular}{r|l}
        Giả thiết   & $X \implies Y$ \\
                    & $Y$ \\
        \headerDivider
        Kết luận & $X$
    \end{tabular}.
\end{center}

Sử dụng bảng giá trị chân lí, chúng ta có ngay điều không chứng minh.
\begin{table}[H]
	\centering
	\caption{Bảng giá trị chân lí của $( X \implies Y ) \land Y \implies X$}
	\begin{tabular}{|c|c|ccccccc|}
		\hline
		$X$ & $Y$ & $(X$ & $\implies$ & $Y)$ & $\land$ & $Y$ & $\implies$ & $X$ \\
		\headerDivider
		Đ & Đ & Đ & Đ & Đ & Đ & Đ & \emph{Đ} & Đ \\
		Đ & S & Đ & S & S & S & S & \emph{Đ} & Đ \\
		S & Đ & S & Đ & Đ & Đ & Đ & \emphcolor{S} & S \\
		S & S & S & Đ & S & S & S & \emph{Đ} & S \\
		\hline
	\end{tabular}
\end{table}

\exercise Cho $X$ và $Y$ là hai mệnh đề thỏa mãn $X \iff Y$. Chứng minh rằng $X$ chứng minh $Y$ và $Y$ chứng minh $X$.

\solution

Viết lại đề bài dưới dạng giả thiết -- kết luận:
\begin{center}
    \begin{tabular}{r|l}
        Giả thiết & $X \iff Y$ \\
        \headerDivider
        Kết luận & $(X \implies Y) \land (Y \implies X)$
    \end{tabular}.
\end{center}

Sử dụng bảng giá trị chân lí:
\begin{table}[H]
	\centering
	\caption{Bảng giá trị chân lí của $( X \iff Y ) \implies ( X \implies Y ) \land ( Y \implies X )$}
	\begin{tabular}{|c|c|ccccccccccc|}
		\hline
		$X$ & $Y$ & $(X$ & $\iff$ & $Y)$ & $\implies$ & $(X$ & $\implies$ & $Y)$ & $\land$ & $(Y$ & $\implies$ & $X)$ \\
		\headerDivider
		Đ & Đ & Đ & Đ & Đ & \emphcolor{Đ} & Đ & Đ & Đ & Đ & Đ & Đ & Đ \\
		Đ & S & Đ & S & S & \emphcolor{Đ} & Đ & S & S & S & S & Đ & Đ \\
		S & Đ & S & S & Đ & \emphcolor{Đ} & S & Đ & Đ & S & Đ & S & S \\
		S & S & S & Đ & S & \emphcolor{Đ} & S & Đ & S & Đ & S & Đ & S \\
		\hline
	\end{tabular}
        \begin{tabular}{|c|c!{\vrule width 1.5pt}c|}
        \hline
        $X$ & $Y$ & $( X \implies Y ) \land (Y \implies X)$ \\
        \headerDivider
        Đ & Đ & \multirow{4}{*}{\huge\vspace{-12pt}Đ} \\
        Đ & S & \\
        S & Đ & \\
        S & S & \\
        \hline
    \end{tabular}
\end{table}
chúng ta có điều phải chứng minh.

\subsection{Quy tắc thay thế trong chứng minh lô-gích}

\ % Lùi đầu dòng

Nếu như mỗi lần chứng minh, chúng ta phải lôi bảng giá trị chân lí thì sẽ gặp vấn đề khi một mệnh đề lớn có quá nhiều mệnh đề con cấu thành. Trong tương lai, chúng ta sẽ xây dựng các kiến trúc toán học cơ bản. Để làm được điều đó, chúng ta sẽ cần rất nhiều các định nghĩa và tiên đề. Chúng ta cần một hệ thống quy tắc để lập luận hoặc chứng minh.

\begin{itemize}
    \item 
    \textcolor{colorEmphasisCyan}{Quy tắc thứ nhất}: Cho $A \implies B$. Khi thay đồng thời các mệnh đề $A$ trong $\lonF$ bởi $B$ để có được mệnh đề $\lonG$ thì có được $\lonF \implies \lonG$ hay $\lonF$ chứng minh $\lonG$.
    \item
    \textcolor{colorEmphasis}{Quy tắc thứ hai}: Cho $A \iff B$. Khi thay đồng thời các mệnh đề $A$ trong $\lonF$ bởi $B$ để có được mệnh đề $\lonG$ thì có được $\lonF \iff \lonG$.
    \item    
    \textcolor{colorEmphasisGreen}{Quy tắc thứ ba}: Cho $\chungF$ có mệnh đề $A$ làm mệnh đề con. Nếu $\chungF$ đúng không phụ thuộc vào giá trị chân lí của $A$ thì mệnh đề $\chungF$ vẫn đúng khi thay đồng thời tất cả các mệnh đề $A$ trong $\chungF$ bằng mệnh đề $B$ bất kì.
\end{itemize}

Để thuận tiện cho việc lập luận, sẽ sử dụng các kết quả đã có của bài \ref{ex:toan_hoc_nen_tang:lo_gich:menh_de_phuc_hop:tc_lo_gich} ở trang \pageref{ex:toan_hoc_nen_tang:lo_gich:menh_de_phuc_hop:tc_lo_gich}.

\exercise Chứng minh rằng $\overline{\neg A \lor B} \lor A \land \neg C \iff A \land \overline{B \land C}$ với $A$, $B$ và $C$ là các mệnh đề bất kì.

\solution
Có $\overline{P \lor Q} \iff \neg P \land \neg Q$ đúng với mọi mệnh đề $P$ và $Q$ theo định luật Đờ Moóc-gơn. Thay mệnh đề $P$ và $Q$ lần lượt bởi $A$ và $B$, chúng ta có
$$\overline{\neg A \lor B} \iff \neg (\neg A) \land \neg B.$$
Lại có $\overline{\neg P} \iff P$ với mọi $P$ theo tính chất phủ định kép cho nên $\neg(\neg A) \iff A$. Do đó
$$\overline{\neg A \lor B} \iff A \land \neg B.$$
Như một hệ quả,
$$\overline{\neg A \lor B} \lor A \land \neg C \iff A \land \neg B \lor A \land \neg C.$$
Theo tính chất phân phối, $P \land Q \lor P \land R \iff P \land (Q \lor R)$ với mọi $P$, $Q$, $R$,
$$A \land \neg B \lor A \land \neg C \iff A \land (\neg B \lor \neg C).$$
Suy ra, $\overline{\neg A \lor B} \lor A \land \neg C \iff A \land (\neg B \lor \neg C)$.

Định luật Đờ Moóc-gơn vẫn còn một hệ thức nữa là $\neg P \lor \neg Q \iff \overline{P \land Q}$, dẫn đến $(\neg B \lor \neg C) \iff \overline{B \land C}$ nếu đổi $P$ bởi $B$ và $Q$ bởi $C$. Do đó $$A \land (\neg B \lor \neg C) \iff A \land \overline{B \land C}.$$

Vậy $\overline{\neg A \lor B} \lor A \land \neg C \iff A \land \overline{B \land C}$. Chúng ta có điều cần phải chứng minh.

\exercise Biết rằng, lô-gích khó hoặc nhiều người thích học nó, và nếu làm khoa học là dễ thì lô-gích là không khó. Sử dụng lập luận diễn dịch, chứng minh rằng nếu không phải rằng nhiều người thích học lô-gích, làm khoa học là không dễ dàng.

\solution
Đặt biến cho các mệnh đề:
\begin{itemize}
    \item $L$: ``Lô-gích khó.'';
    \item $A$: ``Nhiều người thích học lô-gích.'';
    \item $S$: ``Làm khoa học là dễ dàng.''.
\end{itemize}
Chúng ta có bộ giả thiết -- kết luận như sau:
\begin{center}
    \begin{tabular}{r|ll}
        Giả thiết   & $L \lor A$ & (``Lô-gích khó hoặc nhiều người thích học nó.'')\\
                    & $S \implies \neg L$ & (``Nếu làm khoa học là dễ thì lô-gích là không khó.'') \\
        \headerDivider
        Kết luận & $\neg A \implies \neg S$ & (``Nếu không phải rằng nhiều người thích học lô-gích,\\
        & & làm khoa học là không dễ dàng.'')
    \end{tabular}.
\end{center}
Viết lại dưới dạng kí hiệu, chúng ta cần chứng minh rằng
$$(L \lor A) \land (S \implies \neg L) \implies (\neg A \implies \neg S).$$
Trước hết, để ý rằng $A \implies \neg \neg A$ (theo tính chất phủ định kép) nên $$L \lor A \iff L \lor \neg (\neg A).$$
Do tính chất giao hoán của phép tuyển, $$L \lor \neg(\neg A) \iff \neg(\neg A) \lor L.$$
Theo định nghĩa phép kéo theo, $\neg P \lor Q \iff (P \implies Q)$ với mọi $P$ và $Q$ nên $$\neg(\neg A) \lor L \iff (\neg A \implies L).$$
Kết hợp lại ba mệnh đề tương đương này, chúng ta có $$L \lor A \iff (\neg A \implies L).$$

Theo tính chất phản đảo, $$(S \implies \neg L) \iff (\neg (\neg L) \implies \neg S).$$
Kết hợp với tính chất phủ định kép, $$(S \implies \neg L) \iff (L \implies \neg S).$$

Qua đó, xét vế giả thiết của mệnh đề cần chứng minh:
$$(L \lor A) \land (S \implies \neg L) \iff (\neg A \implies L) \land (L \implies \neg S).$$
Ngoài ra, theo tính chất bắc cầu, $$(\neg A \implies L) \land (L \implies \neg S) \iff (\neg A \implies \neg S).$$
Do vậy, $$(L \lor A) \land (S \implies \neg L) \iff (\neg A \implies \neg S).$$ Điều phải chứng minh.

\exercise Xác định xem các kết luận sau có thể được chứng minh từ các giả thiết được cho sử dụng lập luận diễn dịch hay không.
\begin{enumerate}
    \item 
    \begin{tabular}{r|l}
        Giả thiết & Nếu máy chủ hoạt động bình thường và đường truyền mạng ổn định \\
        & thì người dùng có thể truy cập dữ liệu; \\
        & Người dùng đang không truy cập được dữ liệu. \\
        \headerDivider
        Kết luận & Máy chủ đang gặp sự cố.
    \end{tabular}
    \item 
    \begin{tabular}{r|l}
        Giả thiết & Âm nhạc đang làm cho Qua-di thư giãn, hoặc là tiếng ồn làm cho Qua-di đau đầu; \\
        & Nếu Qua-di đeo tai nghe chống ồn, thì tiếng ồn không làm cho Qua-di đau đầu; \\
        & Thực tế là Qua-di đang đeo tai nghe chống ồn. \\
        \headerDivider
        Kết luận & Âm nhạc đang làm cho Qua-di thư giãn.
    \end{tabular}
    \item 
    \begin{tabular}{r|l}
        Giả thiết & Nếu bây giờ là tháng 12, thì tháng liền trước là tháng 11; \\
        & Nếu tháng trước là tháng 11, thì 6 tháng trước (bây giờ) là tháng 6; \\
        & Nếu tháng sau là tháng 1, thì bây giờ là tháng 12; \\
        & Tháng liền trước là tháng 11. \\
        \headerDivider
        Kết luận & Bây giờ đang là tháng 12.
    \end{tabular}
    \item 
    \begin{tabular}{r|l}
        Giả thiết & Chỉ khi hoàng tử tiêu diệt được ma vương và cứu được công chúa \\
        & thì vương quốc mới thái bình; \\
        & Nếu vương quốc thái bình hoặc nhà vua băng hà, bầu trời sẽ không có màu tím; \\
        & Nếu hoàng tử không cứu được công chúa, bầu trời sẽ có màu tím; \\
        & Nhà vua chưa băng hà và bầu trời không có màu tím. \\
        \headerDivider
        Kết luận & Hoàng tử đã cứu được công chúa.
    \end{tabular}
    \item
    \begin{tabular}{r|l}
        Giả thiết & Nếu Tu-ba-na thức khuya học bài và không uống cà phê, \\
        & Tu-ba-na sẽ mệt mỏi vào sáng ngày hôm sau; \\
        & Nếu Tu-ba-na mệt mỏi vào sáng ngày hôm sau, bạn ấy sẽ không thể thi tốt \\
        & hoặc sẽ ngủ gật trong giờ thi; \\
        & Ở giờ thi sáng nay, Tu-ba-na đã không quay cóp nhưng bạn ấy vẫn không thi tốt.\\
        \headerDivider
        Kết luận & Tu-ba-na đã thức khuya học bài vào ngày trước hôm thi.
    \end{tabular}
\end{enumerate}

\solution

\setcounter{subexercise}{1}
\arabic{subexercise}. Đặt biến cho các mệnh đề:
\begin{itemize}
    \item $N$: ``Máy chủ hoạt động bình thường.'';
    \item $C$: ``Đường truyền mạng ổn định.'';
    \item $D$: ``Người dùng có thể truy cập được dữ liệu.''.
\end{itemize}
``Mảy chủ gặp sự cố.'' tương đương với ``Máy chủ đang không hoạt động bình thường.'', hay $\neg N$.

Viết lại hệ thống giả thiết -- kết luận dưói dạng lô-gích:
\begin{center}
    \begin{tabular}{r|l}
        Giả thiết & $ N \land C \implies D$ \\
        & $\neg D$ \\
        \headerDivider
        Kết luận & $\neg N$
    \end{tabular}.
\end{center}

Lập luận này là không sắc đáng, do tồn tại bộ giá trị chân lí cho $C$, $D$, và $N$ khiến cho lập luận sai. Cụ thể, sau khi giải giá trị chân lí của mệnh đề $(N \land C \implies D) \land \neg D \implies \neg N$ với mệnh đề $C$ và $D$ sai và $N$ là mệnh đề đúng ---
\begin{center}
	\begin{tabular}{|c|c|c|ccccccccccc|}
		\hline
		$C$ & $D$ & $N$ & $(N$ & $\land$ & $C$ & $\implies$ & $D)$ & $\land$ & $\neg$ & $D$ & $\implies$ & $\neg$ & $N$ \\
		\headerDivider
		S & S & Đ & Đ & S & S & Đ & S & Đ & Đ & S & \emphcolor{S} & S & Đ \\
		\hline
	\end{tabular},
\end{center}
chúng ta có mệnh đề cần chứng minh sai. Do vậy, kết luận không suy ra được từ giả thiết.

\stepcounter{subexercise}
\arabic{subexercise}. Đặt biến:
\begin{itemize}
    \item $R$: ``Âm nhạc (đang) làm cho Qua-di thư giãn.'';
    \item $B$: ``Tiếng ồn làm cho Qua-di đau đầu.'';
    \item $H$: ``Qua-di (đang) đeo tai nghe chống ồn.''.
\end{itemize}

Kí hiệu hóa:
\begin{center}
    \begin{tabular}{r|l}
        Giả thiết & $ R\lor B$ \\
        & $H \implies \neg B$ \\
        & $H$ \\
        \headerDivider
        Kết luận & $R$
    \end{tabular}.
\end{center}

Do tính chất kết hợp của phép hội nên
$$(R \lor B) \land (H \implies \neg B) \land H \iff (R \lor B) \land ((H \implies \neg B) \land H).$$

Có $(H \implies \neg B) \land H \implies \neg B$ ($(P \implies Q) \land P \implies Q $) nên 
$$(R \lor B) \land ((H \implies \neg B) \land H) \implies (R \lor B) \land \neg B.$$

Từ tính chất phân phối, chúng ta có
$$(R \lor B) \land \neg B \iff R \land \neg B \lor B \land \neg B.$$

Theo tính chất phủ định kép, $B \land \neg B \iff \overline{\neg(B \land \neg B)}$. Lại có $\neg(B \land \neg B)$ luôn đúng do tính chất không mâu thuẫn nên $\overline{\neg(B \land \neg B)}$ sai. Kết hợp lại, có được $B \land \neg B \iff \mathbf{M}$ với $\mathbf{M}$ là một mệnh đề sai nào đó.

Do vậy $(R \lor B) \land \neg B \iff R \land \neg B \lor \mathbf{M}$. Thêm tính chất đồng nhất, có $R \land \neg B \lor \mathbf{M} \iff R \land \neg B$, và theo tính chất rút gọn, $R \land \neg B \implies R$.

Kết hợp tất cả các kết quả đã có, chúng ta có thể có quá trình lập luận như sau:
\begin{align*}
    (R \lor B) \land (H \implies \neg B) \land H &\iff (R \lor B) \land ((H \implies \neg B) \land H) \\
    (R \lor B) \land (H \implies \neg B) \land H &\implies (R \lor B) \land \neg B \\
    (R \lor B) \land (H \implies \neg B) \land H &\implies R \land \neg B \lor B \land \neg B \\
    (R \lor B) \land (H \implies \neg B) \land H &\implies R \land \neg B \\
    (R \lor B) \land (H \implies \neg B) \land H &\implies R.
\end{align*}

Vậy, kết luận có thể suy ra được từ giả thiết.

\stepcounter{subexercise}
\arabic{subexercise}. Trông đáp án của câu này có vẻ hiển nhiên, nhưng thực chất không phải như vậy. Kí hiệu các mệnh đề:
\begin{itemize}
    \item $D$: ``Bây giờ (đang) là tháng 12.'';
    \item $U$: ``Tháng (liền) trước là tháng 11.'';
    \item $S$: ``6 tháng trước (bây giờ) là tháng 6.'';
    \item $M$: ``Tháng (liền) sau là tháng 1.''.
\end{itemize}

Chuyển giả thiết và kết luận sang dưới dạng kí hiệu:
\begin{center}
    \begin{tabular}{r|l}
        Giả thiết & $D \implies U$ \\
        & $U \implies S$ \\
        & $M \implies D$ \\
        & $U$ \\
        \headerDivider
        Kết luận & $D$
    \end{tabular}.
\end{center}

Chỉ có một trường hợp khiến cho phép lập luận thất bại:
\begin{center}
    \begin{tabular}{|c|c|c|c|ccccccccccccc|}
		\hline
		$D$ & $M$ & $S$ & $U$ & $(D$ & $\implies$ & $U)$ & $\land$ & $(U$ & $\implies$ & $S)$ & $\land$ & $(M$ & $\implies$ & $D)$ & $\land$ & $U$ \\
		\headerDivider
		S & S & Đ & Đ & S & Đ & Đ & Đ & Đ & Đ & Đ & Đ & S & Đ & S & Đ & Đ \\
		\hline
	\end{tabular}
    \begin{tabular}{|c|c|c|c|ccc|}
        \hline
        $D$ & $M$ & $S$ & $U$ & $(D \implies U) \land (U \implies S) \land (M \implies D) \land U$ & $\implies$ & $D$ \\
        \headerDivider
        S & S & Đ & Đ & Đ & \emphcolor{S} & S \\
        \hline
    \end{tabular}.
\end{center}
Nhưng chỉ cần một trường hợp là đủ để chứng minh không còn hợp lệ.

Cũng có thể nhìn câu này theo một hướng khác. Lập luận này chỉ hợp lệ nếu có điều kiện đi kèm. Nếu như chúng ta không sử dụng lịch thông thường mà sử dụng một loại lịch đặc biệt mà trước tháng 12 lại có một tháng 11A thì sao? Bài tập này càng nhấn mạnh cần phải có một ``nền tảng'' chung trong việc giải quyết vấn đề trước khi đề cập đến những yếu tố chi tiết.

\stepcounter{subexercise}
\arabic{subexercise}. Đặt biến các mệnh đề:
\begin{itemize}
    \item $L$: ``Hoàng tử tiêu diệt được ma vương.'';
    \item $P$: ``Hoàng tử cứu được công chúa.'';
    \item $N$: ``Vương quốc mới thái bình.'';
    \item $I$: ``Nhà vua băng hà.'';
    \item $R$: ``Bầu trời có màu tím.''.
\end{itemize}

Kí hiệu hóa:

\begin{center}
    \begin{tabular}{r|l}
        Giả thiết & $L \land P \implies N$ \\
        & $N \lor I \implies \neg R$ \\
        & $\neg P \implies R$ \\
        & $\neg I \land \neg R$ \\
        \headerDivider
        Kết luận & $P$        
    \end{tabular}
\end{center}

Viết tắt $(L \land P \implies N) \land (N \lor I \implies \neg R) \land (\neg P \implies R) \land (\neg I \land \neg R)$ bởi $M$. Thực hiện liên tiếp các phép ``biến đổi'' thông qua các tính chất, nhận xét:
\begin{align*}
    M \iff \big((L \land P \implies N) &\land (N \lor I \implies \neg R)\big) \land (\neg P \implies R) \land (\neg I \land \neg R) \equationexplanation{thứ tự tính toán}; \\
    M \iff \big((L \land P \implies N) &\land (N \lor I \implies \neg R)\big) \\
    &\land \big((\neg P \implies R) \land (\neg I \land \neg R)\big) \equationexplanation{tính chất kết hợp của phép hội}; \\
    M \implies (\neg P \implies R) &\land (\neg I \land \neg R) \equationexplanation{tính chất rút gọn}.
\end{align*}
Áp dụng tính chất rút gọn một lần nữa để có $\neg I \land \neg R \implies \neg R$. Điều này dẫn đến
\begin{align*}
    M &\implies (\neg P \implies R) \land \neg R; \\
    M &\implies \overline{\neg P} \equationexplanation{quy tắc phủ định}; \\
    M & \implies P \equationexplanation{tính chất phủ định kép}.
\end{align*}

Qua đó, chúng ta có điều phải chứng minh. Bạn đọc cũng có thể thực hiện bài tập này thông qua bảng giá trị chân lí, nhưng khí đó, bạn đọc sẽ phải xét\dots\ mà thôi, bạn đọc có thể tham khảo bảng \ref{ex:toan_hoc_nen_tang:lo_gich:lap_luan:bai_2_phan_4} nếu bạn đọc không muốn phải ghét bản thân như tác giả đã từng khi thực hiện lập bảng này\footnote{Bạn đọc có thể kiểm tra lại bảng \ref{ex:toan_hoc_nen_tang:lo_gich:lap_luan:bai_2_phan_4} để trải nghiệm cảm giác của tác giả lúc viết đáp án. Khi càng tăng nhiều biến thì số trường hợp cần phải kiểm tra cũng tăng theo theo cấp số nhân.}. Ngoài ra, bạn đọc cũng sẽ mất đi cơ hội được thường thức nghệ thuật của sự lập luận do bị thay thế bởi sự cơ học của phương pháp.

\begin{table}[H]
    \centering
	\caption{Bảng giá trị chân lí của $( L \land P \implies N ) \land ( N \lor I \implies \neg R ) \land ( \neg P \implies R ) \land ( \neg I \land \neg R ) \implies P$}
    \label{ex:toan_hoc_nen_tang:lo_gich:lap_luan:bai_2_phan_4}
	\begin{tabular}{|c|c|c|c|c|cccccccccccc|}
		\hline
		$I$ & $L$ & $N$ & $P$ & $R$ & $(L$ & $\land$ & $P$ & $\implies$ & $N)$ & $\land$ & $(N$ & $\lor$ & $I$ & $\implies$ & $\neg$ & $R)$ \\
		\headerDivider
		Đ & Đ & Đ & Đ & Đ & Đ & Đ & Đ & Đ & Đ & S & Đ & Đ & Đ & S & S & Đ \\
		Đ & Đ & Đ & Đ & S & Đ & Đ & Đ & Đ & Đ & Đ & Đ & Đ & Đ & Đ & Đ & S \\
		Đ & Đ & Đ & S & Đ & Đ & S & S & Đ & Đ & S & Đ & Đ & Đ & S & S & Đ \\
		Đ & Đ & Đ & S & S & Đ & S & S & Đ & Đ & Đ & Đ & Đ & Đ & Đ & Đ & S \\
		Đ & Đ & S & Đ & Đ & Đ & Đ & Đ & S & S & S & S & Đ & Đ & S & S & Đ \\
		Đ & Đ & S & Đ & S & Đ & Đ & Đ & S & S & S & S & Đ & Đ & Đ & Đ & S \\
		Đ & Đ & S & S & Đ & Đ & S & S & Đ & S & S & S & Đ & Đ & S & S & Đ \\
		Đ & Đ & S & S & S & Đ & S & S & Đ & S & Đ & S & Đ & Đ & Đ & Đ & S \\
		Đ & S & Đ & Đ & Đ & S & S & Đ & Đ & Đ & S & Đ & Đ & Đ & S & S & Đ \\
		Đ & S & Đ & Đ & S & S & S & Đ & Đ & Đ & Đ & Đ & Đ & Đ & Đ & Đ & S \\
		Đ & S & Đ & S & Đ & S & S & S & Đ & Đ & S & Đ & Đ & Đ & S & S & Đ \\
		Đ & S & Đ & S & S & S & S & S & Đ & Đ & Đ & Đ & Đ & Đ & Đ & Đ & S \\
		Đ & S & S & Đ & Đ & S & S & Đ & Đ & S & S & S & Đ & Đ & S & S & Đ \\
		Đ & S & S & Đ & S & S & S & Đ & Đ & S & Đ & S & Đ & Đ & Đ & Đ & S \\
		Đ & S & S & S & Đ & S & S & S & Đ & S & S & S & Đ & Đ & S & S & Đ \\
		Đ & S & S & S & S & S & S & S & Đ & S & Đ & S & Đ & Đ & Đ & Đ & S \\
		S & Đ & Đ & Đ & Đ & Đ & Đ & Đ & Đ & Đ & S & Đ & Đ & S & S & S & Đ \\
		S & Đ & Đ & Đ & S & Đ & Đ & Đ & Đ & Đ & Đ & Đ & Đ & S & Đ & Đ & S \\
		S & Đ & Đ & S & Đ & Đ & S & S & Đ & Đ & S & Đ & Đ & S & S & S & Đ \\
		S & Đ & Đ & S & S & Đ & S & S & Đ & Đ & Đ & Đ & Đ & S & Đ & Đ & S \\
		S & Đ & S & Đ & Đ & Đ & Đ & Đ & S & S & S & S & S & S & Đ & S & Đ \\
		S & Đ & S & Đ & S & Đ & Đ & Đ & S & S & S & S & S & S & Đ & Đ & S \\
		S & Đ & S & S & Đ & Đ & S & S & Đ & S & Đ & S & S & S & Đ & S & Đ \\
		S & Đ & S & S & S & Đ & S & S & Đ & S & Đ & S & S & S & Đ & Đ & S \\
		S & S & Đ & Đ & Đ & S & S & Đ & Đ & Đ & S & Đ & Đ & S & S & S & Đ \\
		S & S & Đ & Đ & S & S & S & Đ & Đ & Đ & Đ & Đ & Đ & S & Đ & Đ & S \\
		S & S & Đ & S & Đ & S & S & S & Đ & Đ & S & Đ & Đ & S & S & S & Đ \\
		S & S & Đ & S & S & S & S & S & Đ & Đ & Đ & Đ & Đ & S & Đ & Đ & S \\
		S & S & S & Đ & Đ & S & S & Đ & Đ & S & Đ & S & S & S & Đ & S & Đ \\
		S & S & S & Đ & S & S & S & Đ & Đ & S & Đ & S & S & S & Đ & Đ & S \\
		S & S & S & S & Đ & S & S & S & Đ & S & Đ & S & S & S & Đ & S & Đ \\
		S & S & S & S & S & S & S & S & Đ & S & Đ & S & S & S & Đ & Đ & S \\
		\hline
	\end{tabular}
\end{table}
\begin{table}[H]
    \centering
    \begin{tabular}{|c|c|c|c|c|cccccc|}
		\hline
		$I$ & $L$ & $N$ & $P$ & $R$ & $( L \land P \implies N ) \land ( N \lor I \implies \neg R )$ & $\land$ & $(\neg$ & $P$ & $\implies$ & $R)$ \\
		\headerDivider
		Đ & Đ & Đ & Đ & Đ & S & \emphcolor{S} & S & Đ & Đ & Đ \\
		Đ & Đ & Đ & Đ & S & Đ & \emphcolor{Đ} & S & Đ & Đ & S \\
		Đ & Đ & Đ & S & Đ & S & \emphcolor{S} & Đ & S & Đ & Đ \\
		Đ & Đ & Đ & S & S & Đ & \emphcolor{S} & Đ & S & S & S \\
		Đ & Đ & S & Đ & Đ & S & \emphcolor{S} & S & Đ & Đ & Đ \\
		Đ & Đ & S & Đ & S & S & \emphcolor{S} & S & Đ & Đ & S \\
		Đ & Đ & S & S & Đ & S & \emphcolor{S} & Đ & S & Đ & Đ \\
		Đ & Đ & S & S & S & Đ & \emphcolor{S} & Đ & S & S & S \\
		Đ & S & Đ & Đ & Đ & S & \emphcolor{S} & S & Đ & Đ & Đ \\
		Đ & S & Đ & Đ & S & Đ & \emphcolor{Đ} & S & Đ & Đ & S \\
		Đ & S & Đ & S & Đ & S & \emphcolor{S} & Đ & S & Đ & Đ \\
		Đ & S & Đ & S & S & Đ & \emphcolor{S} & Đ & S & S & S \\
		Đ & S & S & Đ & Đ & S & \emphcolor{S} & S & Đ & Đ & Đ \\
		Đ & S & S & Đ & S & Đ & \emphcolor{Đ} & S & Đ & Đ & S \\
		Đ & S & S & S & Đ & S & \emphcolor{S} & Đ & S & Đ & Đ \\
		Đ & S & S & S & S & Đ & \emphcolor{S} & Đ & S & S & S \\
		S & Đ & Đ & Đ & Đ & S & \emphcolor{S} & S & Đ & Đ & Đ \\
		S & Đ & Đ & Đ & S & Đ & \emphcolor{Đ} & S & Đ & Đ & S \\
		S & Đ & Đ & S & Đ & S & \emphcolor{S} & Đ & S & Đ & Đ \\
		S & Đ & Đ & S & S & Đ & \emphcolor{S} & Đ & S & S & S \\
		S & Đ & S & Đ & Đ & S & \emphcolor{S} & S & Đ & Đ & Đ \\
		S & Đ & S & Đ & S & S & \emphcolor{S} & S & Đ & Đ & S \\
		S & Đ & S & S & Đ & Đ & \emphcolor{Đ} & Đ & S & Đ & Đ \\
		S & Đ & S & S & S & Đ & \emphcolor{S} & Đ & S & S & S \\
		S & S & Đ & Đ & Đ & S & \emphcolor{S} & S & Đ & Đ & Đ \\
		S & S & Đ & Đ & S & Đ & \emphcolor{Đ} & S & Đ & Đ & S \\
		S & S & Đ & S & Đ & S & \emphcolor{S} & Đ & S & Đ & Đ \\
		S & S & Đ & S & S & Đ & \emphcolor{S} & Đ & S & S & S \\
		S & S & S & Đ & Đ & Đ & \emphcolor{Đ} & S & Đ & Đ & Đ \\
		S & S & S & Đ & S & Đ & \emphcolor{Đ} & S & Đ & Đ & S \\
		S & S & S & S & Đ & Đ & \emphcolor{Đ} & Đ & S & Đ & Đ \\
		S & S & S & S & S & Đ & \emphcolor{S} & Đ & S & S & S \\
		\hline
	\end{tabular}
\end{table}
\begin{table}[H]
    \centering
    \begin{tabular}{|c|c|c|c|c|ccccccccc|}
		\hline
		$I$ & $L$ & $N$ & $P$ & $R$ & $\begin{array}{c}
            ( L \land P \implies N ) \land ( N \lor I \implies \neg R ) \\
            \land ( \neg P \implies R )
        \end{array}$ & $\land$ & $(\neg$ & $I$ & $\land$ & $\neg$ & $R)$ & $\implies$ & $P$ \\
		\headerDivider
		Đ & Đ & Đ & Đ & Đ & S & S & S & Đ & S & S & Đ & \emphcolor{Đ} & Đ \\
		Đ & Đ & Đ & Đ & S & Đ & S & S & Đ & S & Đ & S & \emphcolor{Đ} & Đ \\
		Đ & Đ & Đ & S & Đ & S & S & S & Đ & S & S & Đ & \emphcolor{Đ} & S \\
		Đ & Đ & Đ & S & S & S & S & S & Đ & S & Đ & S & \emphcolor{Đ} & S \\
		Đ & Đ & S & Đ & Đ & S & S & S & Đ & S & S & Đ & \emphcolor{Đ} & Đ \\
		Đ & Đ & S & Đ & S & S & S & S & Đ & S & Đ & S & \emphcolor{Đ} & Đ \\
		Đ & Đ & S & S & Đ & S & S & S & Đ & S & S & Đ & \emphcolor{Đ} & S \\
		Đ & Đ & S & S & S & S & S & S & Đ & S & Đ & S & \emphcolor{Đ} & S \\
		Đ & S & Đ & Đ & Đ & S & S & S & Đ & S & S & Đ & \emphcolor{Đ} & Đ \\
		Đ & S & Đ & Đ & S & Đ & S & S & Đ & S & Đ & S & \emphcolor{Đ} & Đ \\
		Đ & S & Đ & S & Đ & S & S & S & Đ & S & S & Đ & \emphcolor{Đ} & S \\
		Đ & S & Đ & S & S & S & S & S & Đ & S & Đ & S & \emphcolor{Đ} & S \\
		Đ & S & S & Đ & Đ & S & S & S & Đ & S & S & Đ & \emphcolor{Đ} & Đ \\
		Đ & S & S & Đ & S & Đ & S & S & Đ & S & Đ & S & \emphcolor{Đ} & Đ \\
		Đ & S & S & S & Đ & S & S & S & Đ & S & S & Đ & \emphcolor{Đ} & S \\
		Đ & S & S & S & S & S & S & S & Đ & S & Đ & S & \emphcolor{Đ} & S \\
		S & Đ & Đ & Đ & Đ & S & S & Đ & S & S & S & Đ & \emphcolor{Đ} & Đ \\
		S & Đ & Đ & Đ & S & Đ & Đ & Đ & S & Đ & Đ & S & \emphcolor{Đ} & Đ \\
		S & Đ & Đ & S & Đ & S & S & Đ & S & S & S & Đ & \emphcolor{Đ} & S \\
		S & Đ & Đ & S & S & S & S & Đ & S & Đ & Đ & S & \emphcolor{Đ} & S \\
		S & Đ & S & Đ & Đ & S & S & Đ & S & S & S & Đ & \emphcolor{Đ} & Đ \\
		S & Đ & S & Đ & S & S & S & Đ & S & Đ & Đ & S & \emphcolor{Đ} & Đ \\
		S & Đ & S & S & Đ & Đ & S & Đ & S & S & S & Đ & \emphcolor{Đ} & S \\
		S & Đ & S & S & S & S & S & Đ & S & Đ & Đ & S & \emphcolor{Đ} & S \\
		S & S & Đ & Đ & Đ & S & S & Đ & S & S & S & Đ & \emphcolor{Đ} & Đ \\
		S & S & Đ & Đ & S & Đ & Đ & Đ & S & Đ & Đ & S & \emphcolor{Đ} & Đ \\
		S & S & Đ & S & Đ & S & S & Đ & S & S & S & Đ & \emphcolor{Đ} & S \\
		S & S & Đ & S & S & S & S & Đ & S & Đ & Đ & S & \emphcolor{Đ} & S \\
		S & S & S & Đ & Đ & Đ & S & Đ & S & S & S & Đ & \emphcolor{Đ} & Đ \\
		S & S & S & Đ & S & Đ & Đ & Đ & S & Đ & Đ & S & \emphcolor{Đ} & Đ \\
		S & S & S & S & Đ & Đ & S & Đ & S & S & S & Đ & \emphcolor{Đ} & S \\
		S & S & S & S & S & S & S & Đ & S & Đ & Đ & S & \emphcolor{Đ} & S \\
		\hline
	\end{tabular}
\end{table}

\stepcounter{subexercise}
\arabic{subexercise}. Đặt biến:
\begin{itemize}
    \item $N$: ``Tu-ba-na thức khuya học bài.'';
    \item $K$: ``Tu-ba-na uống cà phê.'';
    \item $L$: ``Tu-ba-na mệt mỏi (vào sáng ngày thi).'';
    \item $B$: ``Tu-ba-na thi tốt (vào sáng ngày thi).'';
    \item $D$: ``Tu-ba-na ngủ gật trong giờ thi.'';
    \item $Q$: ``Tu-ba-na đã quay cóp.''.
\end{itemize}

Kí hiệu hóa giả thiết và kết luận:

\begin{center}
    \begin{tabular}{r|l}
        Giả thiết & $N \land \neg K \implies L$ \\
        & $L \implies \neg B \lor D$ \\
        & $\neg Q \land \neg B$ \\
        \headerDivider
        Kết luận & $N$
    \end{tabular}.
\end{center}

Không có bất cứ lí do nào để chứng minh này có thể hợp lí được cả, bởi vì nếu chúng ta đặt giá trị chân lí của tất cả các mệnh đề con là sai:
\begin{center}
    \begin{tabular}{l}
        \begin{tabular}{|c|c|c|c|c|c|cccccc|}
            \hline
            $B$ & $D$ & $K$ & $L$ & $N$ & $Q$ & $N$ & $\land$ & $\neg$ & $K$ & $\implies$ & $L$ \\
            \headerDivider
            S & S & S & S & S & S & S & S & Đ & S & Đ & S \\
            \hline
        \end{tabular} \\
        \begin{tabular}{|c|c|c|c|c|c|cccccc|}
            \hline
            $B$ & $D$ & $K$ & $L$ & $N$ & $Q$ & $L$ & $\implies$ & $\neg$ & $B$ & $\lor$ & $D$ \\
            \headerDivider
            S & S & S & S & S & S & S & Đ & Đ & S & S & S \\
            \hline
        \end{tabular} \\
        \begin{tabular}{|c|c|c|c|c|c|ccccc|}
            \hline
            $B$ & $D$ & $K$ & $L$ & $N$ & $Q$ & $\neg$ & $Q$ & $\land$ & $\neg$ & $B$ \\
            \headerDivider
            S & S & S & S & S & S & Đ & S & Đ & Đ & S \\
            \hline
        \end{tabular} \\
        \begin{tabular}{|c|c|c|c|c|c|ccccccc|}
            \hline
            $B$ & $D$ & $K$ & $L$ & $N$ & $Q$ & $(N \land \neg K \implies L)$ & $\land$ & $(L \implies \neg B \lor D)$ & $\land$ & $(\neg Q \land \neg B)$ & $\implies$ & $N$ \\
            \headerDivider
            S & S & S & S & S & S & Đ & Đ & Đ & Đ & Đ & \emphcolor{S} & S \\
            \hline
        \end{tabular}
    \end{tabular}
\end{center}
thì chúng ta có thể thấy rằng kết luận $N$ không được suy ra từ giả thuyết.

Vậy, giả thiết không chứng minh được kết luận.

\subsection{Mâu thuẫn}

\ % Lùi đầu dòng

Từ một số giả thiết cho trước, chúng ta có thể chứng minh các mệnh đề khác, nhưng không có nghĩa là nó thực sự ``đúng''. Hãy tưởng tượng rằng bạn đọc đang đi xem ảo thuật. Ảo thuật gia dẫn dắt bạn đọc thông qua một câu chuyện của một quả bóng mà trong đó có những hành động được miêu tả rất chi tiết. Chiếc cốc úp vào quả bóng, di chuyển trên bàn trực tiếp dưới con mắt của bạn đọc. Bạn đọc nửa tin vào lời của ảo thuật gia, nửa tin vào thị giác của mình rằng quả bóng vẫn ở dưới cái cốc, nhưng cuối cùng, kết quả là quả bóng lại biến mất mà lại xuất hiện trong tay của nhà ảo thuật gia. Đúng là kết luận bạn đọc đưa ra --- quả bóng nằm dưới cái cốc --- là hợp lí từ câu chuyện của nhà ảo thuật gia, nhưng nó chỉ ``đúng'' nếu như ảo thuật gia thực hiện hành động thật như lời nói mà không có sự dụng kĩ thuật tay. Nói cách khác, các giả thiết mà nhà ảo thuật gia đưa ra chứng minh được kết luận của bạn đọc, nhưng kết luận chỉ thật sự hợp lí nếu như giả thiết là không có mâu thuẫn.

\defText{Mâu thuẫn} được định nghĩa là khi giả thiết không thể nhận giá trị chân lí đúng. Viết dưới dạng kí hiệu, $X$ bị mâu thuẫn nếu $X \implies \mathbf{M}$ với $M$ là một mệnh đề sai nào đó. Chúng ta không bao giờ muốn làm việc với giả thiết có mâu thuẫn, do nó có thể chứng minh mọi thứ. Thật vậy, gọi $Y$ là mệnh đề bất kì. Do $X$ luôn sai nên $X \implies Y$ luôn đúng do định nghĩa của phép kéo theo. Vậy $X$ chứng minh $Y$.

``Thế không phải mong ước của con người là khám phá mọi thứ, chứng minh mọi thứ hay sao?'' --- Bạn đọc có thể hỏi. Đúng là như vậy, nhưng sẽ khá vô dụng nếu chúng ta chứng minh được một mệnh đề và cả phủ định của nó nữa. Để ý rằng lập luận trong chứng minh trên không phụ thuộc vào $Y$. Do đó, chúng ta hoàn toàn có thể có $X \implies \neg Y$. Điều này không có nghĩa $Y \land \neg Y$, chỉ là khẳng định rằng phép chứng minh là không có giá trị nếu như điều kiện đầu vào đã sai.

Để chứng minh giả thiết $X$ bị mâu thuẫn, chúng ta chứng minh $X \implies \mathbf{M}$. Để chứng minh giả thiết $X$ không có mâu thuẫn, tương tự như chứng minh thông thường, chúng ta chỉ ra một trường hợp để $X \implies \mathbf{M}$ sai.

\exercise Cho các giả thiết sau. Xét xem những giả thiết đó có bị mâu thuẫn hay không.
\begin{multicols}{2}
    \begin{enumerate}
        \item $P \lor Q$, $\neg P$, $\neg Q$;
        \item $P \uparrow R$, $P$, $Q \land R$;
        \item $P \implies Q \impliedby R$, $\neg P$, $\neg Q$, $\neg R$;
        \item $P \oplus Q$, $P \odot Q$;
        \item $P \oplus Q$, $Q \odot R$, $R \oplus P$;
        \item $A \odot B$, $B \oplus C$, $C \odot D$, $D \impliedby A$.
    \end{enumerate}
\end{multicols}

\solution

Với toàn bộ các phần, gọi $\mathbf{M}$ là mệnh đề sai nào đó.

\setcounter{subexercise}{1}
\arabic{subexercise}. Thực hiện biến đổi lên giả thiết:
\begin{align*}
    (P \lor Q) \land \neg P \land \neg Q &\iff (P \land \neg P \lor Q \land \neg P) \land \neg Q \equationexplanation{tính chất phân phối}; \\
    (P \lor Q) \land \neg P \land \neg Q &\iff (\mathbf{M} \lor \neg P \land Q) \land \neg Q \equationexplanation{tính không mâu thuẫn và tính giao hoán}; \\
    (P \lor Q) \land \neg P \land \neg Q &\iff (\neg P \land Q) \land \neg Q \equationexplanation{tính đồng nhất với phép tuyển}; \\
    (P \lor Q) \land \neg P \land \neg Q &\iff \neg P \land (Q \land \neg Q) \equationexplanation{tính chất kết hợp}; \\
    (P \lor Q) \land \neg P \land \neg Q &\iff \neg P \land \mathbf{M} \equationexplanation{tính chất không mâu thuẫn}; \\
    (P \lor Q) \land \neg P \land \neg Q &\iff \mathbf{M} \equationexplanation{tính chất thống trị}.
\end{align*}

Có $(P \lor Q) \land \neg P \land \neg Q$ chứng minh một mệnh đề sai. Do đó, giả thiết có mâu thuẫn.

\stepcounter{subexercise}
\arabic{subexercise}.
\begin{table}[H]
	\centering
	\caption{Bảng giá trị chân lí của $( P \uparrow R ) \land P \land ( Q \land R )$}
	\begin{tabular}{|c|c|c|ccccccccc!{\vrule width 1.5pt}c|}
		\hline
		$P$ & $Q$ & $R$ & $(P$ & $\uparrow$ & $R)$ & $\land$ & $P$ & $\land$ & $(Q$ & $\land$ & $R)$ & $( P \uparrow R ) \land P \land ( Q \land R )$ \\
		\headerDivider
		Đ & Đ & Đ & Đ & S & Đ & S & Đ & \emphcolor{S} & Đ & Đ & Đ & \multirow{8}{*}{\huge\vspace{-24pt}S}\\
		Đ & Đ & S & Đ & Đ & S & Đ & Đ & \emphcolor{S} & Đ & S & S & \\
		Đ & S & Đ & Đ & S & Đ & S & Đ & \emphcolor{S} & S & S & Đ & \\
		Đ & S & S & Đ & Đ & S & Đ & Đ & \emphcolor{S} & S & S & S & \\
		S & Đ & Đ & S & Đ & Đ & S & S & \emphcolor{S} & Đ & Đ & Đ & \\
		S & Đ & S & S & Đ & S & S & S & \emphcolor{S} & Đ & S & S & \\
		S & S & Đ & S & Đ & Đ & S & S & \emphcolor{S} & S & S & Đ & \\
		S & S & S & S & Đ & S & S & S & \emphcolor{S} & S & S & S & \\
		\hline
	\end{tabular}
\end{table}

Với mọi trường hợp, giả thiết đều có giá trị chân lí sai, cho nên giả thiết bị mâu thuẫn.

Có thể sử dụng lập luận rằng để giả thiết $P \uparrow R$ đúng thì không thể $P$ và $R$ cùng đúng, nhưng giả thiết thứ hai cho $P$, và giả thiết thứ ba --- $Q \land R$ --- hiển nhiên chứng minh $R$. Chúng ta thấy được điều mâu thuẫn.

\stepcounter{subexercise}
\arabic{subexercise}. Giả thiết là không mâu thuẫn, dưới trường hợp sau:
\begin{center}
    \begin{tabular}{|c|c|c|cccccccccccccc|}
		\hline
		$P$ & $Q$ & $R$ & $(P$ & $\implies$ & $Q$ & $\impliedby$ & $R)$ & $\land$ & $\neg$ & $P$ & $\land$ & $\neg$ & $Q$ & $\land$ & $\neg$ & $R$ \\
		\headerDivider
		S & S & S & S & Đ & S & Đ & S & Đ & Đ & S & Đ & Đ & S & \emphcolor{Đ} & Đ & S \\
		\hline
	\end{tabular}.
\end{center}
Đây cũng là trường hợp duy nhất mà giả thiết không có mâu thuẫn.

\stepcounter{subexercise}
\arabic{subexercise}. Để $P \oplus Q$ đúng thì $P$ và $Q$ phải không cùng giá trị chân l, nhưng để $P \odot Q$ đúng thì $P$ và $Q$ phải giống nhau về giá trị chân lí. Vậy, mâu thuẫn.

Lập luận được kiểm tra lại bằng bảng giá trị chân lí \ref{tab:toan_hoc_nen_tang:lo_gich:lap_luan:xor_va_nxor}.

\begin{table}[H]
	\centering
	\caption{Bảng giá trị chân lí của $( P \oplus Q ) \land ( P \odot Q )$}
    \label{tab:toan_hoc_nen_tang:lo_gich:lap_luan:xor_va_nxor}
	\begin{tabular}{|c|c|ccccccc!{\vrule width 1.5pt}c|}
		\hline
		$P$ & $Q$ & $(P$ & $\oplus$ & $Q)$ & $\land$ & $(P$ & $\odot$ & $Q)$ & $( P \oplus Q ) \land ( P \odot Q )$\\
		\headerDivider
		Đ & Đ & Đ & S & Đ & \emphcolor{S} & Đ & Đ & Đ &\multirow{4}{*}{\huge\vspace{-12pt}S}\\
		Đ & S & Đ & Đ & S & \emphcolor{S} & Đ & S & S & \\
		S & Đ & S & Đ & Đ & \emphcolor{S} & S & S & Đ & \\
		S & S & S & S & S & \emphcolor{S} & S & Đ & S & \\
		\hline
	\end{tabular}
\end{table}

\stepcounter{subexercise}
\arabic{subexercise}. Giả thiết là không có mâu thuẫn. Ví dụ:
\begin{center}
    \begin{tabular}{|c|c|c|ccccccccccc|}
		\hline
		$P$ & $Q$ & $R$ & $(P$ & $\oplus$ & $Q)$ & $\land$ & $(Q$ & $\odot$ & $R)$ & $\land$ & $(R$ & $\oplus$ & $P)$ \\
		\headerDivider
		Đ & S & S & Đ & Đ & S & Đ & S & Đ & S & \emphcolor{Đ} & S & Đ & Đ \\
		\hline
	\end{tabular}.
\end{center}

\stepcounter{subexercise}
\arabic{subexercise}. Bảng \ref{tab:toan_hoc_nen_tang:lo_gich:lap_luan:xor_nxor_tuong_duong_hon_hop} cho thấy rằng giả thiết luôn mâu thuẫn.
\begin{table}[H]
	\centering
	\caption{Bảng giá trị chân lí của $( A \odot B ) \land ( B \oplus C ) \land ( C \odot D ) \land ( D \iff A )$}
    \label{tab:toan_hoc_nen_tang:lo_gich:lap_luan:xor_nxor_tuong_duong_hon_hop}
	\begin{tabular}{|c|c|c|c|ccccccccccccccc|}
		\hline
		$A$ & $B$ & $C$ & $D$ & $(A$ & $\odot$ & $B)$ & $\land$ & $(B$ & $\oplus$ & $C)$ & $\land$ & $(C$ & $\odot$ & $D)$ & $\land$ & $(D$ & $\iff$ & $A)$ \\
		\headerDivider
		Đ & Đ & Đ & Đ & Đ & Đ & Đ & S & Đ & S & Đ & S & Đ & Đ & Đ & \emphcolor{S} & Đ & Đ & Đ \\
		Đ & Đ & Đ & S & Đ & Đ & Đ & S & Đ & S & Đ & S & Đ & S & S & \emphcolor{S} & S & S & Đ \\
		Đ & Đ & S & Đ & Đ & Đ & Đ & Đ & Đ & Đ & S & S & S & S & Đ & \emphcolor{S} & Đ & Đ & Đ \\
		Đ & Đ & S & S & Đ & Đ & Đ & Đ & Đ & Đ & S & Đ & S & Đ & S & \emphcolor{S} & S & S & Đ \\
		Đ & S & Đ & Đ & Đ & S & S & S & S & Đ & Đ & S & Đ & Đ & Đ & \emphcolor{S} & Đ & Đ & Đ \\
		Đ & S & Đ & S & Đ & S & S & S & S & Đ & Đ & S & Đ & S & S & \emphcolor{S} & S & S & Đ \\
		Đ & S & S & Đ & Đ & S & S & S & S & S & S & S & S & S & Đ & \emphcolor{S} & Đ & Đ & Đ \\
		Đ & S & S & S & Đ & S & S & S & S & S & S & S & S & Đ & S & \emphcolor{S} & S & S & Đ \\
		S & Đ & Đ & Đ & S & S & Đ & S & Đ & S & Đ & S & Đ & Đ & Đ & \emphcolor{S} & Đ & S & S \\
		S & Đ & Đ & S & S & S & Đ & S & Đ & S & Đ & S & Đ & S & S & \emphcolor{S} & S & Đ & S \\
		S & Đ & S & Đ & S & S & Đ & S & Đ & Đ & S & S & S & S & Đ & \emphcolor{S} & Đ & S & S \\
		S & Đ & S & S & S & S & Đ & S & Đ & Đ & S & S & S & Đ & S & \emphcolor{S} & S & Đ & S \\
		S & S & Đ & Đ & S & Đ & S & Đ & S & Đ & Đ & Đ & Đ & Đ & Đ & \emphcolor{S} & Đ & S & S \\
		S & S & Đ & S & S & Đ & S & Đ & S & Đ & Đ & S & Đ & S & S & \emphcolor{S} & S & Đ & S \\
		S & S & S & Đ & S & Đ & S & S & S & S & S & S & S & S & Đ & \emphcolor{S} & Đ & S & S \\
		S & S & S & S & S & Đ & S & S & S & S & S & S & S & Đ & S & \emphcolor{S} & S & Đ & S \\
		\hline
	\end{tabular}
\end{table}

Để có thể sử dụng lập luận thông thường, có thể thấy là để $A \odot B$ và $C \odot D$ đúng thì mỗi mệnh đề trong cặp $A$ và $B$, $C$ và $D$ phải có giá trị chân lí giống nhau. Thêm vào đó, $B \oplus C$ dẫn đến $B$ và $C$ khác giá trị chân lí. Do vậy $A$ và $D$ khác giá trị chân lí. Tuy nhiên $D \iff A$ khi và chỉ khi $D$ và $A$ cùng đúng hoặc cùng sai, dẫn đến mâu thuẫn.