\subsection{Hàm chẵn và hàm lẻ}

\ % Lùi đầu dòng

Phần tính chất đầu tiên mà chúng ta quan tâm đến là tính đối xứng của hàm số trên đồ thị

Cho $f(x)$ là một hàm số xác định trên $A$. $f(x)$ được gọi là \defText{hàm chẵn} nếu $\defMath{x \in A \implies -x \in A}$ và $$\defMath{f(-x) = f(x)}.$$ Tương tự, $f(x)$ được gọi là \defText{hàm lẻ} nếu $\defMath{x \in A \implies -x \in A}$ và $$\defMath{f(-x) = -f(x)}.$$

Biểu diễn về mặt đồ thị, đối với hàm $f(x)$ chẵn, hai điểm 

\exercise Xác định xem những hàm sau có phải là hàm chẵn, hàm lẻ hay không. Sau đó, vẽ đồ thị của chúng.
\begin{multicols}{2}
   \begin{enumerate}
      \item $f(x) = x^4 - 2x^2 - 3$;
      \item $f(x) = -x^7 + x$;
      \item $f(x) = \frac{x}{x^2 + 1}$;
      \item $f(x) = \frac{x^3 - \frac{1}{x^3}}{x + \frac{1}{x}}$;
      \item $f(x) = |x|^2 - \left|x^3\right| + 1$;
      \item $f(x) = \lfloor x \rfloor - \lfloor -x \rfloor$.
   \end{enumerate}
\end{multicols}

\solution 

\setcounter{subexercise}{1}
\arabic{subexercise}. 