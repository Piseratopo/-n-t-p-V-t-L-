\subsection{Hàm chẵn và hàm lẻ}

\ % Lùi đầu dòng

Phần tính chất đầu tiên mà chúng ta quan tâm đến là tính đối xứng của hàm số trên đồ thị. Nhắc lại một chút kiến thức hình học, một hình có thể có hai kiểu đối xứng là đối xứng trục và đối xứng điểm. Tạm thời, chúng ta chỉ quan tâm đến những trường hợp đối xứng cụ thể. Với đồ thị của một hàm số, một cách khá tự nhiên, chúng ta sẽ xem xét tính đối xứng trục tung hoặc qua điểm gốc tọa độ. 

Đầu tiên là đối xứng qua trục tung. Một hàm số có tính đối xứng như vậy được gọi là \defText{hàm chẵn}. Cụ thể, cho $f$ là một hàm số xác định trên $A$. $f$ là hàm chẵn nếu $x \in A \implies -x \in A$ và $$f(-x) = f(x)$$ với mọi $x \in A$. 

Tương tự, $f$ được gọi là \defText{hàm lẻ} nếu $x \in A \implies -x \in A$ và $$f(-x) = -f(x)$$ với mọi $x \in A$. Khi này, hàm sẽ đối xứng qua gốc tọa độ.

{
   \begin{minipageindent}{0.48\textwidth}
      \begin{figure}[H]
         \centering
         \begin{tikzpicture}
            \draw[->] (-4, 0) -- (4, 0) node[right] {$x$};
            \draw[->, color=colorEmphasis] (0, -4) -- (0, 4)  node[above] {$f(x)$};
            \draw[graph thickness, samples=80, color=colorEmphasisCyan, domain=-1.857:1.857] plot (\x, {(((\x)/1)^4 - 2*((\x)/1)^2 - 1) / 1});
            \filldraw[color=colorEmphasis] ({1.65}, { 0.967006249999999 }) circle (\pointSize) node[right] {$\left(x;f(x)\right)$};
            \filldraw[color=colorEmphasis] ({-1.65}, { 0.967006249999999 }) circle (\pointSize) node[left] {$\left(-x;f(x)\right)$};
            \draw[dashed, color=colorEmphasis] ({1.65}, { 0.967006249999999 }) -- ({-1.65}, { 0.967006249999999 });
         \end{tikzpicture}
         \caption{Đồ thị của một hàm chẵn}
      \end{figure}
   \end{minipageindent}
   \hfill
   \begin{minipageindent}{0.48\textwidth}
      \begin{figure}[H]
         \centering
         \begin{tikzpicture}
            \draw[->] (-4, 0) -- (4, 0) node[right] {$x$};
            \draw[->] (0, -4) -- (0, 4)  node[above] {$f(x)$};
            \draw[graph thickness, samples=80, color=colorEmphasisCyan, domain=-4.000:-1.133] plot (\x, {(((\x)/1) / (((\x)/1)^2 - 1)) / 1});
            \draw[graph thickness, samples=80, color=colorEmphasisCyan, domain=-0.883:0.883] plot (\x, {(((\x)/1) / (((\x)/1)^2 - 1)) / 1});
            \draw[graph thickness, samples=80, color=colorEmphasisCyan, domain=1.133:4.000] plot (\x, {(((\x)/1) / (((\x)/1)^2 - 1)) / 1});
            \filldraw[color=colorEmphasis] ({2.0}, { 0.6666666666666666 }) circle (\pointSize) node[above right] {$\left(x;f(x)\right)$};
            \filldraw[color=colorEmphasis] ({-2.0}, { -0.6666666666666666 }) circle (\pointSize) node[below left] {$\left(-x;f(x)\right)$};
            \filldraw[color=colorEmphasis] (0, 0) circle (\pointSize) node[below] {$\left(0;0\right)$};
            \draw[dashed, color=colorEmphasis] ({2.0}, { 0.6666666666666666 }) -- ({-2.0}, { -0.6666666666666666 });
         \end{tikzpicture}
         \caption{Đồ thị của một hàm lẻ}
      \end{figure}
      
   \end{minipageindent}
}

Bạn đọc, một cách rất tự nhiên, có thể đặt câu hỏi rằng liệu có hàm số nào đối xứng qua trục hoành không. Giả sử tồn tại hàm $f$ như vậy. Khi này, nếu một điểm có tọa độ $\left(x; y\right)$ trên đồ thị của $f$ thì đối xứng của nó qua trục hoành là $\left(x; -y\right)$. Tuy nhiên, do $f$ là hàm số nên chỉ tồn tại một điểm $\left(x; f(x)\right)$ trên đồ thị của $f$. Do đó, $f(x) = y = -y$ hay $f(x) = 0$ với mọi $x$ thuộc tập xác định. Vì chỉ tồn tại một hàm duy nhất đối xứng qua trục hoành và hàm này là hàm hằng cho nên chúng ta sẽ không đặt tên mới cho nó.

\exercise Xác định xem những hàm sau có phải là hàm chẵn, hàm lẻ hay không. Sau đó, vẽ đồ thị của chúng.
\begin{multicols}{2}
   \begin{enumerate}
      \item $f(x) = x^4 - 2x^2 - 3$;
      \item $f(x) = x^5 - x^3 + x$;
      \item $f(x) = \frac{x}{x^2 + 1}$;
      \item $f(x) = \frac{x^3 - \frac{1}{x^3}}{x + \frac{1}{x}}$;
      \item $f(x) = |x|^2 - \left|x^3\right| + 1$;
      \item $f(x) = \lceil x \rceil - \lfloor x \rfloor$.
   \end{enumerate}
\end{multicols}

\solution 

\setcounter{subexercise}{1}
\arabic{subexercise}. Tập xác định của hàm là $\mathbb{R}$. Với mọi $x \in \mathbb{R}$, có $-x \in \mathbb{R}$ và
\begin{align*}
   f(-x) &= (-x)^4 - 2(-x)^2 - 3\\
   &= x^4 - 2x^2 - 3\\
   &= f(x).
\end{align*}
Vậy $f(x)$ là hàm chẵn.

\stepcounter{subexercise}
\arabic{subexercise}. Tập xác định của hàm là $\mathbb{R}$. Với mọi $x \in \mathbb{R}$, có $-x \in \mathbb{R}$ và
\begin{align*}
   f(-x) &= (-x)^5 - (-x)^3 + (-x)\\
   &= -x^5 + x^3 - x\\
   &= -\left(x^5 - x^3 + x\right)\\
   &= -f(x).
\end{align*}
Vậy $f(x)$ là hàm lẻ.

{
   \begin{minipageindent}{0.48\textwidth}
      \begin{figure}[H]
         \centering
         \begin{tikzpicture}
            \draw[->] (-3, 0) -- (3, 0) node[right] {$x$};
            \draw[->] (0, -4) -- (0, 4)  node[above] {$f(x)$};
            \draw[graph thickness, samples=80, color=colorEmphasisCyan, domain=-2.040:2.040] plot (\x, {(((\x)/1)^4 - 2*((\x)/1)^2 - 3) / 1.5});
         \end{tikzpicture}
         \caption{Đồ thị của $x^{4} - 2 x^{2} - 3$}
      \end{figure}
   \end{minipageindent}
   \hfill
   \begin{minipageindent}{0.48\textwidth}
      \begin{figure}[H]
         \centering
         \begin{tikzpicture}
            \draw[->] (-3, 0) -- (3, 0) node[right] {$x$};
            \draw[->] (0, -4) -- (0, 4)  node[above] {$f(x)$};
            \draw[graph thickness, samples=80, color=colorEmphasisCyan, domain=-1.398:1.398] plot (\x, {(((\x)/1)^5 - ((\x)/1)^3 + ((\x)/1)) / 1});
         \end{tikzpicture}
         \caption{Đồ thị của $x^{5} - x^{3} + x$}
      \end{figure}
   \end{minipageindent}
}

\stepcounter{subexercise}
\arabic{subexercise}. Tập xác định của hàm là $\mathbb{R}$. Với mọi $x \in \mathbb{R}$, có $-x \in \mathbb{R}$ và
\begin{align*}
   f(-x) &= \frac{-x}{(-x)^2 + 1}\\
   &= \frac{-x}{x^2 + 1}\\
   &= -\frac{x}{x^2 + 1}\\
   &= -f(x).
\end{align*}
Vậy $f(x)$ là hàm lẻ.

\stepcounter{subexercise}
\arabic{subexercise}. Tìm tập xác định, $x$ làm cho $f(x)$ thỏa mãn khi và chỉ khi
\begin{equation*}
   \begin{cases}
      x \neq 0 \\
      x + \frac{1}{x} \neq 0
   \end{cases}.
\end{equation*}
Từ bất phương trình thứ hai, với điều kiện $x \neq 0$:
\begin{equation*}
   x + \frac{1}{x} \neq 0 \\
   \implies x^2 + 1 \neq 0
\end{equation*}
luôm đúng. Cho nên, tập xác định là $\mathbb{R} \setminus \left\{0\right\}$.

Để ý rằng, với $x$ thuộc tập xác định, thì có $x\neq 0 \iff -x \neq 0$. Cho nên $-x$ cũng thuộc tập xác định và
\begin{align*}
   f(-x) &= \frac{(-x)^3 - \frac{1}{(-x)^3}}{-x + \frac{1}{-x}} \\
         &= \frac{-x^3 - \frac{1}{-x^3}}{-x - \frac{1}{x}} \\
         \displaybreak[2]
         &= \frac{-\left(x^3 - \frac{1}{x^3}\right)}{-\left(x + \frac{1}{x}\right)} \\
         &= \frac{x^3 - \frac{1}{x^3}}{x + \frac{1}{x}} = f(x).
\end{align*}
Vậy $f(x)$ là hàm chẵn.

{
   \begin{minipageindent}{0.48\textwidth}
      \begin{figure}[H]
         \centering
         \begin{tikzpicture}
            \draw[->] (-3, 0) -- (3, 0) node[right] {$x$};
            \draw[->] (0, -4) -- (0, 4)  node[above] {$f(x)$};
            \draw[graph thickness, samples=80, color=colorEmphasisCyan, domain=-3.000:3.000] plot (\x, {(((\x)/1) / (((\x)/1)^2 + 1)) / 0.25});
         \end{tikzpicture}
         \caption{Đồ thị của $\frac{x}{x^{2} + 1}$}
      \end{figure}      
   \end{minipageindent}
   \hfill
   \begin{minipageindent}{0.48\textwidth}
      \begin{figure}[H]
         \centering
         \begin{tikzpicture}
            \draw[->] (-3, 0) -- (3, 0) node[right] {$x$};
            \draw[->] (0, -3) -- (0, 5)  node[above] {$f(x)$};
            \draw[graph thickness, samples=80, color=colorEmphasisCyan, domain=-2.425:-0.510] plot (\x, {((((\x)/1)^3 - 1/((\x)/1)^3) / (((\x)/1) + 1/((\x)/1)))});
            \draw[graph thickness, samples=80, color=colorEmphasisCyan, domain=0.510:2.425] plot (\x, {((((\x)/1)^3 - 1/((\x)/1)^3) / (((\x)/1) + 1/((\x)/1)))});
         \end{tikzpicture}
         \caption{Đồ thị của $\frac{x^{3} - \frac{1}{x^{3}}}{x + \frac{1}{x}}$}
      \end{figure}       
   \end{minipageindent}
}

\stepcounter{subexercise}
\arabic{subexercise}. Tập xác định của hàm là $\mathbb{R}$. Với mọi $x \in \mathbb{R}$, có $-x \in \mathbb{R}$ và
\begin{align*}
   f(-x) &= \left| -x \right|^2 - \left|(-x)^3\right| + 1 \\
         &= \left| x \right|^2 - \left|-x^3 \right| + 1 \\
         &= \left| x \right|^2 - \left| x^3 \right| + 1 \\
         &= f(x).
\end{align*}
Vậy $f(x)$ là hàm chẵn.

\stepcounter{subexercise}
\arabic{subexercise}. Tập xác định của hàm là $\mathbb{R}$.

Nếu \textcolor{colorEmphasisCyan}{$x \in \mathbb{Z}$} thì $\begin{cases}
   \left\lceil x \right\rceil = x \\ 
   \left\lfloor x \right\rfloor = x
\end{cases} \implies \left\lceil x \right\rceil - \left\lfloor x \right\rfloor = 0$.

Trong trường hợp còn lại, \textcolor{colorEmphasis}{$x \notin \mathbb{Z}$}. Đặt $\lfloor x \rfloor = n$ với $n$ nguyên. Điều này tương đương với $n \leq x < n + 1$. Do $x$ không phải là số nguyên nên $n \neq x$, cho nên $n < x < n + 1 \implies n < x \leq n + 1 \iff \left\lceil x \right\rceil = n + 1$. Do đó, $$\left\lceil x \right\rceil - \left\lfloor x \right\rfloor = (n + 1) - n = 1.$$

Cho nên, có thể viết lại hàm đã cho bằng

\begin{equation*}
   f(x) = \left\lceil x \right\rceil - \left\lfloor x \right\rfloor = \begin{cases}
      0 & \text{ nếu } x \in \mathbb{Z} \\
      1 & \text{ nếu } x \notin \mathbb{Z}
   \end{cases}.
\end{equation*}

Hiển nhiên rằng $f(x) = f(-x)$ là hàm chẵn.

{
   \begin{minipageindent}{0.48\textwidth}
      \begin{figure}[H]
         \centering
         \begin{tikzpicture}
            \draw[->] (-3, 0) -- (3, 0) node[right] {$x$};
            \draw[->] (0, -2) -- (0, 2)  node[above] {$f(x)$};
            \draw[graph thickness, samples=80, color=colorEmphasisCyan, domain=-1.864:1.864] plot (\x, {(((\x)/1)^2 - (((\x)/1)^6)^(1/2) + 1) / 1});
         \end{tikzpicture}
         \caption{Đồ thị của $|x|^2 - \left|x^3\right| + 1$}
      \end{figure}
   \end{minipageindent}
   \hfill
   \begin{minipageindent}{0.48\textwidth}
      \begin{figure}[H]
         \centering
         \begin{tikzpicture}
            \draw[->] (-3.5, 0) -- (3.5, 0) node[right] {$x$};
            \draw[->] (0, -1) -- (0, 3)  node[above] {$f(x)$};
            \draw[graph thickness, samples=80, color=colorEmphasisCyan, domain=-3.5:3.5] plot (\x, 1);
            \foreach \x in {-3,-2,-1,0,1,2,3} {
               \draw[color=colorEmphasisCyan, hollow point] (\x, 1) circle (\pointSize);
               \filldraw[color=colorEmphasisCyan] (\x, 0) circle (\pointSize);
            }
         \end{tikzpicture}
         \caption{Đồ thị của $\left\lceil x \right\rceil - \left\lfloor x \right\rfloor$}
      \end{figure}
   \end{minipageindent}
}

\exercise Vẽ đồ thị của hàm $f$ xác định trên tập số thực thỏa mãn điều kiện sau:
\begin{enumerate}
   \item $f(x) = |x - 1|$ với mọi $x \geq 0$ và $f$ là hàm chẵn;
   \item $f(x) = x^2 - x - 1$ với mọi $x < 0$ và $f$ là hàm lẻ.
\end{enumerate}

\solution

\setcounter{subexercise}{1}
\arabic{subexercise}. Theo đúng định nghĩa của hàm chẵn, chúng ta xác định giá trị được giá trị của hàm $f$ tại $x$ âm như sau:
$$
   f(x) = f(-x) = \left|(-x) - 1\right| = \left|x + 1\right|.
$$

Vậy, hàm $f$ có biểu thức là
$$
   f(x) = \begin{cases}
      |x + 1| & \text{ nếu } x \geq 0 \\
      |x - 1| & \text{ nếu } x < 0
   \end{cases}.
$$
Chúng ta có đồ thị của hàm số như sau:
\begin{figure}[H]
   \centering
   \begin{tikzpicture}
      \draw[->] (-4, 0) -- (4, 0) node[right] {$x$};
      \draw[->] (0, -1) -- (0, 4)  node[above] {$f(x)$};
      \draw[graph thickness, samples=80, color=colorEmphasisCyan, domain=0.000:4.000] plot (\x, {((((\x)/1) - 1)^2)^(1/2)});
      \draw[graph thickness, samples=80, color=colorEmphasisCyan, domain=-4.000:0.000] plot (\x, {((((\x)/1) + 1)^2)^(1/2)});
   \end{tikzpicture}
   \caption{Đồ thị cho phần $1$}
\end{figure}

\stepcounter{subexercise}
\arabic{subexercise}. Do $f$ là hàm lẻ, vói $x < 0$, chúng ta có
\begin{align*}
   f(x)  &= -f(-x) \\
         &= -\left((-x)^2 - (-x) - 1\right) \\
         &= -\left(x^2 + x - 1\right) \\
         &= -x^2 - x + 1.
\end{align*}

Ngoài ra, có
\begin{align*}
   f(0) &= -f(-0) = f(0) \\
   \implies f(0) &= 0.
\end{align*}

Vậy, hàm $f$ có biểu thức là
$$
   f(x) = \begin{cases}
      -x^2 - x + 1 & \text{ nếu } x < 0 \\
      0 & \text{ nếu } x = 0 \\
      x^2 - x - 1 & \text{ nếu } x > 0
   \end{cases}.
$$
Cuối cùng, chúng ta vẽ đồ thị:
\begin{figure}[H]
	\centering
	\begin{tikzpicture}
		\draw[->] (-4, 0) -- (4, 0) node[right] {$x$};
		\draw[->] (0, -4) -- (0, 4)  node[above] {$f(x)$};
		\draw[graph thickness, samples=80, color=colorEmphasisCyan, domain=0:2.791] plot (\x, {(((\x)/1)^2-((\x)/1)-1) / 1});
      \draw[graph thickness, samples=80, color=colorEmphasisCyan, domain=-2.791:0] plot (\x, {(-((\x))^2-((\x)/1)+1) / 1});
      \foreach \y in {-1,1} {
         \draw[colorEmphasisCyan, hollow point] (0, \y) circle (\pointSize);
      }
      \filldraw[colorEmphasisCyan] (0, 0) circle (\pointSize);
	\end{tikzpicture}
	\caption{Đồ thị cho phần $2$}
\end{figure}

\exercise Tìm một hàm $f$ xác định trên tập số thực sao cho $f$ vừa mang tính chẵn, vừa mang tính lẻ. Chứng minh tại sao hàm đó là hàm duy nhất thỏa mãn điều kiện này.

\solution 

Hàm $f(x) = 0$ với mọi $x$ là hàm duy nhất thỏa mãn điều kiện này. Thật vậy, giả sử $f$ là một hàm có tính chẵn và lẻ thì với mọi $x$: 
\begin{align*}
   f(x) &= f(-x) = -f(x) \\
   \implies 2f(x) &= 0 \\
   \implies f(x) &= 0.
\end{align*}
Bằng sự quy chiếu đơn giản với định nghĩa, chúng ta có hàm $f$ này thỏa mãn. Vậy, chúng ta có hàm \begin{align*}
   f: \mathbb{R} &\to \{0\} \\
         x &\mapsto 0
\end{align*} thỏa mãn và đã được chứng minh là duy nhất.

\exercise Cho hàm $f$ xác định trên đoạn $[-a; a]$. Chứng minh rằng tồn tại duy nhất một bộ hai hàm số $\left(\chanF; \leF\right)$ sao cho $\chanF$ là hàm chẵn, $\leF$ là hàm lẻ và $f(x) = \chanF(x) + \leF(x)$ với mọi $x \in [-a; a]$.

\solution

Giả sử bộ hai hàm số này tồn tại, khi này
$$
f(-x) = \chanF(-x) + \leF(-x) = \chanF(x) - \leF(x).
$$
Do đó, kết hợp với $f(x) = \chanF(x) + \leF(x)$, thực hiện một số biến đổi đại số để có \begin{equation*}
   \begin{cases}
      \chanF(x) = \frac{f(x) + f(-x)}{2} \\
      \leF(x) = \frac{f(x) - f(-x)}{2}
   \end{cases}.
\end{equation*}

Vậy, nếu bộ hai hàm số này tồn tại thì chỉ có thể nhận giá trị là

{
   \begin{minipageindent}{0.45\textwidth}
      \begin{equation*}
         \begin{array}{rccc}
            \chanF: &[-a; a] &\to &\mathbb{R} \\
            &x &\mapsto &\frac{f(x) + f(-x)}{2}
         \end{array};
      \end{equation*}
   \end{minipageindent}
   và 
   \begin{minipageindent}{0.45\textwidth}
      \begin{equation*}
         \begin{array}{rccc}
            \leF: &[-a; a] &\to &\mathbb{R} \\
            &x &\mapsto &\frac{f(x) - f(-x)}{2}
         \end{array}.
      \end{equation*}
   \end{minipageindent}
}

Để hoàn thành chứng minh, chúng ta sẽ khẳng định rằng hai hàm này thỏa mãn. Hiển nhiên rằng, nếu $x \in [-a; a]$ thì $-x \in [-a; a]$. Ngoài ra, với mọi $x \in [-a; a]$,

\begin{equation*}
   \begin{cases}
      \chanF(-x) &= \frac{f(-x) + f\left(-(-x)\right)}{2} = \frac{f(-x) + f(x)}{2} = \chanF(x) \\
      \leF(-x) &= \frac{f(-x) - f\left(-(-x)\right)}{2} = \frac{f(-x) - f(x)}{2} = -\leF(x)
   \end{cases}.
\end{equation*}
Qua đó, chúng ta có điều phải chứng minh.