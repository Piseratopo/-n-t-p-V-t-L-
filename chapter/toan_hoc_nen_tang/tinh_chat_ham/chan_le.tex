\subsection{Hàm chẵn và hàm lẻ}

\ % Lùi đầu dòng

Cho $f(x)$ là một hàm số xác định trên $A$. $f(x)$ được gọi là \defText{hàm chẵn} nếu $\defMath{x \in A \implies -x \in A}$ và $$\defMath{f(-x) = f(x)}.$$ Tương tự, $f(x)$ được gọi là \defText{hàm lẻ} nếu $\defMath{x \in A \implies -x \in A}$ và $$\defMath{f(-x) = -f(x)}.$$

\exercise Xác định xem những hàm sau có phải là hàm chẵn, hàm lẻ hay không. Sau đó, vẽ đồ thị của chúng.
\begin{multicols}{2}
   \begin{enumerate}
      \item $f(x) = x^4 - 2x^2$;
      \item 
   \end{enumerate}
\end{multicols}