\subsection{Hàm đồng biến và nghịch biến}

\ % Lùi đầu dòng

Thông qua biểu diễn hình học của một hàm số, người ta sẽ thấy hàm số tăng và giảm theo giá trị đầu vào. Từ đó, xây dựng được hai khái niệm là hàm đồng biến và hàm nghịch biến. Cụ thể, $f$ được gọi là \defText{đồng biến} trên tập $D$ nếu với mọi $x_1, x_2 \in D$, có $$\defMath{x_1 < x_2 \implies f\left(x_1\right) < f\left(x_2\right)}.$$ Bằng một định nghĩa tương tự, $f$ được gọi là \defText{nghịch biến} trên tập $D$ nếu với mọi $x_1, x_2 \in D$, có $$\defMath{x_1 < x_2 \implies f\left(x_1\right) > f\left(x_2\right)}.$$ 

{
   \begin{minipageindent}{0.48\textwidth}
      \begin{figure}[H]
         \centering
         \begin{tikzpicture}
            \draw[->] (-2, 0) -- (4, 0) node[right] {$x$};
            \draw[->] (0, -1) -- (0, 4)  node[above] {$f(x)$};
            \foreach \x/\y in {-1/0.707106781, 3/2.838138064} {
               \draw[dashed] (\x,\y) -- (\x,0);
               \draw[dashed] (\x,\y) -- (0,\y);
               \filldraw[color=colorEmphasisCyan] (\x, \y) circle (\pointSize);
               }
            \draw[graph thickness, samples=80, color=colorEmphasisCyan, domain=-2.000:4.000] plot (\x, {(2^((\x)/2)) / 1});
            \node[right] at (0,0.707106781) {$y_1$};
            \node[left] at (0,2.838138064) {$y_2$};
            \node[below] at (-1,0) {$x_1$};
            \node[below] at (3,0) {$x_2$};
      \end{tikzpicture}
      \caption{Ví dụ hàm $f$ đồng biến}
   \end{figure}
   \end{minipageindent}
   \hfill
   \begin{minipageindent}{0.48\textwidth}
      \begin{figure}[H]
         \centering
         \begin{tikzpicture}
            \draw[->] (-2, 0) -- (4, 0) node[right] {$x$};
            \draw[->] (0, -1) -- (0, 4)  node[above] {$f(x)$};
            \draw[dashed] (0.75, 0) -- (0.75, 1.155);
            \draw[dashed] (3, 0) -- (3, 0.5773502691896257);
            \draw[dashed] (0, 1.155) -- (0.75, 1.155);
            \draw[dashed] (0, 0.5773502691896257) -- (3, 0.5773502691896257);
            \filldraw[color=colorEmphasis] ({0.75}, { 1.155 }) circle (\pointSize);
            \filldraw[color=colorEmphasis] ({3.0}, { 0.5773502691896257 }) circle (\pointSize);
            \draw[graph thickness, samples=80, color=colorEmphasis, domain=0.062:4.000] plot (\x, {(1/(((\x)/1)^(1/2))) / 1});
            \node[left] at (0,1.155) {$y_1$};
            \node[left] at (0,0.5773502691896257) {$y_2$};
            \node[below] at (0.75,0) {$x_1$};
            \node[below] at (3,0) {$x_2$};
         \end{tikzpicture}
         \caption{Ví dụ hàm $f$ nghịch biến}
      \end{figure}      
   \end{minipageindent}
}

Bạn đọc hoàn toàn có thể thu hẹp định nghĩa này từ tập $D$ thành một khoảng $\left(a; b\right)$. Lí do để chỉ xét trong một khoảng như vậy là từ ứng dụng trong thực tiễn, ít khi nào người ta xét sự tăng giảm của hàm số trên nhiều khoảng tách biệt với nhau.

\exercise Chứng minh rằng
\begin{enumerate}
   \item $2x$ đồng biến trên $\mathbb{R}$;
   \item $(x-1)^2$ nghịch biến trên $\left(-\infty; 1\right)$;
   \item $|x|\cdot \left|\left|x - 1\right| - 1\right|$ đồng biến trên $(0; 1)$ và nghịch biến trên $(1; 2)$;
   \item $x^2 + mx + \lfloor x \rfloor$ đồng biến trên $\left[-\frac{m}{2}; \infty\right)$ nếu coi $m\in\mathbb{R}$ là tham số thực.
\end{enumerate}

\solution

\setcounter{subexercise}{1}
\arabic{subexercise}. Xét hai giá trị $x_1, x_2 \in \mathbb{R}$ sao cho $x_1 < x_2$. Khi này, hiển nhiên có được $2x_1 < 2x_2$. Do đó, kết luận được $2x$ là đồng biến trên $\mathbb{R}$.

\stepcounter{subexercise}
\arabic{subexercise}. Xét hai giá trị $x_1, x_2 \in \left(-\infty; 1\right)$ sao cho $x_1 < x_2$. Thực hiện một số biến đổi:
\begin{align*}
   &x_1 - 1 < x_2 - 1 < 0 \\
   \iff &x_1 - 1 > x_2 - 1 > 0 \\
   \implies &\begin{cases}
      \left(x_1 - 1\right)^2 > \left(x_1 - 1\right)\left(x_2 - 1\right) \equationexplanation{cùng nhân hai vế với $x_1 - 1$ dương}\\
      \left(x_1 - 1\right)\left(x_2 - 1\right) > \left(x_2 - 1\right)^2 \equationexplanation{cùng nhân hai vế với $x_2 - 1$ dương}
   \end{cases} \\
   \implies &\left(x_1 - 1\right)^2 > \left(x_2 - 1\right)^2.
\end{align*}
Chúng ta dễ dàng thấy điều phải chứng minh.

\stepcounter{subexercise}
\arabic{subexercise}. Trên khoảng $(0; 1)$, $|x| = x$ và $\left|\left|x - 1\right| - 1\right| = \left|1 - x - 1\right| = \left|-x\right| = x$. Do đó, $|x|\cdot \left|\left|x - 1\right| - 1\right| = x^2$. Khi này, với $0 < x_1 < x_2 < 1$, dễ dàng có được $x_1^2 < x_2^2$. Do đó, $|x|\cdot \left|\left|x - 1\right| - 1\right|$ đồng biến trên $(0; 1)$.

Ngoài ra, trên khoảng $(1; 2)$, $|x| = x$ giống như trước. Tuy nhiên, $\left|\left|x - 1\right| - 1\right| = \left|x - 1 - 1\right| = \left|x - 2\right| = 2 - x$. Do đó, với $1 < x_1 < x_2 < 2$, chúng ta có:

\begin{align*}
   0 < x_1 - 1 &< x_2 - 1 \\
   \implies  \left(x_1 - 1\right)^2 &< \left(x_2 - 1\right)^2 \\
   \iff -\left(x_1 - 1\right)^2 &> -\left(x_2 - 1\right)^2 \\
   \iff -x_1^2 + 2x_1 - 1 &> -x_2^2 + 2x_2 - 1 \\
   \iff x_1 \left(2 - x_1\right) &> x_2\left(2-x_2\right) \\
   \iff |x_1|\cdot \left|\left|x_1 - 1\right| - 1\right| &> |x_2|\cdot \left|\left|x_2 - 1\right| - 1\right|.
\end{align*}

Vậy $|x|\cdot \left|\left|x - 1\right| - 1\right|$ nghịch biến trên $(1; 2)$. Chúng ta có điều phải chứng minh.

\stepcounter{subexercise}
\arabic{subexercise}. Với $-\frac{m}{2} \leq x_1 < x_2$, chúng ta có:
\begin{align}
   0 \leq x_1 + \frac{m}{2} &< x_2 + \frac{m}{2} \nonumber\\
   \implies \left(x_1 + \frac{m}{2}\right)^2 &< \left(x_2 + \frac{m}{2}\right)^2 \nonumber\\
   \iff x_1^2 + mx_1 &< x_2^2 + mx_2. \label{eq:toan_hoc_nen_tang:tinh_chat_ham:tang_giam:1.4.1}
\end{align}

Đặt $\begin{cases}
   a = \left\lfloor x_1 \right\rfloor \\
   b = \left\lfloor x_2 \right\rfloor
\end{cases}$. Từ đó, có thể khẳng định được $a$ và $b$ là hai số nguyên. Giả sử $a > b$. Để ý rằng, do $\begin{cases}
   a\leq x_1 < a + 1 \\
   b \leq x_2 < b + 1
\end{cases}$ cho nên có thể viết $$\begin{cases}
   x_1 = a + l_a \\
   x_2 = b + l_b
\end{cases}$$ với phần lẻ $l_a$ và $l_b$ nằm trong nửa đoạn $\left[0; 1\right)$. Với $a > b$ là số nguyên, có thể suy ra được $a \geq b + 1$. Từ đó, có chuỗi $$a + l_a \geq a \geq b + 1 > b + l_b \implies x_1 > x_2.$$ Tuy nhiên, điều này trái với giả thiết trước đó. Cho nên, nếu để $x_1 < x_2$ thì \begin{equation}
   \left\lfloor x_1 \right\rfloor\leq \left\lfloor x_2 \right\rfloor \label{eq:toan_hoc_nen_tang:tinh_chat_ham:tang_giam:1.4.2}
\end{equation}.

Kết hợp \refeq{eq:toan_hoc_nen_tang:tinh_chat_ham:tang_giam:1.4.1} và \refeq{eq:toan_hoc_nen_tang:tinh_chat_ham:tang_giam:1.4.2}, cộng vế theo vế để có $$x_1^2 + mx_1 + \left\lfloor x_1 \right\rfloor < x_2^2 + mx_2 + \left\lfloor x_2 \right\rfloor.$$

Chúng ta đã chứng minh được tính đồng biến như yêu cầu.