\subsection{Hàm bị chặn}

\ % Lùi đầu dòng

Con người luôn có mong muốn tìm ra những kì quan vĩ đại, những kết quả ngày càng lớn. Đi kèm với đó là khao khát nắm trọn được sự vô tận trong lòng bàn tay. Tuy nhiên, không giống như lí thuyết, vùng đất mà con người có thể thỏa trí tưởng tượng và bay bổng, nơi mà con người có thể đếm đến vô tận và xa hơn cả thế, địa điểm mà vô tận chỉ tóm gọn trong ``số $8$ nằm ngang'', thực tiễn không cho phép con người đi đến vô tận. Không có cái gì mãi phồng lên vô cùng lớn hay thu bé vô cùng nhỏ. Nói ngắn gọn, mọi thứ đều bị chặn.

Và, tuy thuộc về phạm trù lí thuyết, một số hàm số vẫn bị chặn. Xét trên tập số thực, một hàm số một biến $f$ có tập xác định $D$ được gọi là \defText{bị chặn trên} nếu tồn tại $M \in \mathbb{R}$ sao cho $f(x) \leq M$ với mọi $x \in D$. Tương tự, $f$ được gọi là \defText{bị chặn dưới} nếu tồn tại $m \in \mathbb{R}$ sao cho $f(x) \geq m$ với mọi $x \in D$. Một hàm vừa bị chặn trên và dưới thì được gọi là \defText{hàm bị chặn}.

\exercise Xác định xem các hàm sau có bị chặn trên, bị chặn dưới hay không.

\begin{multicols}{2}
   \begin{enumerate}
      \item $f(x) = x^3 - 1$ xác định trên tập $\mathbb{R}$;
      \item $f(x) = x^4 - 1$ xác định trên tập $\mathbb{R}$;
      \item $f(x) = \frac{1}{x}$ xác định trên tập $\left(0; \infty\right)$;
      \item $f(x) = \frac{x}{x^2 + 1}$ xác định trên tập $\left(0; \infty\right)$;
      \item $
   \end{enumerate}
\end{multicols}

